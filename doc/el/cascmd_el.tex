\documentclass[a4paper,11pt]{book}
\textheight 23 cm
\usepackage{amsmath}
\usepackage{amssymb}
\usepackage{stmaryrd}
\usepackage{makeidx}

\usepackage[francais,english,greek]{babel}

\usepackage{ifpdf}
\ifpdf
 \usepackage[pdftex,colorlinks]{hyperref}
\else
 \usepackage[ps2pdf,breaklinks=true,colorlinks=true,linkcolor=red,citecolor=green]{hyperref}
 \usepackage{pst-plot}
\fi

\usepackage[utf8]{inputenc}
\usepackage[T1]{fontenc}
\usepackage{latexsym}
\newcommand{\gr}{\selectlanguage{greek}}
\newcommand{\en}{\selectlanguage{english}}
\newcommand{\fr}{\selectlanguage{francais}}
\newcommand{\R}{{\mathbb R}}
\newcommand{\C}{{\mathbb C}}
\newcommand{\Z}{{\mathbb Z}}
\newcommand{\N}{{\mathbb N}}


\title {\textgreek{Συμβολική άλγεβρα και μαθηματικά με το} {\tt\en Xcas}}
\author
{     \textlatin{Ren\'ee De Graeve, Bernard Parisse}\\
      \textgreek{Πανεπιστήμιο} \textlatin{Grenoble I}\\ \\ \\ \\
       {\bf\textgreek{Μετάφραση από τα Αγγλικά: }}\\
      {\textgreek{Βασιλική Αλεξίου, Αχιλλέας Καραβασίλης, Όλγα Μανιάτη}}\\  \\
     \textgreek{{\bf Διασκευή:} Αλκιβιάδης Γ. Ακρίτας}\\
     \textgreek{Πανεπιστήμιο Θεσσαλίας}\\
     \textlatin{akritas@uth.gr}\\
}
\en
\date{}
\makeindex
\usepackage{cutting}

\begin{document}
\newcommand{\asinh}{\,\,\mbox{asinh\,}}
\newcommand{\atanh}{\,\,\mbox{atanh\,}}
\maketitle

\vfill

\en
\copyright\ 2002, 2007 Ren\'ee De Graeve, Bernard Parisse\\
\verb|renee.degraeve@wanadoo.fr|\\
\verb|bernard.parisse@ujf-grenoble.fr|\\



\newpage
\en\printindex
\newpage
\gr\tableofcontents


\chapter{\textgreek{Συναρτήσεις του σύστηματος υπολογιστικής άλγεβρας} }\label{sec:cas}

\section{\textgreek{Συμβολικές σταθερές }: {\tt \textlatin{e pi} \textgreek{άπειρο}
\textlatin{ i}}}\index{e}\index{pi}
\index{i}
\gr\index{+άπειρο}\index{-άπειρο}\index{άπειρο}
\noindent {\en\tt e} είναι ο αριθμός $\exp(1)$;\\ 
{\en\tt pi} είναι ο αριθμός $\pi$.\\
{\tt άπειρο} είναι χωρίς πρόσημο $\infty$.\\
{\tt +άπειρο} είναι $+\infty$.\\
{\tt -άπειρο} είναι $-\infty$.\\
{\en\tt i} είναι ο μιγαδικός αριθός $i$.

\en
\section{\textgreek{Λογικός τύπος δεδομένου} - Booleans}
\subsection{\textgreek{Λογικές τιμές} : {\tt\textlatin{ true false}}}\index{true}\index{false}\index{TRUE}\index{FALSE}
\grΟι λογικές τιμές είναι {\en\tt true} ή {\en\tt false}.\\
Τα συνώνυμα είναι :\\
{\en\tt true} ή {\en\tt TRUE} ή {\tt 1},\\
{\en\tt false} ή {\en\tt FALSE} ή {\tt 0}.\\
Ο έλεγχος ή συνθήκες είναι λογικές συναρτήσεις.

\subsection{\textgreek{Έλεγχος} : {\tt \textlatin{==, !=, >, >=, <, =<}}}\index{==}\index{>}\index{<}\index{>=}\index{<=}\index{\symbol{33}=}
{\en\tt ==, !=, >, >=, <, =<} είναι ενθηματικοί τελεστές.\\
{\en\tt a==b} ελέγχει την ισότητα μεταξύ του {\en\tt a} και του {\en\tt b} και επιστρέφει {\tt 1} 
αν το {\en\tt a} είναι ίσο με το {\en\tt b} ,αλλιώς {\tt 0}.\\ 
{\tt a!=b} επιστρέφει {\tt 1} αν το {\tt a} και το {\tt b} είναι διαφορετίκα, αλλιώς {\tt 0} 
.\\
 {\en\tt a>=b} επιστρέφει {\tt 1} αν το {\en\tt a} είναι μεγαλύτερο ή ισο από το {\en\tt b} 
, αλλιώς {\tt 0}.\\ 
{\en\tt a>b} επιστρέφει {\tt 1} αν το {\en\tt a} είναι μεγαλύτερο από το {\en\tt b}
, αλλιώς{\tt 0}.\\ 
{\en\tt a<=b} επιστρέφει {\tt 1} αν το {\en\tt a} είναι μικρότερο ή ισο από το {\en\tt b} , αλλιώς 
{\tt 0}.\\
{\en\tt a<b} επιστρέφει {\tt 1} αν το {\en\tt a} είναι μικρότερο από το {\en\tt b} 
, αλλιώς {\tt 0}.\\ 
Για να γράψουμε την αλγεβρική συνάρτηση έχοντας το ίδιο αποτέλεσμα με την
{\en\tt if...then...else}, χρησιμοποιούμε την λογική συνάρτηση {\en\tt ifte}.\\
Παράδειγμα :  
\begin{center}{\en\tt f(x):=ifte(x>0,true,false)}\end{center}
ορίζει τη λογική συνάρτηση $f$ όπως η {\en\tt f(x)= true}
\en if 
$x \in ]0;+\infty[$ and {\en\tt f(x)=false} if $x \in ]-\infty;0]$.\\
\grΕίσοδος :
\begin{center}{\en\tt f(0)==0}\end{center}
'Εξοδος :
\begin{center}{\tt 1}\end{center}
{\gr\bfΠροσοχή !}\\
{\en\tt a=b} δεν είναι λογικός τύπος δεδομένου !!!!\\
{\en\tt a==b} είναι λογικός τύπος δεδομένου.\\

\subsection{\textgreek{Λογικοί τελεστές}  : {\tt \textlatin{or xor} \textgreek{και}  \textlatin{not}}}
\index{or|textbf}\index{not|textbf}\index{and|textbf}\index{$\bigparallel$}\index{\&\&|textbf}\index{\symbol{33}=|textbf}\index{xor|textbf}
{\en \tt or} (ή {\en \tt ||}), {\en \tt xor}, {\en \tt and} (ή {\en \tt \&\&}) είναι ενθηματικός τελεστής.\\
{\en \tt not} είναι προθεματικός τελεστής.\\ 
Αν το {\en \tt a} και το {\en \tt b} είναι δύο λογικοί τύποι δεδομένου :\\
{\en \tt (a or b)}  {\en \tt (a || b)}επιστρέφει  {\tt 0} (ή {\en \tt false}) αν το {\en \tt a} και το
{\en \tt b} είναι ίσα με το 0, αλλιώς επιστρέφει {\tt 1} (ή {\en \tt true}) .\\ 
{\en \tt (a xor b)}  επιστρέφει {\tt 1} αν το {\en \tt a} είναι ίσο με το 1 και το {\en \tt b} είναι ίσο με το
0 ή αν το {\en \tt a} είναι ίσο με το 0 και το {\en \tt b} είναι ίσο με το 1 και επιστρέφει 0
 αν το{\en \tt a} και το {\en \tt b} είναι ίσα με το 0
 ή αν το  {\en \tt a} και το {\en \tt b}  είναι ίσα με το 1 (είναι το "αποκλειστικό \textlatin{or}").\\ 
{\en\tt (a and b)} ή {\en \tt (a \&\& b)}  επιστρέφει {\tt 1} (ή {\en \tt true}) αν το {\en \tt a}
 και το {\en \tt b}  είναι ίσα με το 1, αλλιώς {\tt 0} (ή {\en \tt false}).\\
{\en \tt not(a)} επιστρέφει {\tt 1} (ή {\en \tt true}) αν το {\en \tt a}  είναι ίσο με το 0 (ή 
{\en \tt false}) και {\tt 0} (ή {\en \tt false})  αν το {\en \tt a}  είναι ίσο με το 1 (ή 
{\en \tt true}).\\ 
Είσοδος :
\begin{center}{\en \tt 1>=0 or 1<0}\end{center}
Έξοδος :
\begin{center}{\tt 1}\end{center}
Είσοδος :
\begin{center}{\en \tt 1>=0 xor 1>0}\end{center}
Έξοδος :
\begin{center}{\tt 0}\end{center}
Είσοδος :
\begin{center}{\en \tt 1>=0 and 1>0}\end{center}
Έξοδος :
\begin{center}{\tt 1}\end{center}
Είσοδος :
\begin{center}{\en \tt not(0==0)}\end{center}
Έξοδος :
\begin{center}{\tt 0}\end{center}
\gr

\subsection{\textgreek{Μετατροπή λογικής παράστασης σε λίστα}: {\tt \textlatin{exp2list}}}\index{exp2list}
\noindent{\en \tt exp2list} επιστρέφει τη λίστα {\en \tt [expr0,expr1]}
\gr όταν το όρισμα είναι  {\en \tt (var=expr0)}  ή {\en\tt (var=expr1)}.\\
{\en\tt exp2list} χρησιμοποιείται στον τρόπο λειτουργίας {\en \tt TI} για την ευκολότερη εκτέλεση της εντολής {\en\tt solve}.\\
Είσοδος :
\begin{center}{\en\tt exp2list((x=2) or (x=0))}\end{center}
Έξοδος :
\begin{center}{\tt [2,0]}\end{center}
Είσοδος :
\begin{center}{\en\tt exp2list((x>0) or (x<2))}\end{center}
Έξοδος :
\begin{center}{\tt [0,2]}\end{center}
Είσοδος στον τρόπο λειτουργίας {\en \tt TI} :
\begin{center}{\en\tt exp2list(solve((x-1)*(x-2)))}\end{center}
Έξοδος :
\begin{center}{\tt [1,2]}\end{center}

\subsection{{\textgreek{Αποτίμηση λογικού τυπου δεδομένων}} : {\tt \textlatin{evalb}}}\index{evalb}
\noindent Στο \en Maple, \gr η εντολή {\en\tt evalb} αποτιμά μία λογική παράσταση.
Επειδή το {\en\tt Xcas} αποτιμά αυτόματα λογικές παραστάσεις, η {\en\tt evalb} χρησιμοποιείται  μόνο για λόγους συμβατότητας και είναι ισοδύναμη με την {\en\tt eval}\\
Είσοδος :
\begin{center}{\en\tt evalb(sqrt(2)>1.41)}\end{center}
ή :
\begin{center}{\en\tt sqrt(2)>1.41}\end{center}
Έξοδος :
\begin{center}{\tt 1}\end{center}
Είσοδος :
\begin{center}{\en\tt evalb(sqrt(2)>1.42)}\end{center}
ή :
\begin{center}{\en\tt sqrt(2)>1.42}\end{center}
Έξοδος :
\begin{center}{\tt 0}\end{center}

\section{ \textgreek{Τελεστές} \textlatin{bit} \textgreek{ανά} \textlatin{bit}}
\subsection{\textgreek{Τελεστές} {\tt \textlatin{ bitor, bitxor, bitand}}}\index{bitor|textbf}\index{bitxor|textbf}\index{bitand|textbf}
Οι ακέραιοι μπορούν να γραφτούν για είσοδο στο {\en\tt Xcas} στην δεκαεξαδική μορφή ως \en 0x...\gr
για παράδειγμα το \en 0x1f \gr αναπαρίσταται ως 16+15=31 στο δεκαδικό σύστημα. 
Επίσης, οι ακέραιοι μπορούν να γραφονται στην έξοδο του {\en\tt Xcas} στην δεκαεξαδική μορφή. 
(στις Ρυθμίσεις \en Cas \gr --- ή στην μπάρα ρυθμίσεων --- επιλέξτε την βάση ακεραίων που θέλετε).\\
{\en\tt bitor} είναι το λογικό διαζευκτικό {\en\tt or} (\en bit \gr ανά \en bit).\\
\gr
Είσοδος :
\begin{center}{\en\tt bitor(0x12,0x38)}\end{center}
ή :
\begin{center}{\en\tt bitor(18,56)}\end{center}
Έξοδος :
\begin{center}{\tt 58}\end{center}
επειδή :\\
το {\tt 18} γράφεται ως {\en\tt 0x12} στο δεκαεξαδικό ή {\en\tt 0b010010} στο δυαδικό,\\
το {\tt 56} γράφεται ως {\en\tt 0x38} στο δεκαεξαδικό ή {\en\tt 0b111000} στο δυαδικό,\\
συνεπώς  η {\en\tt bitor(18,56)} είναι {\en\tt 0b111010} στο δυαδικό και το αποτέλεσμα είναι ίσο με το
{\tt 58}.\\

{\en\tt bitxor} είναι το λογικό αποκλειστικό {\en\tt or} (\en bit \gr ανά \en bit).\\
\gr
Είσοδος :
\begin{center}{\en\tt bitxor(0x12,0x38)}\end{center}
ή είσοδος 
\begin{center}{\en\tt bitxor(18,56)}\end{center}
Έξοδος :
\begin{center}{\tt 42}\end{center}
επειδή :\\
το {\tt 18} γράφεται ως {\en\tt 0x12} στο δεκαεξαδικό και {\en\tt 0b010010} στο δυαδικό,\\
το {\tt 56} γράφεται ως {\en\tt 0x38} στο δεκαεξαδικό και {\en\tt 0b111000} στο δυαδικό,\\
{\en\tt bitxor(18,56)} γράφεται ως {\en\tt 0b101010} στο δυαδικό και το αποτέλεσμα είναι ίσο με το  
{\tt 42}.\\

{\en\tt bitand} είναι το λογικό {\en\tt and} \en(bit \gr ανά \en bit).\\
\gr
Είσοδος :
\begin{center}{\en\tt bitand(0x12,0x38)}\end{center}
ή είσοδος 
\begin{center}{\en\tt bitand(18,56)}\end{center}
Έξοδος :
\begin{center}{\tt 16}\end{center}
επείδη :\\
το {\tt 18} γράφεται ως {\en\tt 0x12} στο δεκαεξαδικό και {\en\tt 0b010010} στο δυαδικό,\\
το {\tt 56} γράφεται ως {\en\tt 0x38} στο δεκαεξαδικό και {\en\tt 0b111000} στο δυαδικό,\\
{\en\tt bitand(18,56)} γράφεται ως {\en\tt 0b010000} στο δυαδικό και το αποτέλεσμα είναι ίσο με το
{\tt 16}.

\subsection{\textgreek{Απόσταση} \textlatin{Hamming bit} \textgreek{ανά} \textlatin{ bit} : {\tt \textlatin{hamdist}}}\index{hamdist|textbf}
Η απόσταση \en Hamming \gr είναι ο αριθμός των διαφορών των \en bits \gr των δύο ορισμάτων.\\
Είσοδος :
\begin{center}{\en\tt hamdist(0x12,0x38)}\end{center}
ή είσοδος 
\begin{center}{\en\tt
  \item  hamdist(18,56)}\end{center}
Έξοδος :
\begin{center}{\tt 3}\end{center}
επειδή :\\
το {\tt 18} γράφεται ως {\en\tt 0x12} στο δεκαεξαδικό και {\en\tt 0b010010} στο δυαδικό,\\
το {\tt 56} γράφεται ως {\en\tt 0x38} στο δεκαεξαδικό και {\en\tt 0b111000} στο δυαδικό,\\
{\en\tt hamdist(18,56)} γράφεται ως  {\tt 1+0+1+0+1+0} και το αποτέλεσμα είναι ίσο με το {\tt 3}.

\section{\textgreek{Συμβολοσειρές}}
\subsection{\textgreek{Χαρακτήρες και συμβολοσειρές :  {\tt \textlatin{"}}\index{"}}}
\noindent Το {\en\tt "} χρησιμοποιείται για να καθοριστεί μια συμβολοσειρά.
 Ένας χαρακτήρας είναι μια συμβολοσειρά μήκους ένα.\\
Μην συγχέετε το {\en\tt "} με
το {\tt '} (ή {\en\tt quote}) τα οποία χρησιμοποιούνται για να αποφευχθεί η αποτίμηση μιας παράστασης. Για παράδειγμα, το
{\en\tt "a"} επιστρέφει το {\en\tt a} ως μία συμβολοσειρά από έναν χαρακτήρα, 
αλλά το {\tt '{\en a}'} ή το {\en\tt quote(a)} επιστρέφει
τη μεταβλητή {\en\tt a} αδιατίμητη.\\

Όταν μία συμβολοσειρά δίνεται σαν είσοδος στην γραμμή εντολών, αποτιμάται από μόνη της,
άρα η έξοδος είναι η ίδια συμβολοσειρά. Χρησιμοποιήστε το {\en\tt +}
για να ενώσετε δύο συμβολοσειρές ή ένα άλλο αντικείμενο και μία συμβολοσειρά.\\
Παράδειγμα :\\
Είσοδος :
\begin{center}{\en\tt "Hello"}\end{center}
{\en\tt "Hello"} είναι η είσοδος αλλά και η έξοδος.\\
Είσοδος :
\begin{center}{\en\tt "Hello"+", how are you?"}\end{center}
Έξοδος :
\begin{center}{\en\tt "Hello, how are you?"}\end{center}
Ο δείκτης χρησιμοποιείται ώστε να πάρουμε τον \en n\gr-στό χαρακτήρα μίας συμβολοσειράς, 
(όπως και στην λίστα). Οι δείκτες αρχίζουν από 0 στο \en Xcas \gr ,ενώ στα άλλα συστήματα από 1 .\\
Παράδειγμα :\\
Είσοδος :
\begin{center}{\en\tt "Hello"[1]}\end{center}
Έξοδος :
\begin{center}{\en\tt "e"}\end{center}

\subsection{\textgreek{Πρώτος χαρακτήρας, μεσαίος και τελευταίος μιας συμβολοσειράς}: {\tt\textlatin{head mid tail}}}\index{head|textbf} \index{tail|textbf}\index{mid}
\begin{itemize}
\item  {\en\tt head(s)} επιστρέφει το πρώτο χαρακτήρα της συμβολοσειράς {\en\tt s}.\\ 
 Είσοδος :
\begin{center}{\en\tt head("Hello")}\end{center}
Έξοδος :
\begin{center}{\en\tt "H"}\end{center}
\item {\en\tt mid(s,p,q)} επιστρέφει το μέρος της συμβολοσειράς {\en\tt s} 
που αρχίζει με τον χαρακτήρα του δείκτη {\en\tt p} και έχει {\en\tt q} χαρακτήρες.\\
Να θυμάστε ότι ο πρώτος δείκτης είναι το 0 στο \en Xcas.\gr\\
Είσοδος :
\begin{center}{\en\tt mid("Hello",1,3)}\end{center}
Έξοδος :
\begin{center}{\en\tt "ell"}\end{center}
\item {\en\tt tail(s)} επιστρέφει την συμβολοσείρα {\en\tt s} χωρίς τον πρώτο χαρακτήρα.\\ 
Είσοδος :
\begin{center}{\en\tt tail("Hello")}\end{center}
Έξοδος :
\begin{center}{\en\tt "ello"}\end{center}
\end{itemize}

\subsection{\textgreek{Συνένωση ακολουθίας λέξεων}: {\tt \textlatin{cumSum}}}\index{cumSum}
\noindent{\en\tt cumSum} χρησιμοποιείται όπως και στις παραστάσεις για να κάνει μερική συνένωση.\\
{\en\tt cumSum} λαμβάνει ως ορίσματα μία λίστα από συμβολοσειρές.\\
{\en\tt cumSum} επιστρέφει μία λίστα από συμβολοσειρές, όπου κάθε στοιχείο με δείκτη $k$ είναι η συνένωση της συμβολοσειράς με δεικτη $k$, με τις συμβολοσειρές των δεικτών 0 έως $k-1$ .\\
Είσοδος :
\begin{center}{\en\tt cumSum("Hello, ","is ","that ","you?")}\end{center}
Έξοδος :
\begin{center}{\en\tt "Hello, ","Hello, is ","Hello, is that ","Hello, is that you?}\end{center}

\subsection{\textgreek{Κώδικας} \textlatin{ ASCII} \textgreek{ενός χαρακτήρα} : {\tt \textlatin{ord}}}\index{ord|textbf}
\noindent {\en\tt ord} παίρνει σαν όρισμα μία συμβολοσειρά {\en\tt s} (
μία λίστα {\en\tt l} από συμβολοσειρές).\\
{\en\tt ord} επιστρέφει σε κώδικα \en ASCII \grτον πρώτο χαρακτήρα της {\en\tt s} ( την λίστα
σε κώδικα \en ASCII \gr του πρώτου χαρακτήρα των στοιχείων της {\en\tt l}).\\
Είσοδος :
\begin{center}{\en\tt ord("a")}\end{center}
Έξοδος :
\begin{center}{\tt 97}\end{center}
Είσοδος :
\begin{center}{\en\tt ord("abcd")}\end{center}
Έξοδος :
\begin{center}{\tt 97}\end{center} 
Είσοδος :
\begin{center}{\en\tt ord(["abcd","cde"])}\end{center}
Έξοδος :
\begin{center}{\tt [97,99]}\end{center} 
Είσοδος :
\begin{center}{\en\tt ord(["a","b","c","d"])}\end{center}
Έξοδος :
\begin{center}{\tt [97,98,99,100]}\end{center} 

\subsection{\textgreek{Κώδικας} \textlatin{ASCII} \textgreek{μίας συμβολοσειράς} : {\tt \textlatin{asc}}}\index{asc}
\noindent {\en\tt asc} παίρνει σαν όρισμα μία σειμβολοσειρά {\en\tt s}.\\
{\en\tt asc} επιστρέφει την λίστα σε κώδικα \en ASCII, \gr των χαρακτήρων της {\en\tt s}.\\
Έισοδος :
\begin{center}{\en\tt asc("abcd")}\end{center}
Έξοδος :
\begin{center}{\tt [97,98,99,100]}\end{center} 
Έισοδος :
\begin{center}{\en\tt asc("a")}\end{center}
Έξοδος :
\begin{center}{\en\tt [97]}\end{center}

\subsection{\textgreek{Ορισμός συμβολοσειράς από χαρακτήρες σε κώδικα} \textlatin{ ASCII} : {\tt \textlatin{char}}}\index{char}
\noindent {\en\tt char} παίρνει ως όρισμα μία λίστα {\en\tt l} σε κώδικα \en ASCII.\\ \gr 
{\en\tt char} επιστρέφει την συμβολοσειρά των οποιών οι χαρακτήρες είναι τα στοιχεία της λίστας {\en\tt l} τα οποία είναι σε κώδικα \en ASCII.\\ \gr
Έισοδος :
\begin{center}{\en\tt char([97,98,99,100])}\end{center}
Έξοδος :
\begin{center}{\en\tt "abcd"}\end{center} 
Έισοδος :
\begin{center}{\en\tt char(97)}\end{center}
Έξοδος :
\begin{center}{\en\tt "a"}\end{center}
Έισοδος :
\begin{center}{\en\tt char(353)}\end{center}
Έξοδος :
\begin{center}{\en\tt "a"}\end{center}
Αποτέλεσμα 353-256=97.

\subsection{\textgreek{Εύρεση ενός χαρακτήρα σε μία συμβολοσειρά}: {\tt\textlatin{inString}}}\index{inString}
\noindent {\en\tt inString} παίρνει δύο ορίσματα: μία σειμβολοσείρα {\en\tt S} και ένα χαρακτήρα 
 {\en\tt c}.\\
{\en\tt inString} εξετάζει αν ο χαρακτήρας {\en\tt c} είναι στην συμβολοσειρά {\en\tt S}.\\
 {\en\tt inString} επιστρέφει {\tt {\tt \en "}τον δείκτη της πρώτης εμφάνισης{\tt \en "}}
ή {\tt -1} αν {\en\tt c} δεν υπάρχει στο {\en\tt S}.\\
Έισοδος :
\begin{center}{\en\tt inString("abcded","d")}\end{center}
Έξοδος :
\begin{center}{\tt  3}\end{center}
Έισοδος :
\begin{center}{\en\tt inString("abcd","e")}\end{center}
Έξοδος :
\begin{center}{\tt  -1}\end{center}

\subsection{\textgreek{Αλληλουχία αντικειμένων σε  συμβολοσειρά} : {\tt\textlatin{ cat}}}\index{cat|textbf}
\noindent {\en\tt cat} παίρνει σαν όρισμα μία ακολουθία αντικειμένων.\\ 
{\en\tt cat} συνενώνει αυτά τα αντικείμενα σε μία συμβολοσειρά.\\
Είσοδος :
\begin{center}{\en\tt cat("abcd",3,"d")}\end{center}
Έξοδος :
\begin{center}{\en\tt  "abcd3d"}\end{center}
Είσοδος :
\begin{center}{\en\tt c:=5}\end{center}
\begin{center}{\en\tt cat("abcd",c,"e")}\end{center}
Έξοδος :
\begin{center}{\en\tt  "abcd5e"}\end{center}
Είσοδος :
\begin{center}{\en\tt purge(c)}\end{center}
\begin{center}{\en\tt cat(15,c,3)}\end{center}
Έξοδος :
\begin{center}{\en\tt  "15c3"}\end{center}

\subsection{\textgreek{Προσθήκη αντικειμένου σε μία συμβολοσειρά} : {\tt +}}\index{+}
\noindent {\tt +} είναι ενθηματικός τελεστής ( {\tt '+'} είναι προθεματικός τελεστής).\\
Αν {\tt +} ( {\tt '+'}) παίρνει σαν όρισμα μία συμβολοσειρά (αντίστοιχα μία ακολουθία από αντικείμενα με μία συμβολοσειρά ως πρώτο ή δεύτερο όρισμα) , 
το αποτέλεσμα είναι η  συνένωση αυτών των αντικειμένων σε μία συμβολοσειρά.\\
\\
{\gr\bf Προσοχή}\\
{\tt +}  είναι ενθηματικός και {\tt '+'} είναι προθεματικός  τελεστής.\\ 
Είσοδος :
\begin{center}{\tt '+'\en("abcd",3,"d")}\end{center}
ή είσοδος :
\begin{center}{\en\tt "abcd"+3+"d"}\end{center}
Έξοδος :
\begin{center}{\en\tt  "abcd3d"}\end{center}
Είσοδος :
\begin{center}{\en\tt c:=5}\end{center}
και είσοδος :
\begin{center}{\en\tt "abcd"+c+"e"}\end{center}
ή είσοδος:
\begin{center}{\tt '+'\en ("abcd",c,"d")}\end{center}
Έξοδος :
\begin{center}{\en\tt  "abcd5e"}\end{center}

\subsection{\textgreek{Μετασχηματισμός ακέραιου σε συμβολοσειρά}: {\tt \textlatin{cat +}}}\index{+}\index{cat}
\noindent Χρησιμοποιήστε την {\en\tt cat} με τον ακέραιο σαν όρισμα, ή  προσθέστε τον ακέραιο σε μία άδεια συμβολοσειρά\\
Είσοδος :
\begin{center}{\tt ""+123}\end{center}
ή :
\begin{center}{\en\tt cat(123)}\end{center}
Έξοδος :
\begin{center}{\tt  {\tt \en "}123{\tt \en "}}\end{center}

\subsection{\textgreek{Μετασχηματισμός συμβολοσειράς σε αριθμό}: {\tt\textlatin{ expr}}}\index{expr|textbf}\label{sec:expr1}
 Χρησιμοποιήστε την {\en\tt expr}, με μία συμβολοσειρά που αναπαραστά έναν αριθμό. 
\begin{itemize}
\item Για ακεραίους, βάλτε την συμβολοσειρά που παριστάνει τον ακέραιο χωρίς αρχικό ψηφίο 0 για ακεραίους στην βάση 10, με πρόθεμα {\en\tt 0x} στην δεκαεξαδική βάση, με πρόθεμα
{\tt 0} στην οχταδική ή με πρόθεμα {\en\tt 0b} στην δυαδική.\\
Είσοδος :
\begin{center}{\en\tt expr("123")}\end{center}
Έξοδος :
\begin{center}{\tt  123}\end{center}
Είσοδος :
\begin{center}{\en\tt expr("0123")}\end{center}
Έξοδος :
\begin{center}{\tt  83}\end{center}
Επειδή : $1*8^2+2*8+3=83$ \\
Είσοδος :
\begin{center}{\en\tt expr("0x12f")}\end{center}
Έξοδος :
\begin{center}{\tt 303}\end{center}
Επειδή : $1*16^2+2*16+15=303$
\item Για δεκαδικούς αριθμούς, χρησιμοποιείστε μία συμβολοσειρά με {\tt .}  ή {\en\tt e} μέσα της.\\
Είσοδος :
\begin{center}{\en\tt expr("123.4567")}\end{center}
Έξοδος :
\begin{center}{\tt  123.4567}\end{center}
Είσοδος :
\begin{center}{\en\tt expr("123e-5")}\end{center}
Έξοδος :
\begin{center}{\tt 0.00123}\end{center}
\item Σημειώστε ότι η {\en\tt expr} γενικότερα μετατρέπει μία συμβολοσειρά
σε μια εντολή, αν η εντολή υπάρχει.\\
Είσοδος :
\begin{center}{\en\tt expr("a:=1")}\end{center}
Έξοδος :
\begin{center}{\tt 1}\end{center}
Έπειτα, είσοδος :
\begin{center}{\en\tt a}\end{center}
Έξοδος :
\begin{center}{\tt 1}\end{center}
\end{itemize}

\section{\textgreek{Μετατροπή  ακεραίου ως προς την βάση} $b$ : {\tt \textlatin{convert}}}
\en
\label{sec:convertbase}
\index{convert}\index{base@{\sl base}|textbf}
\noindent{\en\tt convert} \gr ή {\en\tt convertir}
\gr
μπορούν να κάνουν διαφορετικού είδους
μετατροπές ανάλογα με την επιλογή που δίνεται ως δεύτερο όρισμα.

Για να μετατρέψετε έναν ακέραιο {\en\tt n} σε μία λίστα με τους συντελεστές του ως προς την βάση {\en\tt b}, η επιλογή είναι {\en\tt base}. Τα ορίσματά του {\en\tt convert} ή
{\en\tt convertir} είναι ένας ακέραιος {\en\tt n}, {\en\tt base} και το {\en\tt b}, η τιμή της βάσης.\\
{\en\tt convert} ή {\en\tt convertir} επιστρέφει την λίστα  των συντελεστών του ακέραιου {\en\tt n} ως προς την βάση {\en\tt b} \\
Είσοδος :
\begin{center}{\en\tt convert(123,base,8)}\end{center}
Έξοδος :
\begin{center}{\en\tt [3,7,1]}\end{center}
Για να ελέγξετε την απάντηση, εισάγετε {\en\tt expr("0173")} ή  {\en\tt horner(revlist([3,7,1]),8)}
ή {\en\tt convert([3,7,1],base,8)} οπότε η απάντηση θα είναι {\tt 123}\\
Είσοδος :
\begin{center}{\en\tt convert(142,base,12)}\end{center}
Έξοδος :
\begin{center}{\tt [10,11]}\end{center}

Για να μετατρέψετε τη λίστα των συντελεστών του ακέραιου {\en\tt n} ως προς την βάση {\en\tt b}, 
η επιλογή είναι επίσης {\en\tt base}. 
{\en\tt convert} ή {\en\tt convertir} επιστρέφει τον ακέραιο {\en\tt n}.\\ 
Είσοδος :
\begin{center}{\en\tt convert([3,7,1],base,8)}\end{center}
ή
\begin{center}{\en\tt horner(revlist([3,7,1]),8)}\end{center}
Έξοδος :
\begin{center}{\tt 123}\end{center}
Είσοδος :
\begin{center}{\en\tt convert([10,11],base,12)}\end{center}
ή
\begin{center}{\en\tt horner(revlist([10,11]),12)}\end{center}
Έξοδος :
\begin{center}{\tt 142}\end{center}

\section{\textgreek{Ακέραιοι (και} \textlatin{ Gaussian} \textgreek{Ακέραιοι})}
Για όλες τις συναρτήσεις σε αύτο το κεφάλαιο, μπορείτε να χρησιμοποιήσετε \en Gaussian
\gr
ακεραίους (αριθμούς της μορφής $a+ib$, όπου $a$ και $b$ ανήκουν στο $\mathbb Z$).

\subsection{\textgreek{Παραγοντικό} : {\tt \textlatin{factorial}}}\index{factorial}
Το {\tt \textlatin{ Xcas}} μπορεί να χειριστεί ακεραίους με απεριόριστη ακρίβεια, όπως οι
εξής:\\ 
Είσοδος :
\begin{center}{\en\tt factorial(100)}\end{center}
Έξοδος :
\begin{verbatim}
   9332621544394415268169923885626670049071596826438162
   1468592963895217599993229915608941463976156518286253
   697920827223758251185210916864000000000000000000000000
\end{verbatim}
\en\subsection{GCD : {\tt gcd igcd}}\index{gcd|textbf}\index{igcd|textbf}\label{sec:igcd}
\gr\noindent{\en\tt gcd} ή {\en\tt igcd} υποδηλώνει τον \en GCD \gr (τον μέγιστο κοινό διαιρέτη, μκδ) 
διαφόρων ακεραίων  (για πολυώνυμα βλέπε \ref{sec:gcd}).\\ 
{\en\tt gcd} ή {\en\tt igcd} επιστρέφει τον {\en\tt GCD} όλων των ακεραίων.\\
Είσοδος :
\begin{center}{\en\tt gcd(18,15)}\end{center}
Έξοδος :
\begin{center}{\tt 3}\end{center} 
Είσοδος :
\begin{center}{\en\tt  gcd(18,15,21,36) }\end{center}
Έξοδος :
\begin{center}{\tt 3}\end{center}
Είσοδος :
\begin{center}{\en\tt  gcd([18,15,21,36])}\end{center}
Έξοδος :
\begin{center}{\tt 3}\end{center}
Μπορούμε επίσης, να θέσουμε ως παραμέτρους δύο λίστες του ιδίου μεγέθους (ή ένα πίνακα με 2 γραμμές), στην περίπτωση αυτή ο \en GCD \gr {\en\tt  gcd} επιστρέφει το μέγιστο κοινό διαιρέτη των στοιχείων με τον ίδιο δείκτη  (ή την ίδια στήλη).
Είσοδος :
\begin{center}{\en\tt  gcd([6,10,12],[21,5,8])}\end{center}
ή :
\begin{center}{\en\tt  gcd([[6,10,12],[21,5,8]])}\end{center}
Έξοδος :
\begin{center}{\tt [3,5,4]}\end{center}
{\bf Παράδειγμα}\\
Βρείτε το μέγιστο κοινό διαιρέτη του $4n+1$ και του $5n+3$ όταν $n \in \mathbb N$.\\
Είσοδος :\\
\begin{center}{\en\tt  f(n):=gcd(4*n+1,5*n+3)}\end{center}
Έπειτα, είσοδος :\\
\en
\begin{verbatim}
  essai(n):={
    local j,a,L; 
    L:=NULL;
    for (j:=-n;j<n;j++) {
      a:=f(j);
      if (a!=1) {
        L:=L,[j,a];
      } 
    }
    return L;
  }
\end{verbatim} 
\gr
Έπειτα, είσοδος :\\
\begin{center}{\en\tt essai(20)}\end{center}
Έξοδος :
\begin{center}{\tt [-16,7],[-9,7],[-2,7],[5,7],[12,7],[19,7]}\end{center}
Έτσι, πρέπει να αποδείξουμε ότι :\\
αν $n \neq 5+k*7$ (για $k \in \mathbb Z$),  ο μέγιστος κοινός διαιρέτης του $4n+1$ και του  $5n+3$ είναι 1,\\
και \\
αν  $n=5+k*7$ (για $k \in \mathbb Z$), ο μέγιστος κοινός διαιρέτης του $4n+1$ 
και του $5n+3$ είναι 7.
\subsection{\textlatin{GCD} : {\tt \textlatin{Gcd}}}\index{Gcd|textbf}
\noindent{\en\tt Gcd} είναι η αδρανής μορφή του {\en\tt gcd}. Ανατρέξτε στην αντίστοιχη ενότητα για πολυώνυμα με συντελεστές που ανήκουν στο $\Z/p\Z$ 
 για τη χρήση αυτής της εντολής.\\
Είσοδος :
\begin{center}{\en\tt Gcd(18,15)}\end{center}
Έξοδος :
\begin{center}{\en\tt gcd(18,15)}\end{center}


\subsection{\textlatin{GCD} \textgreek{λίστας ακεραίων}: {\tt\textlatin{ lgcd}}}\index{lgcd}
\noindent{\en\tt lgcd} έχει μια λίστα ακεραίων (ή μία λίστα πολυωνύμων) 
ως όρισμα.\\ 
{\en\tt lgcd} επιστρέφει τον μέγιστο κοινό διαιρέτη, {\tt \en gcd}, όλων των ακεραίων της λίστας (ή τον {\tt \en gcd} όλων των πολυωνύμων της λίστας).\\
Είσοδος :
\begin{center}{\en\tt lgcd([18,15,21,36])}\end{center}
Έξοδος :
\begin{center}{\tt 3}\end{center} 
{\bf Παρατήρηση}\\
{\en\tt lgcd} δεν δέχεται δύο λίστες (έστω και αν έχουν το ίδιο μέγεθος) ως ορίσματα.

\subsection{\textgreek{Ελάχιστο κοινό πολλαπλάσιο} : {\tt \textlatin{ lcm}}}\index{lcm|textbf}\label{sec:ilcm}
\noindent {\en\tt lcm} επιστρέφει το ελάχιστο κοινό πολλαπλάσιο δύο ακεραίων
 (ή δύο πολυωνύμων, βλέπε επίσης \ref{sec:lcm}).\\
Είσοδος :
\begin{center}{\en\tt lcm(18,15) }\end{center}
Έξοδος :
\begin{center}{\tt 90}\end{center}

\subsection{\textgreek{Διάσπαση σε πρώτους παράγοντες}  : {\tt \textlatin{ifactor}}}\index{ifactor}
\noindent{\en\tt ifactor} έχει έναν ακέραιο ως παράμετρο.\\
{\en\tt ifactor} διασπά έναν ακέραιο σε πρώτους παράγοντες.\\
Είσοδος :
\begin{center}{\en\tt ifactor(90) }\end{center}
Έξοδος :
\begin{center}{\en\tt 2*3\verb|^|2*5}\end{center}
Είσοδος :
\begin{center}{\en\tt ifactor(-90) }\end{center}
Έξοδος :
\begin{center}{\en\tt (-1)*2*3\verb|^|2*5}\end{center}

\subsection{\textgreek{Λίστα  πρώτων παραγόντων} : {\tt \textlatin{ifactors}}}\index{ifactors}
\noindent{\en\tt ifactors} έχει έναν ακέραιο (ή μια λίστα ακεραίων) ως παράμετρο.\\
{\en\tt ifactors} διασπά τον ακέραιο (ή τους ακεραίους της λίστας) σε πρώτους παράγοντες, αλλά το αποτέλεσμα δίνεται ως λίστα (ή μια λίστα λιστών), στην οποία κάθε πρώτος παράγοντας ακολουθείται από την πολλαπλότητα του.\\
Είσοδος :
\begin{center}{\en\tt ifactors(90) }\end{center}
Έξοδος :
\begin{center}{\tt [2,1,3,2,5,1] }\end{center}
Είσοδος :
\begin{center}{\en\tt ifactors(-90) }\end{center}
Έξοδος :
\begin{center}{\tt [-1,1,2,1,3,2,5,1] }\end{center}
Είσοδος :
\begin{center}{\en\tt ifactors([36,52]) }\end{center}
Έξοδος :
\begin{center}{\en\tt [[2,2,3,2],[2,2,13,1]]}\end{center}
\subsection{\textgreek{Πίνακας παραγόντων} : {\tt\textlatin{ maple\_ifactors}}}\index{maple\_ifactors}
\noindent{\en\tt maple\_ifactors} έχει έναν ακέραιο $n$ (ή μία λίστα ακεραίων)
ως παράμετρο.\\
{\en\tt maple\_ifactors} διασπά τον ακέραιο (ή τους ακεραίους της λίστας) σε πρώτους παράγοντες, αλλά η έξοδος ακολουθεί την σύνταξη του \en Maple :\\
\gr
είναι μια λίστα με +1 ή -1 (για το πρόσημο) και έναν πίνακα με 2 στήλες, όπου οι γραμμές  αποτελούν τους πρώτους παράγοντες και την πολλαπλότητα τους (ή μια λίστα 
λιστών...).\\
Είσοδος :
\begin{center}{\en\tt maple\_ifactors(90) }\end{center}
Έξοδος :
\begin{center}{\en\tt [1,[[2,1],[3,2],[5,1]]]}\end{center}
Είσοδος :
\begin{center}{\en\tt maple\_ifactor([36,52]) }\end{center}
Έξοδος :
\begin{center}{\en\tt [[1,[[2,2],[3,2]]],[1,[[2,2],[13,1]]]]}\end{center}
\subsection{\textgreek{Διαιρέτες ενός ακεραίου} : {\tt \textlatin{idivis divisors}}} \index{idivis}\index{divisors}
\noindent{\en\tt idivis} ή {\en\tt divisors} δίνει την λίστα των διαιρετών ενός ακεραίου (ή μιας λίστας ακεραίων).\\
Είσοδος :
\begin{center}{\en\tt idivis(36) }\end{center}
Έξοδος :
\begin{center}{\tt  [1,2,4,3,6,12,9,18,36] }\end{center}
Είσοδος :
\begin{center}{\en\tt idivis([36,22]) }\end{center}
Έξοδος :
\begin{center}{\tt [[1,2,4,3,6,12,9,18,36],[1,2,11,22]]}\end{center}

\subsection{\textgreek{Ακέραιο Ευκλείδειο πηλίκο} : {\tt \textlatin{iquo intDiv}}}\index{iquo}\index{intDiv}
\noindent{\en\tt iquo} (ή {\en\tt intDiv}) επιστρέφει το ακέραιο πηλίκο $q$ της Ευκλείδειας διαίρεσης δύο ακεραίων $a$ και $b$ που δίνονται ώς ορίσματα. 
($a=b*q+r$ με $0\leq r< b$).\\ 
Για \en Gaussian \gr ακεραίους, επιλέγουμε $q$ ώστε $b*q$ να είναι τόσο κοντά στο $a$
όσο είναι αυτό δυνατό και μπορεί να αποδειχθεί ότι το $r$ μπορεί να επιλεγεί έτσι ώστε
$|r|^2 \leq |b|^2/2$.\\
Είσοδος :
\begin{center}{\en\tt iquo(148,5) }\end{center}
Έξοδος :
\begin{center}{\tt 29}\end{center}
{\en\tt iquo} δουλεύει με ακεραίους ή με Γκαουσιανούς ακεραίους.\\
Είσοδος :
\begin{center}{\en \tt iquo(factorial(148),factorial(145)+2 )}\end{center}
Έξοδος :
\begin{center}{\en\tt 3176375}\end{center}
Είσοδος :
\begin{center}{\en\tt iquo(25+12*i,5+7*i) }\end{center}
Έξοδος :
\begin{center}{\en\tt 3-2*i}\end{center}
Εδώ το $a-b*q=-4+i$ και $|-4+i|^2=17<|5+7*i|^2/2=74/2=37$

\subsection{\textgreek{Ακέραιο Ευκλείδειο υπόλοιπο} : {\tt \textlatin{ irem remain smod mods mod \%}}}\index{irem}\index{remain}
\noindent{\en\tt irem} (ή {\en\tt remain}) επιστρέφει το ακέραιο υπόλοιπο $r$  της Ευκλείδειας διαίρεσης δύο ακεραίων $a$ και $b$ που δίνονται ώς ορίσματα. 
($a=b*q+r$ με $0\leq r< b$).\\
Για \en Gaussian \gr ακεραίους, επιλέγουμε το $q$ ώστε $b*q$ να είναι τόσο κοντά στο $a$
όσο είναι αυτό δυνατό και μπορεί να αποδειχθεί ότι το $r$ μπορεί να επιλεγεί έτσι ώστε
$|r|^2 \leq |b|^2/2$.\\
Είσοδος :
\begin{center}{\en\tt irem(148,5) }\end{center}
Έξοδος :
\begin{center}{\tt 3}\end{center}
{\en\tt irem} λειτουργεί με μεγάλου μήκους ακεραίους ή με Γκαουσιανούς ακεραίους.\\
Παράδειγμα :
\begin{center}{\en\tt irem(factorial(148),factorial(45)+2 )}\end{center}
Έξοδος :
\begin{center}{\en\tt 111615339728229933018338917803008301992120942047239639312}\end{center}
Άλλο παράδειγμα
\begin{center}{\en\tt irem(25+12*i,5+7*i) }\end{center}
Έξοδος :
\begin{center}{\en\tt -4+i}\end{center}
Έδω το $a-b*q=-4+i$ και $|-4+i|^2=17<|5+7*i|^2/2=74/2=37$

{\en\tt smod} ή {\en\tt mods}\index{smod|textbf}\index{mods|textbf} είναι μια προθηματική συνάρτηση και έχει δύο ακεραίους $a$ και $b$  ως ορίσματα.\\ 
{\en\tt smod} ή {\en\tt mods} επιστρέφει το συμμετρικό υπόλοιπο $s$ της Ευκλείδειας διαίρεσης των ορισμάτων $a$ και $b$ ($a=b*q+s$ με $-b/2<s \leq b/2$).\\
Είσοδος :
\begin{center}{\en\tt smod(148,5) }\end{center}
Έξοδος :
\begin{center}{\tt -2}\end{center}

{\en\tt mod} (ή {\en\tt \%}) είναι μια ενθηματική συνάρτηση και έχει δύο ακεραίους $a$ και $b$ 
ως ορίσματα.\\
{\en\tt mod} (ή {\en\tt \%}) επιστρέφει $r\% b$ από $Z/bZ$ όπου $r$ είναι το υπόλοιπο 
της Ευκλείδειας διαίρεσης των ορισμάτων $a$ και $b$.\\
Είσοδος :\index{mod}\index{\%}
\begin{center}{\en\tt 148\ mod\ 5 }\end{center}
ή
\begin{center}{\en\tt 148 \%  5 }\end{center}
Έξοδος :
\begin{center}{\en\tt 3 \% 5}\end{center}
Σημειώστε ότι η απάντηση {\en\tt 3 \% 5}  δεν είναι ακέραιος αριθμός (3) αλλά ένα στοιχείο $Z/5Z$ (βλ. \ref{sec:modulaire} για τις δυνατές πράξεις στο $Z/5Z$).

\subsection{\textgreek{Ευκλείδειο πηλίκο και Ευκλείδειο υπόλοιπο δύο ακεραί\-ων} : {\tt \textlatin{iquorem}}}\index{iquorem}\label{sec:iquorem}
\noindent{\en\tt iquorem} επιστρέφει μια λίστα με το πηλίκο $q$ και το
υπόλοιπο $r$ της Ευκλείδειας διαίρεσης μεταξύ των δύο ακεραίων $a$ και $b$ που δίνονται ως ορίσματα ($a=b*q+r$ με $0\leq r< b$).\\
Είσοδος :
\begin{center}{\en\tt iquorem(148,5) }\end{center}
Έξοδος :
\begin{center}{\tt [29,3] }\end{center}

\subsection{\textgreek{Έλεγχος άρτιου ακεραίου} : {\tt\textlatin{even}}}\index{even}
\noindent {\en\tt even} λαμβάνει ως όρισμα έναν ακέραιο {\en\tt n}.\\
{\en\tt even} επιστρέφει {\tt 1} αν το {\en\tt n} είναι άρτιος ή επιστρέφει {\tt 0} αν το {\en\tt n} 
είναι περιττός.\\
Είσοδος :
\begin{center}{\en\tt even(148) }\end{center}
Έξοδος:
\begin{center}{\tt 1 }\end{center}
Είσοδος :
\begin{center}{\en\tt even(149) }\end{center}
Έξοδος :
\begin{center}{\tt 0}\end{center}


\subsection{\textgreek {Έλεγχος περιττού ακεραίου} : {\tt\textlatin{odd}}}\index{odd}
\noindent {\en\tt odd} λαμβάνει ως όρισμα έναν ακέραιο {\en\tt n}.\\
{\en\tt odd} επιστρέφει {\tt 1} αν το {\en\tt n} είναι περιττός ή επιστρέφει {\tt 0} αν το {\en\tt n} είναι ζυγός.\\
Είσοδος :
\begin{center}{\en\tt odd(148) }\end{center}
Έξοδος:
\begin{center}{\tt 0 }\end{center}
Είσοδος :
\begin{center}{\en\tt odd(149) }\end{center}
Έξοδος :
\begin{center}{\tt 1}\end{center}

\subsection{\textgreek{Έλεγχος ψευδο-πρώτου ακεραίου}: {\tt\textlatin{is\_pseudoprime}}}\index{is\_pseudoprime}
\noindent Αν το {\en\tt is\_pseudoprime(n)} επιστρέφει {\tt 2} (αληθής), τότε το
{\en\tt n} είναι πρώτος.\\ 
Εαν επιστρέφει 1, τότε το {\en\tt n} είναι ψευδο-πρώτος (πιθανότατα πρώτος).\\
Εαν επιστρέφει 0, τότε το {\en\tt n} δεν είναι πρώτος. \\
{\sc Διευκρίνιση}:{ Για αριθμούς μικρότερους από $10^{14}$, ο ψευδο-πρώτος και ο πρώτος είναι ισοδύναμοι. Αλλά για αριθμούς μεγαλύτερους από $10^{14}$, ο ψευδο-πρώτος είναι ένας αριθμός με μεγάλη πιθανότητα να είναι πρώτος{({\en cf. Rabin's} Αλγόριθμος και {\en Miller-Rabin's} \gr Αλγόριθμος στο Αλγοριθμικό μέρος (μενού {\en\tt Help -> Manuals -> Programming}))}}.\\
Είσοδος :
\begin{center}{\en\tt is\_pseudoprime(100003) }\end{center}
Έξοδος :\begin{center}{\tt 2}\end{center}
Είσοδος :
\begin{center}{\en\tt is\_pseudoprime(9856989898997) }\end{center}
Έξοδος :
\begin{center}{\tt 2}\end{center} 
Είσοδος :
\begin{center}{\en\tt is\_pseudoprime(14) }\end{center}
Έξοδος :
\begin{center}{\tt 0}\end{center}
Είσοδος :
\begin{center}{\en\tt is\_pseudoprime(9856989898997789789) }\end{center}
Έξοδος :
\begin{center}{\tt 1}\end{center}

\subsection{\textgreek{Έλεγχος πρώτου ακεραίου }: {\tt \textlatin{is\_prime isprime \\isPrime}}}\index{is\_prime}
\noindent {\en\tt is\_prime(n)} επιστρέφει {\tt 1} (αληθής) αν το {\en\tt n} είναι πρώτος και 
{\tt 0} (ψευδής) αν το {\en\tt n} δεν είναι πρώτος.\\
{\en\tt isprime} επιστρέφει {\en\tt true} ή {\en\tt false}.\\
Χρησιμοποιήστε την εντολή {\en\tt pari("isprime",n,1)} για πιστοποιητικό πρώτου αριθμού
(ανατρέξτε στο έγγραφο
\en PARI/GP \gr στο μενού {\en\tt Aide->Manuels->PARI-GP}) και {\en\tt pari("isprime",n,2)} για να χρησιμοποιήσετε το τεστ \en APRCL.
\gr
Είσοδος :
\begin{center}{\en\tt is\_prime(100003)}\end{center}
Έξοδος :
\begin{center}{\tt 1}\end{center}
Είσοδος :
\begin{center}{\en\tt isprime(100003)}\end{center}
Έξοδος :
\begin{center}{\en\tt true}\end{center}
Είσοδος :                    
\begin{center}{\en\tt is\_prime(98569898989987)}\end{center}
Έξοδος :
\begin{center}{\tt 1}\end{center} 
Είσοδος :
\begin{center}{\en\tt is\_prime(14)}\end{center}
Έξοδος:
\begin{center}{\tt 0}\end{center}
Είσοδος :
\begin{center}{\en\tt isprime(14)}\end{center}
Έξοδος :
\begin{center}{\en\tt false}\end{center}
Είσοδος :
\begin{center}{\en\tt pari("isprime",9856989898997789789,1)}\end{center}
Αυτή η εντολή επιστρέφει τους συντελεστές αποδεικνύοντας οτι είναι πρώτος αριθμός με το τεστ 
$p-1$ των {\en Selfridge-Pocklington-Lehmer}~:
\begin{center}
{\tt [[2,2,1],[19,2,1],[941,2,1],[1873,2,1],[94907,2,1]]}
\end{center}
Είσοδος :
\begin{center}{\en\tt isprime(9856989898997789789)}\end{center}
Έξοδος :
\begin{center}{\en\tt true}\end{center}

\subsection{\textgreek{Ο μικρότερος ψευδο-πρώτος μεγαλύτερος του} {\tt \textlatin{ n}} : \\ {\tt\textlatin{ nextprime}}}\index{nextprime}
\noindent{\en\tt nextprime(n)} επιστρέφει τον μικρότερο ψευδο-πρώτο (ή πρώτο)
που είναι μεγαλύτερος από {\en\tt n}. \\
Είσοδος :
\begin{center}{\en\tt  nextprime(75) }\end{center}
Έξοδος :
\begin{center}{\tt 79}\end{center}

\subsection{\textgreek{Ο μεγαλύτερος ψευδο-πρώτος μικρότερος του} {\tt \textlatin{n}} :\\ {\tt \textlatin{prevprime}}}\index{prevprime}
\noindent{\en\tt prevprime(n)} επιστρέφει τον μεγαλύτερο ψευδο-πρώτο (ή πρώτο)
που είναι μικρότερος από {\en\tt n}.\\
Είσοδος:
\begin{center}{\en\tt prevprime(75)}\end{center}
Έξοδος :
\begin{center}{\tt 73}\end{center}

\subsection{Ο {\tt \textlatin{n}}-οστός πρώτος αριθμός: {\tt \textlatin{ithprime}}}\index{ithprime}
\noindent{\en\tt ithprime(n)} επιστρέφει τον {\en\tt n}-οστό πρώτο αριθμό μικρότερο του 10000 (τρέχων περιορισμός).\\
Είσοδος :
\begin{center}{\en\tt ithprime(75)}\end{center}
Έξοδος :
\begin{center}{\tt 379}\end{center}
Είσοδος  :
\begin{center}{\en\tt ithprime(1229)}\end{center}
Έξοδος :
\begin{center}{\tt 9973}\end{center}
Είσοδος  :
\begin{center}{\en\tt ithprime(1230)}\end{center}
Έξοδος :
\begin{center}{\en\tt ithprime(1230)}\end{center}
επειδή ο {\en\tt ithprime(1230)} είναι μεγαλύτερος από το 10000.

\subsection{\textgreek{Ταυτότητα} \textlatin{ Bezout} : {\tt \textlatin{iegcd igcdex}}}\index{iegcd}\index{igcdex}
\noindent{\en\tt iegcd(a,b)} ή  {\en\tt igcdex(a,b)} 
επιστρέφει τους συντελεστές της Ταυτότητας \en Bezout \gr για τους δύο ακεραίους που δίνονται ως ορίσματα.\\
{\en\tt iegcd(a,b)} ή {\en\tt igcdex(a,b)} επιστρέφει {\en\tt [u,v,d]} τέτοια ώστε 
{\en\tt au+bv=d} και {\en\tt d=gcd(a,b)}.\\
Είσοδος :
\begin{center}{\en\tt iegcd(48,30) }\end{center}
Έξοδος :
\begin{center}{\en\tt [2,-3,6]}\end{center}
Αλλιώς :
$$2 \cdot 48+ (-3) \cdot 30 =6$$

\subsection{\textgreek{Επίλυση της } \textlatin{au+bv=c} \textgreek{στο} $\Z$: {\tt \textlatin{iabcuv}}}\index{iabcuv}
\noindent{\en\tt iabcuv(a,b,c)} επιστρέφει {\en\tt [u,v]} ώστε {\en\tt au+bv=c}.\\
{\en\tt c} πρέπει να είναι πολλαπλάσιο του {\en\tt gcd(a,b)} για την ύπαρξη μιας λύσης.\\
Είσοδος :
\begin{center}{\en\tt iabcuv(48,30,18) }\end{center}
Έξοδος :
\begin{center}{\tt [6,-9]}\end{center}

\subsection{\textgreek{Κινεζικά υπόλοιπα} : {\tt \textlatin{ichinrem, ichrem}}}\index{ichinrem}\index{ichrem}
\noindent{\en\tt ichinrem([a,p],[b,q])} ή {\en\tt ichrem([a,p],[b,q])} επιστρέφει μια λίστα 
{\en\tt [c,lcm(p,q)]} με 2 ακεραίους.\\
Ο πρώτος αριθμός {\en\tt c} είναι τέτοιος ώστε 
\[ \forall k \in \mathbb Z, \quad d=c+ k \times \mbox{{\en lcm}}(p,q) \]
έχει τις ιδιότητες
\[ d=a \pmod  p, \quad d=b \pmod q \]
Αν {\en\tt p} και {\en\tt q} είναι πρώτοι μεταξύ τους, μια λύση {\en\tt d} υπάρχει πάντα και όλες οι λύσεις 
είναι ισοδύναμες {\en modulo} \gr {\en\tt p*q}.\\

{\bf Παραδείγματα} : \\
Λύστε :
$${\tt \left \{ \begin{array}{rcl} x&=&3\ (\bmod\ 5)\\ 
x&=&9\ (\bmod\ 13) \end{array}\right.}$$
Είσοδος :
\begin{center}{\en\tt ichinrem([3,5],[9,13])}\end{center}
ή είσοδος :
\begin{center}{\en\tt ichrem([3,5],[9,13])}\end{center}
Έξοδος :
\begin{center}{\en\tt [-17,65] }\end{center}
άρα {\en\tt x=-17 (mod 65)}\\
μπορούμε επίσης για είσοδο να έχουμε :
\begin{center}{\en\tt ichrem(3\%5,9\%13)}\end{center}
Έξοδος :
\begin{center}{\tt -17\%65 }\end{center}
Λύστε :
$${\tt \left \{ \begin{array}{rcl} x&=&3\ (\bmod\ 5)\\ 
x&=&4\ (\bmod\ 7) \\ 
x&=&1\ (\bmod\ 9)\end{array}\right.}$$
Αρχική είσοδος :
\begin{center}{\en\tt tmp:=ichinrem([3,5],[4,7])}\end{center}
ή είσοδος :
\begin{center}{\en\tt tmp:=ichrem([3,5],[4,7])}\end{center}
έξοδος :
\begin{center}{\en\tt [-17,35] }\end{center}
επόμενη είσοδος :
\begin{center}{\en\tt ichinrem([1,9],tmp)}\end{center}
ή είσοδος :
\begin{center}{\en\tt ichrem([1,9],tmp)}\end{center}
Έξοδος :
\begin{center}{\tt [-17,315] }\end{center}
άρα {\en\tt x=-17 (mod 315)}\\
Εναλλακτική λύση :\\
\begin{center}{\en\tt ichinrem([3\%5,4\%7,1\%9])}\end{center}
Έξοδος :
\begin{center}{\en\tt -17\%315 }\end{center}

{\bf Σχόλιο}\\
{\en\tt ichrem} (ή {\en\tt ichinrem}) μπορούν να χρησιμοποιηθούν για να βρεθούν οι συντελεστές πολυωνύμου οι οποίοι είναι γνωστοί {\en modulo} αρκετών ακεραίων, για παράδειγμα να βρείτε
$ax+b$ \en modulo \gr $315=5 \times 7 \times 9$ σύμφωνα με τις παραδοχές:
$${\tt \left \{ \begin{array}{rl} a=&3\ (\bmod\ 5)\\ 
a=&4\ (\bmod\ 7) \\ 
a=&1\ (\bmod\ 9) \end{array}\right.},
\quad 
{\tt \left \{ \begin{array}{rl} b=&1\ (\bmod\ 5)\\ 
b=&2\ (\bmod\ 7) \\ 
b=&3\ (\bmod\ 9) \end{array}\right.}$$
Είσοδος :
\begin{center}{\en\tt ichrem((3x+1)\%5,(4x+2)\%7,(x+3)\%9)}\end{center}
Έξοδος :
\begin{center}{\en\tt (-17\%315$\times$ x+156\%315 }\end{center}
άρα {\en\tt a=-17 (mod 315)} και  {\en\tt b=156 (mod 315)}.

\subsection{\textgreek{Κινεζικά υπόλοιπα για λίστες ακεραίων} : {\tt\textlatin{ chrem}}}\index{chrem}
\noindent{\en\tt chrem} λαμβάνει ως ορίσματα 2 λίστες ακεραίων ίδιου μεγέθους.\\
{\en\tt chrem} επιστρέφει μια λίστα 2 ακεραίων.\\
Για παράδειγμα, {\en\tt chrem([a,b,c],[p,q,r])} επιστρέφει την λίστα 
{\en\tt [x,lcm(p,q,r)]} όπου
{\en\tt x=a mod  p} και {\en\tt x=b mod q} και {\en\tt x=c mod r}.\\
Μία λύση {\en\tt x} πάντα υπάρχει αν {\en\tt p, q, r} 
είναι πρώτοι μεταξύ τους, και όλες οι λύσεις ειναι ίσοδύναμες \en modulo \gr {\en\tt p*q*r}. \\
{\sc Προσοχή} στην σειρά των παραμέτρων, πράγματι:\\
{\en\tt chrem([a,b],[p,q])=ichrem([a,p],[b,q])=\\
ichinrem([a,p],[b,q])}\\
{\bf Παραδείγματα} : \\
Λύστε :
$${\tt \left \{ \begin{array}{rl} x=&3\ (\bmod\ 5)\\ 
x=&9\ (\bmod\ 13) \end{array}\right.}$$
Είσοδος :
\begin{center}{\en\tt chrem([3,9],[5,13])}\end{center}
Έξοδος :
\begin{center}{\tt [-17,65] }\end{center}
άρα, {\en\tt x=-17 (mod 65)}\\
Λύστε :
$${\tt \left \{ \begin{array}{rl} x=&3\ (\bmod\ 5)\\ 
x=&4\ (\bmod\ 6) \\ 
x=&1\ (\bmod\ 9)\end{array}\right.}$$
Είσοδος :
\begin{center}{\en\tt chrem([3,4,1],[5,6,9])}\end{center}
Έξοδος :
\begin{center}{\en\tt [28,90] }\end{center}
άρα {\en\tt x=28 (mod 90)}\\
{\bf Σχόλιο}\\
{\en\tt chrem} μπορεί να χρησιμοποιηθεί για να βρεθούν οι συντελεστές πολυωνύμου οι οποίοι είναι γνωστοί {\en modulo} αρκετών ακεραίων,  για παράδειγμα να βρείτε $ax+b$  \en(modulo) \gr $315=5 \times 7 \times 9$ σύμφωνα με τις παραδοχές:
$${\tt \left \{ \begin{array}{rl} a=&3\ (\bmod\ 5)\\ 
a=&4\ (\bmod\ 7) \\ 
a=&1\ (\bmod\ 9) \end{array}\right.}, \quad
{\tt \left \{ \begin{array}{rl} b=&1\ (\bmod\ 5)\\ 
b=&2\ (\bmod\ 7) \\ 
b=&3\ (\bmod\ 9) \end{array}\right.}$$
Είσοδος :
\begin{center}{\en\tt chrem([3x+1,4x+2,x+3],[5,7,9])}\end{center}
Έξοδος :
\begin{center}{\en\tt [-17x+156),315] }\end{center}
άρα, {\en\tt a=-17 (mod 315)} και {\en\tt b=156 (mod 315)}.

\subsection{\textgreek{Επίλυση της} $a^2+b^2=p$ \textgreek{στο} $\Z$ : {\tt\textlatin{pa2b2}}}\index{pa2b2}
\noindent{\en\tt pa2b2}  αναλύει έναν πρώτο ακέραιο $p$ ισοδύναμο με 1 \en modulo 4, 
\gr σαν άθροισμα τετραγώνων : $p= a^2+b^2$.
Το αποτέλεσμα είναι η λίστα {\en\tt [a,b]}.\\
Είσοδος :
\begin{center}{\en\tt pa2b2(17)}\end{center}
Έξοδος :
\begin{center}{\en\tt [4,1] }\end{center}
πράγματι $17=4^2+1^2$

\subsection{\textgreek{Η συνάρτηση $\varphi$ του} \textlatin{Euler} : {\tt \textlatin{euler phi}}}\index{euler}\index{phi}
\noindent{\en\tt euler} (ή {\en\tt phi}) επιστρέφει τον δείκτη \en Euler \gr 
για έναν ακέραιο. \\
{\en\tt euler(n)} (ή {\en\tt phi(n)}) είναι ίσο με το πλήθος των ακεραίων αριθμών που είναι μικρότεροι
του {\en\tt n} και πρώτοι με αυτό. \\
Είσοδος :
\begin{center}{\en\tt euler(21)}\end{center}
Έξοδος :
\begin{center}{\tt 12}\end{center}
Με άλλα λόγια
 E=\{2,4,5,7,8,10,11,13,15,16,17,19\} είναι το σύνολο των ακεραίων που είναι μικρότεροι από 21 και πρώτοι με 21. Υπάρχουν 12 μέλη σε αυτό το σύνολο , άρα ο πληθικός αριθμός του \en E \gr είναι 12 (\en Cardinal(E)=12).\gr

Ο \en Euler \gr εισήγαγε αυτή τη συνάρτηση για να γενικεύσει το μικρό θεώρημα του Φερμά (\en Fermat little's theorem):\\ \gr
\centerline{Αν $a$ και $n$ είναι πρώτοι μεταξύ τους τότε $a^{euler(n)}=1\ \bmod \ n$}

\subsection{\textgreek{Το σύμβολο του} \textlatin{Legendre}: 
{\tt\textlatin{legendre\_symbol}}}\index{legendre\_symbol}
Αν $n$ είναι πρώτος, ορίζουμε το σύμβολο του \en Legendre \gr του $a$ 
γραμμένο $\left(\frac{a}{n}\right)$ όπου :\\

$$\left(\frac{a}{n}\right)=\left\{\begin{array}{rl}
0 & \mbox{{\en if} }a=0 \bmod n \\
1 & \mbox{{\en if} } a \neq 0 \bmod n \mbox{ και {\en if} } a=b^2 \bmod n \\
-1 & \mbox{{\en if} } a \neq 0 \bmod n \mbox{ και {\en if} } a \neq b^2 \bmod n \\
\end{array}
\right.$$
\gr
\\
Ορισμένες ιδιότητες
\begin{itemize}
\item
Αν $n$ είναι πρώτος :
\[ a^{\frac{n-1}{2}}=\left(\frac{a}{n}\right) \bmod n \]
\item
\begin{eqnarray*}
\left(\frac{p}{q}\right).\left(\frac{q}{p}\right)
&=&(-1)^{\frac{p-1}{2}}.(-1)^{\frac{q-1}{2}}
\mbox{ αν $p$ και $q$ είναι περιττοί και θετικοί} \\
\left(\frac{2}{p}\right)&=&(-1)^{\frac{p^2-1}{8}} \\
\left(\frac{-1}{p}\right)&=&(-1)^{\frac{p-1}{2}}
\end{eqnarray*}
\end{itemize}
{\en\tt legendre\_symbol} παίρνει δύο ορίσματα $a$ και $n$ και επιστρέφει 
το σύμβολο του \en Legendre \gr $\left(\frac{a}{n}\right)$.\\
Είσοδος :
\begin{center}{\en\tt legendre\_symbol(26,17)}\end{center}
Έξοδος :
\begin{center}{\tt 1}\end{center}
Είσοδος :
\begin{center}{\en\tt legendre\_symbol(27,17)}\end{center}
Έξοδος :
\begin{center}{\tt -1}\end{center}
Είσοδος :
\begin{center}{\en\tt legendre\_symbol(34,17)}\end{center}
Έξοδος :
\begin{center}{\tt 0}\end{center}

\subsection{\textgreek{Το σύμβολο} \textlatin{Jacobi}  : {\tt \textlatin{jacobi\_symbol}}}\index{jacobi\_symbol}
Αν $n$ δεν είναι πρώτος, το σύμβολο \en Jacobi \gr του $a$, 
συμβολίζεται με $\left(\frac{a}{n}\right)$ και ορίζεται από το σύμβολο του \en Legendre \gr και από την διάσπαση του $n$ σε πρώτους παράγοντες. 
Για
\[ n=p_1^{\alpha _1}..p_k^{\alpha _k} \] 
όπου $p_j$ είναι πρώτος και $\alpha _j$ είναι ένας ακέραιος για $j=1..k$.
το σύμβολο \en Jacobi \gr του $a$ ορίζεται ως:
\[ \left(\frac{a}{n}\right)=\left(\frac{a}{p_1}\right)^{\alpha _1}...\left(\frac{a}{p_k}\right)^{\alpha _k} \]
{\en\tt jacobi\_symbol} λαμβάνει δύο ορίσματα $a$ και $n$, και επιστρέφει το σύμβολο του  \en Jacobi \gr $\left(\frac{a}{n}\right)$.\\
Είσοδος :
\begin{center}{\en\tt jacobi\_symbol(25,12)}\end{center}
Έξοδος :
\begin{center}{\tt 1}\end{center}
Είσοδος :
\begin{center}{\en\tt jacobi\_symbol(35,12)}\end{center}
Έξοδος :
\begin{center}{\tt -1}\end{center}
Είσοδος :
\begin{center}{\en\tt jacobi\_symbol(33,12)}\end{center}
Έξοδος :
\begin{center}{\tt 0}\end{center}

\section{Συνδυαστική ανάλυση}
\subsection{Παραγοντικό : {\tt \textlatin{factorial} \ !}}\index{factorial|textbf}\index{\symbol{33}|textbf}
\noindent{\en\tt factorial} (προθηματικό) ή {\en\tt !} (επιθηματικό)
παίρνει σαν όρισμα έναν ακέραιο $n$.\\
{\en\tt factorial(n)} ή {\en\tt n!} επιστρέφει $n!$.\\
Είσοδος :
\begin{center}{\en\tt factorial(10)}\end{center}
ή
\begin{center}{\tt 10!}\end{center}
Έξοδος :
\begin{center}{\tt 3628800}\end{center}

\subsection{Διωνυμικοί συντελεστές : {\tt \textlatin{binomial comb nCr}}}\index{binomial}\index{comb|textbf}\index{nCr|textbf}
\noindent{\en\tt comb} ή {\en\tt nCr} ή {\en\tt binomial} παίρνει ως ορίσματα δύο ακεραίους {\en\tt n} και {\en\tt p}.\\
{\en\tt comb(n,p)} ή {\en\tt nCr(n,p)} ή {\en\tt binomial(n,p)}  επιστρέφει
$\left(^n_p\right) =C_n^p$.\\
Είσοδος :
\begin{center}{\en\tt comb(5,2)}\end{center}
Έξοδος :
\begin{center}{\tt 10}\end{center}
{\bf Σχόλιο}\\
{\en\tt binomial} (αλλιώς {\en\tt comb, nCr}) 
μπορεί να έχει ένα τρίτο πραγματικό όρισμα,
στην περίπτωση αυτή η {\en\tt binomial(n,p,a)} επιστρέφει
$\left(^n_p\right) a^p(1-a)^{n-p}$.

\subsection{Διατάξεις : {\tt \textlatin{perm nPr}}}\index{perm}\index{nPr}
\noindent{\en\tt perm} ή {\en\tt nPr} παίρνει σαν ορίσματα δύο ακέραιους $n$ και $p$.\\
{\en\tt perm(n,p)} ή {\en\tt nPr(n,p)} επιστρέφει τον αριθμό $\frac{n!}{(n-p)!}$ των διατάξεων $p$ στοιχείων επιλεγμένων από $n$ στοιχεία.\\
Είσοδος :
\begin{center}{\en\tt perm(5,2)}\end{center}
Έξοδος :
\begin{center}{\en\tt 20}\end{center}

\subsection{Τυχαίοι ακέραιοι : {\tt \textlatin{rand}}}\index{rand}
\index{hasard}
\noindent{\en\tt rand} παίρνει σαν όρισμα έναν ακέραιο $n$ ή δεν παίρνει όρισμα.
\begin{itemize}
\item {\en\tt rand(n)} επιστρέφει ένα τυχαίο ακέραιο $p$ τέτοιο ώστε
 $0 \leq p<n$.\\   
Είσοδος :
\begin{center}{\en\tt rand(10)}\end{center}
Έξοδος για παράδειγμα :
\begin{center}{\tt 8}\end{center}

\item {\en\tt rand()} επιστρέφει ένα τυχαίο ακέραιο $p$ τέτοιο ώστε $0 \leq p<2^{31}$ 
(ή σε αρχιτεκτονική 64 \en bits $0 \leq p<2^{63}$).\\  \gr
Είσοδος :
\begin{center}{\en\tt rand()}\end{center}
Έξοδος για παράδειγμα :
\begin{center}{\tt 846930886}\end{center}
\end{itemize}

\section{Ρητοί αριθμοί}
\subsection{Μετατροπή ενός αριθμού κινητής υποδιαστολής σε ρητό αριθμό : {\tt \textlatin{exact 
float2rational}}}\index{float2rational|textbf}\index{exact|textbf}\index{evalf}
\noindent {\en\tt float2rational} ή {\en\tt exact} παίρνει σαν όρισμα 
έναν αριθμό κινητής υποδιαστολή 
{\en\tt d} και επιστρέφει έναν ρητό αριθμό {\en\tt q} που πλησιάζει το
{\en\tt d} έτσι ώστε {\en\tt abs(d-q)<epsilon}. Το
{\en\tt epsilon} ορίζεται στις Ρυθμίσεις {\en\tt Cas} ή στην μπάρα ρυθμίσεων
 ή με την εντολή {\en\tt cas\_setup}.\\
Είσοδος :
\begin{center}{\en\tt float2rational(0.3670520231)}\end{center}
Έξοδος όταν {\en\tt epsilon=1e-10}:
\begin{center}{\en\tt 127/346}\end{center}
% Input :
% \begin{center}{\tt 123/12+57/21}\end{center}
% Output :
% \begin{center}{\tt 363/28}\end{center}
% Then 
Είσοδος :
\begin{center}{\en\tt evalf(363/28)}\end{center}
Έξοδος
\begin{center}{\en\tt 12.9642857143}\end{center}
Είσοδος :
\begin{center}{\en\tt float2rational(12.9642857143)}\end{center}
Έξοδος
\begin{center}{\en\tt 363/28}\end{center}
Αν δύο παραστάσεις αναμειγνύονται, για παράδειγμα:
\begin{center}{\en\tt 1/2+0.7}\end{center}
ο ρητός μετατρέπεται σε αριθμό κινητής υποδιαστολής, έξοδος :
\begin{center}{\en\tt 1.2}\end{center} 
Είσοδος :
\begin{center}{\en\tt 1/2+float2rational(0.7)}\end{center}
Έξοδος :
\begin{center}{\en\tt 6/5}\end{center}

\subsection{Ακέραιο και κλασματικό μέρος : {\tt \textlatin{propfrac propFrac}}}\index{propfrac}\index{propFrac}\label{sec:ipropfrac}
\noindent{\en\tt propfrac(A/B)} ή {\en\tt propFrac(A/B)} επιστρέφει 
$$q+\frac{r}{b}\ \mbox{ με } \ 0\leq r<b$$ 
όπου  $\displaystyle \frac{A}{B}=\frac{a}{b}$ με 
\en $\mbox{gcd}(a,b)=1$ \gr
και \en $a=bq+r$.\\
\gr
Για ρητά κλάσματα, \en cf.  \ref{sec:propfrac}.\\
\gr
Είσοδος :
\begin{center}{\en\tt propfrac(42/15)}\end{center}
Έξοδος :
\begin{center}{\en\tt 2+4/5}\end{center}
Είσοδος :
\begin{center}{\en\tt  propfrac(43/12)}\end{center}
Έξοδος :
\begin{center}{\en\tt  3+7/12}\end{center}

\subsection{Αριθμητής ενός κλάσματος μετά την απλοποίηση : {\tt \textlatin{numer}} 
{\tt \textlatin{getNum}}}\index{numer|textbf}\index{getNum|textbf}\label{sec:inumer}
\noindent{\en\tt numer} ή {\en\tt getNum} λαμβάνει ως όρισμα ένα κλάσμα και επιστρέφει τον αριθμητή του κλάσματος μετά την απλοποίηση 
(για ρητά κλάσματα,
βλ. \ref{sec:numer}).\\
Είσοδος :
\begin{center}{\en\tt  numer(42/12)}\end{center}
ή :
\begin{center}{\en\tt getNum(42/12)}\end{center}
Έξοδος :
\begin{center}{\tt 7}\end{center}
Για να αποφευχθεί η απλοποίηση, το όρισμα πρέπει να αναφέρεται (για ρητά κλάσματα, βλ. \ref{sec:getnum}).\\
Είσοδος :
\begin{center}{\en\tt  numer({\gr '}42/12{\gr '})}\end{center}
ή :
\begin{center}{\en\tt  getNum({\gr '}42/12{\gr '})}\end{center}
Έξοδος :
\begin{center}{\tt 42}\end{center}


\subsection{Παρονομαστής ενός κλάσματος μετά την απλοποίηση: {\tt \textlatin{denom getDenom}}}\index{denom|textbf}\index{getDenom|textbf}\label{sec:idenom}
\noindent{\en\tt denom} ή {\en\tt getDenom} λαμβάνει ως όρισμα ένα κλάσμα και επιστρέφει τον παρονομαστή του κλάσματος μετά την απλοποίηση (για ρητά κλάσματα,
βλ.\ref{sec:denom}).\\
Είσοδος :
\begin{center}{\en\tt denom(42/12)}\end{center}
ή :
\begin{center}{\en\tt getDenom(42/12)}\end{center}
Έξοδος :
\begin{center}{\en\tt 2}\end{center}
Για να αποφευχθεί η απλοποίηση, το όρισμα πρέπει να αναφέρεται (για ρητά κλάσματα, βλ. \ref{sec:getdenom}).\\
Είσοδος :
\begin{center}{\en\tt denom({\gr '}42/12{\gr '})}\end{center}
ή :
\begin{center}{\en\tt getDenom({\gr '}42/12{\gr '})}\end{center}
Έξοδος :
\begin{center}{\en\tt 12}\end{center}

\subsection{Αριθμητής και παρονομαστής ενός κλάσματος: {\tt \textlatin{f2nd \\ fxnd}}}\index{fxnd}\index{f2nd}\label{sec:ifxnd}
\noindent{\en\tt f2nd} (ή {\en\tt fxnd}) λαμβάνει ως όρισμα ένα κλάσμα και επιστρέφει μια λίστα με τον αριθμητή και τον παρονομαστή του κλάσματος μετά την απλοποίηση (για ρητά κλάσματα,
βλ.\ref{sec:fxnd}).\\
Είσοδος :
\begin{center}{\en\tt  f2nd(42/12)}\end{center}
Έξοδος :
\begin{center}{\en\tt [7,2]}\end{center}

\subsection{Απλοποίηση ζεύγους ακεραίων : {\tt \textlatin{simp2}}}\index{simp2|textbf}\label{sec:isimp2}
\noindent{\en\tt simp2} λαμβάνει ως ορίσματα δύο ακεραίους ή μία λίστα δύο ακεραίων 
που αναπαριστούν ένα κλάσμα (για δύο πολυώνυμα βλ. \ref{sec:simp2}).\\
{\en\tt simp2} επιστρέφει μια λίστα με τον αριθμητή και τον παρονομαστή μιας ανάγωγης αναπαράστασης του κλάσματος
(π.χ. μετά την απλοποίηση).\\
Είσοδος :
\begin{center}{\en\tt simp2(18,15) }\end{center}
Έξοδος :
\begin{center}{\en\tt [6,5]}\end{center} 
Είσοδος :
\begin{center}{\en\tt  simp2([42,12])}\end{center}
Έξοδος :
\begin{center}{\en\tt [7,2]}\end{center}

\subsection{Αναπαράσταση πραγματικού αριθμού σαν συνεχές\\ κλάσμα: 
{\tt \textlatin{dfc}}}\index{dfc}\label{sec:convertdfc}\index{confrac@{\sl confrac}|textbf}
\noindent {\en\tt dfc} λαμβάνει ως όρισματα έναν πραγματικό αριθμό ή έναν ρητό αριθμό ή έναν αριθμό κινητής υποδιαστολής {\en\tt a} και έναν ακέραιο {\en\tt n} 
(ή ένα πραγματικό {\en\tt epsilon}).\\
{\en\tt dfc} επιστρέφει την λίστα του συνεχούς κλάσματος που παριστάνει τον {\en\tt a} τάξης {\en\tt n} (ή με ακρίβεια {\en\tt epsilon} δηλαδή
η αναπαράσταση συνεχούς κλάσματος που προσεγγίζει {\en\tt a} ή το {\en\tt evalf(a)} με ακρίβεια 
{\en\tt epsilon}. Από προεπιλογή  {\en\tt epsilon} είναι η τιμή του {\en\tt epsilon} 
 και ορίζεται στις Ρυθμίσεις  {\en\tt Cas} ή στην μπάρα ρυθμίσεων).\\ 
{\en\tt convert} με την επιλογή {\en\tt confrac} έχει παρόμοια λειτουργία: στην περίπτωση αυτή η
τιμή του {\en\tt epsilon} είναι το {\en\tt epsilon} 
και ορίζεται στις Ρυθμίσεις  {\en\tt Cas} ή στην μπάρα ρυθμίσεων (βλ.
\ref{sec:convert})
και η απάντηση μπορεί να αποθηκευθεί σε ένα προαιρετικό τρίτο όρισμα 

{\bf Σχόλια} 
\begin{itemize}
\item Εάν το τελευταίο στοιχείο από το αποτέλεσμα είναι μια λίστα, η αναπαράσταση
είναι εν τέλη περιοδική, και το τελευταίο στοιχείο είναι η περίοδος. Αυτό σημαίνει
ότι ο πραγματικός αριθμός είναι ρίζα μιας εξίσωσης δευτέρου βαθμού με ακέραιους συντελεστές.
\item Αν το τελευταίο στοιχείο του αποτελέσματος δεν είναι ακέραιος, τότε αναπαριστά ένα υπόλοιπο $r$ ($a=a0+1/....+1/an+1/r$). Να γνωρίζετε
ότι αυτό το υπόλοιπο έχει χάσει το μεγαλύτερο μέρος της ακρίβειάς του. 
\end{itemize}
Αν {\en\tt dfc(a)=[a0,a1,a2,[b0,b1]]} αυτό σημαίνει :
\[
a=a0+\frac{1}{a1+\frac{1}{a2+\frac{1}{b0+\frac{1}{b1+\frac{1}{b0+...}}}}} 
\]
Αν {\en\tt dfc(a)=[a0,a1,a2,r]} αυτό σημαίνει :
\[ a=a0+\frac{1}{a1+\frac{1}{a2+\frac{1}{r}}} \]
Είσοδος :
\begin{center}{\en\tt dfc(sqrt(2),5)}\end{center}
Έξοδος :
\begin{center}{\en\tt [1,2,[2]]}\end{center} 
Είσοδος :
\begin{center}{\en\tt dfc(evalf(sqrt(2)),1e-9)}\end{center}
ή : 
\begin{center}{\en\tt dfc(sqrt(2),1e-9)}\end{center}
Έξοδος :
\begin{center}{\en\tt [1,2,2,2,2,2,2,2,2,2,2,2,2]}\end{center} 
Είσοδος :
\begin{center}{\en\tt convert(sqrt(2),confrac,{\gr '}dev{\gr '})}\end{center}
Έξοδος (αν στο {\en\tt cas} έχουμε ρύθμιση {\en\tt epsilon=1e-9}) :
\begin{center}{\en\tt [1,2,2,2,2,2,2,2,2,2,2,2,2]}\end{center} 
και  {\en\tt [1,2,2,2,2,2,2,2,2,2,2,2,2]} αποθηκεύεται στο {\en\tt dev}.\\
Είσοδος :
\begin{center}{\en\tt dfc(9976/6961,5)}\end{center}
Έξοδος :
\begin{center}{\en\tt [1,2,3,4,5,43/7]}\end{center} 
Εισόδος για την επαλήθευση:\\
{\en\tt 1+1/(2+1/(3+1/(4+1/(5+7/43))))}  \\
Έξοδος :\\
{\en\tt 9976/6961}\\
Είσοδος :
\begin{center}{\en\tt convert(9976/6961,confrac,{\gr '}l{\gr '})}\end{center}
Έξοδος (αν στο {\en\tt cas} έχουμε ρύθμιση  {\en\tt epsilon=1e-9}) :
\begin{center}{\en\tt [1,2,3,4,5,6,7]}\end{center} 
και {\en\tt [1,2,3,4,5,6,7]} αποθηκεύεται στο {\en\tt l}\\
Είσοδος :
\begin{center}{\en\tt dfc(pi,5)}\end{center}
Έξοδος :
\begin{center}{\en\tt [3,7,15,1,292,(-113*pi+355)/(33102*pi-103993)]}\end{center} 
Είσοδος :
\begin{center}{\en\tt dfc(evalf(pi),5)}\end{center}
Έξοδος (αν οι αριθμοί κινητής υποδιαστολής είναι αριθμοί κινητής υποδιαστολής υλικού, π.χ. για {\en Digits}=12) :
\begin{center}{\en\tt [3,7,15,1,292,1.57581843574]}\end{center} 
Είσοδος :
\begin{center}{\en\tt dfc(evalf(pi),1e-9)}\end{center}
ή :
\begin{center}{\en\tt dfc(pi,1e-9)}\end{center}
ή (αν στο {\en\tt cas} έχουμε ρύθμιση  {\en\tt epsilon=1e-9}) :
\begin{center}{\en\tt convert(pi,confrac,{\gr '}ll{\gr '})}\end{center}
Έξοδος :
\begin{center}{\en\tt [3,7,15,1,292]}\end{center} 

\subsection{Μετασχηματισμός  συνεχούς κλάσματος σε πραγματικό αριθμό : {\tt \textlatin{dfc2f}}}\index{dfc2f}
\noindent {\en\tt dfc2f} παίρνει σαν όρισμα μια λίστα, που παριστάνει ένα συνεχές κλάσμα
\begin{itemize}
\item  λίστα ακεραίων για έναν ρητό αριθμό
\item μία λίστα της οποίας το τελευταίο στοιχείο είναι μια λίστα για μία περιοδική αναπαράσταση, δηλαδή
για έναν τετραγωνικό αριθμό, που είναι ρίζα μιας  δευτεροβάθμιας εξίσωσης με ακέραιους συντελεστές
\item ή μία λίστα με υπόλοιπο $r$ σαν τελευταίο στοιχείο
 ($a=a0+1/....+1/an+1/r$).
\end{itemize}
{\en\tt dfc2f} επιστρέφει τον ρητό αριθμό ή τον τετραγωνικό αριθμό .\\
Είσοδος :
\begin{center}{\en\tt dfc2f([1,2,[2]])}\end{center}
Έξοδος :
\begin{center}{\en\tt 1/(1/(1+sqrt(2))+2)+1}\end{center} 
Μετά από απλοποίηση με την συνάρτηση {\en\tt normal}:
\begin{center}{\en\tt sqrt(2)}\end{center} 
Είσοδος :
\begin{center}{\en\tt dfc2f([1,2,3])}\end{center}
Έξοδος :
\begin{center}{\en\tt 10/7}\end{center} 
Είσοδος :
\begin{center}{\en\tt normal(dfc2f([3,3,6,[3,6]]))}\end{center}
Έξοδος :
\begin{center}{\en\tt sqrt(11)}\end{center} 
Είσοδος :
\begin{center}{\en\tt dfc2f([1,2,3,4,5,6,7])}\end{center}
Έξοδος :
\begin{center}{\en\tt 9976/6961}\end{center} 
Είσοδος για επαλήθευση :\\
{\en\tt 1+1/(2+1/(3+1/(4+1/(5+1/(6+1/7)))))} \\
Έξοδος :\\
{\tt 9976/6961}\\
Είσοδος :
\begin{center}{\en\tt dfc2f([1,2,3,4,5,43/7])}\end{center}
Έξοδος :
\begin{center}{\en\tt 9976/6961}\end{center} 
Είσοδος για επαλήθευση :\\
{\en\tt 1+1/(2+1/(3+1/(4+1/(5+7/43))))} \\
Έξοδος :\\
{\en\tt 9976/6961}

\subsection{Ο $n$-οστός αριθμός \textlatin{Bernoulli} : {\tt \textlatin{bernoulli}}}\index{bernoulli}
\noindent {\en\tt bernoulli} παίρνει σαν όρισμα έναν ακέραιο $n$.\\
{\en\tt bernoulli} επιστρέφει τον $n$-οστό \en Bernoulli \gr αριθμό $B(n)$.\\
Οι αριθμοί \en Bernoulli \gr ορίζονται από την σχέση :
\[ \frac{t}{e^t-1}=\sum_{n=0}^{+\infty} \frac{B(n)}{n!}t^n \]
Τα πολυώνυμα \en Bernoulli \gr $B_k$ ορίζονται από τις σχέσεις :
\[ B_0=1, \quad B_k{'}(x)=kB_{k-1}(x), \quad  \int_0^1B_k(x)dx=0 \]
και ισχύει $B(n)=B_n(0)$.\\
Είσοδος :
\begin{center}{\en\tt bernoulli(6)}\end{center}
Έξοδος :
\begin{center}{\en\tt 1/42}\end{center}

\subsection{Προσπέλαση στις εντολές \textlatin{ PARI/GP}: {\tt \textlatin{pari}}}\index{pari}
\begin{itemize}
\item
{\en\tt pari} με μια συμβολοσειρά ως πρώτο όρισμα (το όνομα της εντολής του \en PARI) \gr
εκτελεί την αντίστοιχη εντολή του \en PARI \gr με τα υπόλοιπα ορίσματα. 
Για παράδειγμα {\en\tt pari("weber",1+i)} εκτελεί την εντολή του \en PARI \gr 
{\en\tt weber} με όρισμα {\en\tt 1+i}.
\item
{\en\tt pari} χωρίς ορίσματα εξάγει όλες τις συναρτήσεις του \en PARI/GP  \gr 
\begin{itemize}
\item με το ίδιο όνομα της εντολής αν δεν έχουν ήδη οριστεί στο {\en\tt
    Xcas}
\item με το αρχικό όνομα της εντολής τους με το πρόθεμα {\en\tt pari\_}
\end{itemize}
Για παράδειγμα, μετά την κλήση {\en\tt pari()}, με την εντολή {\en\tt pari\_weber(1+i)} ή
{\en\tt weber(1+i)} θα εκτελεσθεί η εντολή του \en PARI \gr 
{\en\tt weber} με όρισμα {\en\tt 1+i}.
\end{itemize}

Η τεκμηρίωση του \en PARI/GP \gr βρίσκεται στο μενού \en
Help -> Manuals.
\gr
\section{Πραγματικοί αριθμοί}
%%
\subsection{Αποτίμηση πραγματικού αριθμού με αριθμητική ακρίβεια} : {\tt\textlatin{ evalf    Digits, DIGITS}}\index{evalf}\index{Digits}\index{DIGITS}
\begin{itemize}
\item Ένας πραγματικός αριθμός είναι ένας ακριβής αριθμός και η αριθμητική του αναπαράσταση σε δεδομένη
ακρίβεια είναι ένας αριθμός κινητής υποδιαστολής που αναπαρίσταται με βάση το 2.\\
Η ακρίβεια ενός αριθμού κινητής υποδιαστολής είναι ο αριθμός των δυαδικών ψηφίων (\en bits)  \gr της
μαντίσα, και ο οποίος είναι τουλάχιστον 53 (Οι αριθμοί κινητής υποδιαστολής, είναι επίσης γνωστοί και ως {\en\tt double}). Οι αριθμοί κινητής υποδιαστολής γράφονται στο δεκαδικό σύστημα με έναν αριθμό ψηφίων που ελέγχεται από τον χρήστη είτε αλλάζοντας την τιμή στην μεταβλητή {\en\tt Digits} είτε τροποποιώντας τις Ρυθμίσεις \en Cas\gr. 
Από προεπιλογή {\en\tt Digits} είναι 12.
Ο αριθμός των ψηφίων που εμφανίζονται ελέγχουν τον αριθμό των δυαδικών ψηφίων (\en bits)  \gr της μαντίσα, εάν τα ψηφία είναι μικρότερα από 15, 53 \en bits \gr χρησιμοποιούνται, εάν τα ψηφία είναι αυστηρώς μεγαλύτερα από 15, ο αριθμός των \en bits \gr στρογγυλοποιείται προς το γινόμενο {\en\tt Digits} επί τον \en log \gr 10 στην δυαδική βάση.
\item
Μια παράσταση μετατρέπεται σε έναν αριθμό κινητής υποδιαστολής με την εντολή 
 {\en\tt evalf}. Η εντολή {\en\tt evalf} μπορεί να έχει ένα προαιρετικό δεύτερο όρισμα το οποίο θα χρησιμοποιηθεί για την αποτίμηση με συγκεκριμένη ακρίβεια.
\item
Σημειώστε ότι αν μια παράσταση περιέχει έναν αριθμό κινητής υποδιαστολής, θα προσπαθήσει να μετατρέψει και τα άλλα ορίσματα σε αριθμούς κινητής υποδιαστολής, ώστε να αναγκάσει ολόκληρη την παράσταση να γίνει ένας ενιαίος αριθμό κινητής υποδιαστολής.
\end{itemize}
Είσοδος :
\begin{center}{\en\tt 1+1/2}\end{center}
Έξοδος :
\begin{center}{\en\tt 3/2}\end{center}
Είσοδος :
\begin{center}{\en\tt 1.0+1/2}\end{center}
Έξοδος :
\begin{center}{\en\tt 1.5}\end{center}
Είσοδος :
\begin{center}{\en\tt exp(pi*sqrt(20))}\end{center}
Έξοδος :
\begin{center}{\en\tt exp(pi*2*sqrt(5)) }\end{center}
Με την {\en\tt evalf}, είσοδος : :
\begin{center}{\en\tt evalf(exp(pi*2*sqrt(5)))}\end{center}
Έξοδος :
\begin{center}{\en\tt 1263794.75367}\end{center}
Είσοδος :
\begin{center}{\en\tt 1.1\verb|^|{20}}\end{center}
Έξοδος :
\begin{center}{\en\tt 6.72749994933}\end{center}
Είσοδος :
\begin{center}{\en\tt sqrt(2)\verb|^|21}\end{center}
Έξοδος :
\begin{center}{\en\tt sqrt(2)*2\verb|^|10}\end{center}
Είσοδος για ένα αποτέλεσμα με 30 ψηφία :
\begin{center}{\en\tt Digits:=30}\end{center}
Είσοδος για την αριθμητική τιμή του $e^{\pi\sqrt{163}}$:
\begin{center}{\en\tt evalf(exp(pi*sqrt(163)))}\end{center}
Έξοδος :
\begin{center}{\en\tt 0.262537412640768743999999999985e18}\end{center}
Σημειώστε ότι τα {\en\tt Digits} είναι τώρα 30. Εάν δεν θέλετε να αλλάξετε την τιμή των {\en\tt Digits} μπορείτε να βάλλετε ως είσοδο
\begin{center}{\en\tt evalf(exp(pi*sqrt(163)),30)}\end{center}

\subsection{Συνήθεις ενθηματικές συναρτήσεις πραγματικών αριθμών : {\tt\textlatin{ +,-,*,/,\^\ }}}
\index{+,-,*,/,\^\ }
\noindent {\en\tt +,-,*,/,\^\ } είναι οι συνήθεις τελεστές που κάνουν προσθέσεις, αφαιρέσεις, πολλαπλασιασμούς διαιρέσεις και ύψωση αριθμού σε δύναμη.\\
Είσοδος :
\begin{center}{\en\tt 3+2}\end{center}
Έξοδος :
\begin{center}{\en\tt 5}\end{center}
Είσοδος :
\begin{center}{\en\tt 3-2}\end{center}
Έξοδος :
\begin{center}{\en\tt 1}\end{center}
Είσοδος :
\begin{center}{\en\tt 3*2}\end{center}
Έξοδος :
\begin{center}{\en\tt 6}\end{center}
Είσοδος :
\begin{center}{\en\tt 3/2}\end{center}
Έξοδος :
\begin{center}{\en\tt 3/2}\end{center}
Είσοδος :
\begin{center}{\en\tt 3.2/2.1}\end{center}
Έξοδος :
\begin{center}{\en\tt 1.52380952381}\end{center}
Είσοδος :
\begin{center}{\en\tt 3\verb|^|2}\end{center}
Έξοδος :
\begin{center}{\en\tt 9}\end{center}
Είσοδος :
\begin{center}{\en\tt 3.2\verb|^|2.1}\end{center}
Έξοδος :
\begin{center}{\en\tt 11.5031015682}\end{center}

{\bf Σχόλιο }\\
Μπορείτε να χρησιμοποιήσετε το πλήκτρο του τετράγωνου ή το πλήκτρο του κύβου αν έχει το πληκτρολόγιο σας,
για παράδειγμα : ${\tt 3^2}$ επιστρέφει 9.

{\bf Σχόλιο }
\begin{itemize}
\item Αν $x$ δεν είναι ακέραιος αριθμός, τότε $a^x=\exp(x \* \ln(a))$, άρα
$a^x$ είναι καλώς ορισμένη μόνο για $a>0$ αν $x$ δεν είναι ρητός. Αν $x$
είναι ρητός και $a<0$, ο κύριος κλάδος του λογαρίθμου χρησιμοποιείται, και οδηγεί σε έναν μιγαδικό αριθμό.
\item Άρα, προσοχή στην διαφορά μεταξύ $\sqrt[n]{a}$ και $a^{\frac{1}{n}}$ 
όταν $n$ είναι ένας περιττός ακέραιος.\\
Για παράδειγμα, για να σχεδιάσουμε την γραφική παράσταση της $y=\sqrt[3]{x^3-x^2}$, πληκτρολογούμε :
\begin{center}
{\en\tt plotfunc(ifte(x>0,(x\verb|^|3-x\verb|^|2)\verb|^|(1/3),\\
-(x\verb|^|2-x\verb|^|3)\verb|^|(1/3)),x,xstep=0.01)}
\end{center}
Μπορούμε ακόμη να πληκτρολογήσουμε : \\
{\en\tt plotimplicit(y\verb|^|3=x\verb|^|3-x\verb|^|2)} \\
αλλά αυτό είναι πολύ πιο αργό και πολύ λιγότερο ακριβές.
\end{itemize}

\subsection{Συνήθεις προθηματικές συναρτήσεις πραγματικών αριθμών : {\tt \textlatin{rdiv}}}\index{rdiv} 
{\en\tt rdiv} είναι ο προθηματικός τύπος της διαίρεσης της συνάρτησης.\\
Είσοδος :
\begin{center}{\en\tt rdiv(3,2)}\end{center}
Έξοδος :
\begin{center}{\en\tt 3/2}\end{center}
Είσοδος :
\begin{center}{\en\tt rdiv(3.2,2.1}\end{center}
Έξοδος :
\begin{center}{\en\tt 1.52380952381}\end{center}

\subsection{$n$-οστή ρίζα : {\tt \textlatin{root}}}\index{root}
\noindent{\en\tt root} παίρνει 2 ορίσματα : έναν ακέραιο $n$ και έναν αριθμό $a$.\\
{\en\tt root} επιστρέφει τη $n$-στή ρίζα του $a$ (π.χ. $a^{1/n}$).
Αν $a<0$, η $n$-στή ρίζα είναι μιγαδικός αριθμός του ορίσματος $2\pi/n$.\\
Είσοδος :
\begin{center}{\en\tt root(3,2)}\end{center}
'Εξοδος :
\begin{center}{\en\tt 2\verb|^|(1/3)}\end{center}
\noindent Είσοδος :
\begin{center}{\en\tt root(3,2.0)}\end{center}
'Εξοδος :
\begin{center}{\en\tt 1.259921049892}\end{center}
Είσοδος :
\begin{center}{\en\tt root(3,sqrt(2))}\end{center}
'Εξοδος :
\begin{center}{\en\tt 2\verb|^|(1/6)}\end{center}

\subsection{Συνάρτηση σφάλματος : {\tt \textlatin{erf}}}\index{erf}
\noindent{\en\tt erf} παίρνει σαν όρισμα έναν αριθμό $a$.\\
{\en\tt erf} επιστρέφει έναν αριθμό κινητής υποδιαστολής που είναι η συνάρτηση σφάλματος  υπολογισμένη στην τιμή $x=a$, και 
όπου η συνάρτηση σφάλματος ορίζεται από την σχέση :
\en
\[ \mbox{\en erf}(x)=\frac{2}{\sqrt{\pi}}\int_0^{x}e^{-t^2}dt \]
\gr
Η  κανονικοποίηση επιλέγεται έτσι ώστε: 
\en
\[ \mbox{erf}(+\infty)=1, \quad  \mbox{erf}(-\infty)=-1 \]
\gr
επειδή :
\en
\[ \int_0^{+\infty}e^{-t^2}dt=\frac{\sqrt{\pi}}{2} \]
\gr
Είσοδος :
\begin{center}{\en\tt erf(1)}\end{center}
Έξοδος :
\begin{center}{\en\tt 0.84270079295}\end{center}
Είσοδος :
\begin{center}{\en\tt erf(1/(sqrt(2)))*1/2+0.5}\end{center}
'Εξοδος :
\begin{center}{\en\tt 0.841344746069}\end{center}
{\bf  Σχόλιο}\\
Η σχέση μεταξύ {\en\tt erf} και {\en\tt normal\_cdf} είναι :\\
\en
\[ \mbox{\tt normal\_cdf}(x)=\frac{1}{2}+\frac{1}{2}\*\mbox{\en\tt
  erf}(\frac{x}{\sqrt{2}}) \]
\gr
Πράγματι, η αλλαγή της μεταβλητής $t=u*\sqrt{2}$ στο
\en
\[ \mbox{normal\_cdf}(x)=\frac{1}{2}+\frac{1}{\sqrt{2\pi}}\int_0^{x}e^{-t^2/2}dt\]
\gr
δίνει:
\en
\[ \mbox{normal\_cdf}(x)=\frac{1}{2}+\frac{1}{\sqrt{\pi}}\int_0^{\frac{x}{\sqrt{2}}}e^{-u^2}du=\frac{1}{2}+\frac{1}{2}\*\mbox{erf}(\frac{x}{\sqrt{2}})\]
\gr
Επαλήθευση:\\
{\en\tt normal\_cdf(1)=0.841344746069}

\subsection{Συμπληρωματική συνάρτηση σφάλματος: {\tt \textlatin{erfc}}}\index{erfc}
\noindent{\en\tt erfc} παίρνει σαν όρισμα έναν αριθμό $a$.\\
{\en\tt erfc} και επιστρέφει την τιμή της συμπληρωματικής συνάρτησης σφάλματος υπολογισμένη στο
$x=a$, και όπου η συμπληρωματική συνάρτηση σφάλματος ορίζεται από την σχέση :
\[ \mbox{\en erfc}(x)=\frac{2}{\sqrt{\pi}}\int_x^{+\infty}e^{-t^2}dt=1-erf(x) \]
Άρα {\en erfc}$(0)=1$,  με:
\[ \int_0^{+\infty}e^{-t^2}dt=\frac{\sqrt{\pi}}{2} \]
Είσοδος :
\begin{center}{\en\tt erfc(1)}\end{center}
Έξοδος :
\begin{center}{\en\tt 0.15729920705}\end{center}
Είσοδος :
\begin{center}{\en\tt 1- erfc(1/(sqrt(2)))*1/2}\end{center}
Έξοδος :
\begin{center}{\en\tt 0.841344746069}\end{center}
{\bf Σχόλιο}\\
Η σχέση μεταξύ {\en\tt erfc} και {\en\tt normal\_cdf} είναι :
\[ \mbox{\en\tt normal\_cdf}(x)=1-\frac{1}{2}\*\mbox{\en\tt erfc}
(\frac{x}{\sqrt{2}}) \]
Επαλήθευση :\\
{\en\tt normal\_cdf(1)=0.841344746069}

\subsection{Η συνάρτηση $\Gamma$  : {\tt \textlatin{Gamma}}}\index{Gamma}
\noindent{\en\tt Gamma} παίρνει σαν όρισμα έναν αριθμό  $a$.\\
{\en\tt Gamma} επιστρέφει την συνάρτηση $\Gamma$ αποτιμημένη στο $a$, με:
\[ \Gamma(x)=\int_0^{+\infty}e^{-t}t^{x-1}dt, \mbox{\en if } x>0 \]
Αν $x$  είναι θετικός ακέραιος, $\Gamma$ υπολογίζεται αναδρομικά :
\[ \Gamma(x+1)=x*\Gamma(x), \quad \Gamma(1)=1 \]
Άρα:
\[ \Gamma(n+1)=n! \]
Είσοδος :
\begin{center}{\en\tt Gamma(5)}\end{center}
Έξοδος :
\begin{center}{\en\tt 24}\end{center}
% Input :
% \begin{center}{\tt Gamma(1/2)}\end{center}
% Output :
% \begin{center}{\tt sqrt(pi)}\end{center}
Είσοδος :
\begin{center}{\en\tt Gamma(0.7)}\end{center}
Έξοδος :
\begin{center}{\en\tt 1.29805533265}\end{center}
Είσοδος :
\begin{center}{\en\tt Gamma(-0.3)}\end{center}
Έξοδος :
\begin{center}{\en\tt -4.32685110883}\end{center}
Πράγματι : {\en\tt Gamma(0.7)=-0.3*Gamma(-0.3)}\\
Είσοδος :
\begin{center}{\en\tt Gamma(-1.3)}\end{center}
Έξοδος :
\begin{center}{\en\tt 3.32834700679}\end{center}
Πράγματι {\en\tt Gamma(0.7)=-0.3*Gamma(-0.3)=(-0.3)*(-1.3)*Gamma(-1.3)}

\subsection{Η συνάρτηση $\beta$ : {\tt \textlatin{Beta}}}\index{Beta}
\noindent{\en\tt Beta} παίρνει σαν όρισμα 2 πραγματικούς αριθμούς $a,b$.\\
{\en\tt Beta} επιστρέφει την τιμή της συνάρτησης $\beta$ αποτιμημένης στα $a,b \in
\mathbb R$, με:
\[ \beta(x,y)=\int_0^1 t^{x-1} (1-t)^{y-1}
=\frac{\Gamma(x)*\Gamma(y)}{\Gamma(x+y)} \]
Αξιοπρόσεχτες τιμές:
\[ \beta(1,1)=1, \quad \beta(n,1)=\frac{1}{n}, \quad 
\beta(n,2)=\frac{1}{n(n+1)} \]
{\en\tt Beta(x,y)} ορίζεται για $x$ και για $y$ θετικους πραγματικούς
(εξασφαλίζεται έτσι η σύγκλιση) και για $x$ και $y$ αν δεν είναι αρνητικοί ακέραιοι.\\
Είσοδος :
\begin{center}{\en\tt Beta(5,2)}\end{center}
Έξοδος :
\begin{center}{\en\tt 1/30}\end{center}
Είσοδος :
\begin{center}{\en\tt Beta(x,y)}\end{center}
Έξοδος :
\begin{center}{\en\tt Gamma(x)*Gamma(y)/Gamma(x+y)}\end{center}
Είσοδος :
\begin{center}{\en\tt Beta(5.1,2.2)}\end{center}
Έξοδος :
\begin{center}{\en\tt 0.0242053671402}\end{center}

\subsection{Παράγωγοι της συνάρτησης διγάμμα : {\tt \textlatin{Psi}}}\index{Psi}
\noindent{\en\tt Psi} παίρνει σαν όρισμα έναν πραγματικό $a$ και έναν ακέραιο $n$ (από προεπιλογή $n=0$).\\ 
{\en\tt Psi} επιστρέφει την τιμή της $n$-οστής παραγώγου της συνάρτησης διγάμμα
για $x=a$, όπου η συνάρτηση διγάμμα είναι η πρώτη παράγωγος 
της $\ln(\Gamma(x))$. Αυτη η συνάρτηση χρησιμοποιείται για την αποτίμηση αθροίσματος ρητών συναρτήσεων με ακεραίους πόλους.\\
Είσοδος :
\begin{center}{\en\tt Psi(3,1)}\end{center}
Έξοδος :
\begin{center}{\en\tt pi\verb|^|2/6-5/4}\end{center}

Αν {\en\tt n=0}, μπορείτε να χρησιμοποιήσετε την {\en\tt Psi(a)} αντί της {\en\tt Psi(a,0)} 
για να υπολογίσετε την συνάρτηση διγάμμα στο $x=a$.\\ 
Είσοδος :
\begin{center}{\en\tt Psi(3)}\end{center}
Έξοδος :
\begin{center}{\en\tt  Psi(1)+3/2}\end{center}
Είσοδος :
\begin{center}{\en\tt evalf(Psi(3))}\end{center}
Έξοδος :
\begin{center}{\en\tt  .922784335098}\end{center}

\subsection{Η συνάρτηση $\zeta$ : {\tt \textlatin{Zeta}}}\index{zeta}
\noindent{\en\tt Zeta} παίρνει σαν όρισμα έναν πραγματικό αριθμό $x$.\\ 
{\en\tt Zeta} επιστρέφει για $x>1$ : 
\[ \zeta(x)= \sum_{n=1}^{+\infty} \frac{1}{n^x} \]
και για  $x<1$  την μερομορφική της συνάρτηση.\\
Είσοδος :
\begin{center}{ \en\tt Zeta(2)}\end{center}
Έξοδος :
\begin{center}{ \en\tt pi\verb|^|2/6}\end{center}
Είσοδος :
\begin{center}{ \en\tt Zeta(4)}\end{center}
Έξοδος :
\begin{center}{ \en\tt pi\verb|^|4/90}\end{center}

\subsection{Οι συναρτήσεις \textlatin{Airy}: {\tt \textlatin{Airy\_Ai \textgreek{\bf και}  Airy\_Bi}}}\index{Airy\_Ai}\index{Airy\_Bi}
\noindent{\en\tt Airy\_Ai} και {\en\tt Airy\_Bi} παίρνουν σαν όρισμα έναν πραγματικό αριθμό $x$.\\
{\en\tt Airy\_Ai} και {\en\tt Airy\_Bi} είναι δύο ανεξάρτητες λύσεις
της εξίσωσης
\[ y^{\prime\prime}-x*y=0 \]
Ορίζονται από τις σχέσεις :
\begin{eqnarray*}
\en
\mbox{Airy\_Ai}(x) &=& (1/\pi) \int_0^\infty \cos(t^3/3 + x*t) dt \\
\en
\mbox{Airy\_Bi}(x) &=& (1/\pi) \int_0^\infty (e^{- t^3/3} + \sin( t^3/3 +
x*t)) dt
\gr
\end{eqnarray*}
Ιδιότητες :\\
\en
\begin{eqnarray*}
\en \tt \mbox{Airy\_Ai}(x)&=&\mbox{Airy\_Ai}(0)*f(x)+
\mbox{Airy\_Ai}^\prime (0)*g(x) \\
\en\tt \mbox{Airy\_Bi}(x)&=&\sqrt{3}(\mbox{Airy\_Ai}(0)*f(x)
-\mbox{Airy\_Ai}^\prime (0)*g(x) )
\end{eqnarray*}
\gr
όπου $f$ και $g$ είναι οι 2 λύσεις της
\[ w^{\prime\prime}-x*w=0 \]
Ακριβέστερα:
\en
\begin{eqnarray*}
f(x)&=&\sum_{k=0}^\infty 3^k\left (\frac{\Gamma(k+\frac{1}{3})}{\Gamma(\frac{1}{3})}\right ) \frac{x^{3k}}{(3k)!}\\
g(x)&=&\sum_{k=0}^\infty 3^k\left
  (\frac{\Gamma(k+\frac{2}{3})}{\Gamma(\frac{2}{3})}\right )
\frac{x^{3k+1}}{(3k+1)!}
\end{eqnarray*}
\gr
Είσοδος :
\begin{center}{\en\tt Airy\_Ai(1)}\end{center}
Έξοδος :
\begin{center}{\en\tt 0.135292416313}\end{center}
Είσοδος :
\begin{center}{\en\tt Airy\_Bi(1)}\end{center}
Έξοδος :
\begin{center}{\en\tt 1.20742359495}\end{center}
Είσοδος :
\begin{center}{\en\tt Airy\_Ai(0)}\end{center}
Έξοδος :
\begin{center}{\en\tt 0.355028053888}\end{center}
Είσοδος :
\begin{center}{\en\tt Airy\_Bi(0)}\end{center}
Έξοδος :
\begin{center}{\en\tt 0.614926627446}\end{center}

\section{Μεταθέσεις}
Μία μετάθεση $p$ μεγέθους $n$ είναι μια {\en "}αμφιμονοσήμαντη και επί{\en "} απεικόνιση από το $[0..n-1]$ στο
$[0..n-1]$  και αναπαρίσταται από την λίστα:
 $[p(0),p(1),p(2)...p(n-1)]$.\\
Για παράδειγμα, η μετάθεση $p$ που αναπαρίσταται από την λίστα $[1,3,2,0]$ είναι η απεικόνιση από το  $[0,1,2,3]$ στο $[0,1,2,3]$ που ορίζεται ως:
\[ p(0)=1,\ p(1)=3,\ p(2)=2,\  p(3)=0 \]
Ένας κύκλος $c$ μεγέθους $p$ αναπαρίσταται από την λίστα:
$[a_0,...,a_{p-1}]$ ($0\leq a_k\leq n-1$) Eίναι η μετάθεση όπου
\[ c(a_i)=a_{i+1} \mbox{\en for }(i=0..p-2), \quad
c(a_{p-1})=a_0, \quad 
 c(k)=k \mbox{ αλλιώς }\]
Ένας κύκλος $c$ αναπαρίσταται από μια λίστα και η διάσπαση ενός κύκλου
αναπαρίσταται από λίστες λιστών.\\
Για παράδειγμα, ο κύκλος $c$ που αναπαρίσταται από την λίστα $[3,2,1]$ είναι η μετάθεση
 $c$ ορισμένη για $c(3)=2,\ c(2)=1,\ c(1)=3,\ c(0)=0$ (δηλαδή η μετάθεση που αναπαρίσταται από την λίστα $[0,3,1,2]$).
 
\subsection{Τυχαία μετάθεση: {\tt \textlatin{randperm}}}\index{randperm}
\noindent{\en\tt randperm} παίρνει σαν όρισμα έναν ακέραιο αριθμό $n$.\\
{\en\tt randperm} επιστρέφει μία τυχαία μετάθεση των αριθμών $[0..n-1]$.\\
Είσοδος :
\begin{center}{\en\tt randperm(3)}\end{center}
Έξοδος :
\begin{center}{\tt [2,0,1]}\end{center}

\subsection{Διάσπαση μετάθεσης σε γινόμενο ξένων κύκλων: \\
{\tt \textlatin{permu2cycles}}}\index{permu2cycles}
\noindent {\en\tt permu2cycles} παίρνει σαν όρισμα μία μετάθεση.\\
{\en\tt permu2cycles} επιστρέφει την διάσπασή της σε γινόμενο ξένων κύκλων.\\
Είσοδος  :
\begin{center}{\en\tt permu2cycles([1,3,4,5,2,0])}\end{center}
Έξοδος :
\begin{center}{\en\tt [[0,1,3,5],[2,4]]}\end{center}
Στην απάντηση οι κύκλοι που είναι μεγέθους 1 παραλείπονται , εκτός αν το $n-1$ είναι 
σταθερό σημείο της μετάθεσης (αυτό απαιτείται για να βρούμε την τιμή του
$n$ στην διάσπαση κύκλων).\\
Είσοδος :
\begin{center}{\en\tt permu2cycles([0,1,2,4,3,5])}\end{center}
Έξοδος :
\begin{center}{\en\tt [[5],[3,4]]}\end{center}
Είσοδος :
\begin{center}{\en\tt permu2cycles([0,1,2,3,5,4])}\end{center}
Έξοδος :
\begin{center}{\en\tt [[4,5]]}\end{center}

\subsection{Γινόμενο ξένων κύκλων σε μετάθεση: {\tt\textlatin{ cycles2permu}}}\index{cycles2permu}
\noindent{\en\tt cycles2permu} παίρνει σαν όρισμα μία λίστα κύκλων.\\
{\en\tt cycles2permu} επιστρέφει την μετάθεση (μεγέθους $n$ , επιλέγοντάς το όσο γίνεται μικρότερο) που είναι το γινόμενο των κύκλων που δίνονται ως όρισμα
(είναι η αντίστροφη συνάρτηση της {\en\tt permu2cycles}).\\
Είσοδος :
\begin{center}{\en\tt cycles2permu([[1,3,5],[2,4]])}\end{center}
Έξοδος :
\begin{center}{\en\tt [0,3,4,5,2,1]}\end{center}
Είσοδος :
\begin{center}{\en\tt cycles2permu([[2,4]])}\end{center}
Έξοδος :
\begin{center}{\en\tt [0,1,4,3,2]}\end{center}
Είσοδος :
\begin{center}{\en\tt cycles2permu([[5],[2,4]])}\end{center}
Έξοδος :
\begin{center}{\en\tt [0,1,4,3,2,5]}\end{center}

\subsection{Μετατροπή κύκλου σε μετάθεση : {\tt\textlatin{cycle2perm}}}\index{cycle2perm}
\noindent{\en\tt cycle2perm} παίρνει σαν όρισμα έναν κύκλο.\\
{\en\tt cycle2perm} επιστρέφει την μετάθεση μεγέθους $n$ που αντιστοιχεί
στον κύκλο που δίνεται ως όρισμα , όπου $n$ επιλέγεται το μικρότερο δυνατό (βλ. επίσης {\en\tt permu2cycles} και {\en\tt cycles2permu}).\\
Είσοδος :
\begin{center}{\en\tt cycle2perm([1,3,5])}\end{center}
Έξοδος :
\begin{center}{\en\tt [0,3,2,5,4,1]}\end{center}

\subsection{Μετατροπή μετάθεσης σε πίνακα: {\tt\textlatin{permu2mat}}}\index{permu2mat}
\noindent{\en\tt permu2mat} παίρνει σαν όρισμα μία μετάθεση $p$ μεγέθους $n$.\\
{\en\tt permu2mat} επιστρέφει τον πίνακα της μετάθεσης, όπου οι σειρές του πίνακα είναι οι μεταθέσεις των σειρών του ταυτοτικού πίνακα τάξης $n$ με την μετάθεση $p$.\\
Είσοδος :
\begin{center}{\en\tt permu2mat([2,0,1])}\end{center}
Έξοδος :
\begin{center}{\en\tt [[0,0,1],[1,0,0],[0,1,0]]}\end{center}

\subsection{Έλεγχος μετάθεσης: {\tt \textlatin{is\_permu}}}\index{is\_permu}
\noindent{\en\tt is\_permu} είναι μία λογική συνάρτηση.\\
{\en\tt is\_permu} παίρνει σαν όρισμα μία λίστα.\\
{\en\tt is\_permu} επιστρέφει 1 αν το όρισμα είναι μετάθεση, ενώ επιστρέφει 0 αν το όρισμα δεν είναι  μετάθεση.\\
Είσοδος :
\begin{center}{\en\tt is\_permu([2,1,3]) }\end{center}
Έξοδος :
\begin{center}{\en\tt 0}\end{center}
Είσοδος :
\begin{center}{\en\tt is\_permu([2,1,3,0]) }\end{center}
Έξοδος :
\begin{center}{\en\tt 1}\end{center}

\subsection{Έλεγχος κύκλου: {\tt \textlatin{is\_cycle}}}\index{is\_cycle}
\noindent{\en\tt is\_cycle} είναι μία λογική συνάρτηση.\\
{\en\tt is\_cycle} παίρνει σαν όρισμα μία λίστα.\\
{\en\tt is\_cycle} επιστρέφει 1 αν το όρισμα είναι κύκλος, ενώ επιστρέφει 0 αν το όρισμα δεν είναι κύκλος.\\
Είσοδος :
\begin{center}{\en\tt is\_cycle([2,1,3]) }\end{center}
Έξοδος :
\begin{center}{\en\tt 1}\end{center}
Είσοδος :
\begin{center}{\en\tt is\_cycle([2,1,3,2]) }\end{center}
Έξοδος :
\begin{center}{\en\tt 0}\end{center}

\subsection{Γινόμενο δύο μεταθέσεων : {\tt \textlatin{p1op2}}}\index{p1op2}
\noindent{\en\tt p1op2}  παίρνει σαν όρισμα δύο μεταθέσεις.\\
{\en\tt p1op2} επιστρέφει την μετάθεση που προκύπτει από την σύνθεση :

\[ 1^{\mbox{ο}}\mbox{όρισμα} \circ 2^{\mbox{ο}} \mbox{όρισμα} \]

Είσοδος :
\begin{center}{\en\tt p1op2([3,4,5,2,0,1],[2,0,1,4,3,5])}\end{center}
Έξοδος :
\begin{center}{\en\tt [5,3,4,0,2,1]}\end{center}
{\bf Προσοχή}\\
Η σύνθεση γίνεται χρησιμοποιώντας το πρότυπο της μαθηματική σημειογραφίας,
όπου η μετάθεση που δίνεται ως δεύτερο όρισμα εκτελείται πρώτη.

\subsection{Σύνθεση κύκλου και μετάθεσης: {\tt \textlatin{c1op2}}}\index{c1op2}
\noindent{\en\tt c1op2} παίρνει ως ορίσματα έναν κύκλο και μία μετάθεση.\\
{\en\tt c1op2} επιστρέφει την μετάθεση που προκύπτει από την σύνθεση :

\[ 1^{\mbox{ο}}\mbox{όρισμα} \circ 2^{\mbox{ο}} \mbox{όρισμα} \]

Είσοδος :
\begin{center}{\en\tt c1op2([3,4,5],[2,0,1,4,3,5])}\end{center}
Έξοδος :
\begin{center}{\en\tt [2,0,1,5,4,3]}\end{center}
{\bf Προσοχή}\\
Η σύνθεση γίνεται χρησιμοποιώντας το πρότυπο της μαθηματική σημειογραφίας,
όπου η μετάθεση που δίνεται ως δεύτερο όρισμα εκτελείται πρώτη.


\subsection{Σύνθεση μετάθεσης και κύκλου: {\tt \textlatin{p1oc2}}}\index{p1oc2}
\noindent{\en\tt p1oc2} παίρνει ως ορίσματα μία μετάθεση και εναν κύκλο.\\
{\en\tt p1oc2} επιστρέφει την μετάθεση που προκύπτει από την σύνθεση :

\[ 1^{\mbox{ο}}\mbox{όρισμα} \circ 2^{\mbox{ο}} \mbox{όρισμα} \]

Είσοδος :
\begin{center}{\en\tt p1oc2([3,4,5,2,0,1],[2,0,1])}\end{center}
Έξοδος :
\begin{center}{\en\tt [4,5,3,2,0,1]}\end{center}
{\bf Προσοχή}\\
Η σύνθεση γίνεται χρησιμοποιώντας το πρότυπο της μαθηματική σημειογραφίας,
όπου ο κύκλος που δίνεται ως δεύτερο όρισμα εκτελείται πρώτος.

\subsection{Γινόμενο δύο κύκλων: {\tt \textlatin{c1oc2}}}\index{c1oc2}
\noindent {\en\tt c1oc2} παίρνει ως ορίσματα δύο κύκλους .\\
{\en\tt c1oc2} επιστρέφει την μετάθεση που προκύπτει από την σύνθεση :
\[ 1^{\mbox{ο}}\mbox{όρισμα} \circ 2^{\mbox{ο}} \mbox{όρισμα} \]
Είσοδος :
\begin{center}{\en\tt c1oc2([3,4,5],[2,0,1])}\end{center}
Έξοδος :
\begin{center}{\en\tt [1,2,0,4,5,3]}\end{center}
{\bf Προσοχή}\\
Η σύνθεση γίνεται χρησιμοποιώντας το πρότυπο της μαθηματική σημειογραφίας,
όπου ο κύκλος που δίνεται ως δεύτερο όρισμα εκτελείται πρώτος.

\subsection{Ίχνος (πρόσημο) μετάθεσης : {\tt \textlatin{signature}}}\index{signature}
\noindent{\en\tt signature} παίρνει ως όρισμα μία μετάθεση.\\
{\en\tt signature} επιστρέφει το ίχνος της μετάθεσης που δίνεται ως όρισμα.\\
Το ίχνος της μετάθεσης είναι ίσο με:
\begin{itemize}
\item 1 αν η μετάθεση ισούται με άρτιο γινόμενο αναστροφών,
\item -1 αν ο συνδυασμός αριθμών ισούται με περιττό γινόμενο αναστροφών.
\end{itemize}
Το ίχνος κύκλου μεγέθους $k$ είναι : $(-1)^{k+1}$.\\ 
Είσοδος :
\begin{center}{\en\tt signature([3,4,5,2,0,1])}\end{center}
Έξοδος :
\begin{center}{\en\tt -1}\end{center}
Πράγματι {\en\tt permu2cycles([3,4,5,2,0,1])=[[0,3,2,5,1,4]]}.

\subsection{Αντίστροφο μετάθεσης : {\tt \textlatin{perminv}}}\index{perminv}
\noindent{\en\tt perminv} παίρνει ως όρισμα μία μετάθεση.\\
{\en\tt perminv} επιστρέφει την μετάθεση που είναι η αντίστροφη της μετάθεσης 
που δόθηκε σαν όρισμα.\\
Είσοδος :
\begin{center}{\en\tt perminv([1,2,0])}\end{center}
Έξοδος :
\begin{center}{\en\tt [2,0,1]}\end{center}

\subsection{Αντίστροφο κύκλου : {\tt \textlatin{cycleinv}}}\index{cycleinv}
\noindent{\en\tt cycleinv} παίρνει ως όρισμα έναν κύκλο.\\
{\en\tt cycleinv}  επιστρέφει τον κύκλο που είναι ο αντίστροφος του κύκλου που δόθηκε σαν όρισμα.\\
Είσοδος :
\begin{center}{\en\tt cycleinv([2,0,1])}\end{center}
Έξοδος :
\begin{center}{\en\tt [1,0,2]}\end{center}

\subsection{Τάξη μετάθεσης : {\tt \textlatin{permuorder}}}\index{permuorder}
\noindent{\en\tt permuorder} παίρνει ως όρισμα μία μετάθεση.\\
{\en\tt permuorder} επιστρέφει την τάξη $k$ της μετάθεσης $p$ που δίνεται ως όρισμα, δηλαδή τον μικρότερο ακέραιο $m$ , ώστε $p^m$ είναι η ταυτοτική μετάθεση.\\
Είσοδος :
\begin{center}{\en\tt permuorder([0,2,1])}\end{center}
Έξοδος :
\begin{center}{\en\tt 2}\end{center}
Είσοδος :
\begin{center}{\en\tt permuorder([3,2,1,4,0])}\end{center}
Έξοδος :
\begin{center}{\en\tt 6}\end{center}

\subsection{Η ομάδα που παράγεται από δύο μεταθέσεις: {\tt \textlatin{groupermu}}}\index{groupermu}
\noindent{\en\tt groupermu} παίρνει σαν όρισμα δύο μεταθέσεις, {\en\tt a} και
{\en\tt b}.\\
{\en\tt groupermu}  επιστρέφει την ομάδα των μεταθέσεων που παράγονται από τις {\en\tt a} και
{\en\tt b}.\\
Είσοδος :
\begin{center}{\en\tt groupermu([0,2,1,3],[3,1,2,0])}\end{center}
Έξοδος :
\begin{center}{\en\tt [[0,2,1,3],[3,1,2,0],[0,1,2,3],[3,2,1,0]]}\end{center}

\section{Μιγαδικοί αριθμοί}
Σημειώστε ότι οι μιγαδικοί αριθμοί χρησιμοποιούνται επίσης για να αναπαραστήσουν ένα σημείο στο επίπεδο
ή μία 1-\en d \gr γραφική παράσταση.

\subsection{Μιγαδικές συναρτήσεις : {\tt \textlatin{+,-,*,/,\^\ }}}\index{+}\index{'+'}\index{-}\index{'-'}\index{\^\ }
\noindent {\en\tt +,-,*,/,\^\ } είναι οι συνήθεις τελεστές για την εκτέλεση
πρόσθεσης, αφαίρεσης, πολλαπλασιασμού, διαίρεσης και ύψωσης σε ακέραια ή ρητή δύναμη.\\
Είσοδος :
\begin{center}{\en\tt (1+2*i)\verb|^|2}\end{center}
Έξοδος :
\begin{center}{\en\tt -3+4*i}\end{center}

\subsection{Πραγματικό μέρος ενός μιγαδικού αριθμού : {\tt \textlatin{re real}}}\index{re}\index{real}
\noindent{\en\tt re} (ή {\en\tt real}) παίρνει σαν όρισμα έναν μιγαδικό αριθμό (ένα σημείο $A$).\\
{\en\tt re} (ή {\en\tt real}) επιστρέφει το πραγματικό μέρος του μιγαδικού αριθμού (
την προβολή του $A$ στον άξονα των  $x$).\\
Είσοδος :
\begin{center}{\en\tt re(3+4*i)}\end{center}
Έξοδος :
\begin{center}{\en\tt 3}\end{center}

\subsection{Φανταστικό μέρος ενός μιγαδικού αριθμού: {\tt \textlatin{im imag}}}\index{im}\index{imag}
\noindent{\en\tt im} (ή {\en\tt imag}) παίρνει σαν όρισμα ένα μιγαδικό αριθμό (ένα σημείο $A$).\\
{\en\tt im} (ή {\en\tt imag}) επιστρέφει το φανταστικό μέρος ενός μιγαδικού αριθμού  ( 
την προβολή του $A$ στον άξονα των $y$).\\
Είσοδος :
\begin{center}{\en\tt im(3+4*i)}\end{center}
Έξοδος :
\begin{center}{\en\tt 4}\end{center}

\subsection{Μιγαδικός αριθμός στην μορφή {\tt \textlatin{re(z)+i*im(z)}} : {\tt\textlatin{evalc}}}\index{evalc}
\noindent{\en\tt evalc} παίρνει σαν όρισμα έναν μιγαδικό αριθμό {\en\tt z}.\\
{\en\tt evalc}  επιστρέφει τον μιγαδικό αριθμό, γραμμένο ως
{\en\tt re(z)+i*im(z)}.\\
Είσοδος :
\begin{center}{\en\tt evalc(sqrt(2)*exp(i*pi/4))}\end{center}
Έξοδος :
\begin{center}{\en\tt 1+i}\end{center}

\subsection{Μέτρο ενός μιγαδικού αριθμού  : {\tt \textlatin{abs}}}\index{abs}
\noindent{\en\tt abs} παίρνει σαν όρισμα έναν μιγαδικό αριθμό.\\
{\en\tt abs} επιστρέφει το μέτρο (απόλυτη τιμή) του μιγαδικού αριθμού.\\
Είσοδος :
\begin{center}{\en\tt abs(3+4*i)}\end{center}
Έξοδος :
\begin{center}{\en\tt 5}\end{center}

\subsection{Όρισμα ενός μιγαδικού αριθμού : {\tt \textlatin{arg}}}\index{arg|textbf}
\noindent{\en\tt arg} παίρνει σαν όρισμα ένα μιγαδικό αριθμό.\\
{\en\tt arg} επιστρέφει το όρισμα του μιγαδικού αριθμού.\\
Είσοδος :
\begin{center}{\en\tt arg(3+4.i)}\end{center}
Έξοδος :
\begin{center}{\en\tt atan(4/3)}\end{center}

\subsection{Κανονικοποιημένος μιγαδικός αριθμός: {\tt \textlatin{normalize \\ unitV}}}\index{unitV}\index{normalize}
\noindent{\en\tt normalize} ή {\en\tt unitV} παίρνει σαν όρισμα έναν μιγαδικό αριθμό.\\
{\en\tt normalize} ή {\en\tt unitV} επιστρέφει τον μιγαδικό αριθμό διαιρεμένο με το μέτρο του.\\
Είσοδος :
\begin{center}{\en\tt normalize(3+4*i)}\end{center}
Έξοδος :
\begin{center}{\en\tt (3+4*i)/5}\end{center}

\subsection{Συζυγής ενός μιγαδικού αριθμού: {\tt \textlatin{conj}}}\index{conj|textbf}
\noindent{\en\tt conj} παίρνει σαν όρισμα ένα μιγαδικό αριθμό.\\
{\en\tt conj} επιστρέφει τον συζυγή μιγαδικό του μιγαδικού αριθμού.\\
Είσοδος :
\begin{center}{\en\tt conj(3+4*i)}\end{center}
Έξοδος :
\begin{center}{\en\tt 3-4*i}\end{center}

\subsection{Πολλαπλασιασμός με τον συζυγή μιγαδικού:\\
 {\tt \textlatin{mult\_c\_conjugate}}}\index{mult\_c\_conjugate}
\noindent {\en\tt mult\_c\_conjugate} παίρνει σαν όρισμα μία μιγαδική παράσταση.\\ 
Αν αυτή η παράσταση έχει μιγαδικό παρανομαστή,
{\en\tt mult\_c\_conjugate} πολλαπλασιάζει τον αριθμητή και τον παρανομαστή  της παράστασης με τον συζυγή μιγαδικό του παρανομαστή.\\
Αν η παράσταση δεν έχει μιγαδικό παρανομαστή,
{\en\tt mult\_c\_conjugate} πολλαπλασιάζει τον αριθμητή και τον παρανομαστή  της παράστασης με τον συζυγή μιγαδικό του αριθμητή.\\
Είσοδος :
\begin{center}{\en\tt mult\_c\_conjugate((2+i)/(2+3*i))}\end{center}
Έξοδος :
\begin{center}{\en\tt (2+i)*(2+3*(-i))/((2+3*(i))*(2+3*(-i)))}\end{center}
Είσοδος :
\begin{center}{\en\tt mult\_c\_conjugate((2+i)/2)}\end{center}
Έξοδος :
\begin{center}{\en\tt (2+i)*(2+-i)/(2*(2+-i))}\end{center}

\subsection{Βαρύκεντρο μιγαδικών αριθμών: 
{\tt \textlatin{ barycentre}}}\index{barycentre} \label{sec:baryc}
{\bf Δείτε επίσης :}  \ref{sec:barycentre2} και \ref{sec:barycentre3}.\\
\noindent{\en\tt barycentre} παίρνει σαν όρισμα δύο λίστες ίδιου μεγέθους
(πίνακας με δύο στήλες): 
\begin{itemize}
\item τα στοιχεία της πρώτης λίστας (στήλης) 
είναι σημεία $A_j$ ή μιγαδικοί αριθμοί $a_j$ (που αντιστοιχούν στα σημεία $A_j$),
\item τα στοιχεία της δεύτερης λίστας (στήλης)  είναι πραγματικοί συντελεστές
$\alpha_j$ τέτοιοι ώστε $\sum \alpha_j \neq 0$.
\end{itemize}
{\en\tt barycentre} επιστρέφει το βαρύκεντρο από τα σημεία $A_j$ σταθμισμένα 
με τους πραγματικούς συντελεστές  $\alpha_j$.
Αν $\sum \alpha_j = 0$, {\en\tt barycentre} επιστρέφει λάθος.\\
{\bf Προσοχή} Για να έχετε έναν μιγαδικό αριθμό στην έξοδο, η είσοδος θα πρέπει να είναι :\\
{\en\tt affixe(barycentre(...,...))} 
επειδή {\en\tt barycentre(...,...)} επιστρέφει ένα σημείο, όχι ένα μιγαδικό αριθμό.\\
Είσοδος :
\begin{center}{\en\tt affixe(barycentre([1+i,1-i],[1,1]))}\end{center}
ή :
\begin{center}{\en\tt affixe(barycentre([[1+i,1],[1-i,1]]))}\end{center}
Έξοδος :
\begin{center}{\en\tt i}\end{center}

\section{Αλγεβρικές παραστάσεις}
\subsection{Αποτίμηση παράστασης : {\tt \textlatin{eval}}}\index{eval}
\noindent {\en\tt eval} χρησημοποιείται για να αποτιμά παραστάσεις. Επειδή το
{\en\tt Xcas} πάντα αποτιμά παραστάσεις που εγγράφονται στη γραμμη εντολών , το
{\en\tt eval} χρησιμοποιείται κυρίως για να αποτιμήσει μία υπο-παράσταση.\\
Είσοδος :
\begin{center}{\en\tt a:=2}\end{center}
Έξοδος :
\begin{center}{\en\tt 2}\end{center}
Είσοδος :
\begin{center}{\en\tt eval(2+3*a)}\end{center}
ή
\begin{center}{\en\tt 2+3*a}\end{center}
Έξοδος :
\begin{center}{\en\tt 8}\end{center}

\subsection{Αποτίμηση αλγεβρικής παράστασης : {\tt \textlatin{evala}}}
\index{evala}
\noindent  Στο \en Maple,\gr {\en\tt evala} αποτιμά μια παράσταση με αλγεβρικές επεκτάσεις. Στο {\en\tt Xcas}, {\en\tt evala} δεν είναι απαραίτητο, συμπεριφέρεται σαν {\en\tt eval}.\\

\subsection{Πρόληψη αποτίμησης : {\tt \textlatin{quote hold  }}{\tt '}}\index{quote|textbf}\index{hold|textbf}\index{'|textbf}
Μέσα σε μία εντολή, μια υπο-παράσταση είτε με {\tt '} ή με
{\en\tt quote} ή {\en\tt hold} δεν μπορεί να αποτιμηθεί.\\
{\bf Σχόλιο}
{\en\tt a:=quote(a)} (ή {\en\tt a:=hold(a)}) ισοδυναμεί με {\en\tt purge(a)} 
(για λόγους συμβατότητας με το \en Maple \gr). Επιστρέφει
την τιμή της μεταβλητής (ή την υπόθεση που γίνεται για αυτή τη μεταβλητή). \\
Είσοδος :
\begin{center}{\en\tt a:=2;quote(2+3*a)}\end{center}
ή
\begin{center}{\tt {\en a:=2;}{\tt '}2+3*a{\tt '}}\end{center}
Έξοδος :
\begin{center}{\en\tt (2,2+3*a)}\end{center}

\subsection{Εξαναγκασμός αποτίμησης : {\tt \textlatin{unquote}}}\index{unquote}
{\en\tt unquote} χρησιμοποιείται για να αποτιμηθεί μια αναφερθείσα υπο-παράσταση.\\
Για παράδειγμα, σε ένα πρόγραμμα, η μεταβλητή αναφέρεται αυτόματα 
(δεν αποτιμάται) έτσι ώστε ο χρήστης να μην χρειάζεται να την αναφέρει ρητά κάθε φορά που θέλει να αλλάξει την τιμή της. 
Ωστόσο,σε κάποιες περιπτώσεις, ίσως χρειαστεί να αποτιμηθεί.
\\
Είσοδος:
\begin{center}{\en\tt  purge(b);a:=b;unquote(a):=3}\end{center}
Έξοδος :
\begin{center}{\tt b περιέχει το 3, επειδή το {\en\tt a} ισούται με 3}\end{center}

\subsection{Προσεταιριστικότητα : {\tt \textlatin{expand fdistrib}}}\index{fdistrib}\index{expand}
\noindent {\en\tt expand} ή {\en\tt fdistrib} παίρνει ως όρισμα μία παράσταση.\\
{\en\tt expand} ή {\en\tt fdistrib} επιστρέφει την παράσταση όπου έχει εφαρμοσθεί η προσεταιριστικότητα του πολλαπλασιασμού, ως προς την πρόσθεση.\\
Είσοδος :
\begin{center}{\en \tt expand((x+1)*(x-2))}\end{center}
ή :
\begin{center}{\en\tt fdistrib((x+1)*(x-2))}\end{center}
Έξοδος :
\begin{center}{\en\tt x\verb|^|2-2*x+x-2}\end{center} 

\subsection{Κανονική μορφή : {\tt \textlatin{canonical\_form}}}\index{canonical\_form}
\noindent{\en\tt canonical\_form} παίρνει σαν όρισμα ένα τριώνυμο δευτέρου βαθμού.\\ 
{\en\tt canonical\_form} επιστρέφει την κανονική μορφή του ορίσματος.\\
Παράδειγμα :\\
Βρείτε την κανονική μορφή του :
$$x^2-6x+1$$
Είσοδος :
\begin{center}{\en\tt canonical\_form(x\verb|^|2-6*x+1)}\end{center}
Έξοδος :
\begin{center}{\en\tt (x-3)\verb|^|2-8}\end{center}

\subsection{Πολλαπλασιασμός επί συζυγή :
{\tt \textlatin{mult\_conjugate}}}\index{mult\_conjugate} 
\noindent {\en\tt mult\_conjugate} παίρνει ως όρισμα μία παράσταση όπου  
ένας παρονομαστής ή ένας αριθμητής υποτίθεται ότι περιέχει τετραγωνική ρίζα :
\begin{itemize}
\item εάν ο παρονομαστής περιέχει μια τετραγωνική ρίζα,\\
{\en\tt mult\_conjugate} πολλαπλασιάζει τον αριθμητή και τον παρονομαστή
της παράστασης με τον συζυγή του παρονομαστή.
\item αλλιώς, εάν ο αριθμητής περιέχει μια τετραγωνική ρίζα,\\
{\en\tt mult\_conjugate} πολλαπλασιάζει τον αριθμητή και τον παρονομαστή
της παράστασης με τον συζυγή του αριθμητή. 
\end{itemize}
Είσοδος :
\begin{center}{\en\tt mult\_conjugate((2+sqrt(2))/(2+sqrt(3)))}\end{center}
Έξοδος :
\begin{center}{\en\tt (2+sqrt(2))*(2-sqrt(3))/((2+sqrt(3))*(2-sqrt(3)))}\end{center}
Είσοδος :
\begin{center}{\en\tt mult\_conjugate((2+sqrt(2))/(sqrt(2)+sqrt(3)))}\end{center}
Έξοδος :
\begin{center}{\en\tt (2+sqrt(2))*(-sqrt(2)+sqrt(3))/}\end{center}
\begin{center}{\en\tt ((sqrt(2)+sqrt(3))*(-sqrt(2)+sqrt(3)))}\end{center}
Είσοδος :
\begin{center}{\en\tt mult\_conjugate((2+sqrt(2))/2)}\end{center}
Έξοδος :
\begin{center}{\en\tt (2+sqrt(2))*(2-sqrt(2))/(2*(2-sqrt(2)))}\end{center}

\subsection{Διαχωρισμός μεταβλητών : {\tt \textlatin {split}}}\index{split}
\noindent{\en\tt split} παίρνει δύο ορίσματα: μια παράσταση δύο μεταβλητών και την λίστα αυτών των δύο μεταβλητών.\\
Εάν η παράσταση μπορεί να παραγοντοποιηθεί σε δύο παράγοντες
όπου κάθε παράγοντας εξαρτάται  
μόνο από μία μεταβλητή, {\en\tt split} επιστρέφει τη λίστα αυτών των δύο
παραγόντων, αλλιώς επιστρέφει τη λίστα {\en\tt [0]}.\\
Είσοδος :
\begin{center}{\en\tt split((x+1)*(y-2),[x,y])}\end{center}
ή:
\begin{center}{\en\tt split(x*y-2*x+y-2,[x,y])}\end{center}
Έξοδος :
\begin{center}{\en\tt [x+1,y-2]}\end{center} 
Είσοδος :
\begin{center}{\en\tt split((x\verb|^|2*y\verb|^|2-1,[x,y])}\end{center}
Έξοδος :
\begin{center}{\en\tt [0]}\end{center}

\subsection{Παραγοντοποίηση : {\tt \textlatin {factor}}}\index{factor|textbf}\label{sec:factore}
\noindent{\en\tt factor} παίρνει ως όρισμα μία παράσταση.\\
{\en\tt factor} παραγοντοποιεί αυτή την παράσταση στο πεδίο των συντελεστών της,
με την προσθήκη του $i$ όταν ο τρόπος λειτουργίας είναι στους μιγαδικούς. Εάν το {\en\tt sqrt} είναι ενεργοποιημένο
στις Ρυθμίσεις {\en Cas} (ή στην μπάρα ρυθμίσεων {\en Config}), πολυώνυμα δευτέρου βαθμού παραγοντοποιούνται είτε στους 
μιγαδικούς είτε στους πραγματικούς εάν η διακρίνουσα είναι θετική.\\
{\bf Παραδείγματα}
\begin{enumerate}
\item Παραγοντοποιήστε το $x^4-1$ στο $\mathbb Q$.\\
Είσοδος :
\begin{center}{\en\tt factor(x\verb|^|4-1)}\end{center}
Έξοδος :
\begin{center}{\en\tt (x\verb|^|2+1)*(x+1)*(x-1)}\end{center}
Οι συντελεστές είναι ρητοί, οπότε και οι παράγοντες είναι πολυώνυμα με
ρητούς συντελεστές.\\
\item Παραγοντοποιήστε το $x^4-1$ στο $\mathbb Q[i]$ \\
Για την παραγοντοποίηση στους μιγαδικούς, τσεκάρετε το {\en\tt complex} στις Ρυθμίσεις  {\en\tt Cas} (ή στην μπάρα ρυθμίσεων {\en Config}).\\
Είσοδος :
\begin{center}{\en\tt factor(x\verb|^|4-1)}\end{center}
Έξοδος :
\begin{center}{\en\tt -i*(-x+-i)*(i*x+1)*(-x+1)*(x+1)}\end{center}
\item  Παραγοντοποιήστε το  $x^4+1$ στο $\mathbb Q$\\
Είσοδος:
\begin{center}{\en\tt factor(x\verb|^|4+1)}\end{center}
Έξοδος :
\begin{center}{\en\tt x\verb|^|4+1}\end{center}
Πράγματι, το $ x^4+1$ δεν έχει παράγοντα με ρητούς συντελεστές.\\
\item  Παραγοντοποιήστε το  $x^4+1$ στο $\mathbb Q[i]$\\
Τσεκάρετε το {\en\tt complex} στις Ρυθμίσεις  {\en\tt Cas} (ή στην μπάρα ρυθμίσεων {\en Config}).\\
Είσοδος :
\begin{center}{\en\tt factor(x\verb|^|4-1)}\end{center}
Έξοδος :
\begin{center}{\en\tt (x\verb|^|2+i)*(x\verb|^|2+-i)}\end{center}
\item  Παραγοντοποιήστε το  $x^4+1$ στο $\mathbb R$.\\
Πρέπει να εισάγετε την τετραγωνική ρίζα που απαιτείται για την επέκταση του σώματος των
ρητών συντελεστών. Για να το κάνετε αυτό με τη βοήθεια του {\en\tt Xcas}, 
αρχικά τσεκάρετε το {\en\tt complex} στις Ρυθμίσεις  {\en\tt Cas} (ή στην μπάρα ρυθμίσεων {\en Config}) και πληκτρολογείστε :\\
\begin{center}{\en\tt solve(x\verb|^|4+1,x)}\end{center}\index{solve}\index{resoudre}
Έξοδος :
\begin{center}{\en\tt  [sqrt(2)/2+(i)*sqrt(2)/2,sqrt(2)/2+(i)*(-(sqrt(2)/2)),
 -sqrt(2)/2+(i)*sqrt(2)/2,-sqrt(2)/2+(i)*(-(sqrt(2)/2))]}\end{center}
Οι ρίζες εξαρτώνται από το $\sqrt 2$. Αποεπιλέξτε την επιλογή για μιγαδικούς στις Ρυθμίσεις  {\en\tt Cas} και πληκτρολογείστε :
\begin{center}{\en\tt factor(x\verb|^|4+1,sqrt(2))}\end{center}
Έξοδος :
\begin{center}{\en\tt (x\verb|^|2+sqrt(2)*x+1)*(x\verb|^|2+(-(sqrt(2)))*x+1)}\end{center}
Για να παραγοντοποιήσετε στο $\mathbb C$, τσεκάρετε το {\en\tt complex} στις Ρυθμίσεις  {\en\tt Cas} (ή στην μπάρα ρυθμίσεων {\en Config}) και πληκτρολογείστε {\en\tt cFactor(x\verb|^|4+1,sqrt(2))} 
({\en cf} {\en\tt cFactor}).
\end{enumerate}

\subsection{Παραγοντοποίηση στους μιγαδικούς : {\tt \textlatin {cFactor}}}\index{cFactor}
\noindent{\en\tt cFactor} παίρνει ως όρισμα μία παράσταση.\\
{\en\tt cFactor} παραγοντοποεί αυτή την παράσταση στο πεδίο 
$\mathbb Q[i] \subset \mathbb C$ (ή στο μιγαδικο-ποιημένο πεδίο των
συντελεστών του ορίσματος) ακόμη κι αν είστε σε τρόπο λειτουργίας για πραγματικούς.\\
{\bf Παραδείγματα}
\begin{enumerate}
\item Παραγοντοποιήστε το $x^4-1$ στο $\mathbb Z[i]$.\\
Είσοδος :
\begin{center}{\en\tt cFactor(x\verb|^|4-1)}\end{center}
Έξοδος :
\begin{center}{\en\tt -((x+-i)*((-i)*x+1)*((-i)*x+i)*(x+1))}\end{center}
\item Παραγοντοποιήστε το $x^4+1$ στο $\mathbb Z[i]$.\\
Είσοδος :
\begin{center}{\en\tt cFactor(x\verb|^|4+1)}\end{center}
Έξοδος :
\begin{center}{\en\tt (x\verb|^|2+i)*(x\verb|^|2+-i)}\end{center}
\item Για την πλήρη παραγοντοποίηση του $x^4+1$, 
επιλέξτε το κουτί {\en sqrt} στις Ρυθμίσεις  {\en\tt Cas} (ή στην μπάρα ρυθμίσεων {\en Config}) ή πληκτρολογείστε :
\begin{center}{\en\tt cFactor(x\verb|^|4+1,sqrt(2))}\end{center}
Έξοδος :
\begin{center}{\en\tt sqrt(2)*1/2*(sqrt(2)*x+1-i)*(sqrt(2)*x-1+i)*sqrt(2)* 1/2*(sqrt(2)*x+1+i)*(sqrt(2)*x-1-i)}\end{center}
\end{enumerate}


\subsection{Ρίζες μιας παράστασης : {\tt \textlatin {zeros}}}\index{zeros}
\noindent{\en\tt zeros} παίρνει ως όρισμα μια παράσταση του $x$.\\
{\en\tt zeros} επιστρέφει μια λίστα τιμών του $x$ όπου η παράσταση 
μηδενίζεται. Η λίστα μπορεί να μην είναι πλήρης στον ακριβή ({\en exact}) τρόπο λειτουργίας εάν η παράσταση δεν είναι πολυώνυμο ή εάν ενδιάμεσες παραγοντοποιήσεις έχουν ανάγωγους παράγοντες βαθμού $>$ 2.\\
Στον τρόπο λειτουργίας για πραγματικούς, (το κουτάκι για μιγαδικούς είναι αποεπιλεγμένο στις Ρυθμίσεις  {\en\tt Cas} (ή στην μπάρα ρυθμίσεων {\en Config})
ή αν έχουμε δώσει την εντολή {\en\tt complex\_mode:=0}), μόνο πραγματικές ρίζες επιστρέφονται. Με την εντολή  
{\en\tt complex\_mode:=1} επιστρέφονται πραγματικές και μιγαδικές ρίζες. Δείτε 
επίσης {\en\tt cZeros} για να πάρετε μιγαδικές ρίζες στον τρόπο λειτουργίας για πραγματικούς.\\
Είσοδος με τρόπο λειτουργίας για πραγματικούς : 
\begin{center}{\en\tt zeros(x\verb|^|2+4)}\end{center} 
Έξοδος :
\begin{center}{\en\tt []}\end{center} 
Είσοδος με τρόπο λειτουργίας για μιγαδικούς : 
\begin{center}{\en\tt zeros(x\verb|^|2+4)}\end{center} 
Έξοδος  :
\begin{center}{\en\tt [-2*i,2*i]}\end{center} 
Είσοδος με τρόπο λειτουργίας για πραγματικούς : 
\begin{center}{\en\tt zeros(ln(x)\verb|^|2-2)}\end{center} 
Έξοδος :
\begin{center}{\en\tt [exp(sqrt(2)),exp(-(sqrt(2)))]}\end{center} 
Είσοδος με τρόπο λειτουργίας για πραγματικούς : 
\begin{center}{\en\tt zeros(ln(y)\verb|^|2-2,y)}\end{center} 
Έξοδος :
\begin{center}{\en\tt [exp(sqrt(2)),exp(-(sqrt(2)))]}\end{center} 
Είσοδος με τρόπο λειτουργίας για πραγματικούς :  
\begin{center}{\en\tt zeros(x*(exp(x))\verb|^|2-2*x-2*(exp(x))\verb|^|2+4)}\end{center} 
Έξοδος :
 \begin{center}{\en\tt [log(sqrt(2)),2]}\end{center} 

\subsection{Μιγαδικές ρίζες μιας παράστασης : {\tt \textlatin {cZeros}}}\index{cZzeros}
\noindent{\en\tt cZeros} παίρνει ως όρισμα μια παράσταση του $x$.\\
{\en\tt cZeros} επιστρέφει μια λίστα μιγαδικών τιμών του $x$ στις οποίες η παράσταση 
μηδενίζεται. Η λίστα μπορεί να μην είναι πλήρης στον ακριβή ({\en exact}) τρόπο λειτουργίας εάν η παράσταση δεν είναι πολυώνυμο ή εάν ενδιάμεσες παραγοντοποιήσεις έχουν ανάγωγους παράγοντες βαθμού $>$ 2.\\
Είσοδος  με τρόπο λειτουργίας για πραγματικούς ή για μιγαδικούς : 
\begin{center}{\en\tt cZeros(x\verb|^|2+4)}\end{center} 
Έξοδος :
\begin{center}{\en\tt [-2*i,2*i]}\end{center} 
Είσοδος : 
\begin{center}{\en\tt cZeros(ln(x)\verb|^|2-2)}\end{center} 
Έξοδος :
\begin{center}{\en\tt [exp(sqrt(2)),exp(-(sqrt(2)))]}\end{center} 
Είσοδος : 
\begin{center}{\en\tt cZeros(ln(y)\verb|^|2-2,y)}\end{center} 
Έξοδος :
\begin{center}{\en\tt [exp(sqrt(2)),exp(-(sqrt(2)))]}\end{center} 
Είσοδος : 
\begin{center}{\en\tt cZeros(x*(exp(x))\verb|^|2-2*x-2*(exp(x))\verb|^|2+4)}\end{center} 
Έξοδος :
\begin{center}{\en\tt [log(sqrt(2)),log(-sqrt(2)),2]}\end{center} 

\subsection{Κανονική μορφή : {\tt \textlatin {normal}}}\index{normal|textbf}
\noindent{\en\tt normal} παίρνει ως όρισμα μια παράσταση. 
Η παράσταση θεωρείται σαν ένα ρητό κλάσμα όσον αφορά  
γενικευμένες μεταβλητές
(είτε αληθινές μεταβλητές είτε υπερβατικές συναρτήσεις που αντικαθίστανται από 
προσωρινές μεταβλητές) με συντελεστές στο $\mathbb Q$ ή στο $\mathbb Q[i]$
ή σε μια αλγεβρική επέκταση (π.χ. $\mathbb Q[\sqrt{2}]$).
{\en\tt normal} επιστρέφει την ανεπτυγμένη, ανάγωγη, αναπαράσταση 
αυτού του ρητού κλάσματος. Δείτε επίσης το {\en\tt ratnormal} για καθαρά ρητά κλάσματα
ή {\en\tt simplify} εάν οι υπερβατικές συναρτήσεις δεν είναι
αλγεβρικά ανεξάρτητες.\\
Είσοδος :  
\begin{center}{\en\tt normal((x-1)*(x+1))}\end{center}
Έξοδος :
 \begin{center}{\en\tt x\verb|^|2-1}\end{center}  
{\bf Σχόλια}
\begin{itemize}
\item Αντίθετα με το {\en\tt simplify},
το {\en\tt normal} δεν δοκιμάζει να βρει αλγεβρικές σχέσεις μεταξύ
υπερβατικών συναρτήσεων όπως $\cos(x)^2+\sin(x)^2=1$.
\item
Μερικές φορές είναι απαραίτητο να τρέξετε την εντολή {\en\tt normal}  δύο φορές 
για να πάρετε μια πλήρως ανάγωγη μορφή μια παράστασης που 
περιέχει αλγεβρικές επεκτάσεις.
\end{itemize}


\subsection{Απλοποίηση: {\tt \textlatin {simplify}}}\index{simplify|textbf}
\noindent{\en\tt simplify} απλοποεί μια παράσταση. Συμπεριφέρεται όπως 
το {\en\tt normal} για ρητά κλάσματα και αλγεβρικές επεκτάσεις. 
Για παράστασεις
που περιέχουν υπερβατικές συναρτήσεις,το {\en\tt simplify} δοκιμάζει πρώτα να τις ξαναγράψει
σε όρους αλγεβρικά ανεξάρτητων υπερβατικών συναρτήσεων
Για τριγωνομετρικές παράστασεις, αυτό απαιτεί την επιλογή για τα ακτίνια
(τσεκάρετε το {\en\tt radian} στις Ρυθμίσεις  {\en\tt Cas} ή εισάγετε {\en \tt angle\_radian:=1}).\\
Είσοδος :  
\begin{center}{\en\tt simplify((x-1)*(x+1))}\end{center}
Έξοδος  :
 \begin{center}{\en\tt x\verb|^|2-1}\end{center}  
Είσοδος :  
\begin{center}{\en\tt simplify(3-54*sqrt(1/162))}\end{center}
Έξοδος  :
 \begin{center}{\en\tt -3*sqrt(2)+3}\end{center} 
Είσοδος :
\begin{center}{\en\tt simplify((sin(3*x)+sin(7*x))/sin(5*x))}\end{center}
Έξοδος  :
\begin{center}{\en\tt 4*(cos(x))\verb|^|2-2}\end{center}
 
\subsection{Κανονική μορφή για ρητά κλάσματα: {\tt \textlatin {ratnormal}}}\index{ratnormal}
\noindent{\en\tt ratnormal} ξαναγράφει μια παράσταση χρησιμοποιώντας
την ανάγωγη αναπαράστασή της. Η παράσταση θεωρείται
σαν ένα ρητό κλάσμα πολλών μεταβλητών  με 
συντελεστές στο $\mathbb Q$ (ή στο $\mathbb Q[i]$). Οι μεταβλητές είναι
γενικευμένες μεταβλητές οι οποίοι υποτίθεται πως είναι αλγεβρικά ανεξάρτητες.
Αντίθετα με το {\en\tt normal}, μια αλγεβρική επέκταση
θεωρείται σαν μία γενικευμένη μεταβλητή. Επομένως, το {\en\tt ratnormal}
ειναι γρηγορότερο αλλά ίσως να παραλείπει κάποιες απλοποιήσεις
αν η παράσταση περιέχει ριζικά ή αλγεβρικά ανεξάρτητες υπερβατικές 
συναρτήσεις.\\
Είσοδος :  
\begin{center}{\en\tt ratnormal((x\verb|^|3-1)/(x\verb|^|2-1))}\end{center}
Έξοδος :
 \begin{center}{\en\tt (x\verb|^|2+x+1)/(x+1)}\end{center}  
Είσοδος :  
\begin{center}{\en\tt ratnormal((-2x\verb|^|3+3x\verb|^|2+5x-6)/(x\verb|^|2-2x+1))}\end{center}
Έξοδος :
 \begin{center}{\en\tt (-2*x\verb|^|2+x+6)/(x-1)}\end{center} 

\subsection{Αντικατάσταση μιας μεταβλητής με μια τιμή : {\tt \textlatin {subst}}}\index{subst|textbf}\label{sec:subst}
\noindent{\en\tt subst} παίρνει δύο ή τρία ορίσματα : 
\begin{itemize}
\item μια παράσταση που εξαρτάται από μια μεταβλητή,
μια ισότητα (μεταβλητή=τιμή αντικατάστασης) ή μια λίστα από ισότητες.
\item μια παράσταση που εξαρτάται από μια μεταβλητή, μια μεταβλητή ή μια λίστα από μεταβλητές, μία τιμή ή μια λίστα από τιμές για αντικατάσταση.
\end{itemize}
{\en\tt subst} επιστρέφει την παράσταση όπου έχει γίνει η αντικατάσταση.
Να σημειωθεί πως το {\en\tt subst} δεν αναφέρει ({\en quote}) το όρισμά του, γιαυτό το λόγο
σε μια κανονική διαδικασία αποτίμησης, δεν θα πρέπει να έχει γίνει απόδοση τιμής στην μεταβλητή που αντικαθίσταται, αλλιώς θα αντικαθίσται από την τιμή που της αποδόθηκε
πριν να γίνει η αντικατάσταση.\\
Είσοδος :
\begin{center}{\en\tt subst(a\verb|^|2+1,a=2)}\end{center} 
ή :
\begin{center}{\en\tt subst(a\verb|^|2+1,a,2)}\end{center} 
Έξοδος (εάν δεν έχει αποδοθεί τιμή στην {\en\tt a} αλλιώς πρώτα πληκτρολογείστε  {\en\tt purge(a)}) :
\begin{center}{\en\tt 5}\end{center} 
Είσοδος :
\begin{center}{\en\tt subst(a\verb|^|2+b,[a,b],[2,1])}\end{center} 
Ή:
\begin{center}{\en\tt subst(a\verb|^|2+b,[a=2,b=1])}\end{center} 
Έξοδος  (εάν δεν έχουν αποδοθεί τιμές στις {\en\tt a} και {\en\tt b} αλλιώς πρώτα πληκτρολογείστε
{\en\tt purge(a,b)}) :
 \begin{center}{\en\tt 2\verb|^|2+1}\end{center}
{\en\tt subst} μπορεί να χρησιμοποιηθεί για αλλαγή μεταβλητής σε ένα ολοκλήρωμα. 
Σε αυτή την περίπτωση, η εντολή {\en\tt integrate}  θα πρέπει να αναφέρεται
(διαφορετικά, το ολοκλήρωμα θα υπολογίζεται πριν την αντικατάσταση) ή 
να χρησιμοποιείται η αδρανής μορφή {\en\tt Int}.
Και στις δύο περιπτώσεις, το όνομα της μεταβλητής πρέπει να δίνεται ως
όρισμα του {\en\tt Int} ή {\en\tt integrate} ακόμα και αν ολοκληρώνετε ως προς {\en\tt x}.\\
Είσοδος:
\begin{center}{\en\tt subst({\gr\tt '}integrate(sin(x\verb|^|2)*x,x,0,pi/2){\gr\tt '},x=sqrt(t))}\end{center}
Ή  :
\begin{center}{\en\tt subst(Int(sin(x\verb|^|2)*x,x,0,pi/2),x=sqrt(t))}\end{center}
Έξοδος
\begin{center}{\en\tt integrate(sin(t)*sqrt(t)*1/2*1/t*sqrt(t),t,0,(pi/2)\verb|^|2)}\end{center} 
Είσοδος :
\begin{center}{\en\tt subst({\gr\tt '}integrate(sin(x\verb|^|2)*x,x){\gr\tt '},x=sqrt(t))}\end{center}
Ή :
\begin{center}{\en\tt subst(Int(sin(x\verb|^|2)*x,x),x=sqrt(t))}\end{center}
Έξοδος
\begin{center}{\en\tt integrate(sin(t)*sqrt(t)*1/2*1/t*sqrt(t),t)}\end{center} 

\subsection{Αντικατάσταση μιας μεταβλητής με μια τιμή (συμβατότητα με {\textlatin {Maple}} και {\textlatin {Mupad }}) : {\tt \textlatin {subs}}}\index{subs}\label{sec:subs}
\noindent Στο {\en\tt Maple} και στο {\en\tt Mupad}, κάποιος θα χρησιμοποιούσε την εντολή {\en\tt subs}
για να αντικαταστήσει  σε μια παράσταση μια μεταβλητή με μια τιμή. Αλλά η σειρά των ορισμάτων διαφέρει ανάμεσα στο {\en\tt Maple} και στο {\en\tt Mupad}. Επομένως, για να επιτευχθεί η συμβατότητα, η διάταξη των ορισμάτων της εντολής  {\en\tt subs} του {\en\tt Xcas} εξαρτάται από τον τρόπο λειτουργίας του
\begin{itemize}
\item
Στον τρόπο λειτουργίας {\en\tt Maple}, το {\en\tt subs} παίρνει δύο ορίσματα: μια ισότητα
(μεταβλητή=τιμή αντικατάστασης) και την παράσταση.\\
Για την αντικατάσταση πολλών μεταβλητών σε μια παράσταση, χρησιμοποιείστε μια λίστα με ισότητες
(όνομα μεταβλητής  = τιμή αντικατάστασης) ως πρώτο όρισμα.
\item Στον τρόπο λειτουργίας {\en\tt Mupad} ή {\en\tt  Xcas} ή {\en\tt TI}, το {\en\tt subs} 
παίρνει δύο ή τρία ορίσματα : 
μία παράσταση και μια ισότητα (μεταβλητή=τιμή αντικατάστασης) ή 
μια παράσταση, ένα όνομα μεταβλητής και την τιμή αντικατάστασης.\\
Για την αντικατάσταση διαφόρων μεταβλητών, το {\en\tt subs} παίρνει δύο ή τρία ορίσματα :
\begin{itemize}
\item μια παράσταση από μεταβλητές και μια λίστα με ισότητες   
(όνομα μεταβλητής = τιμή αντικατάστασης),\gr
\item
μία παράσταση από μεταβλητές, μια λίστα από μεταβλητές και μία λίστα από τις  
τιμές αντικατάστασης.\gr
\end{itemize}
\end{itemize}
{\en\tt subs} επιστρέφει την παράσταση όπου έχει γίνει η αντικατάσταση.
Να σημειωθεί πως το {\en\tt subs} δεν αναφέρει ({\en quote}) το όρισμά του, γιαυτό το λόγο
σε μια κανονική διαδικασία αποτίμησης, δεν θα πρέπει να έχει γίνει απόδοση τιμής στην μεταβλητή που αντικαθίσταται, αλλιώς θα αντικαθίσται από την τιμή που της αποδόθηκε
πριν να γίνει η αντικατάσταση.\\
Εισάγετε στον τρόπο λειτουργίας {\en\tt Maple} (εάν δεν έχει αποδοθεί τιμή στην {\en\tt a}  αλλιώς εισάγετε
{\en\tt purge(a)}) :
\begin{center}{\en\tt subs(a=2,a\verb|^|2+1)}\end{center}
Έξοδος 
\begin{center}{\en\tt 2\verb|^|2+1}\end{center}
Εισάγετε στον τρόπο λειτουργίας {\en\tt Maple} (εάν δεν έχουν αποδοθεί τιμές στις μεταβλητές {\en\tt a} και {\en\tt b} 
  αλλιώς εισάγετε {\en\tt purge(a,b)}):
\begin{center}{\en\tt subs([a=2,b=1],a\verb|^|2+b)}\end{center} 
Έξοδος  :
\begin{center}{\en\tt 2\verb|^|2+1}\end{center}
Είσοδος :
\begin{center}{\en\tt subs(a\verb|^|2+1,a=2)}\end{center} 
ή :
\begin{center}{\en\tt subs(a\verb|^|2+1,a,2)}\end{center} 
Έξοδος (εάν δεν έχει αποδοθεί τιμή στην μεταβλητή  {\en\tt a}  αλλιώς εισάγετε {\en\tt purge(a)}) :
\begin{center}{\en\tt 5}\end{center} 
Είσοδος :
\begin{center}{\en\tt subs(a\verb|^|2+b,[a=2,b=1])}\end{center} 
ή :
\begin{center}{\en\tt subs(a\verb|^|2+b,[a,b],[2,1])}\end{center} 
Έξοδος (εάν δεν έχουν αποδοθεί τιμές στις μεταβλητές  {\en\tt a} και {\en\tt b} αλλιώς εισάγετε 
{\en\tt purge(a,b)}) :
\begin{center}{\en\tt 2\verb|^|2+1}\end{center}

\subsection{Αποτίμηση αντιπαραγώγου  στα όρια: {\tt \textlatin {preval}}}\index{preval}
\noindent{\en\tt preval} παίρνει τρία ορίσματα : μία παράσταση {\en\tt F} 
που εξαρτάται από την
μεταβλητή {\en\tt x}, και δύο παραστάσεις {\en\tt a} και {\en\tt b}.\\
{\en\tt preval}  υπολογίζει $F_{|x=b}-F_{|x=a}$.\\
 {\en\tt preval} χρησιμοποιείται για τον υπολογισμό ενός ορισμένου ολοκληρώματος 
όταν είναι γνωστή η αντιπαράγωγος (ή βασικό ολοκλήρωμα ή {\en primitive integral}) $F$ της ολοκληρωτέας συνάρτησης $f$. Υποθέστε
για παράδειγμα ότι {\en\tt F:=int(f,x)}, τότε {\en\tt preval(F,a,b)} ισούται 
με {\en\tt int(f,x,a,b)} αλλά δεν απαιτεί τον υπολογισμό εκ νέου της {\en\tt F}
από το {\en\tt f} εάν αλλάξουν οι τιμές του $a$ ή $b$.\\
Είσοδος :
\begin{center}{\en\tt preval(x\verb|^|2+x,2,3)}\end{center}
Έξοδος :
\begin{center}{\en\tt 6}\end{center}

\subsection{Υποπαράσταση μιας παράστασης : {\tt \textlatin {part}}}\index{part}
\noindent{\en\tt part} παίρνει δύο ορίσματα : μια παράσταση και έναν ακέραιο $n$.\\
{\en\tt part} αποτιμά την παράσταση και μετά επιστρέφει την $n$-οστή υποπαράσταση  
αυτής της παράστασης.\\
Είσοδος :
\begin{center}{\en\tt part(x\verb|^|2+x+1,2)}\end{center}
Έξοδος :
\begin{center}{\en\tt x}\end{center}
Είσοδος :
\begin{center}{\en\tt part(x\verb|^|2+(x+1)*(y-2)+2,2)}\end{center}
Έξοδος :
\begin{center}{\en\tt (x+1)*(y-2)}\end{center}
Είσοδος :
\begin{center}{\en\tt part((x+1)*(y-2)/2,2)}\end{center}
Έξοδος :
\begin{center}{\en\tt y-2}\end{center}

\section{Τιμές του $u_n$}
\subsection{Πίνακας τιμών μιας ακολουθίας: {\tt \textlatin {tablefunc}}}\index{tablefunc}
{\en\tt tablefunc} είναι μια εντολή που χρησιμοποιείται σε ένα υπολογστικό φύλλο,
επιστρέφει μια φόρμα προς συμπλήρωση δύο στηλών, με τον πίνακα τιμών μιας συνάρτησης. Εάν η τιμή του βήματος είναι 1, η εντολή
{\en\tt tablefunc(ex,n,n0,1)}, όπου το {\en\tt ex} είναι μια παράσταση
του  {\en\tt n}, θα συμπληρώσει το υπολογιστικό φύλλο με 
τις τιμές της ακολουθίας $u_n=ex$ για $n=n0,\ n0+1,\ n0+2,....$.

{\bf Παράδειγμα} : εμφάνιση των τιμών της ακολουθίας $u_n=\sin(n)$\\
Επιλέξτε ένα κελί του λογιστικού φύλλου (για παράδειγμα {\en\tt C0}) 
και εισάγετε στη γραμμή εντολών : 
\begin{center}{\en\tt tablefunc(sin(n),n,0,1)}\end{center}
Έξοδος :
\begin{center}{\tt δύο στήλες : {\en\tt n} και {\en\tt sin(n)}}\end{center}
\begin{itemize}
\item στη στήλη  {\en C} εμφανίζονται : Το όνομα της μεταβλητής {\en\tt n}, η τιμή του βήματος 
(αυτή η τιμή θα πρέπει να ισούται με 1 για μια ακολουθία),
η τιμή του {\en\tt n0} (εδώ 0), και κατόπιν ένας αναδρομικός  τύπος ({\en\tt C2+C\$1}, ...). 
\item  στη στήλη {\en D} εμφανίζονται: {\en\tt sin(n)}, {\en\tt "Tablefunc"}, και κατόπιν 
ένας αναδρομικός  τύπος.
\item Σε κάθε γραμμή,
οι τιμές της ακολουθίας ${\tt u_n=\sin(n)}$ αντιστοιχούν 
στις τιμές του {\en\tt n} ξεκινώντας από το {\en\tt n=n0} (εδώ 0).\gr
\end{itemize} 

\subsection{Πίνακας τιμών και διάγραμμα μιας αναδρομικής ακολουθίας : {\tt \textlatin {tableseq}} και {\tt \textlatin {plotseq}}}\index{tableseq|textbf}\index{plotseq}
{\en\tt tableseq} είναι μια εντολή που χρησιμοποιείται σε ένα υπολογιστικό φύλλο,
επιστρέφει μια φόρμα προς συμπλήρωση μιας στήλης με
${\tt u_0, \ u_{n+1}=f(u_{n})}$ (αναδρομή ενός όρου) ή
πιο γενικά $u_0,...,u_k, \ \ u_{n+k+1}=f(u_n,u_{n+1},...,u_{n+k})$.
Η φόρμα συμπληρώνει τη στήλη ξεκινώντας από το επιλεγμένο κελί,
ή ξεκινώντας από το 0 εάν έχει επιλεχθεί όλη η στήλη.\\
Δείτε επίσης {\en\tt plotseq} (τμήμα \ref{sec:plotseq}) για την γραφική αναπαράσταση μιας αναδρομικής ακολουθίας ενός όρου.\\
{\bf Παραδείγματα} :
\begin{itemize}
\item εμφάνιση τιμών της ακολουθίας  $u_0=3.5, \ u_n=\sin(u_{n-1})$\\
Επιλέξτε ένα κελί του λογιστικού φύλλου  (για παράδειγμα {\en\tt B0}) και εισάγετε στη γραμμή εντολών :
\begin{center}{\en\tt tableseq(sin(n),n,3.5)}\end{center}
Έξοδος :
\begin{center}{\tt μια στήλη με {\en sin(n), n,} 3.5 και τον τύπο {\en evalf(subst(B\$0,B\$1,B2))}}
\end{center} 
Οι τιμές της ακολουθίας
${\tt u_0=3.5,\ u_n=sin(u_{n-1})}$ εμφανίζονται στην στήλη
{\en\tt B}.
\item 
 εμφάνιση τιμών  της ακολουθίας {\textlatin {Fibonacci }} $u_0=1, u_1=1 \ u_{n+2}=u_n+u_{n+1}$\\
Επιλέξτε ένα κελί, ας πούμε το {\en\tt B0}, και εισάγετε στη γραμμή εντολών
\begin{center}{\en\tt tableseq(x+y,[x,y],[1,1])}\end{center}
Αυτό συμπληρώνει τη στήλη Β του υπολογιστικού φύλλου με
\begin{center}{\tt \begin{tabular}{|l|l|}
\hline
στήλη &B\\
\hline
0 &{\en x+y} \\
\hline
1 & {\en x}\\
\hline
2 & {\en y}\\
\hline
3 & 1\\
\hline
4 & 1\\
\hline
5 & 2\\
\hline
.. &..\\
\hline
7 & 5\\
\hline
.. &..\\
\hline
\end{tabular}}\end{center}
\end{itemize}

\section{Τελεστές ή ενθηματικές συναρτήσεις}
Ένας τελεστής είναι μια ενθηματική συνάρτηση.

\subsection{Συνήθεις τελεστές :{\tt +, -, *, /, \^\ }}\index{+}\index{'+'}\index{-}\index{'-'}\index{\^\ }
{\tt +, -, *, /, \^\ } είναι οι τελεστές που κάνουν προσθέσεις, αφαιρέσεις, πολλαπλασιασμούς, διαιρέσεις και ύψωση σε δύναμη.

\subsection{Τελεστές του {\tt \textlatin {Xcas}}}\index{\@}\index{\@\@}\index{\$}\index{\%} 
\begin{itemize}
\item
{\tt \$ } είναι η ενθηματική εκδοχή  του {\en\tt seq}. Για παράδειγμα :\\
{\en\tt (2\verb|^|k)\$(k=0..3)= seq(2\verb|^|k,k=0..3)=(1,2,4,8)} (μην ξεχάσετε να βάλετε παρενθέσεις γύρω από τα ορίσματα),
\item
{\en\tt mod} ή {\tt \%} ορίζει έναν αριθμό - υπόλοιπο,
\item
{\en\tt @} συνθέτει συναρτήσεις. Για παράδειγμα :
{\en\tt (f@g)(x)=f(g(x))},
\item
{\en\tt @@ }για να συνθέσουμε μία συνάρτηση πολλές φορές (όπως η δύναμη, αντικαθιστώντας
πολλαπλασιασμό με σύνθεση). Για παράδειγμα :
{\en\tt (f@@3)(x)=f(f(f(x)))},
\item
{\en\tt minus union intersect} για τηn διαφορά, την ένωση και την 
τομή δύο συνόλων,
\item
{\en\tt ->} ορίζει μια συνάρτηση,
\item
{\tt :=}  {\tt $ =>$} αποθηκεύουν μια παράσταση σε μια μεταβλητή ($=>$ είναι η ενθηματική εκδοχή του {\en\tt sto} και η θέση του ορίσματος είναι διαφορετική από αυτήν του {\tt :=}).
Για παράδειγμα : {\en\tt a:=2} ή {\en\tt 2$=>$a} ή {\en\tt sto(2,a)}.
\item
{\tt $=<$} αποθηκεύει μια παράσταση σε μια μεταβλητή, αλλά η αποθήκευση γίνεται κατά αναφορά εάν ο στόχος είναι ένα στοιχείο πίνακα  ή ένα στοιχείο λίστας. Αυτό είναι γρηγορότερο αν τροποποιείτε αντικείμενα  μέσα σε μια υπάρχουσα λίστα ή μέσα σε έναν υπάρχοντα πίνακα μεγάλου μεγέθους. Επειδή δεν δημιουργείται αντίγραφο, η τροποποίηση γίνεται επί τόπου. Χρησιμοποιείστε με προσοχή, όλα τα αντικέιμενα που δείχνουν σε αυτόν τον πίνακα ή σε αυτήν την λίστα θα τροποποιηθούν.
\end{itemize}


\subsection{Ορισμός ενός τελεστή:  {\tt \textlatin {user\_operator}}}\index{user\_operator}\index{Binary@{\sl Binary}|textbf}\index{Delete@{\sl Delete}|textbf}
\noindent {\en\tt user\_operator} παίρνει ως όρισμα :
\begin{itemize}
\item ένα αλφαριθμητικό : το όνομα του τελεστή,
\item μία συνάρτηση δύο μεταβλητών με τιμές σε $\mathbb R$ ή σε 
{\en\tt true, false},
\item μια επιλογή {\en\tt Binary} για τον ορισμό ή {\en\tt Delete} για να ακυρώσετε αυτόν τον ορισμό.
\end{itemize}
{\en\tt user\_operator} επιστρέφει 1 εάν ο ορισμός έχει γίνει διαφορετικά επιστρέφει 0.

{\bf Παράδειγμα 1}\\
Έστω ότι ο $R$ ορίζεται στο $\mathbb R$ ως $x\ R \ y= x*y+x+y$.\\
Για να ορίσετε τον κανόνα $R$, πληκτρολογείστε :
\begin{center}{\en\tt user\_operator("R",(x,y)->x*y+x+y,Binary)}\end{center}
Έξοδος :
\begin{center}{\tt 1}\end{center}  
Είσοδος (μην ξεχνάτε να βάλετε κενά γύρω από το  {\en\tt R}) :
\begin{center}{\en\tt 5 R 7}\end{center}
Έξοδος :
\begin{center}{\tt 47}\end{center}
  
{\bf Παράδειγμα 2}\\
Έστω ότι ο $S$ ορίζεται στο $\mathbb N$ ως :\\
για τους ακεραίους  $x$ και $y$ , $x\ S \ y <=> x$ και $y$ δεν είναι πρώτοι μεταξύ τους.\\
Για να ορίσετε τον κανόνα  $S$, πληκτρολογείστε :
\begin{center}{\en\tt user\_operator("S",(x,y)->(gcd(x,y))!=1,Binary)}\end{center}
Έξοδος :
\begin{center}{\tt 1}\end{center}  
Είσοδος (μην ξενχάτε να βάλετε κενά γύρω από το {\en\tt S}) :
\begin{center}{\en\tt 5 S 7}\end{center}
Έξοδος :
\begin{center}{\tt 0}\end{center}  
Είσοδος (μην ξενχάτε να βάλετε κενά γύρω από το {\en\tt S}) :
\begin{center}{\en\tt 8 S 12}\end{center}
Έξοδος :
\begin{center}{\tt 1}\end{center}

\section{Συναρτήσεις και παραστάσεις με συμβολικές μεταβλητές}
\subsection{Διαφορά ανάμεσα σε συνάρτηση και παράσταση}\index{->}\index{:=}
Μία συνάρτηση {\en\tt f} ορίζεται για παράδειγμα από :\\
{\en\tt f(x):=x\verb|^|2-1} ή από {\en\tt f:=x->x\verb|^|2-1} \\
δηλαδή, για όλα τα $x$, $f(x)$ ισούται με την παράσταση $x^2-1$. Σε αυτή την περίπτωση, για να πάρετε την τιμή του
 $f$ για $x=2$, εισάγετε :  {\en\tt f(2)}.\\
Αλλά εάν η είσοδος είναι 
{\en\tt g:=x\verb|^|2-1}, τότε το {\en\tt g} είναι μία μεταβλητή όπου  αποθηκεύεται
η παράσταση $x^2-1$. Σε αυτή την περίπτωση, για να πάρετε την τιμή του $g$ για $x=2$, εισάγετε :
{\en\tt subst(g,x=2)} ($g$ είναι μια παράσταση ως προς  $x$).

Όταν μία εντολή αναμένει μία συνάρτηση ως όρισμα, αυτό το όρισμα θα είναι είτε ο ορισμός της συνάρτησης
 (π.χ. {\en\tt x->x\verb|^|2-1}) είτε το όνομα μιας μεταβλητής που έχει ανατεθεί σε μια συνάρτηση (π.χ. {\en\tt f} που ορίσθηκε προηγουμένως από π.χ. {\en\tt f(x):=x\verb|^|2-1}).\\
Όταν μία εντολή αναμένει μία παράσταση ως όρισμα, αυτό το όρισμα θα είναι είτε ο ορισμός της παράστασης (για παράδειγμα
{\en\tt x\verb|^|2-1}), είτε το όνομα μιας μεταβλητής  που έχει ανατεθεί σε μια  παράσταση (π.χ. {\en\tt g} που ορίσθηκε προηγουμένως , για παράδειγμα , από {\en\tt g:=x\verb|^|2-1}), ή η αποτίμηση μιας συνάρτησης. π.χ. {\en\tt f(x)} εάν το {\en\tt f} είναι μια συνάρτηση που έχει οριστεί προηγουμένως, για παράδειγμα, από {\en\tt f(x):=x\verb|^|2-1}.

\subsection{Μετασχηματισμός μιας παράστασης σε συνάρτηση :\\ {\tt \textlatin {unapply}}}\index{unapply}
\noindent  {\en\tt unapply} χρησιμοποιείεται για να μετασχηματισμό μιας παράστασης σε συνάρτηση.\\
{\en\tt unapply} παίρνει δύο ορίσματα : μία παράσταση και το όνομα μιας μεταβλητής.\\
{\en\tt unapply} επιστρέφει τη συνάρτηση που ορίζεται από αυτή την παράσταση και αυτή την μεταβλητή.

{\bf Προσοχή} όταν ορίζεται μια συνάρτηση, το δεξί μέλος της εντολής δεν αποτιμάται, γι' αυτό τον λόγο {\en\verb|g:=sin(x+1); f(x):=g|} δεν ορίζει την συνάρτηση $f: x \rightarrow sin(x+1)$ αλλα την συνάρτηση $f: x \rightarrow g$. Για να ορίσετε την πρώτη συνάρτηση, θα πρέπει να χρησιμοποιειθεί το {\en\tt unapply} , όπως στο ακόλουθο παράδειγμα:\\
Είσοδος :
\begin{center}{\en\tt g:= sin(x+1); f:=unapply(g,x)}\end{center}
Έξοδος :
\begin{center}{\en\tt (sin(x+1), (x)->sin(x+1))}\end{center} 
Επομένως,  η μεταβλητή {\en\tt g} ανατίθεται σε μια συμβολική παράσταση και η μεταβλητή {\en\tt f} ανατίθεται σε μια συνάρτηση.\\
Είσοδος :
\begin{center}{\en\tt unapply(exp(x+2),x)}\end{center}
Έξοδος :
\begin{center}{\en\tt (x)->exp(x+2)}\end{center} 
Είσοδος :
\begin{center}{\en\tt f:=unapply(lagrange([1,2,3],[4,8,12]),x)}\end{center}
Έξοδος :
\begin{center}{\en\tt (x)->4+4*(x-1)}\end{center} 
Είσοδος :
\begin{center}{\en\tt f:=unapply(integrate(log(t),t,1,x),x)}\end{center}
Έξοδος :
\begin{center}{\en\tt  (x)->x*log(x)-x+1}\end{center} 
Είσοδος:
\begin{center}{\en\tt f:=unapply(integrate(log(t),t,1,x),x)}\end{center}
\begin{center}{\en\tt f(x)}\end{center}
Έξοδος :
\begin{center}{\en\tt  x*log(x)-x+1}\end{center}
{\bf Σχόλιο}
Ας υποθέσουμε πως η $f$ είναι μια συνάρτηση δύο μεταβλητών $f:(x,w)\rightarrow f(x,w)$, 
και ότι η $g$ είναι η συνάρτηση που ορίζεται από $g: w \rightarrow h_w$ όπου η $h_w$ είναι η συνάρτηση που ορίζεται από 
$h_w(x)=f(x,w)$.\\ 
{\en\tt unapply} χρησιμοποιείται επίσης για να ορίσουμε την $g$ με το {\en\tt Xcas}.\\
Είσοδος : 
\begin{center}{\en\tt f(x,w):=2*x+w}\end{center}
\begin{center}{\en\tt g(w):=unapply(f(x,w),x)}\end{center}
\begin{center}{\en\tt g(3)}\end{center}
Έξοδος :
\begin{center}{\en\tt  x->2$\cdot$ x+3}\end{center} 

\subsection{Κορυφή και φύλλα μιας παράστασης : {\tt \textlatin {sommet feuille op}}}\index{sommet|textbf}\index{feuille|textbf}\index{op|textbf}\label{sec:op}
Ένας τελεστής είναι μια ενθηματική συνάρτηση : για παράδειγμα το '+' είναι ένας τελεστής και το  '{\en\tt sin}' είναι μια συνάρτηση.\\
Μία παράσταση μπορεί να παρασταθεί και από ένα δέντρο. Η κορυφή του δέντρου είναι είτε ένας τελεστής, είτε μια συνάρτηση και τα φύλλα του δέντρου είναι τα ορίσματα του τελεστή ή της συνάρτησης (δείτε επίσης \ref{sec:makesuiteop}).\\
Η εντολή {\en\tt sommet} (αντίστοιχα {\en\tt feuille} ή {\en\tt op}) επιστρέφει την κορυφή (αντίστοιχα την λίστα των φύλλων) μιας παράστασης.\\
Είσοδος :
\begin{center}{\en\tt sommet(sin(x+2))}\end{center}
Έξοδος :
\begin{center} '{\en\tt sin}'\end{center}  
Είσοδος :
\begin{center}{\en\tt sommet(x+2*y)}\end{center}
Έξοδος :
\begin{center} '{\en\tt +}'\end{center}  
Είσοδος:
\begin{center}{\en\tt feuille(sin(x+2))}\end{center}
ή :
\begin{center}{\en\tt op(sin(x+2))}\end{center}
Έξοδος :
\begin{center}{\en\tt x+2}\end{center}  
Είσοδος :
\begin{center}{\en\tt feuille(x+2*y)}\end{center}
ή :
\begin{center}{\en\tt op(x+2*y)}\end{center}
Έξοδος :
\begin{center}{\en\tt (x,2*y) }\end{center} 
{\bf Σχόλιο}\\
Ας υποθέσουμε πως μια συνάρτηση ορίζεται από ένα πρόγραμμα, για παράδειγμα ας ορίσουμε τη συνάρτηση {\en\tt pgcd} :
\begin{center}{\en\tt pgcd(a,b):=\{local r; while (b!=0) \{r:=irem(a,b);a:=b;b:=r;\} return a;\}}\end{center}
Μετά εισάγετε :
\begin{center}{\en\tt sommet(pgcd)}\end{center}
Έξοδος :
\begin{center} '{\en\tt program}'\end{center}  
Μετά εισάγετε :
\begin{center}{\en\tt feuille(pgcd)[0]}\end{center} 
Έξοδος :
\begin{center}{\en\tt (a,b)}\end{center}
Μετά εισάγετε :
\begin{center}{\en\tt feuille(pgcd)[1]}\end{center} 
Έξοδος :
\begin{center}{\tt (0,0) ή (15,25)
εάν η τελευταία είσοδος ήταν {\en pgcd}(15,25)}\end{center}
Μετά εισάγετε :
\begin{center}{\en\tt feuille(pgcd)[2]}\end{center} 
Έξοδος :
\begin{center}{\tt Το σώμα του προγράμματος : {\en\{local r;....return(a);\}}}\end{center}

\section{Συναρτήσεις}
\subsection{Συναρτήσεις εξαρτώμενες από τα συμφραζόμενα.}
\subsubsection{Οι τελεστές {\tt +} και {\tt -}}\index{+}\index{'+'}\index{-}\index{'-'}
 \noindent{\en\tt +} (αντίστοιχα {\en\tt -}) είναι μια ενθηματική συνάρτηση και '{\en\tt +}' (αντίστοιχα '{\en\tt -}') είναι μια προθεματική συνάρτηση. Το αποτέλεσμα εξαρτάται από τη φύση των ορισμάτων του .\\
Παραδείγματα με {\tt +} (όλα τα παραδείγματα εκτός από το τελευταίο δουλεύουν επίσης και με {\tt -} αντί για {\tt +}) :
\begin{itemize}
\item είσοδος (1,2)+(3,4) ή (1,2,3)+4 ή 1+2+3+4 ή '+'(1,2,3,4), έξοδος 10,
\item είσοδος  1+{\en\tt i}+2+3*{\en\tt i} ή '+'(1,{\en\tt i},2,3*{\en\tt i}), έξοδος 3+4*{\en\tt i},
\item είσοδος  [1,2,3]+[4,1] ή [1,2,3]+[4,1,0] ή '+'([1,2,3],[4,1]), έξοδος [5,3,3],
\item  είσοδος [1,2]+[3,4] ή '+'([1,2],[3,4]), έξοδος [4,6], 
\item  είσοδος [[1,2],[3,4]]+ [[1,2],[3,4]], έξοδος [[2,4],[6,8]],
\item είσοδος  [1,2,3]+4 ή '+'([1,2,3],4), έξοδος { \en\tt poly}[1,2,7],
\item είσοδος  [1,2,3]+(4,1) ή '+'([1,2,3],4,1), έξοδος { \en\tt poly}[1,2,8],
\item  είσοδος {\tt \textlatin{ "Hel"+"lo"}}  ή '+'{\tt \textlatin{("Hel","lo")}}, έξοδος {\tt \textlatin {"Hello"}}.
\end{itemize}

\subsubsection{Ο τελεστής {\tt *}}\index{*}\index{'*'}
\noindent{\tt *} είναι μια ενθηματική συνάρτηση και  '{\tt *}' είναι μια προθεματική συνάρτηση. Το αποτέλεσμα εξαρτάται από τη φύση των ορισμάτων του .\\
Παραδείγματα με {\tt *} :
\begin{itemize}
\item είσοδος (1,2){\tt *}(3,4) ή (1,2,3){\tt *}4=1{\tt *}2{\tt *}3{\tt *}4  {\en\tt or} '{\tt *}'(1,2,3,4), έξοδος 24,
\item  είσοδος 1{\tt *}{\en\tt i}{\tt *}2{\tt *}3{\tt *}{\en\tt i} ή '{\tt *}'(1,{\en\tt i},2,3{\tt *}{\en\tt i}), έξοδος -6,
\item  είσοδος [10,2,3]{\tt *}[4,1] ή [10,2,3]{\tt *}[4,1,0] ή '+'([10,2,3],[4,1]), έξοδος 42 (εσωτερικό γινόμενο),
\item  είσοδος [1,2]{\tt *}[3,4]='{\tt *}'([1,2],[3,4]), έξοδος 11 (εσωτερικό γινόμενο),
\item  είσοδος [[1,2],[3,4]]{\tt *}[[1,2],[3,4]], έξοδος [[7,10],[15,22]],
\item  είσοδος [1,2,3]{\tt *}4 ή '{\tt *}'([1,2,3],4), έξοδος [4,8,12],
\item  είσοδος [1,2,3]*(4,2) ή '{\tt *}'([1,2,3],4,2) ή [1,2,3]{\tt *}8, έξοδος [8,16,24],
\item  είσοδος (1,2)+{\en\tt i}{\tt *}(2,3) {\en\tt or} 1+2+{\en\tt i}{\tt *}2{\tt *}3, έξοδος 3+6{\tt *}{\en\tt i}.
\end{itemize}

\subsubsection{Ο τελεστής {\tt /}}\index{/}\index{'/'}
\noindent{\tt /} είναι μια ενθηματική συνάρτηση και '{\tt /}' είναι μια προθεματική συνάρτηση. Το αποτέλεσμα εξαρτάται από τη φύση των ορισμάτων του .\\
Παραδείγματα με {\tt /} :
\begin{itemize}
\item είσοδος [10,2,3]/[4,1], έξοδος {\en\tt invalid dim} (= μη έγκυρες διαστάσεις),
\item είσοδος [1,2]/[3,4] ή '{\tt /}'([1,2],[3,4]), έξοδος [1/3,1/2], 
\item είσοδος 1/[[1,2],[3,4]] ή '{\tt /}'(1,[[1,2],[3,4]], έξοδος [[-2,1],[3/2,(-1)/2]],
\item είσοδος [[1,2],[3,4]]{\tt *}1/ [[1,2],[3,4]], έξοδος [[1,0],[0,1]],
\item είσοδος [[1,2],[3,4]]/ [[1,2],[3,4]], έξοδος [[1,1],[1,1]]{\tt \textlatin { (division term by term}=διαίρεση κατά όρους)}.
%\item είσοδος [1,2,3]*4 {\en\tt or} '*'([1,2,3],4), έξοδος [4,8,12],
%\item είσοδος [1,2,3]/(4,2) {\en\tt or} '*'([1,2,3],4,2), έξοδος [1,2,3]*8=[8,16,24].
%\item (1,2)+{\en\tt i}/(2,3)=1+2+{\en\tt i}*2*3=3+6*{\en\tt i}
\end{itemize}

\subsection{Συνήθεις συναρτήσεις}
\begin{itemize}
\item {\en\tt max}\index{max|textbf}  παίρνει ως όρισμα δύο πραγματικούς αριθμούς και επιστρέφει τον μεγαλύτερο,
\item
{\en\tt min}\index{min|textbf}  παίρνει ως όρισμα δύο πραγματικούς αριθμούς και επιστρέφει τον μικρότερο,
\item
{\en\tt abs}\index{abs} παίρνει ως όρισμα ένα μιγαδικό αριθμό και επιστρέφει το μέτρο της μιγαδικής παραμέτρου (την απόλυτη τιμή εάν ο μιγαδικός είναι πραγματικός),
\item
{\en\tt sign}\index{sign|textbf}  παίρνει ως όρισμα ένα πραγματικό αριθμό και επιστρέφει το πρόσημο (+1 εάν είναι θετικός, 0 εάν είναι 0, και -1 εάν είναι αρνητικός),
\item
{\en\tt floor}\index{floor|textbf} (ή {\en\tt iPart}\index{iPart|textbf}) παίρνει ως όρισμα ένα πραγματικό αριθμό $r$, και επιστρέφει τον μεγαλύτερο ακέραιο $\leq r$,
\item
{\en\tt round}\index{round|textbf} παίρνει ως όρισμα ένα πραγματικό αριθμό και επιστρέφει τον κοντινότερο ακέραιό του,
\item
{\en\tt ceil} ή {\en\tt ceiling}\index{ceil|textbf}\index{ceiling|textbf}  παίρνει ως όρισμα ένα πραγματικό αριθμό $r$ και επιστρέφει τον μικρότερο ακέραιο $\geq r$
\item
{\en\tt frac}\index{frac|textbf} (or {\en\tt fPart}\index{fPart|textbf})  παίρνει ως όρισμα ένα πραγματικό αριθμό και επιστρέφει το κλασματικό του μέρος,
\item
{\en\tt trunc}\index{trunc|textbf} παίρνει ως όρισμα ένα πργματικό αριθμό και επιστρέφει τον ακέραιο ίσο με τον πραγματικό χωρίς το κλασματικό του μέρος,
\item
{\en\tt id}\index{id|textbf} είναι η ταυτοτική συνάρτηση,
\item
{\en\tt sq}\index{sq|textbf} είναι η τετραγωνική συνάρτηση,
\item
{\en\tt sqrt}\index{sqrt|textbf} είναι η συνάρτηση τετραγωνικής ρίζας,
\item
{\en\tt exp}\index{exp|textbf} είναι η εκθετική συνάρτηση,
\item
{\en\tt log}\index{log|textbf} ή {\en\tt ln}\index{ln|textbf} είναι η συνάρτηση φυσικού λογάριθμου,
\item 
{\en\tt log10}\index{log10|textbf} είναι η συνάρτηση λογάριθμου με βάση 10,
\item
{\en\tt logb}\index{logb|textbf} είναι η συνάρτηση λογάριθμου όπoυ το δεύτερο όρισμα είναι η βάση του λογάριθμου: 
{\en\tt logb(7,10)=log10(7)=log(7)/log(10)},
\item
{\en\tt sin}\index{sin|textbf}, {\en\tt cos}\index{cosh|textbf}, {\en\tt tan}\index{tan|textbf} είναι η συνάρτηση ημιτόνου, συνημιτόνου, και εφαπτομένης αντίστοιχα,
\item {\en\tt cot, sec, csc} είναι η συνάρτηση συνεφαπτομένης, τέμνουσας και συντέμνουσας αντίστοιχα.
\item {\en\tt asin} ( ή {\en\tt arcsin})\index{asin|textbf}\index{arcsin|textbf},  {\en\tt acos} ( ή {\en\tt arccos})\index{acos|textbf}\index{arccos|textbf}, {\en\tt atan} (ή {\en\tt arctan})\index{atan|textbf}\index{arctan|textbf}, {\en\tt acot, asec, acsc} είναι οι αντίστροφες τριγωνομετρικές συναρτήσεις (δείτε το τμήμα \ref{sec:trigo} για περισσότερες πληροφορίες σχετικά με τις τριγωνομετρικές συναρτήσεις.)
\item{\en\tt sinh}\index{sinh|textbf}, {\en\tt cosh}\index{cosh|textbf}, {\en\tt tanh}\index{tanh|textbf} είναι η συνάρτηση υπερβολικού ημιτόνου,  συνημιτόνου, και εφαπτομένης αντίστοιχα,
\item{\en\tt asinh} ή {\en\tt arcsinh}\index{asinh|textbf}\index{arcsinh|textbf} (αντίστοιχα {\en\tt acosh} ή {\en\tt arccosh}\index{acosh|textbf}\index{arccosh|textbf}, {\en\tt atanh} ή {\en\tt arctanh}\index{atanh|textbf}\index{arctanh|textbf}) είναι η αντίστροφη συνάρτηση του {\en\tt sinh} (αντίστοιχα {\en\tt cosh}, {\en\tt tanh})
\end{itemize}

\subsection{Ορισμός αλγεβρικών συναρτήσεων}
\subsubsection{Ορισμός συνάρτησης από το $\mathbb{R}^p$ στο $\mathbb{R}$}
\noindent  Για $p=1$, π.χ. για $f\ :\ (x)\rightarrow x*\sin(x)$, εισάγετε :
\begin{center}{\en\tt f(x):=x*sin(x)}\end{center}
ή :
\begin{center}{\en\tt f:=x->x*sin(x)}\end{center}
Έξοδος :
\begin{center}{\en\tt  (x)->x*sin(x)}\end{center}
Εάν $p>1$, π.χ. για $f\ :\ (x,y)\rightarrow x*\sin(y)$, εισάγετε :
\begin{center}{\en\tt f(x,y):=x*sin(y)}\end{center}
ή :
\begin{center}{\en\tt f:=(x,y)->x*sin(y)}\end{center}
Έξοδος :
\begin{center}{\en\tt  (x,y)->x*sin(y)}\end{center}
{\bf Προσοχή!!!} η παράσταση μετά το {\en\tt  -> } δεν αποτιμάται. Θα πρέπει να χρησιμοποιήσετε {\en\tt unapply} αν αναμένετε το δεύτερο μέλος να αποτιμάται πριν τον ορισμό της συνάρτησης.

\subsubsection{Ορισμός συνάρτησης από το  $\mathbb{R}^p$ σε $\mathbb{R}^q$}
Για παράδειγμα:
\begin{itemize}
\item  Για να ορίσουμε την συνάρτηση $h\ :\ (x,y)\rightarrow (x*\cos(y),x*\sin(y))$.\\
 Είσοδος :
\begin{center}{\en\tt h(x,y):=(x*cos(y),x*sin(y))}\end{center}
Έξοδος :
\begin{center}{\en\tt " (x,y)->\{ \\ x*cos(y),x*sin(y);\\  \}"}\end{center}
\item  Για να ορίσουμε τη συνάρτηση $h\ :\ (x,y)\rightarrow [x*\cos(y),x*\sin(y)]$.\\
 Είσοδος :
\begin{center}{\en\tt h(x,y):=[x*cos(y),x*sin(y)];}\end{center}
ή :
\begin{center}{\en\tt h:=(x,y)->[x*cos(y),x*sin(y)];}\end{center}
ή :
\begin{center}{\en\tt h(x,y):=\{[x*cos(y),x*sin(y)]\};}\end{center}
ή : 
\begin{center}{\en\tt h:=(x,y)->return[x*cos(y),x*sin(y)];}\end{center}
ή : 
\begin{center}{\en\tt h(x,y):=\{return [x*cos(y),x*sin(y)];\}}\end{center}
Έξοδος :
\begin{center}{\en\tt   (x,y)->\{return([x*cos(y),x*sin(y)]);\}}\end{center}
\end{itemize}
{\bf Προσοχή !!!} η παράσταση μετά το {\en\tt  -> } δεν αποτιμάται.

\subsubsection{Ορισμός οικογενειών μιας συνάρτησης απο το $\mathbb{R}^{p-1}$ στο $\mathbb{R}^q$ χρησιμποποιώντας μια συνάρτηση από το $\mathbb{R}^p$ στο $\mathbb{R}^q$} Έστω ότι η συνάρτηση $f: (x,y) \rightarrow  f(x,y)$ έχει ορισθεί, και θέλουμε να ορίσουμε μια οικογένεια συναρτήσεων $g(t)$ τέτοιες ώστε $g(t)(y):=f(t,y)$ (δηλαδή η $t$ θεωρείται ως παράμετρος). Επειδή η παράσταση μετά το {\tt -> } (ή {\tt :=}) δεν αποτιμάται, δεν θα πρέπει να ορίσουμε τη $g(t)$ μέσω {\en\tt g(t):=y->f(t,y)}, αλλά θα πρέπει να χρησιμοποιήσουμε την εντολή {\en\tt unapply}.

Για παράδειγμα, υποθέτοντας ότι η $f:(x,y)\rightarrow x\sin(y)$ και $g(t): y\rightarrow f(t,y)$, είσοδος :
\begin{center}{\en\tt f(x,y):=x*sin(y);g(t):=unapply(f(t,y),y)}\end{center}
Έξοδος :
\begin{center}{\en\tt ((x,y)->x*sin(y), (t)->unapply(f(t,y),y))}\end{center}
Είσοδος 
\begin{center}{\en\tt g(2)}\end{center}
Έξοδος :
\begin{center}{\en\tt   y->2$\cdot$ sin(y)}\end{center}
Είσοδος 
\begin{center}{\en\tt g(2)(1)}\end{center}
Έξοδος :
\begin{center}{\en\tt   2$\cdot$ sin(1)}\end{center}
Επόμενο παράδειγμα: έστω ότι η συνάρτηση $h: (x,y) \rightarrow  [x*\cos(y),x*\sin(y)]$ έχει ορισθεί, και θέλουμε να ορίσουμε την οικογένεια συναρτήσεων $k(t)$ έχοντας το $t$ σαν παράμετρο και έτσι ώστε $k(t)(y):=h(t,y)$.
Για να ορίσουμε την συνάρτηση $h(x,y)$, εισάγουμε :
\begin{center}{\en\tt h(x,y):=(x*cos(y),x*sin(y))}\end{center}
Για να ορίσουμε κατάλληλα την συνάρτηση  $k(t)$, εισάγουμε :
\begin{center}{\en\tt k(t):=unapply(h(x,t),x)}\end{center}
Έξοδος :
\begin{center}{\en\tt (t)->unapply(h(x,t),x)}\end{center}
Είσοδος
\begin{center}{\en\tt k(2)}\end{center}
Έξοδος :
\begin{center}{\en\tt (x)->(x*cos(2),x*sin(2))}\end{center}
Είσοδος 
\begin{center}{\en\tt k(2)(1)}\end{center}
Έξοδος :
\begin{center}{\en\tt   (2*cos(1),2*sin(1))}\end{center}

\subsection{Σύνθεση δύο συναρτήσεων: \textlatin{\tt @}}\index{\@|textbf}
Με το {\en\tt Xcas}, η σύνθεση των συναρτήσεων γίνεται με τον ενθηματικό τελεστή {\en\tt @}.\\
Είσοδος :
\begin{center}{\en\tt (sq@sin+id)(x)}\end{center}
Έξοδος :
\begin{center}{\en\tt (sin(x))\verb|^|2+x}\end{center}  
Είσοδος :
\begin{center}{\en\tt (sin@sin)(pi/2)}\end{center}
Έξοδος :
\begin{center}{\en\tt sin(1)}\end{center}  

\subsection{Επαναλαμβανόμενη σύνθεση συνάρτησης: \textlatin{\tt @@}}\index{\@\@|textbf}
Με το {\en\tt Xcas}, η επαναλαμβανόμενη σύνθεση μιας συνάρτησης με τον εαυτό της $n \in {\mathbb N}$ φορές γίνεται με τον ενθηματικό τελεστή {\en\tt @@}.\\
Είσοδος :
\begin{center}{\en\tt (sin@@3)(x)}\end{center}
Έξοδος :
\begin{center}{\en\tt sin(sin(sin(x)))}\end{center}  
Είσοδος :
\begin{center}{\en\tt (sin@@2)(pi/2)}\end{center}
Έξοδος :
\begin{center}{\en\tt sin(1)}\end{center} 

\subsection{Ορισμός μιας συνάρτησης από το \textlatin{\tt "}ιστορικό\textlatin{\tt "}  :\\ {\tt \textlatin {as\_function\_of}}}\index{as\_function\_of}
\noindent Εάν με μια εντολή ορίσαμε την μεταβλητή {\en\tt a} και εάν σε μια επόμενη εντολή ορίσαμε την μεταβλητή {\en\tt b} (που υποθέτουμε ότι εξαρτάται από την {\en\tt a}), τότε η εντολή {\en\tt c:=as\_function\_of(b,a)} θα ορίσει μια συνάρτηση {\en\tt c} τέτοια ώστε  {\en\tt c(a)=b}.\\
Είσοδος :
\begin{center}{\en\tt  a:=sin(x)}\end{center}
Έξοδος :
\begin{center}{\en\tt  sin(x)}\end{center}
Είσοδος :
\begin{center}{\en\tt  b:=sqrt(1+a\verb|^|2)}\end{center}
Έξοδος :
\begin{center}{\en\tt  sqrt(1+sin(x)\verb|^|2)}\end{center}
Είσοδος :
\begin{center}{\en\tt  c:=as\_function\_of(b,a)}\end{center}
Έξοδος :
\begin{flushleft}{\en\tt (a)-> \\
\{ local NULL;\\ 
  return(sqrt(1+a\verb|^|2));\\  
\}}\end{flushleft}
Είσοδος :
\begin{center}{\en\tt  c(x)}\end{center}
Έξοδος :
\begin{center}{\en\tt sqrt(1+x\verb|^|2)}\end{center}
Είσοδος :
\begin{center}{\en\tt  a:=2}\end{center}
Έξοδος :
\begin{center}{\en\tt  2}\end{center}
Είσοδος :
\begin{center}{\en\tt  b:=1+a\verb|^|2}\end{center}
Έξοδος :
\begin{center}{\en\tt  5}\end{center}
Είσοδος :
\begin{center}{\en\tt  c:=as\_function\_of(b,a)}\end{center}
Έξοδος :
\begin{flushleft}{\en\tt (a)-> \\
\{ local NULL;\\ 
  return(1+a\verb|^|2);\\  
\}}\end{flushleft}
Είσοδος :
\begin{center}{\en\tt  c(x)}\end{center}
Έξοδος :
\begin{center}{\en\tt 1+x\verb|^|2}\end{center}

{\bf Προσοχή !!}\\
Εάν η μεταβλητή {\en\tt b} έχει ανατεθεί πολλές φορές, τότε θα χρησιμοποιηθεί η πρώτη εντολή της {\en\tt b} που ακολουθεί την τελευταία εντολή της {\en\tt a}. 
Επιπλέον, η διάταξη που χρησιμοποιείται είναι η σειρά με την οποία εκτελούνται οι εντολές, και η οποία μπορεί να μην φαίνεται στην διεπαφή του {\en Xcas} αν διάφορες εντολές έχουν εκτελεσθεί ξανά.  \\
Για παράδειγμα Είσοδος :\\
{\en\tt a:=2} και μετά\\
{\en\tt b:=2*a+1} και μετά\\
{\en\tt b:=3*a+2} και μετά\\
{\en\tt  c:=as\_function\_of(b,a)}\\
Έξοδος :
\begin{center}{\en\tt (a)-> \{local NULL; return(2*a+1);\}}\end{center}
δηλαδή {\en\tt c(x)} ισούται με {\en\tt 2*x+1}. \\
Αλλά, Είσοδος :\\
{\en\tt a:=2} και μετά\\
{\en\tt b:=2*a+1} και μετά\\
{\en\tt a:=2} και μετά\\
{\en\tt b:=3*a+2} και μετά\\
{\en\tt  c:=as\_function\_of(b,a)}\\
Έξοδος :
\begin{center}{\en\tt (a)-> \{local NULL; return(3*a+2);\}}\end{center}
δηλαδή {\en\tt c(x)} ισούται με {\en\tt 3*x+2}. \\


\section{Παραγώγιση και εφαρμογές.}
\subsection{Συνάρτηση παραγώγου  : {\tt\textlatin{ function\_diff}}}\index{function\_diff}
{\tt\textlatin{ function\_diff}} Παίρνει μια συνάρτηση ώς όρισμα.\\
{\tt\textlatin{ function\_diff}} επιστρέφει την συνάρτηση παραγώγου αυτής της συνάρτησης.\\
Είσοδος :
\begin{center}{\en\tt function\_diff(sin)}\end{center}
Έξοδος :
\begin{center}{\en\tt ({\gr `} x{\gr `})-> cos({\gr `} x{\gr `})}\end{center}  
Είσοδος :
\begin{center}{\en\tt function\_diff(sin)(x)}\end{center}
Έξοδος :
\begin{center}{\en\tt cos(x)}\end{center}  
Είσοδος :
\begin{center}{\en\tt f(x):=x\verb|^|2+x*cos(x)}\end{center} 
\begin{center}{\en\tt function\_diff(f)}\end{center}
Έξοδος :
\begin{center}{\en\tt  ({\gr `} x{\gr `})->2*{\gr `} x{\gr `}+cos({\gr `} x{\gr `})+{\gr `} x{\gr `}*(-(sin({\gr `} x{\gr `})))}\end{center}  
Είσοδος :
\begin{center}{\en\tt function\_diff(f)(x)}\end{center}
Έξοδος :
\begin{center}{\en\tt  cos(x)+x*(-(sin(x)))+2*x}\end{center}  
Για να ορίσετε τη συνάρτηση $g$ ως $f'$, εισάγετε :\\
\begin{center}{\en\tt g:=function\_diff(f)}\end{center}
Η εντολή {\tt\textlatin{ function\_diff}} έχει το ίδιο αποτέλεσμα σαν να χρησιμοποιούμε την παράγωγο σε συνδιασμό με  την {\en\tt unapply} :
\begin{center}{\en\tt g:=unapply(diff(f(x),x),x)}\end{center}
\begin{center}{\en\tt g(x)}\end{center} 
Έξοδος :
\begin{center}{\en\tt  cos(x)+x*(-(sin(x)))+2*x}\end{center}  
{\bf Προσοχή !!!}\\
Στον τρόπο λειτουργίας {\en\tt Maple}, για συμβατότητα, η εντολή
{\en\tt D} μπορεί να χρησιμοποιηθεί στη θέση της {\tt\textlatin{ function\_diff}}.
Γι' αυτόν το λόγο, στον τρόπο λειτουργίας {\en\tt Maple} είναι αδύνατο να ορίσουμε μια μεταβλητή που ονομάζεται 
{\en\tt D}   (και επομένως δεν μπορούμε να ονομάσουμε ένα
γεωμετρικό αντικέιμενο {\en\tt D}).

\subsection{Μήκος ενός τόξου : {\tt\textlatin {arcLen}}}\index{arcLen}
\noindent {\en\tt arcLen} παίρνει 4 ορίσματα : μια παράσταση $ex$ (αντίστοιχα, μια λίστα
δύο παραστάσεων $[ex1,ex2]$), το όνομα μιας παραμέτρου και 2 τιμές $a$
και $b$ αυτής της παραμέτρου.\\
{\en\tt arcLen} υπολογίζει το μήκος της καμπύλης που ορίζεται από την εξίσωση 
$y=f(x)=ex$ (αντίστοιχα από τις $x=ex1,y=ex2$) όταν οι τιμές της παραμέτρου κινούνται από $a$ 
μέχρι $b$, χρησιμοποιώντας τον τύπο
{\en\tt arcLen(f(x),x,a,b)=}\\  
{\en\tt integrate(sqrt(diff(f(x),x)\verb|^|2+1),x,a,b)}\\
ή \\
{\en\tt integrate(sqrt(diff(x(t),t)\verb|^|2+diff(y(t),t)\verb|^|2),t,a,b)}.\\

{\bf Παραδείγματα}
\begin{itemize}
\item Υπολογισμός του μήκους της παραβολής $y=x^2$ από $x=0$ μέχρι $x=1$.\\
Είσοδος :
\begin{center}{\en\tt arcLen(x\verb|^|2,x,0,1)}\end{center}
ή
\begin{center}{\en\tt arcLen([t,t\verb|^|2],t,0,1)}\end{center}
Έξοδος :
\begin{center}{\en\tt -1/4*log(sqrt(5)-2)-(-(sqrt(5)))/2}\end{center} 
\item Υπολογισμός του μήκους της καμπύλης $y=\cosh(x)$ από $x=0$ έως 
$x=\ln(2)$.\\
Είσοδος :
\begin{center}{\en\tt arcLen(cosh(x),x,0,log(2))}\end{center}
Έξοδος :
\begin{center}{\en\tt 3/4}\end{center}
\item Υπολογισμός του μήκους του κύκλου $x=\cos(t),y=\sin(t)$ από $t=0$ έως 
$t=2*\pi$.\\
Είσοδος :
\begin{center}{\en\tt arcLen([cos(t),sin(t)],t,0,2*pi)}\end{center}
Έξοδος :
\begin{center}{\en\tt 2*pi}\end{center}
\end{itemize}

\subsection{Μέγιστο και ελάχιστο μιας παράστασης: {\tt\textlatin{ fMax fMin}}}\index{fMax}\index{fMin} 
\noindent{\en\tt fMax} και {\en\tt fMin} παίρνουν ένα ή δύο ορίσματα : μια παράσταση 
μιας μεταβλητής και το όνομα αυτής της μεταβλητής (από προεπιλογή {\en\tt x}).\\
{\en\tt fMax} επιστρέφει την τετμημένη 
ενός μεγίστου της παράστασης.\\
{\en\tt fMin} επιστρέφει την τετμημένη 
ενός ελαχίστου της παράστασης.\\
Είσοδος :
\begin{center}{\en\tt fMax(sin(x),x)}\end{center}
ή :
\begin{center}{\en\tt fMax(sin(x))}\end{center}
ή :
\begin{center}{\en\tt fMax(sin(y),y)}\end{center}
Έξοδος :
\begin{center}{\en\tt pi/2}\end{center} 
Είσοδος :
\begin{center}{\en\tt fMin(sin(x),x)}\end{center}
ή :
\begin{center}{\en\tt fMin(sin(x))}\end{center}
ή :
\begin{center}{\en\tt fMin(sin(y),y)}\end{center}
Έξοδος :
\begin{center}{\en\tt -pi/2}\end{center} 
Είσοδος :
\begin{center}{\en\tt fMin(sin(x)\verb|^|2,x)}\end{center}
Έξοδος :
\begin{center}{\en\tt 0}\end{center}

\subsection{Πίνακας τιμών και γράφημα : {\tt\textlatin{ tablefunc}} {\tt\textlatin{ plotfunc}}}\index{tablefunc|textbf}\index{plotfunc}   
{\en\tt tablefunc} είναι μια ειδική εντολή που θα πρέπει να εκτελείται μέσα 
σε υπολογιστικό φύλλο. Επιστρέφει την αποτίμηση μιας παράστασης $ex$ 
του $x$ για \textlatin{ $x=x_0,\ x_0+h,....$~}:
\begin{center}
{\en\tt tablefunc(ex,x,x\_0,h)} ή \en{\tt tablefunc(ex,x)}
\end{center}
Στην τελευταία περίπτωση, η προεπιλεγμένη τιμή για το ${\tt x_0}$
είναι η ελάχιστη προεπιλεγμένη τιμή του $x$ από τη γραφική διαμόρφωση
και η προεπιλεγμένη τιμή για το βήμα $h$ είναι 0.1 επί την διαφορά
ανάμεσα στις προεπιλεγμένες τιμές του μέγιστου και ελάχιστου του $x$ (από τη
γραφική διαμόρφωση).\\
Παράδειγμα: Aνοίξτε ένα υπολογιστικό φύλλο αν κανένα δεν ήταν ανοιχτό πριν.
Μετά επιλέξτε ένα κελλι του υπολογιστικού φύλλου (για παράδειγμα το {\en\tt C0}) και για να πάρετε τον πίνακα
του {\en\tt "sinus"}, εισάγετε στη γραμμή εντολών του υπολογιστικού φύλλου : 
\begin{center}{\en\tt tablefunc(sin(x),x)}\end{center}
Αυτό θα γεμίσει 2 στήλες με τις αριθμητικές τιμές του {\en\tt x} και 
{\en\tt sin(x)} :
\begin{itemize}
\item στην πρώτη στήλη εμφανίζονται: η μεταβλητή {\en\tt x}, 
η τιμή του βήματος {\en\tt h}
(1.0),  η ελάχιστη τιμή του $x$ (-5.0), μετά ένας τύπος, για παράδειγμα 
{\en\tt=C2+C1}, και οι υπόλοιπες γραμμές της στήλης συμπληρώνονται κάνοντας επικόλληση αυτόν τον τύπο.
\item στην επόμενη στήλη εμφανίζονται : η συνάρτηση {\en\tt sin(x)}, η λέξη
\tt\textlatin{"Tablefunc"}, ένας τύπος, για παράδειγμα {\en\tt =evalf(subst(D\$0,C\$0,C2))}, και οι υπόλοιπες γραμμές
της στήλης συμπληρώνονται κάνοντας επικόλληση αυτόν τον τύπο.
\end{itemize}
Επομένως, οι τιμές της {\en\tt sin(x)} είναι στις ίδιες γραμμές με τις τιμές της 
 {\en\tt x}. Σημειωστε ότι το βήμα, η αρχική τιμή και η παράσταση μπορούν να αλλαχτούν εύκολα τροποποιώντας το αντίστοιχο κελί.

Η γραφική αναπαράσταση μπορεί να αναπαρασταθεί γραφικά με την εντολή {\en\tt plotfunc}  

\subsection{Παράγωγος και μερική παράγωγος}\index{diff|textbf}\index{derive|textbf}\index{deriver|textbf}
\noindent{\en\tt diff} ή {\en\tt derive} μπορεί να έχουν ένα ή δύο ορίσματα για τον υπολογισμό της παραγώγου πρώτης τάξης (ή της μερικής παραγώγου πρώτης τάξης) μιας παράστασης ή μιας λίστας παραστάσεων, ή περισσότερα ορίσματα για τον υπολογισμό της $n$-οστής μερικής παραγώγου μιας παράστασης ή μιας λίστας παραστάσεων.

\subsubsection{Παράγωγος και μερική παράγωγος πρώτης τάξης : {\tt\textlatin{ diff derive deriver}}}
{\en\tt diff} (ή {\en\tt derive}) παίρνει δύο ορίσματα : μια παράσταση και μια μεταβλητή (αντίστοιχα, ένα διάνυσμα των ονομάτων των μεταβλητών) (δείτε συνάρτησεις πολλών μεταβλητών στο {\ref{sec:plusvar}}). Εάν δίνεται μόνο ένα όρισμα, η παράγωγος υπολογίζεται ως προς $x$\\
{\en\tt diff} (ή {\en\tt derive}) επιστρέφει την παράγωγο (αντίστοιχα ένα διάνυσμα παραγώγων ) της παράστασης ως προς τη μεταβλητή (αντίστοιχα ως προς κάθε μεταβλητή) που δίνεται ως δεύτερο όρισμα.\\
Παραδείγματα :
\begin{itemize}
\item Υπολογίστε :
$$\frac {\partial (x.y^2.z^3+x.y.z)}{\partial z}$$
Είσοδος :
\begin{center}{\en\tt  diff(x*y \verb|^|2*z\verb|^|3+x*y*z,z)}\end{center}
Έξοδος :
\begin{center}{\en\tt x*y\verb|^|2*3*z\verb|^|2+x*y}\end{center}
\item Υπολογίστε τις  3 μερικές παραγώγους πρώτης τάξης της παράστασης $x*y^2*z^3+x*y*z$.\\
Είσοδος :
\begin{center}{\en\tt  diff(x*y\verb|^|2*z\verb|^|3+x*y*z,[x,y,z])}\end{center}
Έξοδος :
\begin{center}{\en\tt [y\verb|^|2*z\verb|^|3+y*z, x*2*y*z\verb|^|3+x*z, x*y\verb|^|2*3*z\verb|^|2+x*y]}\end{center}
% \item Compute :
% $$\frac {\partial^3 (x.y^2.z^3+x.y.z)}{\partial y\partial^2 z}$$
% Είσοδος :
% \begin{center}{\en\tt  diff(x*y \verb|^|2*z\verb|^|3+x*y*z,y,z\$2)}\end{center}
% Έξοδος :
% \begin{center}{\en\tt x*2*y*3*2*z}\end{center}
\end{itemize}

\subsubsection{Παράγωγος και μερική παράγωγος $n$-οστής τάξης : {\en\tt diff derive deriver}}\index{\$}
\noindent {\en\tt derive} (ή {\en\tt diff}) μπορεί να πάρει παραπάνω από δύο ορίσματα : μια παράσταση και τα ονόματα των μεταβλητών παραγώγισης (κάθε μεταβλητή μπορεί να ακολουθείται από \$$n$ για να υποδεικνύει τον αριθμό $n$ των παραγωγίσεων ).\\
{\en\tt diff} επιστρέγει τη μερική παράγωγο της παράστασης ως προς τις μεταβλητές που δίνονται μετά το πρώτο όρισμα.

Ο συμβολισμός \$ είναι χρήσιμος εάν θέλετε να παραγωγίσετε $k$ φορές ως προς την ίδια μετβλητή. Αντί να εισάγετε $k$ φορές το ίδιο όνομα της μεταβλητής,  εισάγετε το όνομα μεταβλητής ακολουθούμενο από {\en\tt \$k}, για παράδειγμα {\en\tt x\$3} αντί του {\en\tt(x,x,x)}. 
Κάθε μεταβλητή μπορεί να ακολουθείται από ένα \$, για παράδειγμα 
{\tt\textlatin{ diff(exp(x*y),x\$3,y\$2,z)}} είναι το ίδιο με 
{\tt\textlatin{ diff(exp(x*y),x,x,x,y,y,z)}}\\
{\bf Παραδείγματα}
\begin{itemize}
\item Υπολογίστε :
$$\frac {\partial^2 (x.y^2.z^3+x.y.z)}{\partial x\partial z}$$
Είσοδος :
\begin{center}{\en\tt diff(x*y \verb|^|2*z\verb|^|3+x*y*z,x,z)}\end{center}
Έξοδος  :
\begin{center}{\en\tt y\verb|^|2*3*z\verb|^|2+y}\end{center}
\item Υπολογίστε :
$$\frac {\partial^3 (x.y^2.z^3+x.y.z)}{\partial x\partial^2 z}$$
Είσοδος :
\begin{center}{\en\tt  diff(x*y \verb|^|2*z\verb|^|3+x*y*z,x,z,z)}\end{center}
ή είσοδος :
\begin{center}{\en\tt  diff(x*y \verb|^|2*z\verb|^|3+x*y*z,x,z\$2)}\end{center}
Έξοδος  :
\begin{center}{\en\tt y\verb|^|2*3*2*z}\end{center}
\item Υπολογίστε την τρίτη παράγωγο της :
$$\frac{1}{x^2+2}$$
Είσοδος :
\begin{center}{\en\tt  normal(diff((1)/(x\verb|^|2+2),x,x,x))}\end{center}
ή :
\begin{center}{\en\tt  normal(diff((1)/(x\verb|^|2+2),x\$3))}\end{center}
Έξοδος  :
\begin{center}{\en\tt (-24*x\verb|^|3+48*x)/(x\verb|^|8+8*x\verb|^|6+24*x\verb|^|4+32*x\verb|^|2+16)}\end{center}
\end{itemize}
{\bf Σχόλια}
\begin{itemize}
\item 
Σημειώστε τη διαφορά μεταξύ {\en\tt diff(f,x,y)} και {\en\tt  diff(f,[x,y])} :\\
{\en\tt diff}$(f,x,y)$ επιστρέφει $\displaystyle \frac{\partial^2(f)}{\partial x\partial y}$ ενώ\\
{\en\tt diff}$(f,[x,y])$ επιστρέφει
$\displaystyle[\frac{\partial(f)}{\partial x},\frac{\partial
  (f)}{\partial y}]$ 
\item Μην ορίζετε ποτέ την συνάρτηση παραγώγου με {\en\tt
    f1(x):=diff(f(x),x)}.
Πράγματι, το {\en\tt x} θα σήμαινε δύο διαφορετικά πράγματα με τα οποία το\tt\textlatin{ Xcas} αδυνατεί να χειρισθεί : το όνομα της μεταβλητής για τον ορισμό της συνάρτησης $f_1$ και τη μεταβλητή παραγώγισης. Ο σωστός τρόπος για τον ορισμό μιας συνάρτησης παραγώγου είναι είτε με την εντολή {\en\tt
  function\_diff} είτε:
\begin{center}
{\en\tt f1:=unapply(diff(f(x),x),x)}
\end{center}
\end{itemize}

\section{Ολοκλήρωση}
\subsection{Αντιπαράγωγος και ορισμένο ολοκλήρωμα : {\tt\textlatin{integrate\\ int Int}}}\index{integrate}\index{Int}\index{int}
\noindent{\en\tt integrate} (ή {\en\tt int}) υπολογίζουν ένα αόριστο ή ένα ορισμένο ολοκλήρωμα. Μια διαφορά ανάμεσα στις δύο εντολές είναι ότι, εάν εισάγετε {\en\tt quest()}, αμέσως μετά την αποτίμηση του {\en\tt  integrate}, η απάντηση γράφεται με το σύμβολο  $\int$.

{\en\tt integrate}  (ή {\en\tt int} ή {\en\tt Int}) παίρνει ένα, δύο , ή τέσσερα ορίσματα.
\begin{itemize}
\item  με ένα ή δύο ορίσματα\\
μια παράσταση ή μια παράσταση και το όνομα μιας μεταβλητής (από προεπιλογή το {\en\tt x}),
{\en\tt integrate} ( ή {\en\tt int}) επιστρέφει το αόριστο ολοκλήρωμα της παράστασης  ως προς την μεταβλητή που δίνεται ως δεύτερο όρισμα.\\
Είσοδος :
\begin{center}{\en\tt integrate(x\verb|^|2)}\end{center}
Έξοδος  :
\begin{center}{\en\tt x\verb|^|3/3}\end{center}
Είσοδος :
\begin{center}{\en\tt integrate(t\verb|^|2,t)}\end{center}
Έξοδος  :
\begin{center}{\en\tt t\verb|^|3/3}\end{center}
\item με τέσσερα ορίσματα :\\
μια παράσταση, το όνομα μιας μεταβλητής και τα όρια του ορισμένου ολοκληρώματος,\\ 
{\en\tt integrate} ( ή {\en\tt int}) επιστρέφει την ακριβή τιμή του ορισμένου ολοκληρώματος εάν ο υπολογισμός ήταν επιτυχής ή ένα μη αποτιμημένο ολοκλήρωμα διαφορετικά.\\
Είσοδος :
\begin{center}{\en\tt integrate(x\verb|^|2,x,1,2)}\end{center}
Έξοδος  :
\begin{center}{\en\tt 7/3}\end{center}
Είσοδος :
\begin{center}{\en\tt integrate(1/(sin(x)+2),x,0,2*pi)}\end{center}
Έξοδος μετά από απλοποίηση (με την εντολή {\en\tt simplify} ) :
\begin{center}{\en\tt 2*pi*sqrt(3)/3}\end{center}
\end{itemize}

{\en\tt Int} είναι η αδρανής μορφή του {\en\tt integrate}, αποτρέπει την αποτίμηση για παράδειγμα για την αποφυγή ενός υπολογισμού που ίσως να μην είναι επιτυχής εάν θέλετε απλά μια αριθμητική αποτίμηση.\\
Είσοδος :
\begin{center}{\en\tt evalf(Int(exp(x\verb|^|2),x,0,1))}\end{center}
ή :
\begin{center}{\en\tt evalf(int(exp(x\verb|^|2),x,0,1))}\end{center}
Έξοδος  :
\begin{center}{\en\tt 1.46265174591}\end{center}

{\bf Άσκηση 1}\\
Έστω $$f(x)=\frac {x}{x^2-1}+\ln(\frac {x+1}{x-1})$$
Βρείτε το αόριστο ολοκλήρωμα  της $f$.\\
Είσοδος :
\begin{center}{\en\tt int(x/(x\verb|^|2-1)+ln((x+1)/(x-1)))}\end{center}
Έξοδος : 
\begin{center}{\en\tt x*log((x+1)/(x-1))+log(x\verb|^|2-1)+1/2*log(2*x\verb|^|2/2-1)}\end{center}
Ή πρώτα ορίσετε την συνάρτηση {\en\tt f}, εισάγοντας :
\begin{center}{\en\tt f(x):=x/(x\verb|^|2-1)+ln((x+1)/(x-1))}\end{center}
και μετά εισάγετε :
\begin{center}{\en\tt int(f(x))}\end{center}
Το  αποτέλεσμα φυσικά είναι το ίδιο.\\
{\bf Προσοχή}\\
Στο {\en\tt Xcas},το {\en\tt log} είναι ο φυσικός λογάριθμος (όπως {\en\tt ln}),
επειδή το {\en\tt log10} είναι ο λογάριθμος με βάση 10.

{\bf Άσκηση 2}\\
Υπολογίστε :
$$\int \frac {2}{x^6+2 \cdot x^4+x^2} \ dx $$
Είσοδος :
\begin{center}{\en\tt int(2/(x\verb|^|6+2*x\verb|^|4+x\verb|^|2))}\end{center}
Έξοδος :
\begin{center}{\en\tt 2*((3*x\verb|^|2+2)/(-(2*(x\verb|^|3+x)))+-3/2*atan(x))}\end{center}

{\bf Άσκηση 3}\\
Υπολογίστε :
$$\int \frac {1}{\sin(x)+\sin(2 \cdot x )} \ dx $$
Είσοδος :
\begin{center}{\en\tt integrate(1/(sin(x)+sin(2*x )))}\end{center}
Έξοδος :
\begin{center}{\en\tt (1/-3*log((tan(x/2))\verb|^|2-3)+1/12*log((tan(x/2))\verb|^|2))*2}\end{center} 

% \subsection{Primitive και ορισμένο ολοκλήρωμα : {\en\tt risch}}\index{risch}
% \noindent{\en\tt risch}  υπολογίζει ένα \tt\textlatin{primitive} ή ένα ορισμένο ολοκλήρωμα χρησιμοποιώντας τον
% αλγόριθμο \en\tt Risch .\\
% Είσοδος :
% \begin{center}{\en\tt risch(x\verb|^|2)}\end{center}
% Έξοδος  :
% \begin{center}{\en\tt x\verb|^|3/3}\end{center}
% Είσοδος :
% \begin{center}{\en\tt risch(x\verb|^|2,x,0,1)}\end{center}
% Έξοδος  :
% \begin{center}{\en\tt 1/3}\end{center}
% Είσοδος :
% \begin{center}{\en\tt risch(exp(-x\verb|^|2))}\end{center}
% Έξοδος  :
% \begin{center}{\en\tt $\int$ exp(x\verb|^|2) dx}\end{center}
% είναι να πούμε ότι $\exp(-x^2)$ δεν έχει πρώιμη παράσταση με
% συνήθεις συναρτήσεις.

\subsection{Διακριτή άθροιση: {\tt\textlatin{sum}}}\index{sum|textbf}
\noindent{\en\tt sum} παίρνει δύο ή τέσσερα ορίσματα :
\begin{itemize}
\item τέσσερα ορίσματα\\
μια παράσταση, το όνομα της μεταβλητής (για παράδειγμα {\en\tt n}), και τα όρια (για παράδειγμα {\en\tt a} και {\en\tt b}).\\
{\en\tt sum} επιστρέφει το διακριτό άθροισμα της παράστασης ως προς τη μεταβλητή από $a$ έως $b$.\\
Είσοδος :
\begin{center}{\en\tt sum(1,k,-2,n) }\end{center}
Έξοδος  :
\begin{center}{\en\tt n+1+2}\end{center}
Είσοδος :
\begin{center}{\en\tt normal(sum(2*k-1,k,1,n))}\end{center}
Έξοδος  :
\begin{center}{\en\tt n\verb|^|2}\end{center}
Είσοδος :
\begin{center}{\en\tt sum(1/(n\verb|^|2),n,1,10)}\end{center}
Έξοδος  :
\begin{center}{\en\tt 1968329/1270080}\end{center} 
Είσοδος :
\begin{center}{\en\tt sum(1/(n\verb|^|2),n,1,+(infinity)) }\end{center}
Έξοδος  :
\begin{center}{\en\tt pi\verb|^|2/6}\end{center}
Είσοδος :
\begin{center}{\en\tt sum(1/(n\verb|^|3-n),n,2,10) }\end{center}
Έξοδος  :
\begin{center}{\en\tt 27/110}\end{center} 
Είσοδος :
\begin{center}{\en\tt sum(1/(n\verb|^|3-n),n,1,+(infinity)) }\end{center}
Έξοδος  :
\begin{center}{\en\tt 1/4}\end{center}
Αυτό το αποτέλεσμα προέρχεται από τη διάσπαση ${\en\tt 1/(n\verb|^|3-n)}$.\\
Είσοδος :
\begin{center}{\en\tt partfrac(1/(n\verb|^|3-n)) }\end{center}
Έξοδος  :
\begin{center}{\en\tt 1/(2*(n+1))-1/n+1/(2*(n-1))}\end{center}
 Επομένως :\\
$\displaystyle \sum_{n=2}^N -\frac{1}{n}=-\sum_{n=1}^{N-1} \frac{1}{n+1}=-\frac{1}{2}-\sum_{n=2}^{N-2} \frac{1}{n+1}-\frac{1}{N}$\\
$\displaystyle \frac{1}{2}*\sum_{n=2}^N \frac{1}{n-1}=\frac{1}{2}*(\sum_{n=0}^{N-2} \frac{1}{n+1})=\frac{1}{2}*(1+\frac{1}{2}+\sum_{n=2}^{N-2}\frac{1}{n+1})$\\
$\displaystyle \frac{1}{2}*\sum_{n=2}^N \frac{1}{n+1}=\frac{1}{2}*(\sum_{n=2}^{N-2} \frac{1}{n+1}+\frac{1}{N}+\frac{1}{N+1})$\\
Έπειτα από απλοποίηση με $\sum_{n=2}^{N-2}$, απομένει :\\
 $\displaystyle -\frac{1}{2}+\frac{1}{2}*(1+\frac{1}{2})-\frac{1}{N}+\frac{1}{2}*(\frac{1}{N}+\frac{1}{N+1})=\frac{1}{4}-\frac{1}{2N(N+1)}$\\
Συνεπώς :
\begin{itemize}
\item για $N=10$ το άθροισμα ισούται με : $1/4-1/220=27/110$
\item για $N=+\infty$ το άθροισμα ισούται με : $1/4$ επειδή $\frac{1}{2N(N+1)}$ 
πλησιάζει στο 0 όταν το $N$ πλησιάζει το άπειρο.
\end{itemize}

\item δύο ορίσματα \\
μια παράσταση μιας μεταβλητής (για παράδειγμα $f$) και το όνομα αυτής της μεταβλητής
 (για παράδειγμα $x$).\\
{\en\tt sum} επιστρέφει την διακριτή αντιπαράγωγο της παράστασης, δηλαδή 
μια παράσταση $G$ τέτοια ώστε $G_{|x=n+1}-G_{|x=n}=f_{|x=n}$.\\ 
Είσοδος :
\begin{center}{\en\tt sum(1/(x*(x+1)),x)}\end{center}
Έξοδος  :
\begin{center}{\en\tt -1/x}\end{center}
\end{itemize}

\subsection{Άθροισμα \textlatin{ Riemann} : {\tt\textlatin{ sum\_riemann}}}\index{sum\_riemann}
\noindent{\tt\textlatin{ sum\_riemann}} παίρνει δύο ορίσματα : μια παράσταση που εξαρτάται από δύο μεταβλητές και τη λίστα από τα ονόματα των δύο μεταβλητών.\\ 
{\tt\textlatin{ sum\_riemann(expression(n,k),[n,k])}} επιστρέφει στην περιοχή του
 $ n=+\infty$ μία παράσταση ισοδύναμη του $\sum_{k=1}^n {\en expression}(n,k)$ (ή του
$ \sum_{k=0}^{n-1} {\en expression}(n,k)$ ή του $ \sum_{k=1}^{n-1} expression(n,k)$) 
όπου το άθροισμα θεωρείται σαν ένα άθροισμα {\en Riemann}  συσχετιζόμενο με μια συνεχή 
συνάρτηση ορισμένη στο [0,1] ή επιστρέφει  
{\tt\textlatin{"}}πιθανόν δεν είναι ένα άθροισμα{\tt\textlatin{ Riemann"}} όταν η προσπάθεια είναι ανεπιτυχής.\\
{\bf Άσκηση 1}\\
Υποθέστε ότι $\displaystyle S_n=\sum_{k=1}^n \frac{k^2}{n^3}$.\\
Υπολογίστε το $\displaystyle \lim_{n \rightarrow  +\infty} S_n$.\\
Είσοδος :
\begin{center}{\en\tt sum\_riemann(k\verb|^|2/n\verb|^|3,[n,k])}\end{center}
Έξοδος  :
\begin{center}{\en\tt 1/3}\end{center}
{\bf Άσκηση 2}\\
Υποθέστε ότι $\displaystyle S_n=\sum_{k=1}^n \frac{k^3}{n^4}$.\\
Υπολογίστε το $\displaystyle \lim_{n \rightarrow  +\infty} S_n$.\\
Είσοδος :
\begin{center}{\en\tt sum\_riemann(k\verb|^|3/n\verb|^|4,[n,k])}\end{center}
Έξοδος  :
\begin{center}{\en\tt 1/4}\end{center}
{\bf Exercise 3}\\
Υπολογίστε 
$\displaystyle \lim_{n \rightarrow  +\infty}(\frac{1}{n+1}+\frac{1}{n+2}+...+\frac{1}{n+n})$.\\
Είσοδος :
\begin{center}{\en\tt sum\_riemann(1/(n+k),[n,k])}\end{center}
Έξοδος :
\begin{center}{\en\tt log(2)}\end{center}
{\bf Άσκηση 4}\\
Υποθέστε ότι $\displaystyle S_n=\sum_{k=1}^n \frac{32n^3}{16n^4-k^4}$.\\
Υπολογίστε $\displaystyle \lim_{n \rightarrow  +\infty} S_n$.\\
Είσοδος :
\begin{center}{\en\tt sum\_riemann(32*n\verb|^|3/(16*n\verb|^|4-k\verb|^|4),[n,k])}\end{center}
Έξοδος :
\begin{center}{\en\tt 2*atan(1/2)+log(3)}\end{center}

\subsection{Ολοκλήρωση κατά μέλη : {\tt\textlatin {ibpdv}} και {\tt\textlatin {ibpu}}}
\subsubsection{\en\tt ibpdv}\index{ibpdv}
\noindent{\en\tt ibpdv} χρησιμοποιείται για την εύρεση του αορίστου ολοκληρώματος μιας παράστασης γραμμένης 
ως $u(x).v'(x)$.\\
{\en\tt ibpdv} παίρνει δύο ορίσματα :
\begin{itemize}
\item μια παράσταση 
 $u(x).v'(x)$ και $v(x)$ (ή μια λίστα δύο παραστάσεων 
$[F(x), u(x)*v'(x)]$ και $v(x)$),
\item ή μια παράσταση $g(x)$ και $0$ (ή μια λίστα δύο παραστάσεων 
$[F(x), g(x)]$ και $0$).
\end{itemize}
{\en\tt ibpdv} επιστρέφει :
\begin{itemize}
\item εάν $v(x) \neq 0$, την λίστα $[u(x).v(x),-v(x).u'(x)]$ (ή 
$[F(x)+u(x).v(x),-v(x).u'(x)]$),
\item εάν το δεύτερο όρισμα είναι 0, το αόριστο ολοκλήρωμα του πρώτου ορίσματος 
$g(x)$ (ή $F(x)$+το αόριστο ολοκλήρωμα του $g(x)$) :\\
επομένως, η εντολή {\en\tt ibpdv(g(x),0)} επιστρέφει το αόριστο ολοκλήρωμα {\en\tt G(x)} της {\en\tt g(x)} και \\
η εντολή {\en\tt ibpdv([F(x),g(x)],0)} επιστρέφει {\en\tt F(x)+G(x)} όπου {\en\tt diff(G(x))=g(x)}.
\end{itemize}
Επομένως, {\en\tt ibpdv} επιστρέφει τους όρους που υπολογίζονται με ολοκλήρωση κατά μέλη, με την πιθανότητα να γίνουν πολλαπλά {\en\tt ibpdv} διαδιοχικά.\\
Αφού υπολογίσουμε το αποτέλεσμα του {\en\tt ibpdv(u(x)*v$'$(x),v(x))}, για να βρούμε το αόριστο ολοκλήρωμα του $u(x).v'(x)$, απομένει να υπολογίσουμε το ολοκλήρωμα του δεύτερου όρου του ολοκληρώματος και μετά, να αθροίσουμε αυτό το ολοκλήρωμα με τον πρώτο όρο του αποτελέσματος : για να το κάνουμε αυτό, απλά χρησιμοποιούμε την εντολή {\en\tt ibpdv} με το αποτέλεσμα ως πρώτο όρισμα και μια νέα $v(x)$ (ή $0$ για να τερματίσουμε την ολοκλήρωση) ως δεύτερο όρισμα.\\ 
Είσοδος :
\begin{center}{\en\tt ibpdv(ln(x),x) }\end{center}
Έξοδος :
\begin{center}{\en\tt [x.ln(x),-1]}\end{center}
μετά :
\begin{center}{\en\tt ibpdv([x.ln(x),-1],0) }\end{center}
Έξοδος :
\begin{center}{\en\tt -x+x.ln(x)}\end{center}
{\bf Σχόλιο}\\
 Όταν το πρώτο όρισμα του {\en\tt ibpdv} είναι μια λίστα δύο στοιχείων, η εντολή {\en\tt ibpdv} δουλεύει μόνο με το τελευταίο στοιχείο της λίστας και προσθέτει τον όρο που έχει ολοκληρωθεί στο πρώτο στοιχείο αυτής της λίστας. (Συνεπώς είναι δυνατόν να γίνουν πολλαπλά {\en\tt ibpdv} διαδοχικά).\\
Για παράδειγμα :\\
{\en\tt ibpdv((log(x))\verb|^|2,x) = [x*(log(x))\verb|^|2,-(2*log(x))]}\\ 
απομένει να ολοκληρώσουμε το {\en\tt -(2*log(x))}. Εισάγετε :\\
{\en\tt ibpdv(ans(),x)} ή εισάγετε :\\
{\en\tt ibpdv([x*(log(x))\verb|^|2,-(2*log(x))],x)}\\
Έξοδος :\\
{\en\tt [x*(log(x))\verb|^|2+x*(-(2*log(x))),2]}\\
και απομένει να ολοκληρώσουμε το {\en\tt 2}, γι'αυτό εισάγετε {\en\tt ibpdv(ans(),0)} ή\\
{\en\tt ibpdv([x*(log(x))\verb|^|2+x*(-(2*log(x))),2],0)}.\\
Έξοδος :
{\en\tt x*(log(x))\verb|^|2+x*(-(2*log(x)))+2*x}
\subsubsection{\en\tt ibpu}\index{ibpu}
\noindent{\en\tt ibpu} χρησιμοποιείται για την εύρεση του αορίστου ολοκληρώματος μιας παράστασης γραμμένης  
ως $u(x).v'(x)$\\
{\en\tt ibpu} παίρνει δύο ορίσματα  : 
\begin{itemize}
\item μια παράσταση $u(x).v'(x)$ και $u(x)$ (ή μια λίστα δύο παραστάσεων 
$[F(x), u(x)*v'(x)]$ και $u(x)$),
\item μια παράσταση $g(x)$ και $0$ (ή μια λίστα δύο παραστάσεων $[F(x) g(x)]$ 
και $0$).
\end{itemize}
{\en\tt ibpu} επιστρέφει :
\begin{itemize}
\item εάν $u(x) \neq 0$, τη λίστα $[u(x).v(x),-v(x).u'(x)]$ 
(ή επιστρέφει τη λίστα $[F(x)+u(x).v(x),-v(x).u'(x)]$),
\item εάν το δεύτερο όρισμα είναι 0, το αόριστο ολοκλήρωμα του πρώτου ορίσματος $g(x)$
(ή $F(x)$+το αόριστο ολοκλήρωμα της $g(x)$):\\ 
{\en\tt ibpu(g(x),0)} επιστρέφει {\en\tt G(x)} όπου {\en\tt diff(G(x))=g(x)} ή\\
 {\en\tt ibpu([F(x),g(x)],0)} επιστρέφει {\en\tt F(x)+G(x)} όπου {\en\tt diff(G(x))=g(x)}.
\end{itemize}
Επομένως, η εντολή {\en\tt ibpu} επιστρέφει τους όρους που υπολογίζονται με ολοκλήρωση κατά μέλη, με την πιθανότητα να γίνουν πολλαπλά {\en\tt ibpu} διαδοχικά.\\
Αφού υπολογίσουμε το αποτέλεσμα του {\en\tt ibpu(u(x)*v$'$(x),u(x))}, για να βρούμε το αόριστο ολοκλήρωμα του $u(x).v'(x)$, απομένει να υπολογίσουμε το ολοκλήρωμα του δεύτερου όρου του ολοκληρώματος και μετά, να αθροίσουμε αυτό το ολοκλήρωμα με τον πρώτο όρο του αποτελέσματος : για να το κάνουμε αυτό, απλά χρησιμοποιούμε την εντολή {\en\tt ibpu} με το αποτέλεσμα ως πρώτο όρισμα και μια νέα $u(x)$ (ή $0$ για να τερματίσουμε την ολοκλήρωση) ως δεύτερο όρισμα.\\ 
Είσοδος :
\begin{center}{\en\tt ibpu(ln(x),ln(x)) }\end{center}
Έξοδος :
\begin{center}{\en\tt [x.ln(x),-1]}\end{center}
μετά
\begin{center}{\en\tt ibpu([x.ln(x),-1],0) }\end{center}
Έξοδος :
\begin{center}{\en\tt -x+x.ln(x)}\end{center}
{\bf Σχόλιο}\\
Όταν το πρώτο όρισμα του {\en\tt ibpu} είναι μια λίστα δύο στοιχείων, η εντολή {\en\tt ibpu} δουλεύει μόνο με το τελευταίο στοιχείο της λίστας και προσθέτει τον όρο που έχει ολοκληρωθεί στο πρώτο στοιχείο της λίστας. (Συνεπώς είναι δυνατόν να γίνουν πολλαπλά {\en\tt ibpu} διαδοχικά).\\
Για παράδειγμα :\\
{\en\tt ibpu((log(x))\verb|^|2,log(x)) = [x*(log(x))\verb|^|2,-(2*log(x))]}\\ 
απομένει να ολοκληρώσουμε το {\en\tt -(2*log(x))}, γι' αυτό εισάγουμε : \\
{\en\tt ibpu(ans(),log(x))}
 ή εισάγουμε :\\
{\en\tt ibpu([x*(log(x))\verb|^|2,-(2*log(x))],log(x))}\\
Έξοδος :\\
{\en\tt [x*(log(x))\verb|^|2+x*(-(2*log(x))),2]}\\
απομένει να ολοκληρώσουμε το  {\en\tt 2}, γι' αυτό εισάγουμε :\\
{\en\tt ibpu(ans(),0)} ή εισάγουμε :\\
{\en\tt ibpu([x*(log(x))\verb|^|2+x*(-(2*log(x))),2],0)}.\\
Έξοδος :
{\en\tt x*(log(x))\verb|^|2+x*(-(2*log(x)))+2*x}

\subsection{Αλλαγή μεταβλητών : {\tt\textlatin{ subst}}}
Δείτε την εντολή {\en\tt subst} σο τμήμα \ref{sec:subst}. 

\section{Όρια}
\subsection{Όρια : {\tt\textlatin{ limit}}}\index{limit|textbf}\label{sec:limit}

{\en\tt limit} υπολογίζει το όριο μιας παράστασης σε ένα πεπερασμένο σημείο ή στο άπειρο.
Είναι επίσης πιθανό με προαιρετικό όρισμα να υπολογίσετε ένα μονοκατευθυντικό
όριο (1 για όριο από δεξιά  και -1 για όριο από αριστερά ).\\
{\en\tt limit} παίρνει τρία ή τέσσερα ορίσματα:\\
μια παράσταση, το όνομα μιας μεταβλητής (για παράδειγμα {\en\tt x}), το οριακό σημείο 
(για παράδειγμα {\en\tt a}) και ένα προαιρετικό όρισμα, από προεπιλογή {\tt 0},  
για να υποδείξει αν το όριο είναι μονοκατευθυντικό. 
Αυτό το όρισμα ισούται με 
{\tt -1} για όριο από αριστερά ({\en\tt x<a}) ή ισούται με  {\tt  1} 
για όριο από δεξιά ({\en\tt x>a}) ή ισούται με {\tt 0} για ένα όριο.\\
{\en\tt limit} επιστρέφει το όριο της παράστασης όταν η μεταβλητή (για παράδειγμα
{\en\tt x}) πλησιάζει το οριακό σημείο (για παράδειγμα {\en\tt a}).\\
{\bf Σχόλιο}\\
Είναι επιπλέον δυνατόν να βάλουμε το {\en\tt x=a} ως όρισμα αντί των {\en\tt x,a}. Επομένως :
{\en\tt limit} παίρνει επίσης ως ορίσματα μια παράσταση που εξαρτάται από μια μεταβλητή, 
μια ισότητα (μεταβλητή = τιμή του οριακού σημείου ) και ίσως 1 ή -1 για να 
υποδείξει την κατεύθυνση.\\ 
Είσοδος :
\begin{center}{\en\tt limit(1/x,x,0,-1)}\end{center}
ή
\begin{center}{\en\tt limit(1/x,x=0,-1)}\end{center}
Έξοδος :
\begin{center}{\en\tt -(infinity)}\end{center} 
Είσοδος :
\begin{center}{\en\tt limit(1/x,x,0,1)}\end{center}
ή
\begin{center}{\en\tt limit(1/x,x=0,1)}\end{center}
Έξοδος :
\begin{center}{\en\tt +(infinity)}\end{center} 
Είσοδος :
\begin{center}{\en\tt limit(1/x,x,0,0)}\end{center}
ή
\begin{center}{\en\tt limit(1/x,x,0)}\end{center}
ή
\begin{center}{\en\tt limit(1/x,x=0)}\end{center}
Έξοδος :
\begin{center}{\en\tt infinity}\end{center} 
Επομένως, {\en\tt abs(1/x)} πλησιάζει $+\infty$ όταν το $x$ πλησιάζει $0$.

{\bf Ασκήσεις} :
\begin{itemize}
\item Για $n>2$, και  όταν το $x$ πλησιάζει το $0$ βρείτε το όριο της παράστασης :
$$ \frac{n\tan(x)-\tan(nx)}{\sin(nx)-n\sin(x)}$$
Είσοδος :
\begin{center}{\en\tt limit((n*tan(x)-tan(n*x))/(sin(n*x)-n*sin(x)),x=0)}\end{center}
Έξοδος :
\begin{center}{\en\tt 2 }\end{center}
\item Βρείτε το όριο όταν το  $x$ πλησιάζει το $+\infty$ :
$$\sqrt{x+\sqrt{x+\sqrt x}}-\sqrt x$$ 
Είσοδος :
\begin{center}{\en\tt limit(sqrt(x+sqrt(x+sqrt(x)))-sqrt(x),x=+infinity)}\end{center}
Έξοδος :
\begin{center}{\tt 1/2 }\end{center}
\item Βρείτε το όριο όταν το $x$ πλησιάζει το 0 :
$$\frac{\sqrt{1+x+x^2/2}-\exp(x/2)}{(1-\cos(x))\sin(x)}$$ 
Είσοδος :
\begin{center}{\en\tt limit((sqrt(1+x+x\verb|^|2/2)-exp(x/2))/((1-cos(x))*sin(x)),x,0)}\end{center}
Έξοδος :
\begin{center}{\tt -1/6 }\end{center}
\end{itemize}

{\bf Σχόλιο}\\
Για τον υπολογισμό του ορίου, είναι καλύτερο μερικές φορές να αναφέρουμε το πρώτο όρισμα.\\  
Είσοδος  :
\begin{center}{\en\tt limit($'$(2*x-1)*exp(1/(x-1))$'$,x=+infinity)}\end{center}
Να σημειωθεί πως το πρώτο όρισμα αναφέρεται,  γιατί είναι προτιμότερο
αυτό το όρισμα να μην απλοποιείται (δηλαδή να μην αποτιμάται).\\
Έξοδος :
\begin{center}{\en\tt +(infinity)}\end{center}

\subsection{Ολοκλήρωμα και όριο}\index{limit} \index{limite} 
Δυο παραδείγματα :\\
\begin{itemize}
\item Βρείτε το όριο της ακόλουθης παράστασης, όταν το $a$ πλησιάζει το $+\infty$ :
$$  \int _2^a \frac {1}{x^2}\ dx$$
Είσοδος :
\begin{center}{\en\tt limit(integrate(1/(x\verb|^|2),x,2,a),a,+(infinity))}\end{center}
Έξοδος (εάν το {\en\tt a} δεν είναι τυπικό τότε εισάγετε  {\en\tt purge(a)}) :
\begin{center}{\tt 1/2}\end{center}
\item Βρείτε το όριο της ακόλουθης παράστασης, οταν το $a$ πλησιάζει το $+\infty$ :
$$  \int _2^a (\frac {x}{x^2-1}+\ln(\frac {x+1}{x-1}))\ dx$$
Είσοδος :
\begin{center}{\en\tt limit(integrate(x/(x\verb|^|2-1)+log((x+1)/(x-1)),x,2,a),}\end{center} 
\begin{center}{\en\tt a,+(infinity))}\end{center} 
Έξοδος (εάν το {\en\tt a} δεν είναι τυπικό τότε εισάγετε {\en\tt purge(a)}) :
\begin{center}{\en\tt +(infinity)}\end{center}
\end{itemize}

\section{Επανεγγραφή υπερβατικών και τριγωνομετρικών παραστάσεων}
\subsection{Ανάπτυγμα υπερβατικών και τριγωνομετρικών παραστάσεων : {\tt\textlatin{ texpand tExpand}}}\index{texpand|textbf}\index{tExpand|textbf}
\noindent{\en\tt texpand} ή {\en\tt tExpand} παίρνει ως όρισμα μια
παράσταση που περιέχει υπερβατικές ή τριγωνομετρικές συναρτήσεις.\\
{\en\tt texpand} ή {\en\tt tExpand} αναπτύσσει αυτές τις συναρτήσεις. Είναι σαν να καλούνται ταυτόχρονα οι
{\en\tt expexpand}, {\en\tt lnexpand} και {\en\tt trigexpand}, όπου
για παράδειγμα, η $\ln(x^n)$ γίνεται $n\ln(x)$, η $\exp(nx)$ 
γίνεται $\exp(x)^n$, και η $\sin(2x)$ γίνεται $2\sin(x)\cos(x)$...\\
{\bf Παραδείγματα} :\\
\begin{itemize}
\item
\begin{enumerate}
\item Αναπτύξτε την $\cos(x+y)$.\\
Είσοδος :
\begin{center}{\en\tt texpand(cos(x+y))}\end{center}
Έξοδος :
\begin{center}{\en\tt cos(x)*cos(y)-sin(x)*sin(y)}\end{center}
\item Αναπτύξτε την $\cos(3x)$.\\
Είσοδος :
\begin{center}{\en\tt texpand(cos(3*x))}\end{center}
Έξοδος :
\begin{center}{\en\tt 4*(cos(x))\verb|^| 3-3*cos(x)}\end{center}
\item Αναπτύξτε την $\displaystyle \frac{\sin(3*x)+\sin(7*x)}{\sin(5*x)}$.\\
Είσοδος :
\begin{center}{\en\tt texpand((sin(3*x)+sin(7*x))/sin(5*x))}\end{center}
Έξοδος
\begin{center}{\en\tt (4*(cos(x))\verb|^|2-1)*(sin(x)/(16*(cos(x))\verb|^|4- 12*(cos(x))\verb|^|2+1))/sin(x)+(64*(cos(x))\verb|^|6- 80*(cos(x))\verb|^|4+24*(cos(x))\verb|^|2-1)*sin(x)/ (16*(cos(x))\verb|^|4-12*(cos(x))\verb|^|2+1)/sin(x)}\end{center}
Έξοδος, μετά από απλοποίηση με την {\en\tt normal(ans())} :
\begin{center}{\en\tt 4*(cos(x))\verb|^|2-2}\end{center}
\end{enumerate}

\item \begin{enumerate}
\item Αναπτύξτε την $\exp(x+y)$.\\
Είσοδος :
\begin{center}{\en\tt texpand(exp(x+y))}\end{center}
Έξοδος :
\begin{center}{\en\tt exp(x)*exp(y)}\end{center}
\item Αναπτύξτε την $\ln(x\times y)$.\\
Είσοδος :
\begin{center}{\en\tt texpand(log(x*y))}\end{center}
Έξοδος :
\begin{center}{\en\tt log(x)+log(y)}\end{center}
\item Αναπτύξτε την $\ln(x^n)$.\\
Είσοδος :
\begin{center}{\en\tt  texpand(ln(x\verb|^|n))}\end{center}
Έξοδος :
\begin{center}{\en\tt n*ln(x)}\end{center}
\item Αναπτύξτε την  $\ln((e2)+\exp(2*\ln(2))+exp(\ln(3)+\ln(2)))$.\\
Είσοδος :
\begin{center}{\en\tt texpand(log(e\verb|^|2)+exp(2*log(2))+exp(log(3)+log(2)))}\end{center}
Έξοδος :
\begin{center}{\en\tt 6+3*2}\end{center}
Ή εισάγετε :
\begin{center}{\en\tt texpand(log(e\verb|^|2)+exp(2*log(2)))+ lncollect(exp(log(3)+log(2)))}\end{center}
Έξοδος :
\begin{center}{\en\tt 12}\end{center}
\end{enumerate}
\item 
Αναπτύξτε την $\exp(x+y)+\cos(x+y)+\ln(3x2)$.\\
Είσοδος :
\begin{center}{\en\tt texpand(exp(x+y)+cos(x+y)+ln(3*x\verb|^|2))}\end{center}
Έξοδος :
\begin{center}{\en\tt cos(x)*cos(y)-sin(x)*sin(y)+exp(x)*exp(y)+ ln(3)+2*ln(x)}\end{center}
\end{itemize}

\subsection{Συνδυασμός όρων του ιδίου τύπου  : {\tt\textlatin{ combine}}}\index{combine}\index{exp@{\sl exp}|textbf}\index{log@{\sl log}|textbf}\index{ln@{\sl ln}|textbf}\index{sin@{\sl sin}|textbf}\index{cos@{\sl cos}|textbf}\index{trig@{\sl trig}|textbf}
\noindent{\en\tt combine} παίρνει δύο ορίσματα : μια παράσταση και 
και το όνομα μιας συνάρτησης ή κλάσης συναρτήσεων
{\en\tt exp, log, ln, sin, cos, trig}.\\
Όποτε είναι δυνατόν, η εντολή {\en\tt combine} συνδυάζει υποπαραστάσεις που αντιστιχούν στο
δεύτερο όρισμα:
\begin{itemize}
\item {\en\tt combine(expr,ln)} ή {\en\tt combine(expr,log)} δίνει το ίδιο αποτέλεσμα με 
την  {\en\tt lncollect(expr)}
\item
{\en\tt combine(expr,trig)} ή {\en\tt combine(expr,sin)}  ή {\en\tt combine(expr,cos)}
δίνει το ίδιο αποτέλεσμα με την {\en\tt tcollect(expr)}.
\end{itemize}
Είσοδος :
\begin{center}{\en\tt combine(exp(x)*exp(y)+sin(x)*cos(x)+ln(x)+ln(y),exp)}\end{center}
Έξοδος :
\begin{center}{\en\tt exp(x+y)+sin(x)*cos(x)+ln(x)+ln(y)}\end{center}
Είσοδος :
\begin{center}{\en\tt combine(exp(x)*exp(y)+sin(x)*cos(x)+ln(x)+ln(y),trig)}\end{center}
ή
\begin{center}{\en\tt combine(exp(x)*exp(y)+sin(x)*cos(x)+ln(x)+ln(y),sin)}\end{center}
ή
\begin{center}{\en\tt combine(exp(x)*exp(y)+sin(x)*cos(x)+ln(x)+ln(y),cos)}\end{center}
Έξοδος :
\begin{center}{\en\tt exp(y)*exp(x)+(sin(2*x))/2+ln(x)+ln(y)}\end{center}
Είσοδος :
\begin{center}{\en\tt combine(exp(x)*exp(y)+sin(x)*cos(x)+ln(x)+ln(y),ln)}\end{center}
ή
\begin{center}{\en\tt combine(exp(x)*exp(y)+sin(x)*cos(x)+ln(x)+ln(y),log)}\end{center}
Έξοδος :
\begin{center}{\en\tt exp(x)*exp(y)+sin(x)*cos(x)+ln(x*y)}\end{center}

\section{Τριγωνομετρία}
\subsection{Τριγωνομετρικές συναρτήσεις}\label{sec:trigo}
\begin{itemize}
\item {\en\tt sin} \index{sin} είναι η συνάρτηση ημιτόνου ,
\item {\en\tt cos} \index{cos}  είναι η συνάρτηση συνημιτόνου,
\item  {\en\tt tan} \index{tan} είναι η συνάρτηση εφαπτομένης ({\en\tt tan(x)= sin(x)/cos(x)}),
\item 
{\en\tt cot} \index{cot|textbf}  είναι η συνάρτηση συνεφαπτομένης ({\en\tt cot(x)=cos(x)/sin(x)}),
\item 
{\en\tt sec} \index{sec|textbf}  είναι η συνάρτηση τέμνουσας ({\en\tt sec(x)=1/cos(x)}),
\item
{\en\tt csc} \index{csc|textbf}  είναι η συνάρτηση συντέμνουσας (\en{\tt csc(x) = 1/sin(x)}),
\item
{{\tt\textlatin{ asin }} {\gr ή} {\tt{ arcsin}}\index{asin}\index{arcsin}, {\tt\textlatin{ acos } {\gr ή} {\tt{ arccos}}\index{acos}\index{arccos}, {\tt\textlatin{ atan }} {\gr ή} {\tt{ arctan}}\index{atan}\index{arctan}, {\tt{ acot}}\index{acot|textbf}, {\tt{ asec}}\index{asec|textbf},  {\tt\textlatin{ acsc }}\index{acsc|textbf}}}  {\gr είναι οι αντίστροφες τριγωνομετρικές συναρτήσεις.  Οι τρεις τελευταίες ορίζονται ως εξής} : 
\begin{enumerate}
\item {\en\tt asec(x) = acos(1/x)}, 
\item
{\en\tt acsc(x) = asin(1/x)},
\item
{\en\tt  acot(x) = atan(1/x)}. 
\end{enumerate}
\end{itemize}

\subsection{Ανάπτυγμα  τριγωνομετρικών παραστάσεων :{\tt\textlatin{trigexpand}}}\index{trigexpand}
\noindent{\en\tt trigexpand} παίρνει ως όρισμα μια παράσταση
που περιέχει τριγωνομετρικές συναρτήσεις.\\
Η {\en\tt trigexpand} αναπτύσσει αθροίσματα, διαφορές και γινόμενα επί έναν ακέραιο
μέσα στις τριγωνομετρικές συναρτήσεις. \\
Είσοδος :
\begin{center}{\en\tt trigexpand(cos(x+y))}\end{center}
Έξοδος :
\begin{center}{\en\tt cos(x)*cos(y)-sin(x)*sin(y)}\end{center}


\subsection{Γραμμικοποίηση τριγωνομετρικών παραστάσεων : {\tt\textlatin{tlin}}}\index{tlin}
\noindent{\en\tt tlin} παίρνει ως όρισμα μια παράσταση
που περιέχει τριγωνομετρικές συναρτήσεις.\\
{\en\tt tlin} γραμμικοποιεί τα γινόμενα και τις ακέραιες δυνάμεις τριγωνομετρικών
συναρτήσεων (π.χ. ως προς $\sin(n.x)$ και 
$\cos(n.x)$)\\
{\bf Παραδείγματα}
\begin{itemize}
\item Γραμμικοποιείστε την $\cos(x)*\cos(y)$.\\
Είσοδος :
\begin{center}{\en\tt tlin(cos(x)*cos(y))}\end{center}
Έξοδος :
\begin{center}{\en\tt 1/2*cos(x-y)+1/2*cos(x+y)}\end{center}
\item Γραμμικοποιείστε την $\cos(x)^{3}$.\\
Είσοδος :
\begin{center}{\en\tt tlin(cos(x)\verb|^|3)}\end{center}
Έξοδος :
\begin{center}{\en\tt 3/4*cos(x)+1/4*cos(3*x)}\end{center}
\item Γραμμικοποιείστε την $4\cos(x)2-2$.\\
Είσοδος :
\begin{center}{\en\tt tlin(4*cos(x)\verb|^|2-2)}\end{center}
Έξοδος :
\begin{center}{\en\tt 2*cos(2*x)}\end{center}
\end{itemize}

\subsection{Συλλογή των ημιτόνων και συνημιτόνων της ίδιας γωνίας : {\tt\textlatin{ tcollect tCollect}}}\index{tcollect}\index{tCollect}
\noindent{\en\tt tcollect} ή {\en\tt tCollect} παίρνει ως όρισμα 
μια παράσταση που περιέχει τριγωνομετρικές συναρτήσεις.\\
{\en\tt tcollect} πρώτα γραμμικοποιεί την παράσταση 
(π.χ.  ως προς $\sin(n.x)$ και  $\cos(n.x)$),  
και μετά, συλλέγει τα ημίτονα και συνημίτονα της ίδιας γωνίας.\\
Είσοδος :
\begin{center}{\en\tt tcollect(sin(x)+cos(x))}\end{center}
Έξοδος :
\begin{center}{\en\tt sqrt(2)*cos(x-pi/4)}\end{center}
Είσοδος :
\begin{center}{\en\tt tcollect(2*sin(x)*cos(x)+cos(2*x))}\end{center}
Έξοδος :
\begin{center}{\en\tt sqrt(2)*cos(2*x-pi/4)}\end{center}

\subsection{Απλοποίηση παραστάσεων : {\tt\textlatin{ simplify}}\index{simplify}}
\noindent{{\en\tt simplify} απλοποιεί μια παράσταση.}\\
Όπως με όλες τις αυτόματες απλοποιήσεις, μην περιμένετε θαύματα. Αν δεν δουλέψει θα πρέπει να χρησιμοποιήσετε συγκεκριμένους κανόνες αναγραφής.\\ 
Είσοδος :
\begin{center}{\en\tt simplify((sin(3*x)+sin(7*x))/sin(5*x))}\end{center}
Έξοδος :
\begin{center}{\en\tt 4*(cos(x))\verb|^|2-2}\end{center}
{\bf Προσοχή}{ Η {\en\tt simplify} είναι πιο αποδοτική στην επιλογή {\en\tt radian} (Επιλέξτε 
{\en\tt radian} στις Ρυθμίσεις {\en\tt Cas}  
  ή εισάγετε {\en\tt angle\_radian:=1})}.

\subsection{Μετασχηματισμός {\tt\textlatin{arccos}} σε 
{\tt\textlatin{arcsin}} : {\tt\textlatin{ acos2asin}}\index{acos2asin}}
\noindent{{\en\tt acos2asin} παίρνει ως όρισμα μια παράσταση που περιέχει 
αντίστροφες τριγωνομετρικές συναρτήσεις.}\\
{\en\tt acos2asin}  αντικαθιστά την $\arccos(x)$ με 
$\displaystyle \frac{\pi}{2}-\arcsin(x)$, σε αυτή την παράσταση.\\
Είσοδος :
\begin{center}{\en\tt acos2asin(acos(x)+asin(x))}\end{center}
Έξοδος μετά από απλοποίηση :
\begin{center}{\en\tt pi/2}
\end{center}

\subsection{Μετασχηματισμός {\tt\textlatin{arccos}} σε {\tt\textlatin{arctan}} : {\tt\textlatin{ acos2atan}}\index{acos2atan}}
\noindent{\en\tt acos2atan} παίρνει ως όρισμα μια παράσταση που περιέχει 
αντίστροφες τριγωνομετρικές συναρτήσεις.\\
{\en\tt acos2atan}  αντικαθιστά την $\arccos(x)$ με την
$\displaystyle \frac{\pi}{2}-\arctan(\frac{x}{\sqrt{1-x2}})$, σε αυτή την  
παράσταση.\\
Είσοδος :
\begin{center}{\en\tt acos2atan(acos(x))}\end{center}
Έξοδος :
\begin{center}{\en\tt  pi/2-atan(x/sqrt(1-x\verb|^|2))}\end{center}

\subsection{Μετασχηματισμός {\tt\textlatin{arcsin}} σε 
{\tt\textlatin{arccos}} : {\tt\textlatin{ asin2acos}}}\index{asin2acos}
\noindent {\en\tt asin2acos} παίρνει ως όρισμα μια παράσταση που περιέχει 
αντίστροφες τριγωνομετρικές συναρτήσεις.\\
{\en\tt asin2acos} αντικαθιστά την $\arcsin(x)$  με την
$\displaystyle \frac{\pi}{2}-\arccos(x)$, σε αυτή την παράσταση.\\
Είσοδος :
\begin{center}{\en\tt asin2acos(acos(x)+asin(x))}\end{center}
Έξοδος μετά από απλοποίηση :
\begin{center}{\en\tt pi/2}\end{center}

\subsection{Μετασχηματισμός {\tt\textlatin{arcsin}} σε {\tt\textlatin{arctan}} : {\tt\textlatin{ asin2atan}}}\index{asin2atan}
\noindent{\en\tt asin2atan} παίρνει ως όρισμα μια παράσταση που περιέχει 
αντίστροφες τριγωνομετρικές συναρτήσεις.\\
{\en\tt asin2atan} αντικαθιστά την $\arcsin(x)$  με την 
$\displaystyle \arctan(\frac{x}{\sqrt{1-x2}})$, σε αυτή την παράσταση.\\
Είσοδος :
\begin{center}{\en\tt asin2atan(asin(x))}\end{center}
Έξοδος :
\begin{center}{\en\tt atan(x/sqrt(1-x\verb|^|2))}\end{center}

\subsection{Μετασχηματισμός {\tt\textlatin{arctan}} σε {\tt\textlatin{arcsin}} : {\tt\textlatin{ atan2asin}}}\index{atan2asin}
\noindent{\en\tt atan2asin} παίρνει ως όρισμα μια παράσταση που περιέχει 
αντίστροφες τριγωνομετρικές συναρτήσεις.\\
{\en\tt atan2asin} αντικαθιστά την $\arctan(x)$ με την 
$\displaystyle \arcsin(\frac{x}{\sqrt{1+x2}})$, σε αυτή την παράσταση.\\
Είσοδος :
\begin{center}{\en\tt atan2asin(atan(x))}\end{center}
Έξοδος :
\begin{center}{\en\tt asin(x/sqrt(1+x\verb|^|2))}\end{center}

\subsection{Μετασχηματισμός {\tt\textlatin{arctan}} σε {\tt\textlatin{arccos}} : {\tt\textlatin{ atan2acos}}}\index{atan2acos}
\noindent {\en\tt atan2acos} παίρνει ως όρισμα μια παράσταση που περιέχει 
αντίστροφες τριγωνομετρικές συναρτήσεις.\\
{\en\tt atan2acos} αντικαθιστά την $\arctan(x)$ με την 
$\displaystyle \frac{\pi}{2}-\arccos(\frac{x}{\sqrt{1+x2}})$, 
σε αυτή την παράσταση.\\
Είσοδος :
\begin{center}{\en\tt atan2acos(atan(x))}\end{center}
Έξοδος :
\begin{center}{\en\tt pi/2-acos(x/sqrt(1+x\verb|^|2))}\end{center}

\subsection{Μετασχηματισμός μιγαδικών εκθετικών σε {\tt\textlatin{sin}} και {\tt\textlatin{cos}} : {\tt\textlatin{ sincos exp2trig}}}\index{sincos}\index{exp2trig}
\noindent {\en\tt sincos}  ή {\en\tt exp2trig} παίρνει ως όρισμα μια παράσταση που περιέχει 
σύνθετα εκθετικά.\\
{\en\tt sincos} ή {\en\tt exp2trig} αναγράφει αυτήν την παράσταση ως προς 
$\sin$ και  $\cos$.\\
Είσοδος :
\begin{center}{\en\tt sincos(exp(i*x))}\end{center}
Έξοδος :
\begin{center}{\en\tt cos(x)+(i)*sin(x)}\end{center}
Είσοδος :
\begin{center}{\en\tt exp2trig(exp(-i*x))}\end{center}
Έξοδος :
\begin{center}{\en\tt cos(x)+(i)*(-(sin(x)))}\end{center}
Είσοδος :
\begin{center}{\en\tt simplify(sincos(((i)*(exp((i)*x))\verb|^|2-i)/(2*exp((i)*x))))}\end{center}
ή :
\begin{center}{\en\tt simplify(exp2trig(((i)*(exp((i)*x))\verb|^|2-i)/(2*exp((i)*x))))}\end{center}
Έξοδος :
\begin{center}{\en\tt -sin(x)}\end{center}

\subsection{Μετασχηματισμός {\tt\textlatin{tan(x)}} σε {\tt\textlatin{sin(x)/cos(x)}} : \\{\tt\textlatin{ tan2sincos}}}\index{tan2sincos}
\noindent {\en\tt tan2sincos} παίρνει ως όρισμα μια παράσταση που περιέχει 
τριγωνομετρικές συναρτήσεις.\\
{\en\tt tan2sincos} αντικαθιστά την $\tan(x)$ με την
 $\displaystyle \frac{\sin(x)}{\cos(x)}$, σε αυτήν την παράσταση.\\
Είσοδος :
\begin{center}{\en\tt tan2sincos(tan(2*x))}\end{center}
Έξοδος :
\begin{center}{\en\tt sin(2*x)/cos(2*x)}\end{center}

\subsection{Αναγραφή της {\tt\textlatin{tan(x)}} ως προς {\tt\textlatin{sin(2x)}} και  {\tt\textlatin{cos(2x)}} : {\tt\textlatin{ tan2sincos2}}}\index{tan2sincos2}
\noindent {\en\tt tan2sincos2} παίρνει ως όρισμα μια παράσταση που περιέχει 
τριγωνομετρικές συναρτήσεις.\\
{\en\tt tan2sincos2} αντικαθιστά την $\tan(x)$ με την 
$\displaystyle \frac{\sin(2.x)}{1+\cos(2.x)}$, σε αυτή την παράσταση.\\
Είσοδος :
\begin{center}{\en\tt tan2sincos2(tan(x))}\end{center}
Έξοδος :
\begin{center}{\en\tt sin(2*x)/(1+cos(2*x))}\end{center}

\subsection{Αναγραφή της {\tt\textlatin{tan(x)}} ως προς {\tt\textlatin{cos(2x)}} και {\tt\textlatin{sin(2x)}} : {\tt\textlatin{ tan2cossin2}}}\index{tan2cossin2}
\noindent{\en\tt tan2cossin2} παίρνει ως όρισμα μια παράσταση που περιέχει 
τριγωνομετρικές συναρτήσεις.\\
{\en\tt tan2cossin2} αντικαθιστά την $\tan(x)$ με την 
$\displaystyle \frac{1-\cos(2.x)}{\sin(2.x)}$ , σε αυτή την παράσταση.\\
Είσοδος :
\begin{center}{\en\tt tan2cossin2(tan(x))}\end{center}
Έξοδος :
\begin{center}{\en\tt (1-cos(2*x))/sin(2*x)}\end{center}

\subsection{Αναγραφή των {\tt\textlatin{sin, cos, tan}} ως προς {\tt\textlatin{tan(x/2)}} : \\{\tt\textlatin{halftan}}}\index{halftan}
\noindent {\tt\textlatin{halftan}} παίρνει ως όρισμα μια παράσταση που περιέχει
τριγωνομετρικές συναρτήσεις.\\
{\tt\textlatin{halftan}} αναγράφει τις $\sin(x),\ \cos(x)$ και  $ \tan(x)$ 
ως προς $\tan(\frac{x}{2})$.\\
Είσοδος :
\begin{center}{\en\tt halftan(sin(2*x)/(1+cos(2*x)))}\end{center}
Έξοδος :
\begin{center}{\en\tt 2*tan(2*x/2)/((tan(2*x/2))\verb|^|2+1)/}\end{center}
\begin{center}{\en\tt (1+(1-(tan(2*x/2))\verb|^|2)/((tan(2*x/2))\verb|^|2+1))}\end{center}
Έξοδος, μετά από απλοποίηση με την {\en{\tt normal(ans())}} :
\begin{center}{\en\tt tan(x)}\end{center}
Είσοδος :
\begin{center}{\en\tt halftan(sin(x)\verb|^|2+cos(x)\verb|^|2)}\end{center}
Έξοδος :
\begin{center}{\en\tt (2*tan(x/2)/((tan(x/2))\verb|^|2+1))\verb|^|2+}\end{center}
\begin{center}{\en\tt ((1-(tan(x/2))\verb|^|2)/((tan(x/2))\verb|^|2+1))\verb|^|2}\end{center}
Έξοδος, μετά από απλοποίηση με την {\en{\tt\textlatin normal(ans())}} :
\begin{center}{\tt 1}\end{center}

\subsection{Aναγραφή τριγωνομετρικών συναρτήσεων ως προς\\ {\tt\textlatin{ tan(x/2)}} 
και υπερβολικών συναρτήσεων ως προς {\tt\textlatin {exp(x)}}: \\
{\tt\textlatin{ halftan\_hyp2exp}}}\index{halftan\_hyp2exp}
\noindent{{\en\tt halftan\_hyp2exp} παίρνει ως όρισμα μια τριγωνομετρική και μια  
υπερβολική παράσταση.}\\
{\en\tt halftan\_hyp2exp} αναγράφει τις $\sin(x), \cos(x), \tan(x)$ 
ως προς $\tan(\frac{x}{2})$ 
και τις $\sinh(x), \cosh(x), \tanh(x)$ ως προς $\exp(x)$.\\
Είσοδος :
\begin{center}{\en\tt halftan\_hyp2exp(tan(x)+tanh(x))}\end{center}
Έξοδος :
\begin{center}{\en\tt (2*tan(x/2))/((1-(tan(x/2))\verb|^|2))+(((exp(x))\verb|^|2-1))/ (((exp(x))\verb|^|2+1))}\end{center}
Είσοδος :
\begin{center}{\en\tt halftan\_hyp2exp(sin(x)\verb|^|2+cos(x)\verb|^|2-sinh(x)\verb|^|2+cosh(x)\verb|^|2)}\end{center}
Έξοδος, μετά από απλοποίηση με την {\en\tt normal(ans())} :
\begin{center}{\tt 2}\end{center}

\subsection{Μετασχηματισμός αντίστροφων τριγωνομετρικών συναρτήσεων σε λογαρίθμους : {\tt\textlatin{atrig2ln}}}\index{atrig2ln}
\noindent{{\en\tt atrig2ln}  παίρνει ως όρισμα μια παράσταση που περιέχει αντίστροφες
τριγωνομετρικές συναρτήσεις.}\\
{\en\tt atrig2ln} αναγράφει αυτές τις συναρτήσεις ως προς μιγαδικούς λογαρίθμους.\\
Είσοδος :
\begin{center}{\en\tt atrig2ln(asin(x))}\end{center}
Έξοδος :
\begin{center}{\en\tt i*log(x+sqrt(x\verb|^|2-1))+pi/2}\end{center}

\subsection{Μετασχηματισμός τριγωνομετρικών συναρτήσεων σε μιγαδικά εκθετικά  : {\tt\textlatin{trig2exp}}}\index{trig2exp}
\noindent{{\en\tt trig2exp} παίρνει ως όρισμα μια παράσταση που περιέχει
τριγωνομετρικές συναρτήσεις.}\\
Η {\en\tt trig2exp} αναγράφει αυτές τις τριγωνομετρικές συναρτήσεις ως προς μιγαδικά εκθετικά
({\sc χωρίς} γραμμικοποίηση).\\ 
Είσοδος :
\begin{center}{\en\tt trig2exp(tan(x))}\end{center}
Έξοδος :
\begin{center}{\en\tt ((exp((i)*x))\verb|^|2-1)/((i)*((exp((i)*x))\verb|^|2+1))}\end{center}
Είσοδος :
\begin{center}{\en\tt trig2exp(sin(x))}\end{center}
Έξοδος :
\begin{center}{\en\tt (exp((i)*x)-1/(exp((i)*x)))/(2*i)}\end{center}

\subsection{Απλοποίηση και αναγραφή κατά προτίμηση ως προς \tt\textlatin{sine} : {\tt\textlatin{trigsin}}}\index{trigsin}
\noindent{\en\tt trigsin} παίρνει ως όρισμα μια παράσταση που περιέχει
τριγωνομετρικές συναρτήσεις.\\
{\en\tt trigsin} απλοποιεί αυτή την παράσταση με τους τύπους :\\
$\sin(x)2+\cos(x)2=1$, $\displaystyle \tan(x)=\frac{\sin(x)}{\cos(x)}$ και 
προσπαθεί να την αναγράψει  μόνο ως προς {\en sine}.\\ 
Είσοδος :
\begin{center}{\en\tt trigsin(sin(x)\verb|^|4+cos(x)\verb|^|2+1)}\end{center}
Έξοδος :
\begin{center}{\en\tt sin(x)\verb|^|4-sin(x)\verb|^|2+2}\end{center}

\subsection{Απλοποίηση και αναγραφή κατά προτίμηση ως προς \\{\tt\textlatin{cosine}} : {\tt\textlatin{trigcos}}}\index{trigcos}
\noindent{{\en\tt trigcos} παίρνει ως όρισμα μια παράσταση που περιέχει
τριγωνομετρικές συναρτήσεις.}\\
{\en\tt trigcos} απλοποιεί αυτή την παράσταση με τους τύπους :\\
$\sin(x)2+\cos(x)2=1$, $\displaystyle \tan(x)=\frac{\sin(x)}{\cos(x)}$ και 
προσπαθεί να την αναγράψει  μόνο ως προς {\en cosine}.\\ 
Είσοδος :
\begin{center}{\en\tt trigcos(sin(x)\verb|^|4+cos(x)\verb|^|2+1)}\end{center}
Έξοδος :
\begin{center}{\en\tt cos(x)\verb|^|4-cos(x)\verb|^|2+2}\end{center}

\subsection{Απλοποίηση και αναγραφή κατά προτίμηση ως προς \\{\tt\textlatin{tangents}} : {\tt\textlatin{trigtan}}}\index{trigtan}
\noindent{{\en\tt trigtan} παίρνει ως όρισμα μια παράσταση που περιέχει
τριγωνομετρικές συναρτήσεις.}\\
{\en\tt trigtan} απλοποιεί αυτήν την παράσταση με τους τύπους :\\
$\sin(x)2+\cos(x)2=1$, $\displaystyle \tan(x)=\frac{\sin(x)}{\cos(x)}$ και 
προσπαθεί να την αναγράψει μόνο ως προς {\tt\textlatin{tangents.}}\\ 
Είσοδος :
\begin{center}{\en\tt trigtan(sin(x)\verb|^|4+cos(x)\verb|^|2+1)}\end{center}
Έξοδος :
\begin{center}{\en\tt((tan(x))\verb|^|2/(1+(tan(x))\verb|^|2))\verb|^|2+1/(1+(tan(x)\verb|^|2)+1}\end{center}
Έξοδος, μετά από απλοποίηση με την {\en\tt normal} :
\begin{center}{\en\tt (2*tan(x)\verb|^|4+3*tan(x)\verb|^|2+2)/(tan(x)\verb|^|4+2*tan(x))\verb|^|2+1)}\end{center}

\subsection{Αναγραφή παράστασης με διάφορες επιλογές : {\tt\textlatin {convert convertir}}}\index{convert|textbf}\index{convertir|textbf}\index{sin@{\sl sin}|textbf}\index{cos@{\sl cos}|textbf}\index{sincos@{\sl sincos}|textbf}\index{exp@{\sl exp}|textbf}\index{tan@{\sl tan}|textbf}\index{ln@{\sl ln}|textbf}\index{expln@{\sl expln}|textbf}\index{string@{\sl string}|textbf}\index{matrix@{\sl matrix}|textbf}\index{polynom@{\sl polynom}}\index{parfrac@{\sl parfrac}|textbf}\index{partfrac@{\sl partfrac}|textbf}\index{fullparfrac@{\sl fullparfrac}|textbf}\label{sec:convert}
\noindent{{\en\tt convert} παίρνει ως όρισμα μια παράσταση και μια επιλογή.}\\
{\en\tt convert} αναγράφει αυτήn την παράσταση εφαρμόζοντας κανόνες που εξαρτώνται από 
την επιλογή. Έγκυρες επιλογές είναι:
\begin{itemize}
\item{\en\tt sin} μετατρέπει μια παράσταση όπως {\en\tt trigsin}.
\item{\en\tt cos}  μετατρέπει μια παράσταση όπως {\en\tt trigcos}.
\item{\en\tt sincos}  μετατρέπει μια παράσταση όπως {\en\tt sincos}.
\item{\en\tt trig}  μετατρέπει μια παράσταση όπως {\en\tt sincos}.
\item{\en\tt tan}  μετατρέπει μια παράσταση όπως {\en\tt halftan}.
\item{\en\tt exp} μετατρέπει μια παράσταση όπως {\en\tt trig2exp}.
\item{\en\tt ln}  μετατρέπει μια παράσταση όπως {\en\tt trig2exp}.
\item{\en\tt expln}  μετατρέπει μια παράσταση όπως {\en\tt trig2exp}.
\item{\en\tt string}  μετατρέπει μια παράσταση σε μια συμβολοσειρά
\item{\en\tt matrix} μετατρέπει μια λίστα από λίστες σε ένα πίνακα.
\item{\en\tt polynom} μετατρέπει μια ακολουθία {\en Taylor} σε ένα πολυώνυμο 
αφαιρώντας το υπόλοιπο (βλέπε \ref{sec:convertpoly}).
\item{\en\tt parfrac} ή {\en\tt partfrac} ή {\en\tt fullparfrac} μετατρέπει ένα ρητό κλάσμα 
στην ανάλυση μερικών κλασμάτων (βλέπε \ref{sec:convertparf}).
\end{itemize}
{\en\tt convert} μπορεί επίσης να :
\begin{itemize}
\item μετατρέπει μονάδες, για παράδειγμα 
{\en\tt convert(1000\_g,\_kg)=1.0\_kg} (βλέπε \ref{sec:convertunit}).
\item γράφει έναν πραγματικό σαν ένα συνεχές κλάσμα  : 
{\en\tt convert(a,confrac,$'$fc$'$)} γράφει το {\en\tt a} σαν ένα συνεχές κλάσμα
αποθηκευμένο στο {\en\tt fc}. Μην ξεχάσετε να αναφέρεται το τελευταίο όρισμα αν του είχε  ανατεθεί τιμή
προηγουμένως.\\
Για παράδειγμα, {\en\tt convert(1.2,confrac,$'$fc$'$)=[1,5]} και  {\en\tt fc} περιέχει 
το συνεχές κλάσμα που είναι ίσο με  1.2 (βλέπε \ref{sec:convertdfc}). 
\item  μετατρέπει έναν ακέραιο στη λίστα των ψηφίων του ως προς κάποια
βάση, αρχίζοντας με το ψηφίο των μονάδων (και αντίστροφα)
\begin{itemize}
\item
{\en\tt convert(n,base,b)} μετατρέπει τον ακέραιο {\en\tt n} στην λίστα των 
των ψηφίων του ως προς την βάση {\en\tt b}  ξεκινώντας από το αρχίζοντας με το ψηφίο των μονάδων.\\ Για παράδειγμα,
{\en\tt convert(123,base,10)=[3,2,1]} και αντίστροφα  
\item
{\en\tt convert(l,base,b)} μετατρέπει την λίστα {\en\tt l} στον ακέραιο {\en\tt n} 
που έχει την {\en\tt l} σαν λίστα των ψηφίων του ως προς την βάση {\en\tt b} αρχίζοντας με το μοναδιαίο ψηφίο.\\ Για παράδειγμα, 
{\en\tt convert([3,2,1],base,10)=123} (βλέπε \ref{sec:convertbase}). 
\end{itemize}
\end{itemize}

\section{Μετασχηματισμός \tt\textlatin{Fourier}}
\subsection{Συντελεστές {\tt\textlatin{Fourier}} : {\tt\textlatin{fourier\_an}} και  {\tt\textlatin{fourier\_bn}} ή {\tt\textlatin{fourier\_cn}}} \index{integer}
Έστω $f$ μια $T$-περιοδική συνάρτηση στο
$\mathbb{R}$ η οποία είναι συνεχής εκτός ίσως από ένα πεπερασμένο αριθμό σημείων.
Μπορεί να αποδειχθεί ότι εάν η $f$ είναι συνεχής στο $x$, τότε :
\begin{eqnarray*}
f(x)&=&\frac{a_0}{2}+\sum _{n=1}^{+\infty} a_n \cos(\frac{2\pi
  nx}{T})+b_n \sin(\frac{2\pi nx}{T}) \\
 &=&\sum _{n=-\infty}^{+\infty} c_n e^{\frac{2i\pi nx}{T}}
\end{eqnarray*}
όπου οι συντελεστές $a_n,\ b_n$, $n\in N$, (ή $c_n$, $n \in Z$) είναι οι 
συντελεστές {\tt\textlatin{Fourier}} της $f$.
Η εντολή {\en\tt fourier\_an} και {\en\tt fourier\_bn} ή {\en\tt fourier\_cn} 
υπολογίζει αυτούς τους συντελεστές.   

\subsubsection{\en\tt fourier\_an}\index{fourier\_an}\label{sec:fourier_an}
\noindent{\en\tt fourier\_an} παίρνει 4 η 5 ορίσματα : μια παράσταση $expr$
που εξαρτάται από μια μεταβλητή, το όνομα της μεταβλητής (για παράδειγμα $x$), την
περίοδο $T$, έναν ακέραιο $n$ και έναν πραγματικό $a$ (ως προεπιλογή $a=0$).\\
{\en\tt fourier\_an(expr,x,T,n,a)} επιστρέφει τον συντελεστή {\en Fourier} $a_n$ μιας 
συνάρτησης $f$ 
μεταβλητής $x$ που ορίζεται στο $[a,a+T[$ από $f(x)=expr$ και τέτοια ώστε
$f$ να είναι περιοδική με περίοδο $T$:
$$\displaystyle a_n=\frac{2}{T}\int_a^{a+T}f(x)\cos(\frac{2\pi nx }{T})dx$$
Για να απλοποιήσουμε τους υπολογισμούς, μπορούμε να εισάγουμε {\en\tt assume(n,integer)} 
πριν  καλέσουμε την {\en\tt fourier\_an} για να βεβαιώσουμε 
ότι ο $n$ είναι ένας ακέραιος.\\
{\bf Παράδειγμα : } Έστω η συνάρτηση $f$, με περίοδο $T=2$, που ορίζεται στο $[-1,1[$ από την 
$f(x)=x^2$.\\
Για να υπολογίσουμε τον συντελεστή $a_0$ εισάγουμε:
\begin{center}{\en\tt fourier\_an(x\verb|^|2,x,2,0,-1)}\end{center}
Έξοδος :
\begin{center}{\en\tt 1/3}\end{center}
Για τον συντελεστή $a_n$ ($n\neq 0$) εισάγουμε:
 \begin{center}{\en\tt assume(n,integer);fourier\_an(x\verb|^|2,x,2,n,-1)}\end{center}
Έξοδος :
\begin{center}{\en\tt 4*(-1)\verb|^|n/(pi\verb|^|2*n\verb|^|2)}\end{center}

\subsubsection{\en\tt fourier\_bn}\index{fourier\_bn}\label{sec:fourier_bn}
\noindent{\en\tt fourier\_bn} παίρνει 4 ή 5 ορίσματα : μια παράσταση $expr$
που εξαρτάται από μια μεταβήτή, το όνομα της μεταβλητής (για παράδειγμα $x$), την  
περίοδο $T$, έναν ακέραιο $n$ και έναν πραγματικό $a$ (απο προεπιλογή $a=0$).\\
{\en\tt fourier\_bn(expr,x,T,n,a)} επιστρέφει τον συντελεστή {\en Fourier} $b_n$ μιας 
συνάρτησης $f$ μεταβλητής $x$ που ορίζεται στο $[a,a+T[$  από $f(x)=expr$ και περιοδική με περίοδο $T$:
$$\displaystyle b_n=\frac{2}{T}\int_a^{a+T}f(x)\sin(\frac{2\pi nx}{T})dx$$
Για να απλοποιήσουμε τους υπολογισμούς, μπορούμε να εισάγουμε {\en\tt assume(n,integer)} 
πριν καλέσουμε την {\en\tt fourier\_bn} για να βεβαιώσουμε 
ότι ο $n$ είναι ένας ακέραιος.\\
{\bf Παραδείγματα} 
\begin{itemize}
\item Έστω η συνάρτηση $f$, περιόδου $T=2$, ορισμένη στο $[-1,1[$ από την 
$f(x)=x^2$.\\
Για τον συντελεστή $b_n$ ($n\neq 0$) εισάγετε :
 \begin{center}{\en\tt  assume(n,integer);fourier\_bn(x\verb|^|2,x,2,n,-1)}\end{center}
Έξοδος :
\begin{center}{\en\tt  0}\end{center}

\item Έστω η συνάρτηση $f$, περιόδου $T=2$, ορισμένη στο $[-1,1[$ από την
$f(x)=x^3$.\\
Για τον συντελεστή  $b_1$ εισάγετε :
 \begin{center}{\en\tt fourier\_bn(x\verb|^|3,x,2,1,-1)}\end{center}
Έξοδος :
\begin{center}{\en\tt (2*pi\verb|^|2-12)/pi\verb|^|3}\end{center}
\end{itemize}

\subsubsection{\en\tt fourier\_cn}\index{fourier\_cn}\label{sec:fourier_cn}
\noindent{\en\tt fourier\_cn} παίρνει 4 ή 5 ορίσματα : μια παράσταση $expr$
που εξαρτάται από μια μεταβήτή, το όνομα της μεταβλητής (για παράδειγμα $x$), την 
περίοδο $T$, έναν ακέραιο $n$ και έναν πραγματικό $a$ (από προεπιλογή $a=0$).\\
{\en\tt fourier\_cn(expr,x,T,n,a)} επιστρέφει τον συντελεστή {\en Fourier} $c_n$ μιας συνάρτησης
$f$ μεταβλητής $x$ ορισμένης στο $[a,a+T[$ από την $f(x)=expr$ και περιοδική με περίοδο $T$:
$$\displaystyle c_n=\frac{1}{T}\int_a^{a+T}f(x)e^{\frac{-2i\pi nx}{T}}dx$$
Για να απλοποιήσουμε τους υπολογισμούς,  μπορούμε να εισάγουμε
την {\en\tt assume(n,integer)} πριν να καλέσουμε την {\en\tt fourier\_cn}
για να βεβαιώσουμε ότι ο $n$ είναι ένας ακέραιος.\\
{\bf Παραδείγματα}
\begin{itemize}
\item Βρείτε τους συντελεστές {\tt\textlatin{Fourier} $c_n$} της περιοδικής συνάρτησης $f$
περιόδου $2$ και ορισμένης στο $[-1,1[$ ως $ f(x)=x^2$.\\ 
Εισάγετε, για να πάρετε $c_0$ :
\begin{center}{\en\tt fourier\_cn(x\verb|^|2,x,2,0,-1)}\end{center}
Έξοδος:
\begin{center}{\en\tt 1/3}\end{center}
Εισάγετε, για να πάρετε $c_n$ :
\begin{center}{\en\tt assume(n,integer)}\end{center}
\begin{center}{\en\tt fourier\_cn(x\verb|^|2,x,2,n,-1)}\end{center}
Έξοδος:
\begin{center}{\en\tt 2*(-1)\verb|^|n/(pi\verb|^|2*n\verb|^|2)}\end{center}

\item Βρείτε τους συντελεστές {\tt\textlatin{Fourier} $c_n$} της περιοδικής συνάρτησης $f$, περιόδου 
$2$, και ορισμένης στο $[0,2[$ ως $ f(x)=x^2$.\\ 
Εισάγετε, για να πάρετε $c_0$ :
\begin{center}{\en\tt fourier\_cn(x\verb|^|2,x,2,0)}\end{center}
Έξοδος:
\begin{center}{\tt 4/3}\end{center}
Εισάγετε, για να πάρετε $c_n$ :
\begin{center}{\en\tt assume(n,integer)}\end{center}
\begin{center}{\en\tt fourier\_cn(x\verb|^|2,x,2,n)}\end{center}
Έξοδος:
\begin{center}{\en\tt ((2*i)*pi*n+2)/(pi\verb|^|2*n\verb|^|2)}\end{center}

\item Βρείτε τους συντελεστές {\tt\textlatin{Fourier}} $c_n$ της περιοδικής συνάρτησης $f$  
περιόδου $2.\pi$ και ορισμένης στο $[0,2.\pi[$ ως $ f(x)=x^2$.\\ 
Είσοδος \index{assume} :
\begin{center}{\en\tt assume(n,integer)}\end{center}
\begin{center}{\en\tt fourier\_cn(x\verb|^|2,x,2*pi,n)}\end{center}
Έξοδος :
\begin{center}{\en\tt ((2*i)*pi*n+2)/n\verb|^|2}\end{center}
Αν δεν βεβαιώσετε ότι $n$ είναι ακέραιος με την εντολή {\en\tt assume(n,integer)}, το αποτέλεσμα  δεν θα μπορεί 
να απλοποιηθεί :
\begin{center}{\en\tt ((2*i)*pi\verb|^|2*n\verb|^|2*exp((-i)*n*2*pi)+2*pi*n*exp((-i)*n*2*pi)+}\end{center}
\begin{center}{\en\tt (-i)*exp((-i)*n*2*pi)+i)/(pi*n\verb|^|3)}\end{center}
Θα μπορούσαμε να απλοποιήσουε αυτήν την παράσταση αντικαθιστώντας το
{\en\tt exp((-i)*n*2*pi)} με {\tt 1}, Εισαγωγή :
\begin{center}{\en\tt subst(ans(),exp((-i)*n*2*pi)=1)}\end{center}
Έξοδος :
\begin{center}{\en\tt ((2*i)*pi\verb|^|2*n\verb|^|2+2*pi*n+-i+i)/pi/n\verb|^|3}\end{center}
Αυτή η παράσταση απλοποιείται τότε με την {\en\tt normal}, και το τελικό αποτέλεσμα είναι :
\begin{center}{\en\tt ((2*i)*pi*n+2)/n\verb|^|2}\end{center}
Επομένως, για $n \neq 0$, $\displaystyle c_n=\frac{2in\pi+2}{n^2}$.
Όπως φαίνεται σε αυτό το παράδειγμα, ειναι προτιμότερο να εισάγουμε  {\en\tt
  assume(n,integer)} πριν καλέσουμε {\en\tt fourier\_cn}.\\
Πρέπει να υπολογίσουμε επίσης $c_n$ για $n=0$, εισάγουμε :
\begin{center}{\en\tt fourier\_cn(x\verb|^|2,x,2*pi,0)}\end{center}
Έξοδος :
\begin{center}{\en\tt  4*pi\verb|^|2/3}\end{center}
Επομένως, για  $n= 0$, $\displaystyle c_0=\frac{4.{\pi}^2}{3}$. 
\end{itemize}
{\bf Σχόλια }:
\begin{itemize}
\item Εισάγουμε {\en\tt purge(n)}\index{purge} για να άρουμε την υπόθεση που έγινε
  για το $n$.
\item
Εισάγουμε {\en\tt about(n)}\index{about} ή {\en\tt assume(n)}\index{assume}, για να μάθουμε ποια υπόθεση έγινε
για το $n$.
\end{itemize}

\subsection{Διακριτός Μετασχηματισμός \tt\textlatin{Fourier} }
Έστω $N$ ένας ακέραιος.
Ο Διακριτός Μετασχηματισμός {\tt\textlatin{Fourier (DFT)}} είναι ένας μετασχηματισμός $F_N$ που ορίζεται στο σύνολο των περιοδικών ακολουθιών περιόδου $N$. Εξαρτάται απ' την επιλογή
μιας αρχικής $N$-οστής ρίζας της μονάδας $\omega_N$. Εάν ο
{\tt\textlatin{DFT}} ορίζεται σε ακολουθίες με μιγαδικούς συντελεστές, έχουμε:
\[ \omega_N=e^{\frac{2 i \pi}{N}}\]
Εάν $x$ είναι μια περιοδική ακολουθία περιόδου
$N$, που ορίζεται από το διάνυσμα $x=[x_0,x_1,...x_{N-1}]$ τότε ο 
$F_N(x)=y$ είναι μια περιοδική ακολουθία περιόδου $N$, που ορίζεται από:
\[ {(F_{N,\omega_N}(x))}_k=y_k=\sum_{j=0}^{N-1}x_j\omega_N^{-k\cdot j}, k=0..N-1 \]
όπου $\omega_N$ είναι μια αρχική $N$-οστή ρίζα της μονάδας. 
Mε τον Γρήγορο Μετασχηματισμό {\tt\textlatin{Fourier (FFT)}} o διακριτός Μετασχηματισμός {\tt\textlatin{Fourier}} μπορεί να υπολογιστεί γρηγορότερα από τον υπολογισμό
του καθενός $y_k$ ατομικά.
Το {\en\tt Xcas} υλοποιεί τον αλγόριθμο {\tt\textlatin{FFT}} για να υπολογίσει
τον διακριτό μετασχηματισμό {\tt\textlatin{Fourier}} μόνο εάν το $N$ είναι δύναμη του 2.

\subsubsection{Ιδιότητες του διακριτού μετασχηματισμού {\tt\textlatin{Fourier}}}
Ο Διακριτός Μετασχηματισμός {\tt\textlatin{Fourier}} $F_N$ είναι ένας αμφιμονοσήμαντος και επί μετασχηματισμός 
σε περιοδικές ακολουθίες τέτοιες ώστε : 
\begin{eqnarray*}
 F_{N,\omega_N}^{-1}&=&\frac{1}{N} F_{N,\omega_N^{-1}} \\
&=&\frac{1}{N} \overline{F_{N}} \quad \mbox{ στο } \mathbb C
\end{eqnarray*}
δηλαδή :
\[ {(F_N^{-1}(x))}_k=\frac{1}{N}\sum_{j=0}^{N-1}x_j\omega_N^{k\cdot j} \]
Στο {\en\tt Xcas} ο Διακριτός Μετασχηματισμός {\tt\textlatin{Fourier}} και ο αντίστροφός του
δηλώνονται με {\en\tt fft} και  {\en\tt ifft}:
\begin{center}
{\en\tt fft(x)}=$\displaystyle F_N(x)$, \ {\en\tt ifft(x)}=$\displaystyle F_N^{-1}(x)$
\end{center}
{\bf Ορισμοί}\\
Έστω $x$ και  $y$ δύο περιοδικές ακολουθίες με περίοδο $N$.
\begin{itemize}
\item Το γινόμενο {\tt\textlatin{Hadamard}}  (συμβολισμός $\cdot$) ορίζεται με:
\[ {(x \cdot y)}_k = x_k y_k \]
\item το γινόμενο συνέλιξης (συμβολισμός $*$) ορίζεται με:
\[ {(x * y)}_k=\sum_{j=0}^{N-1}x_jy_{k-j} \]
\end{itemize}
{\bf Ιδιότητες} :
\begin{eqnarray*}
N*F_N(x \cdot y)&=&F_N(x) * F_N(y)\\
F_N(x * y)&=&F_N(x) \cdot F_N(y)
\end{eqnarray*}

\subsubsection{Εφαρμογές}
\begin{enumerate}
\item Τιμή ενός πολυωνύμου\\ 
Ορίζουμε ένα πολυώνυμο $P(x)=\sum_{j=0}^{N-1}c_jx^j$ με το διάνυσμα
των συντελεστών του $c:=[c_0,c_1,..c_{N-1}]$, όπου μηδενικά μπορούν να προστεθούν έτσι ώστε το
$N$ να είναι δύναμη του 2.
\begin{itemize}
\item Υπολογισμός των τιμών του $P(x)$ στα σημεία
\[ x=a_k=\omega_N^{-k}=\exp(\frac{-2ik\pi}{N}), \quad k=0..N-1 \]
Αυτός είναι απλά ο διακριτός μετασηματισμός {\tt\textlatin{Fourier}} του $c$ αφού,
\[ P(a_k)=\sum_{j=0}^{N-1}c_j(\omega_N^{-k})^j=F_N(c)_k \]
Εισάγετε, για παράδειγμα :
\begin{center}
{\en\tt P(x):=x+x\verb|^|2;  w:=i}
\end{center}
Εδώ οι συντελεστές του $P$ είναι [0,1,1,0],  
$N=4$ και $\omega=\exp(2i\pi/4)=i$.\\
Είσοδος :\\
{\en\tt fft([0,1,1,0])}\\
Έξοδος :\\
{\en\tt [2,-1-i,0,-1+i]}\\
επομένως,
\begin{itemize}
\item {\en\tt P(0)=2},
\item {\en\tt P(-i)=P(w\verb|^|-1)=-1-i},
\item {\en\tt P(-1)=P(w\verb|^|-2)=0},
\item {\en\tt P(i)=P(w\verb|^|-3)=-1+i}.
\end{itemize}

\item Υπολογισμός των τιμών του $P(x)$ στα
\[ x=b_k=\omega_N^{k}=\exp(\frac{2ik\pi}{N}), \quad k=0..N-1 \]
Αυτό είναι $N$ επί τον αντίστροφο μετασχηματισμό {\tt\textlatin{Fourier}}  του $c$ αφού,
\[ P(a_k)=\sum_{j=0}^{N-1}c_j(\omega_N^{k})^j=NF_N^{-1}(c)_k \]
Εισάγετε, για παράδειγμα :\\
{\en\tt P(x):=x+x\verb|^|2} και {\en\tt w:=i}\\
Εδώ, οι συντελεστές του $P$ είναι [0,1,1,0], 
$N=4$ και $\omega=\exp(2i\pi/4)=i$.\\
Είσοδος  :\\
{\en\tt 4*ifft([0,1,1,0])}\\
Έξοδος :\\
{\en\tt [2,-1+i,0,-1-i]}\\
επομένως : \begin{itemize}
\item {\en\tt P(0)=2},
\item {\en\tt  P(i)=P(w\verb|^|1)=-1+i},
\item {\en\tt  P(-1)=P(w\verb|^|2)=0},
\item {\en\tt  P(-i)=P(w\verb|^|3)=-1-i}.
\end{itemize}
Ασφαλώς, βρίσκουμε τις ίδιες τιμές όπως και παραπάνω...
\end{itemize}

\item Τριγωνομετρική παρεμβολή\\
Έστω $f$ μια περιοδική συνάρτηση περιόδου $2\pi$, και έστω ότι $f(2k\pi/N)=f_k$ 
for $k=0..(N-1)$. Βρείτε ένα τριγωνομετρικό πολυώνυμο $p$ που παρεμβάλει την $f$ 
στα $x_k=2k\pi/N$, δηλαδή βρείτε $p_j, j=0..N-1$ τέτοια ώστε
\[  p(x)= \sum_{j=0}^{N-1} p_j \exp(ijx), \quad p(x_k)=f_k\]
Αντικαθιστώντας στο $p(x)$ το $x_k$ με την τιμή του έχουμε:
\[ \sum_{j=0}^{N-1} p_j \exp(i\frac{j2k\pi}{N}) = f_k\]
Μ' άλλα λόγια,  $(f_k)$ είναι ο αντίστροφος μετασχηματισμός {\tt\textlatin{Fourier DFT}} του $(p_k)$, και επομένως
\[ (p_k)= \frac{1}{N} F_N( \ (f_k) \ ) \]
Εάν η συνάστηση $f$ είναι πραγματική, $p_{-k}=\overline p_k$, και επομένως ανάλογα αν το $N$ είναι  άρτιο ή περιττό:
\begin{eqnarray*}
p(x)&=&p_0+
2 \Re(\sum_{k=0}^{\frac{N}{2}-1}p_k\exp(ikx))+\Re(p_{\frac{N}{2}} \exp(i\frac{Nx}{2})) \\
p(x)&=&p_0+ 2 \Re(\sum_{k=0}^{\frac{N-1}{2}}p_k\exp(ikx))
\end{eqnarray*}

\item  Σειρές {\tt\textlatin{Fourier}}\\
Έστω $f$ μια περιοδική συνάρτηση περιόδου $2\pi$, τέτοια ώστε :
\[ f(x_k)=y_k, \quad x_k=\frac{2k\pi}{N}, k=0..N-1 \]
Έστω επίσης ότι η σειρά {\tt\textlatin{Fourier}} της $f$ συγκλίνει στην $f$ (αυτό θα ισχύει εάν για παράδειγμα $f$ είναι συνεχής). Εάν το $N$ είναι μεγάλο,
μια καλή προσέγγιση της $f$ θα δίνεται από:
\[ \sum_{-\frac{N}{2} \leq n<\frac{N}{2}} c_n \exp(inx) \]
Επομένως θέλουμε μια αριθμητική προσέγγιση των 
\[ c_n=\frac{1}{2\pi} \int_0^{2\pi}f(t)\exp(-int)dt \]
Η αριθμητική τιμή του ολοκληρώματος $\int_0^{2\pi}f(t)\exp(-int)dt$ μπορεί να υπολογιστεί από τον τραπεζοειδή κανόνα (σημειώνουμε ότι ο αλγόριθμος {\tt\textlatin{Romberg}} δεν θα δούλευε εδώ, επειδή το ανάπτυγμα {\tt\textlatin{Euler Mac Laurin}}
 έχει μηδενικούς συντελεστές, αφού η ολοκληρωτέα συνάρτηση είναι περιοδική, και επομένως όλες της οι παράγωγοι έχουν την ίδια τιμή στο $0$ και στο $2\pi$).
Εάν $\tilde{c_n}$ είναι η αριθμητική τιμή της $c_n$ που παίρνουμε από 
τον τραπεζοειδή κανόνα, τότε 
\[
\tilde{c_n}=\frac{1}{2\pi}\frac{2\pi}{N}\sum_{k=0}^{N-1}y_k\exp(-2i\frac{nk\pi}{N}),
\quad  -\frac{N}{2} \leq n<\frac{N}{2} \]
Πράγματι, αφού $x_k=2k\pi/N$ και  $f(x_k)=y_k$:
\begin{eqnarray*} 
f(x_k)\exp(-inx_k)&=&y_k\exp(-2i\frac{nk\pi}{N}), \\
f(0)\exp(0)=f(2\pi)\exp(-2i\frac{nN\pi}{N})&=&y_0=y_N 
\end{eqnarray*}
Επομένως :
\[
[\tilde{c}_0,..\tilde{c}_{\frac{N}{2}-1},\tilde{c}_{\frac{-N}{2}},..c_{-1}]=
\frac{1}{N}F_N([y_0,y_1...y_{(N-1)}]) \]
<αφού
\begin{itemize}
\item εάν $n\geq0$, $\tilde{c}_n=y_n$ 
\item εάν $n<0$ $\tilde{c}_n=y_{n+N}$
\item $\omega_N=\exp(\frac{2i\pi}{N})$, 
τότε $\omega_N^n=\omega_N^{n+N}$ 
\end{itemize}

{\bf Ιδιότητες}
\begin{itemize}
\item Οι συντελεστές του τριγωνομετρικού πολυωνύμου  που παρεμβάλει την $f$ 
στα $x=2k\pi/N$ είναι
\[ p_n=\tilde{c}_n, \quad -\frac{N}{2} \leq n<\frac{N}{2} \]
\item
Εάν η $f$ είναι ένα τριγωνομετρικό πολυώνυμο $P$ βαθμού $m\leq \frac{N}{2}$, 
τότε
\[ f(t)=P(t)=\sum_{k=-m}^{m-1}c_k\exp(2ik\pi t) \]
το τριγωνομετρικό πολυώνυμο που παρεμβάλει την $f=P$ είναι το $P$, η αριθμητική  
προσέγγιση των συντελεστών είναι στην πραγματικότητα ακριβής ($\tilde{c}_n=c_n$).
\item Πιο γενικά, μπορούμε να υπολογίσουμε το $\tilde{c}_n-c_n$.\\
Υποθέτοντας ότι η $f$ ισούται με την σειρά {\en Fourier} της, δηλαδή ότι :\\
\[ f(t)=\sum_{m=-\infty}^{+\infty}c_m\exp(2i\pi mt), \quad
\sum_{m=-\infty}^{+\infty}|c_m|<\infty \]
Τότε :
\[ f(x_k)=f(\frac{2k\pi}{N})=y_k=\sum_{m=-\infty}^{+\infty}c_m\omega_N^{km},
\quad
\tilde{c_n}=\frac{1}{N}\sum_{k=0}^{N-1}y_k\omega_N^{-kn} \]
Αντικαταστείστε το $y_k$ με την τιμή του στο $\tilde{c_n}$:
\[
\tilde{c_n}=\frac{1}{N}\sum_{k=0}^{N-1}\sum_{m=-\infty}^{+\infty}
c_m\omega_N^{km}\omega_N^{-kn} \]
Εάν $m\neq n \pmod N$, $\omega_N^{m-n}$ είναι μια $N$-στη ρίζα της μονάδος διάφορη
του 1, και επομένως:
\[ \omega_N^{(m-n)N}=1, \quad \sum_{k=0}^{N-1}\omega_N^{(m-n)k}=0 \]
Άρα, εάν $m-n$ είναι ένα πολλαπλάσιο του $N$ ($m=n+l\cdot N$) τότε 
$\sum_{k=0}^{N-1}\omega_N^{k(m-n)}=N$, διαφορετικά
$\sum_{k=0}^{N-1}\omega_N^{k(m-n)}=0$.
Αντιστρέφοντας τα δυο αθροίσματα, έχουμε
\begin{eqnarray*}
\tilde{c_n}&=&\frac{1}{N}\sum_{m=-\infty}^{+\infty}c_m\sum_{k=0}^{N-1}\omega_N^{k(m-n)} \\
&=&\sum_{l=-\infty}^{+\infty}c_{(n+l\cdot N)} \\
&=&...c_{n-2\cdot N}+c_{n-N}+c_{n}+c_{n+N}+c_{n+2\cdot
  N}+.....
\end{eqnarray*}
Συμπέρασμα: εάν $|n|<N/2$, η διαφορά $\tilde{c_n}-c_n$ είναι ένα άθροισμα του $c_j$ μεγάλων  δεικτών
(τουλάχιστον  $N/2$ σε απόλυτη τιμή), και θα είναι μικρή (εξαρτάται από 
την τάξη σύγκλισης των σειρών {\en Fourier)}.
\end{itemize}
{\bf Παράδειγμα}, εισάγετε
\begin{center}
{\en\tt f(t):=cos(t)+cos(2*t)}\\
{\en\tt x:=f(2*k*pi/8)\$(k=0..7)}
\end{center}
Έξοδος :
\begin{center}
 ${\en\tt x=\{2,(\sqrt
  2)/2,-1,-((\sqrt2)/2),0,-((\sqrt2)/2),-1,(\sqrt2)/2\}}$ \\
{\en\tt fft(x)=[0.0,4.0,4.0,0.0,0.0,0.0,4.0,4.0]}
\end{center}
Μετά από διαίρεση με $N=8$, έχουμε :
\begin{center} $c_0=0,c_1=4.0/8,c_2=4.0/2,c_3=0.0$,\\
$c_{-4}=0,c_{-3}=0,c_{-2}=4.0/8,=c_{-1}=4.0/8$
\end{center}
Επομένως, $b_k=0$ και  $a_k=c_{-k}+c_k$ ισούται με 1 εάν $k=1, 2$ και 0 σε κάθε άλλη περίπτωση. 

\item Γινόμενο Συνέλιξης\\
Εάν $P(x)=\sum_{j=0}^{n-1}a_jx^j$  
και $Q(x)=\sum_{j=0}^{m-1}b_jx^j$ 
δίνονται από τα διανύσματα των συντελεστών τους 
$a=[a_0,a_1,..a_{n-1}]$ and  $b=[b_0,b_1,..b_{m-1}]$, υπολογίζουμε
το γινόμενο αυτών των δύο πολυωνύμων χρησιμοποιώντας τον {\tt\textlatin{DFT}}.
Το γινόμενο των δύο πολυωνύμων είναι το γινόμενο συνέλιξης 
της περιοδικής σειράς των συντελεστών τους 
εάν η περίοδος είναι μεγαλύτερη ή ίση από  
$(n+m)$. Επομένως συμπληρώνουμε το $a$ (αντίστοιχα το $b$) με $m+p$ 
(αντίστοιχα $n+p$) μηδενικά, όπου
$p$ επιλέγεται τέτοιο ώστε το $N=n+m+p$ να είναι δύναμη του 2.
Εάν $a=[a_0,a_1,..a_{n-1},0..0]$ και  $b=[b_0,b_1,..b_{m-1},0..0]$, τότε:
\[ P(x)Q(x)=\sum_{j=0}^{n+m-1}(a*b)_jx^j \]
Υπολογίζουμε τα $F_N(a)$, $F_N(b)$, και μετά $ab=F_N^{-1}(F_N(a)\cdot F_N(b))$
χρησιμοποιώντας τις ιδιότητες
\[ NF_N(x \cdot y)=F_N(x) * F_N(y), \quad
F_N(x * y)=F_N(x) \cdot F_N(y) \]
\end{enumerate}

\subsection{Γρήγορος Μετασχηματισμός \textlatin{Fourier}  : {\tt   \textlatin{fft}}}\index{fft}
\noindent{\en\tt fft} παίρνει ως όρισμα μια λίστα (ή μια ακολουθία)
${\tt [a_0,..a_{N-1}]}$ όπου {\tt N} είναι μια δύναμη του 2.\\
{\en\tt fft} επιστρέφει τη λίστα ${\tt [b_0,..b_{N-1}]}$ τέτοια ώστε, 
για {\tt k=0..N-1} 
\[ {\tt 
  {fft([a_0,..a_{N-1}])}[k]=b_k=\sum_{j=0}^{N-1}x_j\omega_N^{-k\cdot
    j}} \]
όπου $\omega_N$ είναι μια αρχική $N$-στη ρίζα της μονάδος.\\
Είσοδος :
\begin{center}{\en\tt fft(0,1,1,0)}\end{center}
Έξοδος :
\begin{center}{\tt [2.0, -1-i, 0.0, -1+i]}\end{center}

\subsection{Αντίστροφος Γρήγορος Μετασχηματισμός \textlatin{Fourier} :{\tt   \textlatin{\tt ifft}}}\index{ifft}
\noindent{\en\tt ifft} παίρνει ως όρισμα μια λίστα ${\tt [b_0,..b_{N-1}]}$ όπου
{\tt N} είναι δύναμη του δύο.\\
{\en\tt ifft} επιστρέφει τη λίστα ${\tt [a_0,..a_{N-1}]}$ τέτοια ώστε
\[ {\tt fft([a_0,..a_{N-1}])=[b_0,..b_{N-1}]} \]
Είσοδος :
\begin{center}{\en\tt ifft([2,-1-i,0,-1+i])}\end{center}
Έξοδος :
\begin{center}{\tt [0.0, 1.0, 1.0, 0.0]}\end{center}

\subsection{Μια  άσκηση με {\tt   \textlatin{fft}}}
Στον ακόλουθο πίνακα είναι οι θερμοκρασίες $T$, σε βαθμούς {\en Celsius}, την χρονική στιγμή $t$ :
\begin{center}
\begin{tabular}{|r|rrrrrrrr|}
\hline
$t$ & 0 & 3 & 6 & 9 &12 & 15 & 19 & 21\\
\hline
$T$ & 11 & 10 & 17 & 24 & 32 & 26 & 23 & 19\\
\hline
\end{tabular}
\end{center}
Ποια ήταν η θερμοκρασία στις 13:45 ?

Εδώ $N=8=2*m$. Το πολυώνυμο παρεμβολής είναι :
\[ p(t)=\frac{1}{2} p_{-m}(\exp(-2i\frac{\pi mt}{24})+
\exp(2i\frac{\pi mt}{24}))+
\sum_{k=-m+1}^{m-1}p_k \exp(2i\frac{\pi kt}{24}) \]
και
\[ p_k=\frac{1}{N} \sum_{k=j}^{N-1}T_k \exp(2i\frac{\pi k}{N}) \]
Είσοδος :\\
{\en\tt q:=1/8*fft([11,10,17,24,32,26,23,19])}\\
Έξοδος :\\
{\en\tt q:=[20.25,-4.48115530061+1.72227182413*i,-0.375+0.875*i,\\
-0.768844699385+0.222271824132*i,0.5,\\
-0.768844699385-0.222271824132*i,\\
-0.375-0.875*i,-4.48115530061-1.72227182413*i]}\\
Επομένως, :
\begin{itemize}
\item $p_0=20.25$
\item $p_1=-4.48115530061+1.72227182413*i=\overline{p_{-1}}$,
\item $p_2=0.375+0.875*i=\overline{p_{-2}}$,
\item $p_3=-0.768844699385+0.222271824132*i=\overline{p_{-3}}$,
\item $p_{-4}=0.5$
\end{itemize}
Πράγματι,
\[
q=[q_0,...q_{N-1}]=[p_0,..p_{\frac{N}{2}-1},p_{-\frac{N}{2}},..,p_{-1}]=\frac{1}{N}F_N([y_0,..y_{N-1}])={\tt
  \frac{1}{N}fft(y)} \]
Είσοδος :\\
{\en\tt pp:=[q[4],q[5],q[6],q[7],q[0],q[1],q[2],q[3]]}\\
Εδώ, $p_k=pp[k+4]$ για $k=-4...3$.
Απομένει να υπολογίσουμε την τιμή του πολυωνύμου παρεμβολής στο σημείο
$t0=13,75=55/4$, εισάγουμε\\
{\en\tt t0(j):=exp(2*i*pi*(13+3/4)/24*j)}\\
{\en\tt T0:=1/2*pp[0]*(t0(4)+t0(-4))+sum(pp[j+4]*t0(j),j,-3,3)}\\
{\en\tt evalf(re(T0))}\\
Έξοδος :\\
{\tt 29.4863181684}\\
Η θερμοκρασία υπολογίζεται ίση με 29.49 βαθμούς {\en Celsius}.\\
Είσοδος :\\
{\en\tt q1:=[q[4]/2,q[3],q[2],q[1],q[0]/2]}\\
{\en\tt a:=t0(1)} ({\gr ή} {\en\tt a:=-exp(i*pi*7/48)})\\
{\en\tt g(x):=r2e(q1,x)}\\
{\en\tt evalf(2*re(g(a)))}\\
ή\\
{\en\tt 2.0*re(q[0]/2+q[1]*t0(1)+q[2]*t0(2)+q[3]*t0(3)+q[4]/2*t0(4))}\\
Έξοδος :\\
{\tt 29.4863181684}\\

{\bf Σχόλιο}\\
Χρησιμοποιώντας το πολυώνυμο παρεμβολής {\en Lagrange} (το πολυώνυμο δεν είναι περιοδικό),
εισάγουμε :\\
{\en\tt l1:=[0,3,6,9,12,15,18,21]}\\
{\en\tt l2:=[11,10,17,24,32,26,23,19]}\\
{\en\tt subst(lagrange(l1,l2,13+3/4),x=13+3/4)}\\
Έξοδος :\\
${\tt \displaystyle \frac{8632428959}{286654464}\simeq 30.1144061688}$

\section{Εκθετικά και Λογάριθμοι}
\subsection{Αναγραφή υπερβολικών συναρτήσεων ως εκθετικά :\\ {\tt \textlatin{hyp2exp}}}\index{hyp2exp}
\noindent{\en\tt hyp2exp} παίρνει ως όρισμα μια υπερβολική παράσταση.\\
{\en\tt hyp2exp} αναγράφει κάθε υπερβολική συνάρτηση με εκθετικά
(σαν εάν ρητό κλάσμα ενός εκθετικού,
δηλαδή {\sc χωρίς} γραμμικοποίηση).\\ 
Είσοδος :
\begin{center}{\en\tt hyp2exp(sinh(x))}\end{center}
Έξοδος :
\begin{center}{\en\tt (exp(x)-1/(exp(x)))/2}\end{center}

\subsection{Ανάπτυγμα εκθετικών : {\tt \textlatin{expexpand}}}\index{expexpand}
\noindent{\en\tt expexpand} παίρνει ως όρισμα μια παράσταση με εκθετικά.\\
{\en\tt expexpand} αναπτύσει αυτήν την παράσταση (αναγράφει το {\en exp} αθροισμάτων
σαν γινόμενο  {\en exp}).\\
Είσοδος :
\begin{center}{\en\tt expexpand(exp(3*x)+exp(2*x+2))}\end{center}
Έξοδος :
\begin{center}{\en\tt exp(x)\verb|^|3+exp(x)\verb|^|2*exp(2)}\end{center}

\subsection{Ανάπτυγμα λογαρίθμων : {\tt \textlatin{lnexpand}}}\index{lnexpand}
\noindent{\en\tt lnexpand} παίρνει ως όρισμα μια παράσταση με λογαρίθμους.\\
{\en\tt lnexpand} αναπτύσει αυτήν την παράσταση (αναγράφει το {\en ln} γινομένων
σαν άθροισμα  {\en ln}).\\
Είσοδος :
\begin{center}{\en \tt lnexpand(ln(3*x\verb|^|2)+ln(2*x+2))}\end{center}
Έξοδος :
\begin{center}{\en \tt  ln(3)+2*ln(x)+ln(2)+ln(x+1)}\end{center}

\subsection{Γραμμικοποίηση  εκθετικών : {\tt \textlatin{lin}}}\index{lin}
\noindent{\en\tt lin} παίρνει ως όρισμα μια παράσταση με
εκθετικά.\\
{\en \tt lin} αναγράφει υπερβολικές συναρτήσεις σαν εκθετικά εάν απαιτείται,
έπειτα γραμμικοποιεί αυτήν την παράσταση (δηλαδή αντικαθιστά γινόμενο
εκεθτικών με εκθετικό αθροισμάτων).\\
{\bf Παραδείγματα}
\begin{itemize}
\item Είσοδος :
\begin{center}{\en\tt lin(sinh(x)\verb|^|2)}\end{center}
Έξοδος :
\begin{center}{\en\tt 1/4*exp(2*x)+1/-2+1/4*exp(-(2*x))}\end{center}

\item Είσοδος :
\begin{center}{\en\tt lin((exp(x)+1)\verb|^|3)}\end{center}
Έξοδος :
\begin{center}{\en\tt exp(3*x)+3*exp(2*x)+3*exp(x)+1}\end{center}
\end{itemize}

\subsection{Συλλογή λογαρίθμων : {\tt\textlatin{ lncollect}}}\index{lncollect}
\noindent{{\en\tt lncollect} παίρνει ως όρισμα μια παράσταση με λογάριθμους.}\\
{\en\tt lncollect} συλλέγει τους λογαρίθμους (αναγράφει αθροίσματα λογαρίθμων ως λογάριθμο γινομένων). Είναι καλά να παραγοντοποιήσετε την παράσταση με {\en \tt factor} πριν την συλλογή με {\en\tt lncollect}. \\
Είσοδος :
\begin{center}{\en\tt lncollect(ln(x+1)+ln(x-1))}\end{center}
Έξοδος :
\begin{center}{\en\tt log((x+1)*(x-1))}\end{center}
Είσοδος :
\begin{center}{\en\tt lncollect(exp(ln(x+1)+ln(x-1)))}\end{center}
Έξοδος :
\begin{center}{\en\tt (x+1)*(x-1)}\end{center}
{\bf Προσοχή!!!}  Για το {\en\tt Xcas}, το {\en\tt log=ln} (Χρησιμοποιείστε {\en\tt log10}
για  λογάριθμο με βάση το 10).

\subsection{Ανάπτυγμα δυνάμεων : {\tt\textlatin{ powexpand}}}\index{powexpand}
\noindent{{\en\tt powexpand} αναγράφει μια δύναμη σε άθροισμα σαν ένα γινόμενο δυνάμεων.}\\
Είσοδος :
\begin{center}{\en\tt powexpand(a\verb|^|(x+y))}\end{center}
Έξοδος :
\begin{center}{\en\tt a\verb|^|x*a\verb|^|y}\end{center}


\subsection{Αναγραφή δύναμης σε εκθετικό : {\tt\textlatin{ pow2exp}}}\index{pow2exp}
\noindent{{\en\tt pow2exp} αναγράφει μια δύναμη σαν εκθετικό.}\\
Είσοδος :
\begin{center}{\en\tt  pow2exp(a\verb|^|(x+y))}\end{center}
Έξοδος :
\begin{center}{\en\tt exp((x+y)*ln(a))}\end{center}

\subsection{Αναγραφή του {\tt\textlatin {exp(n*ln(x))}} σε δύναμη : {\tt\textlatin {exp2pow}}}\index{exp2pow}
\noindent{{\en\tt exp2pow} αναγράφει παράσταση της μορφής $\exp(n*\ln(x))$ σε μια δύναμη του $x$.}\\
Είσοδος :
\begin{center}{\en\tt exp2pow(exp(n*ln(x)))}\end{center}
Έξοδος :
\begin{center}{\en\tt x\verb|^|n}\end{center}
Σημειώστε τη διαφορά με την {\en\tt lncollect} :\\
{\en\tt lncollect(exp(n*ln(x))) = exp(n*log(x))}\\
{\en\tt lncollect(exp(2*ln(x))) = exp(2*log(x))}\\
{\en\tt exp2pow(exp(2*ln(x))) = x\verb|^|2 }\\
Αλλά :\\
{\en\tt lncollect(exp(ln(x)+ln(x))) = x\verb|^|2}\\
{\en\tt exp2pow(exp(ln(x)+ln(x))) = x\verb|^|(1+1)}\\

\subsection{Απλοποίηση μιγαδικών εκθετικών : {\tt\textlatin{ tsimplify}}}\index{tsimplify}
\noindent{{\en\tt tsimplify} απλοποιεί υπερβατικές παραστάσεις 
αναγράφοντας την παράσταση με μιγαδικά εκθετικά.\\
Είναι καλά να δοκιμάζουμε άλλες τεχνικές απλοποίησης
και να καλέσουμε την {\en\tt tsimplify} αν οι άλλες δεν δουλέυουν.}\\ 
Είσοδος :
\begin{center}{\en\tt tsimplify((sin(7*x)+sin(3*x))/sin(5*x))}\end{center}
Έξοδος :
\begin{center}{\en\tt ((exp((i)*x))\verb|^|4+1)/(exp((i)*x))\verb|^|2 }\end{center}


\section{Πολυώνυμα}
Ένα πολυώνυμο μιας μεταβλητής αναπαριστάται είτε 
από μια συμβολική παράσταση είτε από την λίστα των συντελεστών
του διατεταγμένων σε φθίνουσα  τάξη των δυνάμεων (πυκνή αναπαράσταση).
Στην τελευταία περίπτωση, για να αποφύγουμε τη σύγχυση με άλλου είδους λίστες
\begin{itemize}
\item χρησιμοιείστε {\en\tt \verb|poly1[…]|} για οριοθέτες στην είσοδο
\item προσέξτε τα  $\talloblong \ \talloblong$ στη έξοδο του {\en\tt Xcas}.
\end{itemize}
Σημειώστε ότι πολυώνυμα που αναπαριστώνται ως λίστες συντελεστών
γράφονται πάντοτε σε φθίνουσα τάξη δυνάμεων ακόμα κι αν στις Ρυθμίσεις του {\en\tt Cas} έχει τσεκαρισθεί η {\en\tt "}αύξουσα δύναμη{\en\tt "}.

Ένα πολυώνυμο πολλών μεταβλητών αναπαριστάται :
\begin{itemize}
\item είτε ως μια συμβολική παράσταση
\item είτε ως μια πυκνή αναδρομική μονοδιάστατη αναπαράσταση όπως παραπάνω
\item είτε ως ένα άθροισμα
μονωνύμων με μη μηδενικούς συντελεστές (κατανεμημένη αραιή 
αναπαράσταση).\\ \\
Ένα μονώνυμο με πολλές μεταβλητές αναπαριστάται από έναν συντελεστή και μια λίστα
 ακεραίων (που αντιστοιχούν στις δυνάμεις της λίστας μεταβλητών). Οι οριοθέτες 
για τα μονώνυμα είναι
{\tt \%\%\%\{} και {\tt \%\%\%\}}, για παράδειγμα το $3x^2y$ αναπαριστάται ως 
{\tt \%\%\%\{3,[2,1]\%\%\%\}} κατ' αντιστοιχία με τη λίστα μεταβλητών {\en\tt [x,y]}).
\end{itemize} 


\subsection{Μετατροπή σε  συμβολικό πολυώνυμο :{\tt\textlatin{r2e poly2symb}}}\index{r2e}\index{poly2symb}
\noindent{{\en\tt r2e}  ή {\en\tt poly2symb} παίρνει ως όρισμα} 
\begin{itemize}
\item μια λίστα από
συντελεστές ενός πολυωνύμου (με φθίνουσα σειρά) και ένα συμβολικό
όνομα μεταβλητής 
(εξ' ορισμού {\en\tt x})
\item ή ένα άθροισμα μονωνύμων {\en\tt \%\%\%\{coeff,[n1,....nk] \%\%\%\}} 
και ένα διάνυσμα συμβολικών μεταβλητών {\en\tt [x1,...,xk]}).
\end{itemize}
{\en\tt r2e} ή {\en\tt poly2symb} μετασχηματίζει το όρισμα σε ένα συμβολικό
πολυώνυμο.\\
Παράδειγμα πολυωνύμων μιας μεταβλητής, εισάγετε :
\begin{center}{\en\tt r2e([1,0,-1],x)}\end{center}
ή :
\begin{center}{\en\tt r2e([1,0,-1])}\end{center}
ή :
\begin{center}{\en\tt poly2symb([1,0,-1],x)}\end{center}
Έξοδος :
\begin{center}{\en\tt  x*x-1}\end{center}
Παράδειγμα με αραιά πολυώνυμα πολλαπλών μεταβλητών, εισάγετε:
\begin{center}{\en\tt poly2symb(\%\%\%\{1,[2]\%\%\%\}+\%\%\%\{-1,[0]\%\%\%\},[x])}\end{center}
ή :
\begin{center}{\en\tt r2e(\%\%\%\{1,[2]\%\%\%\}+\%\%\%\{-1,[0]\%\%\%\},[x])}\end{center}
Έξοδος :
\begin{center}{\en\tt  x\verb|^2|-1}\end{center}
Είσοδος :
\begin{center}{\en\tt r2e(\%\%\%\{1,[2,0]\%\%\%\}+\%\%\%\{-1,[1,1]\%\%\%\}+\%\%\%\{2,[0,1]\%\%\%\},[x,y])}\end{center}
ή :
\begin{center}{\en\tt poly2symb(\%\%\%\{1,[2,0]\%\%\%\}+\%\%\%\{-1,[1,1]\%\%\%\}+\%\%\%\{2,[0,1]\%\%\%\},[x,y])}\end{center}
Έξοδος :
\begin{center}{\en\tt  x\verb|^|2-x*y+2*y}\end{center}

\subsection{Μετατροπή από  συμβολικό πολυώνυμο :{\tt\textlatin{e2r symb2poly}}}\index{e2r}\index{symb2poly}
\noindent{{\en\tt e2r} ή {\en\tt symb2poly} παίρνει ως όρισμα ένα συμβολικό πολυώνυμο και
είτε το όνομα μιας συμβολικής μεταβλητής (εξ' ορισμού {\en\tt x}) ή
μια λίστα ονομάτων συμβολικών μεταβλητών.}\\
{\en\tt e2r} ή {\en\tt symb2poly} μετασχηματίζει το πολυώνυμο σε μια λίστα
(πυκνή αναπαράσταση πολυωνύμων μιας μεταβλητής, όπου οι συντελεστές είναι γραμμένοι
σε φθίνουσα σειρά) ή σε ένα άθροισμα μονωνύμων (αραιή 
αναπαράσταση πολυωνύμων με πολλαπλές μεταβλητές).\\
Είσοδος :
\begin{center}{\en\tt e2r(x\verb|^|2-1)}\end{center}
ή :
\begin{center}{\en\tt symb2poly(x\verb|^|2-1)}\end{center}
ή :
\begin{center}{\en\tt symb2poly(y\verb|^|2-1,y)}\end{center}
ή :
\begin{center}{\en\tt e2r(y\verb|^|2-1,y)}\end{center}
Έξοδος :
\begin{center}{\en\tt $\talloblong$1,0,-1$\talloblong$}\end{center}
Είσοδος :
\begin{center}{\en\tt e2r(x\verb|^|2-x*y+y, [x,y])}\end{center}
ή :
\begin{center}{\en\tt symb2poly(x\verb|^|2-x*y+2*y, [x,y])}\end{center}
Έξοδος :
\begin{center}{\en\tt \%\%\%\{1,[2,0]\%\%\%\}+\%\%\%\{-1,[1,1]\%\%\%\}+\%\%\%\{2,[0,1]\%\%\%\}}\end{center}

\subsection{Συντελεστές πολυωνύμου: {\tt\textlatin{ coeff coeffs}}}\index{coeff}\index{coeffs}
\noindent{{\en\tt coeff} ή {\en\tt coeffs} παίρνει τρία ορίσματα : το πολυώνυμο, 
το όνομα της μεταβλητής (ή την λίστα των ονομάτων των μεταβλήτών) και
τον βαθμό (ή την λίστα των βαθμών των μεταβλητών).}\\
{\en\tt coeff} ή {\en\tt coeffs} επιστρέφει τον συντελεστή του πολυωνύμου
του βαθμού που δίνεται ως τρίτο όρισμα. 
Εάν δεν είχε καθοριστεί κανένας βαθμός, η {\en\tt coeffs} επιστρέφει
την λίστα των συντελεστών του πολυωνύμου, συμπεριλμβανομένου και του 0 στην περίπτωση
της πυκνής παράστασης πολυωνύμων μιας μεταβλητής και εξαιρώντας το 0 στην περίπτωση της αραιής παράστασης πολυωνύμων πολλαπλών μεταβλητών.\\
Είσοδος :
\begin{center}{\en\tt coeff(-x\verb|^|4+3*x*y\verb|^|2+x,x,1)}\end{center}
Έξοδος :
\begin{center}{\en\tt 3*y\verb|^|2+1}\end{center}  
Είσοδος :
\begin{center}{\en\tt coeff(-x\verb|^|4+3x*y\verb|^|2+x,y,2)}\end{center}
Έξοδος :
\begin{center}{\en\tt 3*x}\end{center} 
Είσοδος :
\begin{center}{\en\tt coeff(-x\verb|^|4+3x*y\verb|^|2+x,[x,y],[1,2])}\end{center}
Έξοδος :
\begin{center}{\en\tt 3}\end{center} 

\subsection{Βαθμός πολυωνύμου : {\tt\textlatin{ degree}}}\index{degree}
\noindent{{\en\tt degree} παίρνει ως όρισμα ένα πολυώνυμο που δίνεται από την συμβολική 
αναπαράσταση ή από την λίστα των συντελεστών του.}\\
{\en\tt degree} επιστρέφει τον βαθμό αυτού του πολυωνύμου (τον υψηλότερο βαθμό των μη μηδενικών μονωνύμων του).\\ 
Είσοδος :
\begin{center}{\en\tt degree(x\verb|^|3+x)}\end{center}
Έξοδος :
\begin{center}{\en\tt 3}\end{center} 
Είσοδος :
\begin{center}{\en\tt degree([1,0,1,0])}\end{center}
Έξοδος :
\begin{center}{\en\tt 3}\end{center}

\subsection{Μικρότερος βαθμός πολυωνύμου :{\tt\textlatin{valuation ldegree}}}\index{valuation}\index{ldegree}
\noindent{{\en\tt valuation} ή {\en\tt ldegree} παίρνει ως όρισμα ένα πολυώνυμο που δίνεται από μια 
συμβολική παράσταση ή από μια λίστα των συντελεστών του.}\\
{\en\tt valuation} ή {\en\tt ldegree} επιστρέφει την αποτίμηση αυτού του πολυωνύμου ({\en\tt valuation}),
που είναι ο μικρότερος βαθμός των μη μηδενικών του μονωνύμων.\\
Είσοδος :
\begin{center}{\en\tt valuation(x\verb|^|3+x)}\end{center}
Έξοδος :
\begin{center}{\tt 1}\end{center} 
Είσοδος :
\begin{center}{\en\tt valuation([1,0,1,0])}\end{center}
Έξοδος :
\begin{center}{\tt 1}\end{center}

\subsection{Κύριος συντελεστής πολυωνύμου : {\tt\textlatin{ lcoeff}}}\index{lcoeff}
\noindent{{\en\tt lcoeff}  παίρνει ως όρισμα ένα πολυώνυμο που δίνεται από μια
συμβολική παράσταση ή από την λίστα των συντελεστών του.}\\
{\en\tt lcoeff} επιστρέφει τον κύριο συντελεστή αυτού του πολυωνύμου,
που είναι ο συντελεστής του μονωνύμου με τον μεγαλύτερο βαθμό.\\
Είσοδος :
\begin{center}{\en\tt lcoeff([2,1,-1,0])}\end{center}
Έξοδος :
\begin{center}{\tt  2}\end{center}
Είσοδος :
\begin{center}{\en\tt lcoeff(3*x\verb|^|2+5*x,x)}\end{center}
Έξοδος :
\begin{center}{\tt  3}\end{center}
Είσοδος :
\begin{center}{\en\tt lcoeff(3*x\verb|^|2+5*x*y\verb|^|2,y)}\end{center}
Έξοδος :
\begin{center}{\en\tt  5*x}\end{center}
 \subsection{Τελευταίος συντελεστής πολυωνύμου : {\tt\textlatin{ tcoeff}}}\index{tcoeff}
\noindent{{\en\tt tcoeff}  παίρνει ως όρισμα ένα πολυώνυμο που δίνεται από μια
συμβολική παράσταση 
ή από μια λίστα των συντελεστών του.}\\
{\en\tt tcoeff}  επιστρέφει τον συντελεστή του μονωνύμου με τον μικρότερο βαθμό σε 
αυτό το πολυώνυμο ({\en\tt tcoeff =  trailing coefficient}).\\
Είσοδος :
\begin{center}{\en\tt tcoeff([2,1,-1,0])}\end{center}
Έξοδος :
\begin{center}{\tt  -1}\end{center}
Είσοδος :
\begin{center}{\en\tt tcoeff(3*x\verb|^|2+5*x,x)}\end{center}
Έξοδος :
\begin{center}{\tt  5}\end{center}
Είσοδος :
\begin{center}{\en\tt tcoeff(3*x\verb|^|2+5*x*y\verb|^|2,y)}\end{center}
Έξοδος :
\begin{center}{\en\tt  3*x\verb|^|2}\end{center}

\subsection{Αποτίμηση πολυωνύμου : {\tt\textlatin{ peval polyEval}}}\index{peval}
\index{polyEval}
\noindent{{\en\tt peval} ή {\en\tt polyEval} παίρνει ως όρισμα ένα πολυώνυμο 
{\en\tt p} που δίνεται από τη λίστα των συντελεστών του και έναν πραγματικό αριθμό {\en\tt a} .\\
{\en\tt peval} ή {\en\tt polyEval} επιστρέφει την ακριβή ή αριθμητική τιμή του 
{\en\tt p(a)} χρησιμοποιώντας τον αλγόριθμο των {\en\tt Ruffini-Horner}.}\\ 
 Είσοδος :
\begin{center}{\en\tt peval([1,0,-1],sqrt(2))}\end{center}
Έξοδος :
\begin{center}{\en\tt sqrt(2)*sqrt(2)-1}\end{center} 
Τότε :
\begin{center}{\en\tt normal(sqrt(2)*sqrt(2)-1)}\end{center} 
Έξοδος :
\begin{center}{\tt {\tt 1}}\end{center} 
Είσοδος :
\begin{center}{\en\tt peval([1,0,-1],1.4)}\end{center}
Έξοδος :
\begin{center}{\tt  0.96}\end{center} 

\subsection{Παραγοντοποίηση του $x^n$ σε ένα πολυώνυμο : \\{\tt\textlatin{ factor\_xn}}}\index{factor\_xn}
\noindent{{\en\tt factor\_xn} παίρνει ως όρισμα ένα πολυώνυμο {\en\tt P}.\\
{\en\tt factor\_xn} επιστρέφει το πολυώνυμο {\en\tt P} γραμμένο
σαν το γινόμενο του μονωνύμου με τον μεγαλύτερο βαθμό,  του $x^n$ ({\en\tt n=degree}),
με ένα ρητό κλάσμα που έχει ένα μη μηδενικό, πεπερασμένο, όριο στο άπειρο.}\\ 
 Είσοδος:
\begin{center}{\en\tt factor\_xn(-x\verb|^|4+3)}\end{center}
Έξοδος :
\begin{center}{\en\tt x\verb|^|4*(-1+3*x\verb|^|-4)}\end{center} 

\subsection{Μέγιστος κοινός διαιρέτης ({\tt\textlatin{GCD}} ή ΜΚΔ) των συντελεστών ενός πολυωνύμου : {\tt\textlatin{ content}}}\index{content|textbf}
\noindent{{\en\tt content} παίρνει ως όρισμα ένα πολυώνυμο {\en\tt P} που δίνεται από 
μια συμβολική παράσταση ή από την λίστα των συντελεστών του.}\\
{\en\tt content} επιστρέφει το περιεχόμενο του {\en\tt P},
που είναι ο ΜΚΔ  (Μέγιστος Κοινός Διαιρέτης) ({\tt\textlatin{GCD}}) των συντελεστών του 
{\en\tt P}.\\
Είσοδος :
\begin{center}{\en\tt content(6*x\verb|^|2-3*x+9)}\end{center}
ή:
\begin{center}{\en\tt content([6,-3,9],x))}\end{center}
Έξοδος :
\begin{center}{\tt  3}\end{center} 

\subsection{Αρχικό μέρος ενός πολυωνύμου : {\tt\textlatin{ primpart}}}\index{primpart}
\noindent{{\en\tt primpart} παίρνει ως όρισμα ένα πολυώνυμο {\en\tt P} που δίνεται από 
μια συμβολική παράσταση ή από την λίστα των συντελεστών του.}\\
{\en\tt primpart} επιστρέφει το αρχικός μέρος του {\en\tt P},
που είναι το {\en\tt P} διαιρεμένο από τον {\tt\textlatin{GCD}} 
(Μέγιστο Κοινό Διαιρέτη) των συντελεστών του.\\
Είσοδος :
\begin{center}{\en\tt primpart(6x\verb|^|2-3x+9)}\end{center}
ή:
\begin{center}{\en\tt  primpart([6,-3,9],x))}\end{center}
Έξοδος :
\begin{center}{\en\tt 2*x\verb|^|2-x+3}\end{center} 

\subsection{Παραγοντοποίηση : {\tt\textlatin{ collect}}}\index{collect}
\noindent{{\en\tt collect} παίρνει ως όρισμα ένα πολυώνυμο ή μια λίστα 
πολυωνύμων και προεραιτικά μια αλγεβρική παράσταση όπως {\en\tt sqrt(n)}
(για $\sqrt{n}$).}\\
{\en\tt collect} παραγοντοποιεί το πολυώνυμο (ή τα πολυώνυμα της 
λίστας) στο πεδίο των συντελεστών του (για παράδειγμα $\mathbb Q$)
ή στη μικρότερη επέκταση που περιέχει το προαιρετικό δεύτερο όρισμα (π.χ.
$\mathbb Q[\sqrt{n}]$). Στoν τρόπο λειτουργίας  {\en\tt "}στους μιγαδικούς{\en\tt "} (στις Ρυθμίσεις του {\en\tt Cas}), το πεδίο είναι αυτό των μιγαδικών αριθμών.\\
{\bf Παραδείγματα} :
\begin{itemize}
\item Παραγοντοποιείστε το $x^2-4$ στους ακεραίους. Εισαγωγή :
\begin{center}{\en\tt collect(x\verb|^|2-4)}\end{center}
Έξοδος στoν τρόπο λειτουργίας για πραγματικούς αριθμούς:
\begin{center}{\en\tt (x-2)*(x+2)}\end{center}
\item Παραγοντοποιείστε το $x^2+4$ στους ακεραίους. Εισαγωγή :
\begin{center}{\en\tt collect(x\verb|^|2+4)}\end{center}
Έξοδος στoν τρόπο λειτουργίας για πραγματικούς αριθμούς:
\begin{center}{\en\tt x\verb|^|2+4}\end{center}
Έξοδος στoν τρόπο λειτουργίας  {\en\tt "}στους μιγαδικούς{\en\tt "}:
\begin{center}{\en\tt (x+2*i)*(x-2*i)}\end{center}
\item Παραγοντοποιείστε το $x^2-2$ στους ακεραίους. Εισαγωγή :
\begin{center}{\en\tt collect(x\verb|^|2-2)}\end{center}
Έξοδος στoν τρόπο λειτουργίας για πραγματικούς αριθμούς:
\begin{center}{\en\tt x\verb|^|2-2}\end{center}
Αλλά εάν εισάγετε :
\begin{center}{\en\tt collect((x\verb|^|2-2),sqrt(2))}\end{center}
Έξοδος :
\begin{center}{\en\tt (x-sqrt(2))*(x+sqrt(2))}\end{center}
\item Παραγοντοποιείστε στους ακεραίους τα πολυώνυμα : 
$$x^3-2x^2+1 \mbox{ και } x^2-x$$
Είσοδος :
 \begin{center}{\en\tt collect([x\verb|^|3-2*x\verb|^|2+1,x\verb|^|2-x])}\end{center}
Έξοδος :
\begin{center}{\en\tt  [(x-1)*(x\verb|^|2-x-1),x*(x-1)]}\end{center}
Αλλά, αν εισάγετε :
 \begin{center}{\en\tt collect((x\verb|^|3-2*x\verb|^|2+1)*sqrt(5))}\end{center}
Έξοδος :
\begin{center}${\en\tt \sqrt{5} (x+\frac{-\left(\sqrt{5}\right)-1}{2}) (x-1) (x+\frac{\sqrt{5}-1}{2})}$\end{center}
Ή, αν εισάγετε :
\begin{center}{\en\tt collect(x\verb|^|3-2*x\verb|^|2+1,sqrt(5))}\end{center}
Έξοδος :
\begin{center}${\en\tt (x+\frac{-\left(\sqrt{5}\right)-1}{2}) (x-1) (x+\frac{\sqrt{5}-1}{2}) }$\end{center}


\end{itemize}

\subsection{Παραγοντοποίηση : {\tt\textlatin{ factor factoriser}}}\index{factor}\index{factoriser}\label{sec:factor}
\noindent{{\en\tt factor} παίρνει ως όρισμα ένα πολυώνυμο ή μια λίστα 
πολυωνύμων και προεραιτικά μια αλγεβρική επέκταση, π.χ. {\en\tt sqrt(n)}.}\\
{\en\tt factor} παραγοντοποιεί το πολυώνυμο (ή τα πολυώνυμα της λίστας) στο πεδίο
των συντελεστών του (στoν τρόπο λειτουργίας  {\en\tt "}στους μιγαδικούς{\en\tt "} (στις Ρυθμίσεις του {\en\tt Cas}), το πεδίο είναι αυτό των μιγαδικών αριθμών)
ή στη μικρότερη επέκταση που περιέχει το προαιρετικό δεύτερο όρισμα.
Αντίθετα με την {\en\tt collect},
η {\en\tt factor} θα παραγοντοποιήσει περαιτέρω κάθε παράγοντα βαθμού 2
εάν  τσεκάρουμε την {\en\tt Sqrt} στις Ρυθμίσεις του {\en\tt Cas} 
(δείτε επίσης  \ref{sec:factore}).
Μπορείτε να ελεγξετε τον τρέχοντα τρόπο λειτουργίας στην μπάρα Ρυθμίσεων του {\en\tt Cas}  και να αλλάξετε τις ρυθμίσεις.\\
Είσοδος :
 \begin{center}{\en\tt factor(x\verb|^|2+2*x+1)}\end{center}
Έξοδος :
\begin{center}{\en\tt (x+1)\verb|^|2}\end{center}
Είσοδος :
\begin{center}{\en\tt factor(x\verb|^|4-2*x\verb|^|2+1)}\end{center}
Έξοδος :
\begin{center}{\en\tt (-x+1)\verb|^|2*(x+1)\verb|^|2}\end{center}
Είσοδος :
 \begin{center}{\en\tt factor(x\verb|^| 3-2*x\verb|^|2+1)}\end{center}
Έξοδος εάν η {\en\tt Sqrt} δεν τσεκάρεται στη διαμόρφωση {\en\tt cas} :
\begin{center}{\en\tt  (x-1)*(x\verb|^|2-x-1)}\end{center}
Έξοδος εάν η {\en\tt Sqrt} έχει τσεκαρισθεί στις Ρυθμίσεις του {\en\tt Cas}  :
\begin{center}{\en\tt (x-1)*(x+(sqrt(5)-1)/2)*(x+(-sqrt(5)-1)/2)}\end{center}
Είσοδος :
 \begin{center}{\en\tt factor(x\verb|^| 3-2*x\verb|^|2+1,sqrt(5))}\end{center}
Έξοδος :
\begin{center}${\en\tt (x+\frac{-\left(\sqrt{5}\right)-1}{2}) (x-1) (x+\frac{\sqrt{5}-1}{2}) }$\end{center}
Είσοδος :
 \begin{center}{\en\tt factor(x\verb|^|2+1)}\end{center}
Έξοδος στoν τρόπο λειτουργίας για πραγματικούς αριθμούς:
\begin{center}{\en\tt  x\verb|^|2+1}\end{center}
Έξοδος στoν τρόπο λειτουργίας  {\en\tt "}στους μιγαδικούς{\en\tt "} :
\begin{center}{\en\tt  ((-i)*x+1)*((i)*x+1)}\end{center}

\subsection{Παραγοντοποίηση \tt\textlatin{Square-free} : {\tt\textlatin{ sqrfree}}}\index{sqrfree}
\noindent{{\en\tt sqrfree} παίρνει ως όρισμα ένα πολυώνυμο.}\\
{\en\tt sqrfree} παραγοντοποιεί αυτό το πολυώνυμο σαν ένα γινόμενο
δυνάμεων  παραγόντων πρώτων μεταξύ τους, όπου κάθε παράγοντας έχει ρίζες πολλαπλότητας 1
(με άλλα λόγια, ένα παράγοντας και η παράγωγός του είναι πρώτοι μεταξύ τους).\\
Είσοδος : 
\begin{center}{\en\tt sqrfree((x\verb|^|2-1)*(x-1)*(x+2))}\end{center} 
Έξοδος :
 \begin{center}{\en\tt (x\verb|^|2+3*x+2)*(x-1)\verb|^|2}\end{center} 
Είσοδος : 
\begin{center}{\en\tt sqrfree((x\verb|^|2-1)\verb|^|2*(x-1)*(x+2)\verb|^|2)}\end{center} 
Έξοδος :
 \begin{center}{\en\tt (x\verb|^|2+3*x+2)*(x-1)\verb|^|3}\end{center} 

 \subsection{Λίστα παραγόντων : {\tt\textlatin{ factors}}}\index{factors|textbf}
\noindent{{\en\tt factors}  έχει είτε ένα πολυώνυμο είτε  μια λίστα πολυωνύμων σαν
όρισμα.\\
{\en\tt factors} επιστρέφει μια λίστα που περιέχει τους παράγοντες του πολυωνύμου
και τους εκθέτες τους.}\\
Είσοδος :
 \begin{center}{\en\tt factors(x\verb|^|2+2*x+1)}\end{center}
Έξοδος :
\begin{center}{\en\tt  [x+1,2]}\end{center}
Είσοδος :
 \begin{center}{\en\tt factors(x\verb|^|4-2*x\verb|^|2+1)}\end{center}
Έξοδος :
\begin{center}{\en\tt [x+1,2,x-1,2]}\end{center}
Είσοδος :
 \begin{center}{\en\tt factors([x\verb|^|3-2*x\verb|^|2+1,x\verb|^|2-x])}\end{center}
Έξοδος :
\begin{center}{\en\tt [[x-1,1,x\verb|^|2-x-1,1],[x,1,x-1,1]]}\end{center}
Είσοδος :
 \begin{center}{\en\tt factors([x\verb|^|2,x\verb|^|2-1])}\end{center}
Έξοδος:
\begin{center}{\en\tt  [[x,2],[x+1,1,x-1,1]]}\end{center}

\subsection{Αποτίμηση πολυωνύμου : {\tt\textlatin{ horner}}}\index{horner}
\noindent{{\en\tt  horner} παίρνει 2 ορίσματα : ένα πολυώνυμο {\en\tt P} που δίνεται από τη 
συμβολική του παράσταση ή από τη λίστα των συντελεστών του και έναν αριθμό {\en\tt a}.\\ 
{\en\tt  horner} επιστρέφει το {\en\tt P(a)} που υπολογίζεται χρησιμοποιώντας τον αλγόριθμο των {\tt\textlatin{Ruffini-Horner.}}}\\
Είσοδος :
\begin{center}{\en\tt  horner(x\verb|^|2-2*x+1,2)}\end{center}
ή  :
\begin{center}{\en\tt  horner([1,-2,1],2)}\end{center}
Έξοδος :
\begin{center}{\en\tt 1}\end{center}

\subsection{Αναγραφή ως προς δυνάμεις του {\tt\textlatin{(x-a)}} : {\tt\textlatin{ ptayl}}}\index{ptayl}
{\en\tt ptayl} χρησιμοποιείται για να αναγράψουμε ένα πολυώνυμο {\en\tt P} μεταβλητής {\en\tt x}
ως προς δυνάμεις του {\en\tt (x-a)} 
({\en\tt ptayl} σημαίνει πολυώνυμο {\tt\textlatin{Taylor)}}\\
{\en\tt ptayl} παίρνει 2 ορίσματα: ένα πολυώνυμο {\en\tt P} που δίνεται από τη
συμβολική του παράσταση ή από τη λίστα των συντελεστών του και 
ένα αριθμό {\en\tt a}.\\
{\en\tt ptayl} επιστρέφει το πολυώνυμο {\en\tt Q} τέτοιο ώστε {\en\tt Q(x-a)=P(x)}\\
Είσοδος :
\begin{center}{\en\tt ptayl(x\verb|^|2+2*x+1,2)}\end{center}
Έξοδος, το πολυώνυμο Q:
\begin{center}{\en\tt  x\verb|^|2+6*x+9}\end{center}
Είσοδος :
\begin{center}{\en\tt  ptayl([1,2,1],2)}\end{center}
Έξοδος :
\begin{center}{\en\tt [1,6,9]}\end{center}
{\bf Σχόλιο}
\begin{center}{\en\tt P(x)=Q(x-a)}\end{center}
π.χ. για παράδειγμα :\\
$x^2+2x+1=(x-2)^2+6(x-2)+9$

\subsection{Υπολογισμός με την ακριβή ρίζα  πολυωνύμου :{\tt\textlatin{rootof}}}\index{rootof}
Έστω $P$ και $Q$ δύο πολυώνυμα που δίνονται από τη λίστα των συντελεστών τους.
Τότε η {\en\tt rootof(P,Q)} δηλώνει την τιμή του $P(\alpha)$ όπου $\alpha$ είναι η  
ρίζα του $Q$ με το μεγαλύτερο πραγματικό μέρος (και το μεγαλύτερο φανταστικό μέρος σε περίπτωση
ισότητας).\\
Σε ακριβείς υπολογισμούς, το {\en\tt Xcas} θα αναγράψει ρητές αποτιμήσεις 
της {\en\tt rootof} σαν μια μοναδική {\en\tt rootof}  με {\en\tt $degree(P)<degree(Q)$}.
Εάν η τελική {\tt\textlatin{rootof}} είναι η λύση μιας εξίσωσης 2ου βαθμού,
θα απλοποιηθεί.

{\bf Παράδειγμα}\\
Έστω $\alpha$ η ρίζα με το μεγαλύτερο φανταστικό
μέρος του $Q(x)=x^4+10x^2+1$ (όλες οι ρίζες του $Q$ έχουν πραγματικό μέρος ίσο με 0).
\begin{itemize}
\item Να υπολογισθεί το $\displaystyle \frac{1}{\alpha}$. Είσοδος :
\begin{center}{\en\tt normal(1/rootof([1,0],[1,0,10,0,1])) }\end{center}
Το $P(x)=x$ παριστάνεται από το [1,0] και το $\alpha$
από το {\en\tt rootof([1,0],[1,0,10,0,1])}.\\
Έξοδος :
\begin{center}{\en\tt rootof([[-1,0,-10,0],[1,0,10,0,1]])}\end{center}
δηλαδή :
\[  \frac{1}{\alpha}=-\alpha^3-10\alpha \]
\item Να υπολογισθεί το $\alpha^2$. Είσοδος :
\begin{center}{\en\tt
    normal(rootof([1,0],[1,0,10,0,1])\verb|^|2)}\end{center}
ή (αφού το $P(x)=x^2$ παριστάνεται από το [1,0,0]) είσοδος
\begin{center}{\en\tt normal(rootof([1,0,0],[1,0,10,0,1]))}\end{center}
Έξοδος :
\begin{center}{\en\tt -5-2*sqrt(6)}\end{center}
\end{itemize}

\subsection{Ακριβείς ρίζες πολυωνύμου : {\tt\textlatin{ roots}}}\index{roots}
\noindent{{\en\tt roots} παίρνει ως όρισμα μια συμβολική 
πολυωνυμική παράσταση και το όνομα της μεταβλητής της.\\
{\en\tt roots} επιστρέφει έναν δίστηλο πίνακα : κάθε γραμμή είναι 
μία λίστα που περιέχει μια ρίζα του πολυωνύμου και την πολλαπλότητά της.}\\
{\bf Παραδείγματα}
\begin{itemize}
\item Βρείτε τις ρίζες του $P(x)=x^5-2x^4+x^3$.\\
Είσοδος :
\begin{center}{\en\tt roots(x\verb|^|5-2*x\verb|^|4+x\verb|^|3) }\end{center}
Έξοδος :
\begin{center}{\en\tt [[8+3*sqrt(7),1],[8-3*sqrt(7),1],[0,3]]}\end{center}
\item  Βρείτε τις ρίζες του
$x^{10}-15x^8+90x^6-270x^4+405x^2-243=(x^2-3)^5$.\\
 Είσοδος :
\begin{center}{\en\tt roots(x\verb|^|10-15*x\verb|^|8+90*x\verb|^|6-270*x\verb|^|4+405*x\verb|^|2-243)}\end{center}
Έξοδος :
\begin{center}{\en\tt[[sqrt(3),5],[-(sqrt(3)),5]]}\end{center}
\item  Βρείτε τις ρίζες του $(t^3-3)$.\\
Είσοδος :
\begin{center}{\en\tt roots(t\verb|^|3-1,t)}\end{center}
Έξοδος :
\begin{center}{\en\tt[[(-1+(i)*sqrt(3))/2,1],[(-1-(i)*sqrt(3))/2,1],[1,1]]}\end{center}
\end{itemize}

\subsection{Συντελεστές πολυωνύμου που ορίζεται από τις ρίζες του : {\tt\textlatin{ pcoeff pcoef}}}\index{pcoeff}\index{pcoef}
\noindent {{\en\tt pcoeff} (ή {\en\tt pcoef}) παίρνει ως όρισμα μια λίστα 
από ρίζες ενός πολυωνύμου $P$.}\\
{\en\tt pcoeff} (ή {\en\tt pcoef}) επιστρέφει το μονομεταβλητό πολυώνυμο που έχει  αυτές τις
ρίζες,
και που αναπαρίσταται ως λίστα των συντελεστών του με φθίνουσα σειρά.\\
Είσοδος :
\begin{center}{\en\tt pcoef([1,2,0,0,3])}\end{center}
Έξοδος :
\begin{center}{\en\tt [1,-6,11,-6,0,0]}\end{center}
δηλαδή $(x-1)(x-2)(x^2)(x-3)=x^5-6x^4+11x^3-6x^2$.

\subsection{Αποκοπή βαθμού $n$ : {\tt\textlatin{ truncate}}}\index{truncate}
\noindent{{\en\tt truncate} παίρνει ως όρισμα ένα πολυώνυμο και έναν ακέραιο 
{\en\tt n}.\\
{\en\tt truncate} αποκόπτει αυτό το πολυώνυμο στον βαθμό {\en\tt n} (αφαιρώντας
όλους τους όρους βαθμού μεγαλύτερου ή ίσου με τον {\en\tt n+1}).\\
Η εντολή {\en\tt truncate} μπορεί να χρησιμοποιηθεί για να μετατρέψει ένα ανάπτυγμα σε σειρά σε 
ένα πολυώνυμο ή για να υπολογίσει ένα ανάπτυγμα σε σειρά βήμα προς βήμα.}\\
Είσοδος :
\begin{center}{\en\tt truncate((1+x+x\verb|^|2/2)\verb|^|3,4)}\end{center}
Έξοδος :
\begin{center}{\en\tt (9*x\verb|^|4+16*x\verb|^|3+18*x\verb|^|2+12*x+4)/4}\end{center}
Είσοδος :
\begin{center}{\en\tt truncate(series(sin(x)),4)}\end{center}
Έξοδος :
\begin{center}{\en\tt (-x\verb|^|3-(-6)*x)/6}\end{center}
Σημειώσατε πως το επιστρεφόμενο πολυώνυμο είναι κανονικοποιημένο.

\subsection{Μετατροπή αναπτύγματος σε σειρά, σε  πολυώνυμο :\\ {\tt\textlatin{ convert convertir}}}\index{convert}\index{convertir}\index{polynom@{\sl polynom}|textbf}\label{sec:convertpoly}
\noindent{{\en\tt convert}, με την επιλογή {\en\tt polynom}, μετατρέπει μια σειρά {\tt\textlatin{Taylor}} 
σε πολυώνυμο. Χρησιμοποιείται για λειτουργίες όπως 
ο σχεδιασμός γραφημάτων σειράς {\tt\textlatin{Taylor}} μιας συνάρτησης κοντά σ' ένα σημείο.\\
{\en\tt convert} παίρνει 2 ορίσματα : μια παράσταση
και την επιλογή {\en\tt polynom}.\\
{\en\tt convert} αντικαθιστά τον όρο βαθμού {\en\tt order\_size} με 0 μέσα στην 
παράσταση.}\\
Είσοδος :
\begin{center}{\en\tt convert(taylor(sin(x)),polynom)}\end{center}
Έξοδος :
\begin{center}{\en\tt x+1/-6*x\verb|^|3+1/120*\verb|x^|5+x\verb|^|6*0}\end{center}
Είσοδος :
\begin{center}{\en\tt convert(series(sin(x),x=0,6),polynom)}\end{center}
Έξοδος :
\begin{center}{\en\tt x+1/-6*x\verb|^|3+1/120*\verb|x^|5+x\verb|^|7*0}\end{center}

\subsection{Τυχαίο πολυώνυμο : {\tt\textlatin{ randpoly randPoly}}}\index{randpoly}\index{randPoly}
\noindent{{\en\tt randpoly} (ή {\en\tt randPoly}) παίρνει 2 ορίσματα: το όνομα μιας
μεταβλητής (από προεπιλογή {\en\tt x}) και έναν ακέραιο {\en\tt n} (η διάταξη των ορισμάτων
δεν είναι σημαντική).\\
{\en\tt randpoly} επιστρέφει ένα πολυώνυμο ως προς την μεταβλητή που δίνεται ως  
όρισμα (ή ως προς {\en\tt x} εάν δεν έχει εισαχθεί καμία), 
και βαθμού όπως ορίζεται από το 2ο όρισμα, με συντελεστές
τυχαίους ακεραίους ομοιόμορφα κατανεμημένους στο διάστημα -99..+99.}\\ 
Είσοδος :
\begin{center}{\en\tt randpoly(t,4)}\end{center}
Έξοδος για παράδειγμα:
\begin{center}{\en\tt -8*t\verb|^|4-87*t\verb|^|3-52*t\verb|^|2+94*t+80}\end{center}
Είσοδος :
\begin{center}{\en\tt randpoly(4)}\end{center}
Έξοδος για παράδειγμα:
\begin{center}{\en\tt 70*x\verb|^|4-46*x\verb|^|3-7*x\verb|^|2-24*x+52}\end{center}
Είσοδος :
\begin{center}{\en\tt randpoly(4,u)}\end{center}
Έξοδος για παράδειγμα:
\begin{center}{\en\tt 2*u\verb|^|4+33*u\verb|^|3-6*u\verb|^|2-92*u-12}\end{center}

\subsection{Αλλαγή διάταξης των μεταβλητών : {\tt\textlatin{ reorder}}}\index{reorder}
\noindent{{\en\tt reorder}  παίρνει 2 ορίσματα : μια παράσταση και ένα διάνυσμα 
με  ονόματα μεταβλητών.\\
{\en\tt reorder} αναπτύσσει την παράσταση σύμφωνα με τη σειρά των μεταβλητών
που δίνονται σαν 2ο όρισμα.}\\
Είσοδος :
\begin{center}{\en\tt reorder(x\verb|^|2+2*x*a+a\verb|^|2+z\verb|^|2-x*z,[a,x,z])}\end{center}
Έξοδος :
\begin{center}{\en\tt a\verb|^|2+2*a*x+x\verb|^|2-x*z+z\verb|^|2}\end{center}
{\bf Προσοχή} :\\
Οι μεταβλητές πρέπει να είναι συμβολικές (να μην τους έχει αποδοθεί τιμή). Σε αντίθετη περίπτωση, προβείτε σε εκκαθάρισή τους με την {\en\tt purge} πριν καλέσετε την
{\en\tt reorder})

\subsection{Τυχαία λίστα : {\tt\textlatin{ ranm}}}\index{ranm}\label{sec:ranm1}
\noindent{{\en\tt ranm} παίρνει σαν όρισμα έναν ακέραιο {\en\tt n}.\\
{\en\tt ranm} επιστρέφει μια λίστα από {\en\tt n} τυχαίους ακεραίους (μεταξύ -99 και  +99).
Αυτή η λίστα μπορεί να θεωρηθεί ως οι συντελεστές ενός μονομεταβλητού 
πολυωνύμου βαθμού {\en\tt n-1}
(δείτε επίσης \ref{sec:ranm2} και  \ref{sec:ranm3}).}\\
Είσοδος :
\begin{center}{\en\tt ranm(3)}\end{center}
Έξοδος :
\begin{center}{\en\tt [68,-21,56]}\end{center}

\subsection{Πολυώνυμο \tt\textlatin{Lagrange}  : {\tt\textlatin{ lagrange interp}}}\index{lagrange}\index{interp}
\noindent{{\en\tt lagrange} παίρνει ως όρισμα 2 λίστες μεγέθους {\en\tt n} (αντίστοιχα έναν
πίνακα 2 γραμμών και {\en\tt n} στηλών) και το όνομα μιας μεταβλητής 
{\en\tt var} (από προεπιλογή {\en\tt x}).\\
Η 1η λίστα (αντίστοιχα γραμμή) αντιστοιχεί στις τιμές της τετμημένης $x_k$ ($k=1..n$), 
και η 2η λίστα (αντίστοιχα γραμμή) αντιστοιχεί στις τιμές της τεταγμένης $y_k$ 
($k=1..n$).\\
{\en\tt lagrange} επιστρέφει μια πολυωνυμική παράσταση {\en\tt P} 
μεταβλητής {\en\tt var} και βαθμού 
{\en\tt n-1}, τέτοια ώστε $P(x_i)=y_i$.}\\
Είσοδος :
\begin{center}{\en\tt lagrange([[1,3],[0,1]])}\end{center}
ή :
\begin{center}{\en\tt lagrange([1,3],[0,1])}\end{center}
Έξοδος :
\begin{center}{\en\tt (x-1)/2}\end{center}
αφού $\frac{x-1}{2}=0$ για $x=1$,  και  $\frac{x-1}{2}=1$ για $x=3$.\\ 
Είσοδος :
\begin{center}{\en\tt lagrange([1,3],[0,1],y)}\end{center}
Έξοδος :
\begin{center}{\en\tt (y-1)/2}\end{center}
{\bf Προσοχή}\\
Η {\en\tt f:=lagrange([1,2],[3,4],y)} δεν επιστρέφει μια συνάρτηση 
αλλά μια παράσταση ως προς $y$.
Για να ορίσετε την $f$ σαν συνάρτηση, εισάγετε
\begin{center}
{\en\tt f:=unapply(lagrange([1,2],[3,4],x),x)}
\end{center}
Αποφύγετε την εντολή {\en\tt f(x):=lagrange([1,2],[3,4],x)} αφού το
πολυώνυμο {\tt\textlatin{Lagrange}} θα υπολογίζεται κάθε φορά που η {\en\tt f} καλείται
(πράγματι στον ορισμό μιας συνάρτησης, το 2ο μέλος της εντολής
δεν υπολογίζεται).\\
Σημειώστε επίσης ότι η εντολή 
{\en\tt g(x):=lagrange([1,2],[3,4])} δεν θα δούλευε
αφού από προεπιλογή το όρισμα της {\en\tt lagrange} (δηλαδή η μεταβλητή {\en\tt x})
θα ήταν καθολικό, και επομένως διάφορο από την τοπική μεταβλητή
που χρησιμοποιείται στον ορισμό της {\en\tt g}.

\subsection{Φυσικές (συναρτήσεις) \tt\textlatin{splines}: {\tt\textlatin{ spline}}}\index{spline|textbf}
\subsubsection{Ορισμός}
Έστω $\sigma_n$ μια υποδιαίρεση  του πραγματικού διαστήματος $[a,b]$~:
\[ a=x_0,\quad x_1,\quad...,\quad x_n=b \]
Η $s$ είναι μια συνάρτηση {\en\tt spline} βαθμού $l$, εάν η $s$ είναι μια απεικόνιση του $[a,b]$ 
στο $\mathbb R$ τέτοια ώστε ~:
\begin{itemize}
\item  Η $s$ έχει συνεχείς παραγώγους τάξης μέχρι $l-1$,
\item σε κάθε διάστημα της υποδιαίρεσης, το $s$ 
είναι ένα πολυώνυμο βαθμού μικρότερου ή ίσου με $l$.
\end{itemize}

\subsubsection{Θεώρημα }
Το σύνολο των συναρτήσεων {\en\tt spline} βαθμού  $l$ στο $\sigma_n$ είναι ένας
 διανυσματικός υπόχωρος του $\mathbb R$ διάστασης $n+l$.

{\bf Απόδειξη}\\
Στο $[a,x_1]$, το $s$ είναι ένα πολυώνυμο $A$ βαθμού μικρότερου ή ίσου με 
$l$, και επομένως στο $[a,x_1]$, $s=A(x)=a_0+a_1x+...a_lx^l$ που σημαίνει ότι το  $A$ είναι ένας γραμμικός 
συνδυασμός των $1,x,...x^l$.\\
Στο $[x_1,x_2]$, το $s$ είναι ένα πολυώνυμο $B$ βαθμού μικρότερου ή ίσου με
$l$, γι' αυτό και στο $[x_1,x_2]$, $s=B(x)=b_0+b_1x+...b_lx^l$.\\
Η συνάρτηση $s$ έχει συνεχείς παραγώγους τάξης μέχρι και $l-1$, γι 'αυτό :
\[ \forall 0 \leq j \leq l-1, \quad  B^{(j)}(x_1)-A^{(j)}(x_1)=0\]
και επομένως $B(x)-A(x)=\alpha_1(x-x_1)^l$ ή $B(x)=A(x)+\alpha_1(x-x_1)^l$.\\
Ορίστε την συνάρτηση :
\[\mbox{\en\tt q}_1(x)  \mbox{ = }
\left\{
\begin{array}{rcl}
0 & \mbox{στο} & [a,x_1] \\
(x-x_1)^l  & \mbox{στο} & [x_1,b]\\
\end{array} 
\right.
\]
Επομένως :
\[ s|_{[a,x_2]}=a_0+a_1x+...a_lx^l+\alpha_1q_1(x) \]
Στο $[x_2,x_3]$, το $s$ είναι ένα πολυώνυμο $C$ βαθμού μικρότερου ή ίσου με
$l$, γι' αυτό στο $[x_2,x_3]$, $s=C(x)=c_0+c_1x+...c_lx^l$.\\
Η συνάρτηση $s$ έχει συνεχείς παραγώγους τάξης μέχρι και $l-1$, γι' αυτό :  
\[ \forall 0 \leq j \leq l-1, \quad  C^{(j)}(x_2)-B^{(j)}(x_2)=0\]
και επομένως $C(x)-B(x)=\alpha_2(x-x_2)^l$ or $C(x)=B(x)+\alpha_2(x-x_2)^l$.\\
Ορίστε τη συνάρτηση:
\[\mbox{\en\tt q}_2(x)  \mbox{ = }
\left\{
\begin{array}{rcl}
0 & \mbox{στο} & [a,x_2] \\
(x-x_2)^l  & \mbox{στο} & [x_2,b]\\
\end{array} 
\right.
\]
Επομένως :
$s|_{[a,x_3]}=a_0+a_1x+...a_lx^l+\alpha_1q_1(x)+\alpha_2q_2(x)$\\
Και ούτω καθεξής , οι συναρτήσεις ορίζονται από :
\[\forall 1 \leq j \leq n-1, \mbox{\en\tt q}_j(x)  \mbox{ = }
\left\{
\begin{array}{rcl}
0 & \mbox{στο} & [a,x_j] \\
(x-x_j)^l  & \mbox{στο} & [x_j,b]\\
\end{array} 
\right.
\]
Επομένως,
\[ s|_{[a,b]}=a_0+a_1x+...a_lx^l+\alpha_1q_1(x)+....+\alpha_{n-1}q_{n-1}(x) \]
και το $s$ είναι ένας γραμμικός συνδυασμός από $n+l$ ανεξάρτητες συναρτήσεις 
$1,x,..x^l,q_1,..q_{n-1}$.

\subsubsection{Παρεμβολή με συναρτήσεις {\en\tt spline}}
Αν θέλουμε να παρεμβάλουμε μια συνάρτηση $f$ στο $\sigma_n$ από μια συνάρτηση {\en\tt spline}
 $s$ βαθμού $l$, τότε το $s$ πρέπει να επαληθεύει τις συνθήκες $s(x_k)=y_k=f(x_k)$ για όλα τα 
$0\geq k\geq n$. Γι' αυτό υπάρχουν $n+1$ συνθήκες, και $l-1$ βαθμοί 
ελευθερίας. Μπορούμε επομένως να προσθέσουμε $l-1$ συνθήκες, οι οποίες είναι στις 
παραγώγους της $s$ στα $a$ και $b$.

Η Ερμιτιανή παρεμβολή, η φυσική παρεμβολή και η περιοδική παρεμβολή
είναι 3 είδη παρεμβολής που παίρνουμε συγκεκριμενοποιώντας 3 είδη 
περιορισμών. Η μοναδικότητα της λύσης  
του προβλήματος της παρεμβολής μπορεί να αποδειχθεί για το κάθε είδος
περιορισμών.

Αν το $l$ είναι περιττό ($l=2m-1$), υπάρχουν $2m-2$ βαθμοί
ελευθερίας. Οι περιορισμοί ορίζονται από :
\begin{itemize}
\item Ερμιτιανή Παρεμβολή
\[ \forall 1\leq j\leq m-1, \quad s^{(j)}(a)=f^{(j)}(a),
s^{(j)}(b)=f^{(j)}(b) \]
\item  Φυσική παρεμβολή
\[ \forall m \leq j \leq 2m-2, \quad s^{(j)}(a)=s^{(j)}(b)=0 \]
\item Περιοδική παρεμβολή 
\[\forall 1\leq j\leq 2m-2, \quad s^{(j)}(a)=s^{(j)}(b) \]
\end{itemize}

Αν το $l$ είναι ζυγός ($l=2m$), υπάρχουν $2m-1$ βαθμοί 
ελευθερίας. Οι περιορισμοί ορίζονται από :
\begin{itemize}
\item Ερμιτιανή Παρεμβολή
\[ \forall 1\leq j\leq m-1, \quad s^{(j)}(a)=f^{(j)}(a),
s^{(j)}(b)=f^{(j)}(b) \] 
και
\[s^{(m)}(a)=f^{(m)}(a)\] 
\item Φυσική παρεμβολή
\[ \forall m \leq j \leq 2m-2, \quad s^{(j)}(a)=s^{(j)}(b)=0 \]
και
\[s^{(2m-1)}(a)=0\] 
\item  Περιοδική παρεμβολή 
\[\forall 1\leq j\leq 2m-1, \quad s^{(j)}(a)=s^{(j)}(b) \]
\end{itemize}
Μια φυσική {\tt\textlatin{spline}} 
είναι μια συνάρτηση {\tt\textlatin{spline}} η οποία επαληθεύει τους περιορισμούς της φυσικής παρεμβολής.

{\en\tt spline} παίρνει ως ορίσματα μια λίστα τετμημένων (σε αύξουσα σειρά), 
μια λίστα τεταγμένων, το όνομα μιας μεταβλητής, και ένα βαθμό.\\
{\en\tt spline} επιστρέφει την φυσική συνάρτηση {\tt\textlatin{spline}} (με συγκεκριμενοποιημένα τον βαθμό
και τα σημεία από τα οποία διέρχεται) σαν μια λίστα πολυωνύμων, με κάθε
πολυώνυμο έκγυρο σε ένα διάστημα.

Παραδείγματα:
\begin{enumerate}
\item Μια φυσική {\tt\textlatin{spline}} βαθμού 3, που διέρχεται από τα σημεία
$x_0=0,y_0=1$, $x_1=1,y_1=3$ και  $x_2=2, y_2=0$, Είσοδος~:
\begin{center}
{\en\tt spline([0,1,2],[1,3,0],x,3)}
\end{center}
Έξοδος είναι μια λίστα 2 πολυωνυμικών παραστάσεων του $x$~:
\[[ -5*x3/4+13*x/4+1, \quad 5*(x-1)3/4-15*(x-1)2/4+(x-1)/-2+3 ]\]
που ορίζονται αντίστοιχα στα διαστήματα $[0,1]$ και $[1,2]$.
\item Μια φυσική {\tt\textlatin{spline}} βαθμού 4, που διέρχεται από τα σημεία
$x_0=0,y_0=1$, $x_1=1,y_1=3$, $x_2=2, y_2=0$ και  $x_3=3, y_3=-1$, 
Είσοδος~:
\begin{center}
{\en\tt spline([0,1,2,3],[1,3,0,-1],x,4)}
\end{center}
Έξοδος είναι μια λίστα 3 πολυωνυμικών συναρτήσεων του $x$~:
\[ [(-62*x4+304*x)/121+1,\] 
\[(201*(x-1)4-248*(x-1)3-372*(x-1)2+56*(x-1))/121+3,\] 
\[(-139*(x-2)4+556*(x-2)3+90*(x-2)2+-628*(x-2))/121]\]
που ορίζονται αντίστοιχα στα διαστήματα $[0,1]$, $[1,2]$ και  $[2,3]$.
\item Η φυσική παρεμβολή {\tt\textlatin{spline}} του  $\cos$ στο 
$[0,\pi/2,3\pi/2]$, Είσοδος ~:
\begin{center}
{\en\tt spline([0,pi/2,3*pi/2],cos([0,pi/2,3*pi/2]),x,3)}
\end{center}
Έξοδος ~:
\[
[((3*\pi3+(-7*\pi2)*x+4*x3)*1/3)/(\pi3),\]
\[((15*\pi3+(-46*\pi2)*x+36*\pi*x2-8*x3)*1/12)/(\pi3)]
\]
\end{enumerate}



\section{Aριθμητική πολυωνύμων}
Τα πολυώνυμα αντιπροσωπεύονται από παραστάσεις ή μια λίστα συντελεστών
σε φθίνουσα τάξη δύναμης. Στην 1η περίπτωση, για εντολές που απαιτείται
μια κύρια μεταβλητή (όπως στους υπολογισμούς για τον επεκταμένο Ευκλείδειο αλγόριθμο), η μεταβλητή
που χρησιμοποιείται από προεπιλογή είναι $x$ αν δεν ορίζεται αλλιώς. Για  συντελεστές
στον δακτύλιο $\Z/n\Z$, χρησιμοποιείστε {\en\tt \% n } για κάθε συντελεστή {\en\tt n } της λίστας
ή εφαρμόστε το στην παράσταση που ορίζει το πολυώνυμο.

\subsection{Διαιρέτες πολυωνύμου : {\tt\textlatin{ divis}}}\index{divis}
\noindent{{\en\tt divis} παίρνει ως όρισμα ένα πολυώνυμο (ή μια λίστα  
πολυωνύμων) και επιστρέφει μια λίστα 
των διαιρετών του πολυωνύμου(ων).}\\ 
Είσοδος : 
\begin{center}{\en\tt divis(x\verb|^|4-1)}\end{center}
Έξοδος :
\begin{center}{\en\tt [1,x\verb|^|2+1,x+1,(x\verb|^|2+1)*(x+1),x-1,(x\verb|^|2+1)*(x-1),}\end{center}
\begin{center}{\en\tt (x+1)*(x-1),(x\verb|^|2+1)*(x+1)*(x-1)]}\end{center}
Είσοδος : 
\begin{center}{\en\tt divis([x\verb|^|2,x\verb|^|2-1])}\end{center}
Έξοδος :
\begin{center}{\en\tt [[1,x,x\verb|^|2],[1,x+1,x-1,(x+1)*(x-1)]]}\end{center}

\subsection{Ευκλείδειο πηλίκο : {\tt\textlatin{ quo}}}\index{quo|textbf}
\noindent{{\en\tt quo} επιστρέφει το Ευκλείδειο πηλίκο  $q$ της
της Ευκλείδειας διαίρεσης μεταξύ δύο πολυωνύμων.
Εάν τα πολυώνυμα παριστάνονται ως 
παραστάσεις, η μεταβλητή μπορεί να ορισθεί σαν 3ο
όρισμα.}\\
Είσοδος :
\begin{center}{\en\tt quo(x\verb|^|2+2*x +1,x)}\end{center}
Έξοδος :
\begin{center}{\en\tt x+2}\end{center}
Είσοδος :
\begin{center}{\en\tt quo(y\verb|^|2+2*y +1,y,y)}\end{center}
Έξοδος :
\begin{center}{\en\tt y+2}\end{center}
Στην αναπαράσταση λιστών, για να υπολογίσουμε το πηλίκο της διαίρεσης του $x^2+2x+4$ με το $x^2+x+2$ 
εισάγουμε :
\begin{center}{\en\tt quo([1,2,4],[1,1,2])}\end{center}
Έξοδος :
\begin{center}{\en\tt [1]}\end{center}
αυτό είναι δηλαδή το πολυώνυμο {\en\tt 1}.

\subsection{Ευκλίδειο πηλίκο : {\tt\textlatin{ Quo}}}\index{Quo|textbf}
\noindent{{\en\tt Quo} είναι η αδρανής μορφή του {\en\tt quo}.\\
{\en\tt Quo} επιστρέφει το ευκλίδειο πηλίκο ανάμεσα σε δύο πολυώνυμα
 χωρίς αποτίμηση. 
Χρησιμοποιείται όταν το {\en\tt Xcas} είναι στον τρόπο λειτουργίας {\tt\textlatin{Maple}} για να υπολογίσουμε
το ευκλίδειο πηλίκο της διαίρεσης των δύο πολυωνύμων
με συντελεστές στο $\Z/p\Z$ χρησιμοποιώντας συντακτικό του {\tt\textlatin{Maple}}.\\
Στον τρόπο λειτουργίας {\en\tt Xcas}, αν εισάγετε :
\begin{center}{\en\tt Quo(x\verb|^|2+2*x+1,x)}\end{center}
Έξοδος :
\begin{center}{\en\tt quo(x\verb|^|2+2*x+1,x)}\end{center}
Στον τρόπο λειτουργίας {\en\tt Maple}, αν εισάγετε :
\begin{center}{\en\tt Quo(x\verb|^|3+3*x,2*x\verb|^|2+6*x+5) mod 5}\end{center}
Έξοδος :
\begin{center}{\en\tt -(2)*x+1)}\end{center}
Η διαίρεση έγινε χρησιμοποιώντας αριθμητική υπολοίπων, σε αντίθεση με το
\begin{center}{\en\tt quo(x\verb|^|3+3*x,2*x\verb|^|2+6*x+5) mod 5}\end{center}
όπου  η διαίρεση γίνεται στο $\Z[X]$ και μειώνεται έπειτα σε:
\begin{center}{\en\tt 3*x-9}\end{center}
Εάν το {\en\tt Xcas} δεν είναι σε τρόπο λειτουργίας {\en\tt Maple}, πολυωνυμική διαίρεση 
στο $\Z/p\Z[X]$  γίνεται π.χ. με~:
\begin{center}{\en\tt
\verb|quo((x^3+3*x)% 5,(2x^2+6x+5)%5)|}
\end{center}

\subsection{Ευκλίδειο υπόλοιπο : {\tt\textlatin{ rem}}}\index{rem|textbf}
\noindent{{\en\tt rem} επιστρέφει το Ευκλείδειο υπόλοιπο μεταξύ δύο πολυωνύμων
(διαίρεση φθίνουσας δύναμης). Εάν τα πολυώνυμα αναπαρίστανται ως 
παραστάσεις, η μεταβλητή μπορεί να ορισθεί σαν 3ο
όρισμα.}\\
Είσοδος :
\begin{center}{\en\tt rem(x\verb|^|3-1,x\verb|^|2-1)}\end{center}
Έξοδος :
\begin{center}{\en\tt x-1}\end{center}
Για να έχουμε το υπόλοιπο της διαίρεσης του $x^2+2x+4$ με το $x^2+x+2$ μπορούμε επίσης να εισάγουμε :
\begin{center}{\en\tt  rem([1,2,4],[1,1,2])}\end{center}
Έξοδος :
\begin{center}{\en\tt [1,2]}\end{center}
δηλαδή το πολυώνυμο $x+2$.

\subsection{Ευκλίδειο υπόλοιπο: {\tt\textlatin{ Rem}}}\index{Rem|textbf}
\noindent{{\en\tt Rem}  είναι η αδρανής μορφή της {\en\tt rem}.\\
{\en\tt Rem} επιστρέφει το Ευκλείδειο υπόλοιπο ανάμεσα σε δύο πολυώνυμα
(διαίρεση φθίνουσας δύναμης) χωρίς αποτίμηση. 
Χρησιμοποιείται όταν το {\en\tt Xcas} είναι σε τρόπο λειτουργίας {\en\tt Maple} για να υπολογίσουμε
το Ευκλείδειο υπόλοιπο της διαίρεσης δύο
πολυωνύμων με συντελεστές στο $\Z/p\Z$ χρησιμοποιώντας συντακτικό του {\tt\textlatin{Maple}}.}\\
Στον τρόπο λειτουργίας {\en\tt Xcas} , αν εισάγετε :
\begin{center}{\en\tt Rem(x\verb|^|3-1,x\verb|^|2-1)}\end{center}
Έξοδος :
\begin{center}{\en\tt rem(x\verb|^|3-1,x\verb|^|2-1)}\end{center}
Στον τρόπο λειτουργίας {\en\tt Maple}, αν εισάγετε  :
\begin{center}{\en\tt Rem(x\verb|^|3+3*x,2*x\verb|^|2+6*x+5) mod 5}\end{center}
Έξοδος :
\begin{center}{\en\tt 2*x}\end{center}
Η διαίρεση έγινε χρησιμοποιώντας αριθμητική υπολοίπων, σε αντίθεση με
\begin{center}{\en\tt rem(x\verb|^|3+3*x,2*x\verb|^|2+6*x+5) mod 5}\end{center}
όπου η διαίρεση γίνεται στο $\Z[X]$ και μειώνετα έπειτα σε:
\begin{center}{\en\tt 2*x}\end{center}
Εάν το {\en\tt Xcas} δεν είναι σε τρόπο λειτουργίας {\tt\textlatin{Maple}}, πολυωνυμική διαίρεση
στο $\Z/p\Z[X]$ γίνεται π.χ. με~:
\begin{center}{\en\tt \verb|rem((x^3+3*x)% 5,(2x^2+6x+5)%5)|}
\end{center}

\subsection{Πηλίκο και υπόλοιπο : {\tt\textlatin{ quorem divide}}}\index{quorem|textbf}\index{divide|textbf}\label{sec:quorem} 
\noindent{{\en\tt quorem} (ή {\en\tt divide}) επιστρέφει μια λίστα με το πηλίκο και 
και το υπόλοιπο της ευκλίδειας διαίρεσης (σε φθίνουσα δύναμη) δύο
πολυωνύμων.}\\
Είσοδος :
\begin{center}{\en\tt quorem([1,2,4],[1,1,2]) }\end{center}
Έξοδος :
\begin{center}{\en\tt [poly1[1],poly1[1,2]]}\end{center}
Είσοδος :
\begin{center}{\en\tt quorem(x\verb|^|3-1,x\verb|^|2-1)}\end{center}
Έξοδος :
\begin{center}{\en\tt [x,x-1]}\end{center}

\subsection{Μέγιστος Κοινός Διαιρέτης (ΜΚΔ ή {\tt\textlatin{GCD}}) δύο πολυωνύμων με τον Ευκλείδειο αλγόριθμο: {\tt\textlatin{ gcd}}}\index{gcd}\label{sec:gcd}
\noindent {\en\tt gcd} δηλώνει τον μέγιστο κοινό διαιρέτη ({\tt\textlatin{gcd}}) δύο πολυωνύμων
(ή μιας λίστας πολυωνύμων ή μιας ακολουθίας πολυωνύμων) 
(δείτε επίσης {\tt\textlatin{\ref{sec:igcd}}} για τον ΜΚΔ ({\tt\textlatin{GCD}}) ακεραίων).}

{\bf Παραδείγματα}\\ 
Είσοδος :
\begin{center}{\en\tt gcd(x\verb|^|2+2*x+1,x\verb|^|2-1)}\end{center}
Έξοδος :
\begin{center}{\en\tt x+1 }\end{center} 
Είσοδος :
\begin{center}{\en\tt gcd(x\verb|^|2-2*x+1,x\verb|^|3-1,x\verb|^|2-1,x\verb|^|2+x-2)}\end{center}
ή
\begin{center}{\en\tt gcd([x\verb|^|2-2*x+1,x\verb|^|3-1,x\verb|^|2-1,x\verb|^|2+x-2])}\end{center}
Έξοδος :
\begin{center}{\en\tt x-1}\end{center}

Για πολυώνυμα με  συντελεστές σε πεδία υπολοίπων ({\tt\textlatin{modular}} συντελεστές), εισάγετε π.χ. :
\begin{center}{\en\tt gcd((x\verb|^|2+2*x+1) mod 5,(x\verb|^|2-1) mod 5)}\end{center} 
Έξοδος :
\begin{center}{\en\tt x \% 5}\end{center}
Σημειώστε ότι :
\begin{center}{\en\tt gcd(x\verb|^|2+2*x+1,x\verb|^|2-1) mod 5}\end{center} 
θα έχει ως έξοδο :
\begin{center}{\en\tt 1}\end{center}
επειδή η πράξη {\tt\textlatin{mod}} εκτελείται αφού ο {\tt\textlatin{GCD}} έχει υπολογισθεί στο $\Z[x]$.

\subsection{Μέγιστος Κοινός Διαιρέτης (ΜΚΔ ή {\tt\textlatin{GCD}}) δύο πολυωνύμων με τον Ευκλείδειο αλγόριθμο : {\tt\textlatin{Gcd}}}\index{Gcd}
\noindent{{\en\tt Gcd}  είναι η αδρανής μορφή της {\en\tt gcd}.
{\en\tt Gcd} επιστρέφει τον μέγιστο κοινό διαιρέτη ({\tt\textlatin{gcd}}) δύο πολυωνύμων
(ή μιας λίστας πολυωνύμων ή μιας ακολουθίας πολυωνύμων) χωρίς
αποτίμηση. Χρησιμοποιείται όταν το {\en\tt Xcas} είναι στον τρόπο λειτουργίας {\tt\textlatin{Maple}} για να υπολογίσουμε το
{\en\tt gcd} των πολυωνύμων με συντελεστές στο $\Z/p\Z$ χρησιμοποιώντας συντακτικό του {\tt\textlatin{Maple}}.}\\
Εισάγετε στον τρόπο λειτουργίας {\en\tt Xcas}  :
\begin{center}{\en\tt Gcd(x\verb|^|3-1,x\verb|^|2-1)}\end{center}
Έξοδος :
\begin{center}{\en\tt gcd(x\verb|^|3-1,x\verb|^|2-1)}\end{center}
Εισάγετε στον τρόπο λειτουργίας {\en\tt Maple}  :
\begin{center}{\en\tt Gcd(x\verb|^|2+2*x,x\verb|^|2+6*x+5) mod 5}\end{center}
Έξοδος :
\begin{center}{\tt 1}\end{center}

\subsection{Επιλογή του αλγορίθμου για την εύρεση του ΜΚΔ ({\tt\textlatin{GCD}})  δύο πολυωνύμων : \tt\textlatin{ezgcd heugcd modgcd psrgcd}}\index{ezgcd}\index{psrgcd}\index{modgcd}\index{heugcd}
\noindent{\tt\textlatin{ ezgcd heugcd modgcd psrgcd}} δηλώνει τον μέγιστο κοινό διαιρέτη ({\tt\textlatin{gcd}}) δύο μονομεταβλητών ή πολυμεταβλητών
πολυωνύμων με συντελεστές
στο $\Z$ ή $\Z[i]$ χρησιμοποιώντας έναν συγκεκριμένο αλγόριθμο ~: 
\begin{itemize}
\item {\en\tt ezgcd} {\tt\textlatin{ezgcd algorithm}},
\item {\en\tt heugcd} {\tt\textlatin{heuristic gcd algorithm}},
\item {\en\tt modgcd} {\tt\textlatin{modular algorithm}},
\item {\en\tt psrgcd} {\tt\textlatin{sub-resultant algorithm}}.
\end{itemize}
Είσοδος :
\begin{center}{\en\tt ezgcd(x\verb|^|2-2*x*y+y\verb|^|2-1,x-y)}\end{center}
ή
\begin{center}{\en\tt heugcd(x\verb|^|2-2*x*y+y\verb|^|2-1,x-y)}\end{center}
ή
\begin{center}{\en\tt modgcd(x\verb|^|2-2*x*y+y\verb|^|2-1,x-y)}\end{center}
ή
\begin{center}{\en\tt psrgcd(x\verb|^|2-2*x*y+y\verb|^|2-1,x-y)}\end{center}
Έξοδος :
\begin{center}{\en\tt 1 }\end{center} 
Είσοδος :
\begin{center}{\en\tt ezgcd((x+y-1)*(x+y+1),(x+y+1)\verb|^|2)}\end{center}
ή
\begin{center}{\en\tt heugcd((x+y-1)*(x+y+1),(x+y+1)\verb|^|2)}\end{center}
ή
\begin{center}{\en\tt modgcd((x+y-1)*(x+y+1),(x+y+1)\verb|^|2)}\end{center}
Έξοδος :
\begin{center}{\en\tt x+y+1}\end{center}
Είσοδος :
\begin{center}{\en\tt psrgcd((x+y-1)*(x+y+1),(x+y+1)\verb|^|2)}\end{center}
Έξοδος :
\begin{center}{\en\tt -x-y-1}\end{center}
Είσοδος :
\begin{center}{\en\tt ezgcd((x+1)\verb|^|4-y\verb|^|4,(x+1-y)\verb|^|2)}\end{center}
Έξοδος :
\begin{center}{\en\tt "GCD not successfull Error: Bad Argument Value"}\end{center}
 Αλλά είσοδος :
\begin{center}{\en\tt heugcd((x+1)\verb|^|4-y\verb|^|4,(x+1-y)\verb|^|2)}\end{center}
ή 
\begin{center}{\en\tt modgcd((x+1)\verb|^|4-y\verb|^|4,(x+1-y)\verb|^|2)}\end{center}
ή 
\begin{center}{\en\tt psrgcd((x+1)\verb|^|4-y\verb|^|4,(x+1-y)\verb|^|2)}\end{center}
Έξοδος :
\begin{center}{\en\tt x-y+1 }\end{center} 

\subsection{Ελάχιστο Κοινό Πολλαπλάσιο (ΕΚΠ ή {\tt\textlatin{ LCM}}) δύο πολυωνύμων : {\tt\textlatin{ lcm}}}\index{lcm}\label{sec:lcm}
\noindent{{\en\tt lcm} επιστρέφει το Ελάχιστο Κοινό Πολλαπλάσιο  (ΕΚΠ ή {\tt\textlatin{ LCM}}) δύο πολυωνύμων
(ή μιας λίστας πολυωνύμων ή μιας ακολουθίας πολυωνύμων)
(δείτε \ref{sec:ilcm} για ΕΚΠ ακεραίων).}\\
Είσοδος :
\begin{center}{\en\tt lcm(x\verb|^|2+2*x+1,x\verb|^|2-1)}\end{center}
Έξοδος :
\begin{center}{\en\tt  (x+1)*(x\verb|^|2-1)}\end{center}
Είσοδος :
\begin{center}{\en\tt lcm(x,x\verb|^|2+2*x+1,x\verb|^|2-1)}\end{center}
ή
\begin{center}{\en\tt lcm([x,x\verb|^|2+2*x+1,x\verb|^|2-1])}\end{center}
Έξοδος :
\begin{center}{\en\tt (x\verb|^|2+x)*(x\verb|^|2-1)}\end{center}

\subsection{Ταυτότητα {\tt\textlatin{B\'ezout}} : {\tt\textlatin{ egcd gcdex}}}\index{egcd}\index{gcdex}
Αυτή η συνάρτηση υπολογίζει τους πολυωνυμικούς συντελεστές της ταυτότητας {\tt\textlatin{B\'ezout}} 
 (επίσης γνωστής και ως Επεκταμένος Μέγιστος Κοινός Διαιρέτης --- {\tt\textlatin{Extended Greatest Common Divisor}}). 
Δοθέντων δύο πολυωνύμων $A(x),B(x)$, η {\en\tt egcd} υπολογίζει 3 πολυώνυμα
$U(x),V(x)$ και $D(x)$ τέτοια ώστε~:
\[    U(x)*A(x)+V(x)*B(x)=D(x)=GCD(A(x),B(x)) \]
{\en\tt egcd} παίρνει 2 ή 3 ορίσματα: τα πολυώνυμα $A$ και $B$ σαν
παραστάσεις μιας μεταβλητής (εάν η μεταβλητή δεν δίνεται ορίζεται από προεπιλογή το $x$). Εναλλακτικά, τα $A$ και $B$ μπορεί να δίνονται και
σαν λίστες πολυωνύμων.\\
Είσοδος :
\begin{center}{\en\tt egcd(x\verb|^|2+2*x+1,x\verb|^|2-1)}\end{center}
Έξοδος :
\begin{center}{\en\tt  [1,-1,2*x+2]}\end{center}
Είσοδος :
\begin{center}{\en\tt egcd([1,2,1],[1,0,-1])}\end{center}
Έξοδος :
\begin{center}{\en\tt  [[1],[-1],[2,2]]}\end{center}
Είσοδος :
\begin{center}{\en\tt egcd(y\verb|^|2-2*y+1,y\verb|^|2-y+2,y)}\end{center}
Έξοδος :
\begin{center}{\en\tt  [y-2,-y+3,4]}\end{center}
Είσοδος :
\begin{center}{\en\tt egcd([1,-2,1],[1,-1,2])}\end{center}
Έξοδος :
\begin{center}{\en\tt [[1,-2],[-1,3],[4]]}\end{center}

\subsection{Επίλυση της {\tt\textlatin{au+bv=c}} στα πολυώνυμα: {\tt\textlatin{ abcuv}}}\index{abcuv}
{\en\tt abcuv} επιλύνει την πολυωνυμική εξίσωση
\[ C(x)=U(x)*A(x)+V(x)*B(x) \]
όπου $A,B,C$ είναι δοθέντα πολυώνυμα και $U$ και $V$ είναι άγνωστα
πολυώνυμα. Το $C$ πρέπει να έιναι ένα πολλαπλάσιο του {\tt\textlatin{gcd}} του $A$ και του $B$
για να υπάρχει λύση. Η {\en\tt abcuv} παίρνει 3 παραστάσεις σαν όρισμα,
και μια προαιρετική μεταβλητή (από προεπιλογή $x$)
και επιστρέφει μια λίστα δύο παραστάσεων ($U$ και $V$). Εναλλακτικά, τα
πολυώνυμα $A,B,C$ μπορούν να εισαχθούν σαν λίστες πολυωνύμων.

Είσοδος :
\begin{center}{\en\tt abcuv(x\verb|^|2+2*x+1 ,x\verb|^|2-1,x+1)}\end{center}
Έξοδος :
\begin{center}{\en\tt [1/2,1/-2]}\end{center}
Είσοδος :
\begin{center}{\en\tt abcuv(x\verb|^|2+2*x+1 ,x\verb|^|2-1,x\verb|^|3+1)}\end{center}
Έξοδος :
\begin{center}{\en\tt [1/2*x\verb|^|2+1/-2*x+1/2,-1/2*x\verb|^|2-1/-2*x-1/2]}\end{center}
Είσοδος :
\begin{center}{\en\tt abcuv([1,2,1],[1,0,-1],[1,0,0,1])}\end{center}
Έξοδος :
\begin{center}{\en\tt [poly1[1/2,1/-2,1/2],poly1[1/-2,1/2,1/-2]]}\end{center}

% 

\subsection{Κινέζικα υπόλοιπα : {\tt\textlatin{ chinrem}}}\index{chinrem}
\noindent{\en\tt{chinrem}} παίρνει δύο λίστες ως ορίσματα, κάθε λίστα αποτελούμενη από δύο 
πολυώνυμα (είτε παραστάσεις είτε  λίστες συντελεστών σε φθίνουσα
τάξη).
 Εάν τα πολυώνυμα είναι παραστάσεις, ένα προαιρετικό 3ο
όρισμα μπορεί  να παρέχεται για να καθορίσει την κύρια μεταβλητή --- από προεπιλογή
χρησιμοποιείται η {\en\tt x} .\\
{\en\tt chinrem([A,R],[B,Q])} επιστρέφει μια λίστα δύο πολυωνύμων,
του {\en\tt P} και του {\en\tt S}, τέτοια ώστε :
\[  S=R.Q, \quad  P=A \pmod R, \quad P=B \pmod Q \]
Εάν {\en\tt R} και {\en\tt Q} είναι πρώτα μεταξύ τους, μια λύση {\en\tt P} υπάρχει πάντα 
και όλες οι λύσεις είναι ισοδύναμες {\en\tt modulo S=R*Q}.
Για παράδειγμα, υποθέστε ότι θέλουμε να λύσουμε : 
\[ {\en\tt \left\{ \begin{array}{rlr} P(x)=&x\ &\bmod\ (x^2+1)\\
      P(x)=&x-1\ &\bmod\ (x^2-1) \end{array}\right.} \]
Είσοδος :
\begin{center}{\en\tt chinrem([[1,0],[1,0,1]],[[1,-1],[1,0,-1]])}\end{center}
Έξοδος :
\begin{center}{\en\tt [[1/-2,1,1/-2],[1,0,0,0,-1]]}\end{center}
ή :
\begin{center}{\en\tt chinrem([x,x\verb|^|2+1],[x-1,x\verb|^|2-1])}\end{center}
Έξοδος :
\begin{center}{\en\tt [1/-2*x\verb|^|2+x+1/-2,x\verb|^|4-1]}\end{center}
Επομένως, $\displaystyle P(x)=-\frac{x^2-2.x+1}{2} \ (\bmod\  x^4-1)$\\
Άλλο παράδειγμα, εισάγετε:
\begin{center}{\en\tt chinrem([[1,2],[1,0,1]],[[1,1],[1,1,1]])}\end{center}
Έξοδος :
\begin{center}{\en\tt [[-1,-1,0,1],[1,1,2,1,1]]}\end{center}
ή :
\begin{center}{\en\tt chinrem([y+2,y\verb|^|2+1],[y+1,y\verb|^|2+y+1],y)}\end{center}
Έξοδος :
\begin{center}{\en\tt [-y\verb|^|3-y\verb|^|2+1,y\verb|^|4+y\verb|^|3+2*y\verb|^|2+y+1]}\end{center}

\subsection{Κυκλοτομικά πολυώνυμα : {\tt\textlatin {cyclotomic}}}\index{cyclotomic}
\noindent{{\en\tt cyclotomic}  παίρνει ένα ακέραιο $n$ σαν όρισμα και
και επιστρέφει την λίστα των συντελεστών του κυκλοτομικού
πολυωνύμου τάξης $n$.  Αυτό 
είναι το πολυώνυμο που έχει για ρίζες τις $n$-οστές αρχικές ρίζες της μονάδος (μια  $n$-στή ρίζα της μονάδος είναι αρχική εάν το σύνολο των δυνάμεών 
της είναι το σύνολο όλων των $n$-στών ριζών της μονάδος).}

Για παράδειγμα, έστω $n=4$, οι τέταρτες ρίζες της μονάδος, είναι:
$\{ 1,i,-1,-i\}$ και οι αρχικές ρίζες είναι: $\{i,-i\}$.
Γι' αυτό , το κυκλοτομικό πολυώνυμο τάξης $4$ είναι $(x-i).(x+i)=x^2+1$.
Επαλήθευση:
\begin{center}{\en\tt cyclotomic(4)}\end{center}
Έξοδος :
\begin{center}{\en\tt [1,0,1]}\end{center}
Άλλο παράδειγμα, εισάγετε :
\begin{center}{\en\tt cyclotomic(5)}\end{center}
Έξοδος :
\begin{center}{\en\tt [1,1,1,1,1]}\end{center}
Γι' αυτό , το κυκλοτομικό πολυώνυμο τάξης $5$ είναι $x^4+x^3+x^2+x+1$
το οποίο διαιρεί το  $x^5-1$ αφού $(x-1)*(x^4+x^3+x^2+x+1)=x^5-1$.\\
Είσοδος :
\begin{center}{\en\tt cyclotomic(10)}\end{center}
Έξοδος :
\begin{center}{\en\tt [1,-1,1,-1,1]}\end{center}
Γι' αυτό, το κυκλοτομικό πολυώνυμο τάξης $10$ είναι $x^4-x^3+x^2-x+1$ και
\[ (x^5-1)*(x+1)*(x^4-x^3+x^2-x+1)=x^{10}-1 \]
Είσοδος :
\begin{center}{\en\tt cyclotomic(20)}\end{center}
Έξοδος :
\begin{center}{\en\tt [1,0,-1,0,1,0,-1,0,1]}\end{center}
Γι' αυτό, το κυκλοτομικό πολυώνυμο τάξης $20$ είναι $x^8-x^6+x^4-x^2+1$ και
\[ (x^{10}-1)*(x^2+1)*(x^8-x^6+x^4-x^2+1)=x^{20}-1 \]

\subsection{Ακολουθίες {\tt\textlatin{Sturm}}  και ο αριθμός μεταβολών προσήμου 
του $P$ στο $]a;\ b]$ : {\tt\textlatin{ sturm}}}\index{sturm}
\noindent{{\en\tt sturm} πάιρνει δύο ή τέσσερα ορίσματα :το $P$ μια πολυωνυμική παράσταση
ή το $P/Q$ ένα ρητό κλάσμα και ένα όνομα μεταβλητής ή το $P$ μια πολυωνυμική
παράσταση, ένα όνομα μεταβλητής και δύο πραγματικούς ή μιγαδικούς αριθμούς $a$ και $b$.}

Εάν η {\en\tt sturm} παίρνει δύο ορίσματα, η {\en\tt sturm} επιστρέφει την λίστα των ακολουθιών 
{\tt\textlatin{Sturm}}  των {\tt\textlatin{square-free}} παραγόντων του $P$ (ή του
$P/Q$) (σ' αυτή την περίπτωση η {\en\tt sturm} συμπεριφέρεται σαν την {\en\tt sturmseq}). (Σημείωση: Οι {\tt\textlatin{square-free}} παράγοντες έχουν (απλές) ρίζες με πολλαπλότητα 1.)

Εάν η {\en\tt sturm} παίρνει 4 ορίσματα, συμπεριφέρεται σαν την {\en\tt sturmab}~:
\begin{itemize}
\item εάν $a$ και $b$ είναι πραγματικοί, 
η {\en\tt sturm} επιστρέφει τον αριθμό των μεταβολών προσήμου του $P$ στο $]a;\ b]$
\item εάν $a$ και $b$ είναι μιγαδικοί, 
η {\en\tt sturm} επιστρέφει τον αριθμό των μιγαδικών ριζών του $P$ στο τετράγωνο 
που έχει το $a$ και το $b$ σαν αντίθετες κορυφές .
\end{itemize} 
Είσοδος :
\begin{center}{\en\tt sturm(2*x\verb|^|3+2,x)}\end{center}
Έξοδος :
\begin{center}{\en\tt [2,[[1,0,0,1],[3,0,0],-9],1]}\end{center}
Είσοδος :
\begin{center}{\en\tt sturm((2*x\verb|^|3+2)/(x+2),x)}\end{center}
Έξοδος :
\begin{center}{\en\tt [2,[[1,0,0,1],[3,0,0],-9],1,[[1,2],1]]}\end{center}
Είσοδος :
\begin{center}{\en\tt sturm(x\verb|^|2*(x\verb|^|3+2),x,-2,0)}\end{center}
Έξοδος :
\begin{center}{\en\tt 1}\end{center}

\subsection{Αριθμός ριζών στο $]a,\ b]$ : {\tt\textlatin{ sturmab}}}\index{sturmab}
\noindent{\en\tt sturmab} παίρνει τέσσερα ορίσματα: μια πολυωνυμική παράσταση $P$, το 
όνομα μιας μεταβλητής και δύο πραγματικούς ή μιγαδικούς αριθμούς $a$ και $b$
\begin{itemize}
\item εάν $a$ και $b$ είναι πραγματικοί, 
η {\en\tt sturmab} επιστρέφει τον αριθμό των μεταβολών προσήμου του $P$ στο $]a,\ b]$ (ανοιχτό στο $a$  και κλειστό στο $b$).
Με άλλα λόγια, επιστρέφει τον αριθμό των πραγματικών ριζών στο $]a,\ b]$ του 
πολυωνύμου  $P/G$ όπου $G=\mbox{\en\tt gcd}(P,\mbox{\en\tt diff}(P))$.
\item εάν $a$ ή $b$ είναι μιγαδικοί, 
η {\en\tt sturmab} επιστρέφει τον αριθμό των μιγαδικών ριζών του $P$ στο τετράγωνο 
που έχει το $a$ και $b$ σαν αντίθετες κορυφές.
\end{itemize} 
Είσοδος :
\begin{center}{\en\tt sturmab(x\verb|^|2*(x\verb|^|3+2),x,-2,0)}\end{center}
Έξοδος :
\begin{center}{\en\tt 1}\end{center}
Είσοδος :
\begin{center}{\en\tt sturmab(x\verb|^|3-1,x,-2-i,5+3i}\end{center}
Έξοδος :
\begin{center}{\tt 3}\end{center}
Είσοδος :
\begin{center}{\en\tt sturmab(x\verb|^|3-1,x,-i,5+3i}\end{center}
Έξοδος :
\begin{center}{\tt 1}\end{center}
{\bf Προσοχή !!!!}\\
\begin{itemize}

\item Στην εντολή {\en\tt sturmab} το $P$ ορίζεται με συμβολική παράσταση.\\
Είσοδος  :\\
{\en\tt sturmab([1,0,0,2,0,0],x,-2,0)},\\ 
Έξοδος :\\
{\en\tt Bad argument type {\gr\tt (= Κακός τύπος ορίσματος)}}.

\item {\en\tt sturmab} επιστρέφει -1 όταν δεν υπάρχουν ρίζες στο $]a,\ b]$ και το πολυώνυμο είναι αρνητικό.\\
Είσοδος  :\\
{\en\tt sturmab((x-1)*(x-5),x,2,3)},\\ 
Έξοδος :\\
-1

\end{itemize}


\subsection{Ακολουθίες {\tt\textlatin{Sturm}} : {\tt\textlatin{ sturmseq}}}\index{sturmseq}
\noindent{{\en\tt sturmseq} παίρνει ως όρισμα, μια πολυωνυμική παράσταση $P$ ή ένα 
ρητό κλάσμα $P/Q$ και επιστρέφει τη λίστα των ακολουθιών {\tt\textlatin{Sturm}} 
των {\tt\textlatin{square-free}} παραγόντων  του $P$ (ή του $P/Q$). (Σημείωση: Οι {\tt\textlatin{square-free}} παράγοντες έχουν (απλές) ρίζες με πολλαπλότητα 1.)
Αν $F$ ένας {\en\tt square-free} παράγοντας, η ακολουθία {\tt\textlatin{Sturm}} 
$R_1,R_2,...$ γίνεται από τα $F$, $F'$ από μια αναδρομική σχέση
~:}
\begin{itemize}
\item 
Το $R_1$ είναι το αντίθετο του υπολοίπου της Ευκλείδειας διαίρεσης του $F$ από το
$F'$ μετά,
\item
το $R_2$ είναι το αντίθετο του υπολοίπου της Ευκλείδειας διαίρεσης του  $F'$ από το
$R_1$, 
\item ... 
\item και ούτω καθεξής μέχρι $R_k=0$.
\end{itemize}
Είσοδος :
\begin{center}{\en\tt sturmseq(2*x\verb|^|3+2)}\end{center}
ή 
\begin{center}{\en\tt sturmseq(2*y\verb|^|3+2,y)}\end{center}
Έξοδος :
\begin{center}{\en\tt [2,[[1,0,0,1],[3,0,0],-9],1]}\end{center}
Ο πρώτος όρος δίνει το {\en\tt "}περιεχόμενο{\en\tt "} (τον {\en\tt gcd} των συντελεστών) του αριθμητή (εδώ 2), 
έπεται η ακολουθία {\tt\textlatin{Sturm}}  $[x^3+1,3x^2,-9]$ και ο τρίτος όρος είναι το περιεχόμενο του παρανομαστή (εδώ 1).\\
Είσοδος :
\begin{center}{\en\tt sturmseq((2*x\verb|^|3+2)/(3*x\verb|^|2+2),x)}\end{center}
Έξοδος :
\begin{center}{\en\tt [2,[[1,0,0,1], [3,0,0],-9], 1, [[3,0,2],[6,0],-72]]}\end{center}
Ο πρώτος όρος δίνει το περιεχόμενο του αριθμητή (εδώ 2), 
έπεται η ακολουθία {\tt\textlatin{Sturm}} του αριθμητή ([[1,0,0,1], [3,0,0], -9]), 
μετά είναι το περιεχόμενο του παρονομαστή (εδώ 1) και στο τέλος έχουμε την ακολουθία {\tt\textlatin{Sturm}} 
του παρονομαστή ([[3,0,2], [6,0], -72]). Σαν παραστάσεις,
η $[x^3+1,3x^2, -9]$ είναι η ακολουθία {\tt\textlatin{Sturm}} του αριθμητή και 
η $[3x^2+2,6x,-72]$ είναι η ακολουθία {\tt\textlatin{Sturm}} του παρονομαστή.\\
Είσοδος :
\begin{center}{\en\tt sturmseq((x\verb|^|3+1)\verb|^|2,x)}\end{center}
Έξοδος :
\begin{center}{\en\tt [1,1]}\end{center}
Είσοδος :
\begin{center}{\en\tt sturmseq(3*(3*x\verb|^|3+1)/(2*x+2),x)}\end{center}
Έξοδος :
\begin{center}{\en\tt[3,[[3,0,0,1],[9,0,0],-81],2,[[1,1],1]]}\end{center}
Ο πρώτος όρος δίνει το περιεχόμενο του αριθμητή 
(εδώ {\tt 3}),\\
ο δεύτερος όρος δίνει την ακολουθία {\tt\textlatin{Sturm}} του αριθμητή
(εδώ {\en\tt 3x\verb|^|3+1, 9x\verb|^|2, -81}),\\
ο τρίτος όρος δίνει το περιεχόμενο του παρονομαστή (εδώ 
{\tt 2}),\\
ο τέταρτος όρος δίνει την ακολουθία {\tt\textlatin{Sturm}} του παρονομαστή 
(εδώ {\en\tt x+1,1}).\\
{\bf Προσοχή !!!!}\\
Το $P$ ορίζεται απ' τη συμβολική του παράσταση.\\
Είσοδος  :\\  
{\en\tt sturmseq([1,0,0,1],x)},\\ 
Έξοδος :\\
{\en\tt Bad argument type ( = {\gr\tt Κακός τύπος ορίσματος})}.

\subsection{Πίνακας {\tt\textlatin{Sylvester}} δύο πολυωνύμων  : {\tt\textlatin{ sylvester}}}\index{sylvester}
\noindent{{\en\tt sylvester} παίρνει δύο πολυώνυμα σαν ορίσματα.\\
{\en\tt sylvester} επιστρέφει τον πίνακα {\tt\textlatin{Sylvester}}  $S$ αυτών των δύο πολυωνύμων.}\\
Εάν $A(x)=\sum_{i=0}^{i=n} a_ix^i$ και 
$B(x)=\sum_{i=0}^{i=m}b_ix^i$ είναι δύο πολυώνυμα, ο πίνακας {\tt\textlatin{Sylvester}} τους 
$S$ είναι ένας τετραγωνικός πίνακας μεγέθους {\en\tt m+n} όπου {\en\tt m=degree(B(x))} και 
{\en\tt n=degree(A(x))}. Οι {\en\tt m} πρώτες γραμμές γίνονται με τους συντελεστές του  $A(x)$,
έτσι ώστε  :
$$\left(\begin{array}{ccccccc}
s_{11}=a_n & s_{12}=a_{n-1}& \cdots & s_{1(n+1)}=a_0 & 0 & \cdots & 0\\
s_{21}=0 & s_{22}=a_{n}& \cdots & s_{2(n+1)}=a_1 & s_{2(n+2)}=a_0 & \cdots & 0\\
\vdots &\vdots &\vdots &\ddots &\vdots &\ddots &\vdots\\
s_{m1}=0 & s_{m2}=0& \cdots & s_{m(n+1)}=a_{m-1} & s_{m(n+2)}=a_{m-2} & \cdots&a_0 
\end{array}\right)$$
και οι επόμενες {\en\tt n} γραμμές  γίνονται με τους συντελεστές του $B(x)$,
έτσι ώστε  :
$$\left(\begin{array}{ccccccc}
s_{(m+1)1}=b_m & s_{(m+1)2}=b_{m-1}& \cdots & s_{(m+1)(m+1)}=b_0 & 0 & \cdots & 0\\
\vdots &\vdots &\vdots &\ddots &\vdots &\ddots &\vdots\\
s_{(m+n)1}=0 & s_{(m+n)2}=0& \cdots & s_{(m+n)(m+1)}=b_{n-1}  & b_{n-2}  &\cdots&b_0 
\end{array}\right)$$
Είσοδος :
\begin{center}{\en\tt sylvester(x\verb|^|3-p*x+q,3*x\verb|^|2-p,x)}\end{center}
Έξοδος :
\begin{center}{\en\tt [[1,0,-p,q,0],[0,1,0,-p,q],[3,0,-p,0,0], [0,3,0,-p,0],[0,0,3,0,-p]]}\end{center}
Είσοδος :
\begin{center}{\en\tt det([[1,0,-p,q,0],[0,1,0,-p,q],[3,0,-p,0,0], [0,3,0,-p,0],[0,0,3,0,-p]])}\end{center}
Έξοδος :
\begin{center}{\en\tt -4*p\verb|^|3--27*q\verb|^|2}\end{center}
 
\subsection{Απαλοίφουσα δύο πολυωνύμων : {\tt\textlatin{ resultant}}}\index{resultant}
\noindent{{\en\tt resultant} παίρνει σαν όρισμα δύο πολυώνυμα και
επιστρέφει την απαλοίφουσα των  πολυωνύμων αυτών.}\\
Η απαλοίφουσα τνω δύο πολυωνύμων είναι η ορίζουσα του
πίνακα {\tt\textlatin{Sylvester}} $S$. 
Ο πίνακας {\tt\textlatin{Sylvester}} $S$ δύο πολυωνύμων $A(x)=\sum_{i=0}^{i=n} a_ix^i$
και $B(x)=\sum_{i=0}^{i=m} b_ix^i$
είναι ένας τετραγωνικός πίνακας με $m+n$ γραμμές και στήλες; οι πρώτες $m$ γραμμές του
γίνονται από τους συντελεστές του  $A(X)$:
$$\left(\begin{array}{ccccccc}
s_{11}=a_n & s_{12}=a_{n-1}& \cdots & s_{1(n+1)}=a_0 & 0 & \cdots & 0\\
s_{21}=0 & s_{22}=a_{n}& \cdots & s_{2(n+1)}=a_1 & s_{2(n+2)}=a_0 & \cdots & 0\\
\vdots &\vdots &\vdots &\ddots &\vdots &\ddots &\vdots\\
s_{m1}=0 & s_{m2}=0& \cdots & s_{m(n+1)}=a_{m-1} & s_{m(n+2)}=a_{m-2} & \cdots&a_0 
\end{array}\right)$$
και οι επόμενες $n$ γραμμές γίνονται με τον ίδιο τρόπο από τους 
συντελεστές του $B(x)$ :
$$\left(\begin{array}{ccccccc}
s_{(m+1)1}=b_m & s_{(m+1)2}=b_{m-1}& \cdots & s_{(m+1)(m+1)}=b_0 & 0 & \cdots & 0\\
\vdots &\vdots &\vdots &\ddots &\vdots &\ddots &\vdots\\
s_{(m+n)1}=0 & s_{(m+n)2}=0& \cdots & s_{(m+n)(m+1)}=b_{n-1}  & b_{n-2}  &\cdots&b_0 
\end{array}\right)$$

Εάν $A$ και $B$ έχουν ακέραιους συντελεστές με μη μηδενική απαλοίφουσα $r$, 
τότε η πολυωνυμική εξίσωση 
\[ AU+BV=r\]
έχει μοναδική λύση $U,V$ τέτοια ώστε {\en \tt degree$(U) < $degree$(B)$} και 
{\en \tt degree$(V) \\< $degree$(A)$}, και αυτή η λύση έχει ακέραιους συντελεστές.

Είσοδος :
\begin{center}{\en\tt resultant(x\verb|^|3-p*x+q,3*x\verb|^|2-p,x)}\end{center}
Έξοδος :
\begin{center}{\en\tt -4*p\verb|^|3--27*q\verb|^|2}\end{center}
{\bf Σχόλιο}\\
{\en \tt discriminant(P)=resultant(P,P{\gr \tt '}).}

{\bf Ένα παράδειγμα χρήσης της απαλοίφουσας ({\tt\textlatin{resultant}})}\\
Έστω, $F1$ και $F2$ δύο σταθερά σημεία στο επίπεδο και
$A$, ένα μεταβλητό σημείο στον κύκλο κέντρου $F1$ και ακτίνας $2a$.
Βρείτε την καρτεσιανή εξίσωση του συνόλου των σημείων $M$, της τομής της 
γραμμής $F1A$ και της διχοτόμου του ευθυγράμμου τμήματος $F2A$.

Γεωμετρική απάντηση~:
\[ MF1+MF2=MF1+MA=F1A=2a\] 
γι' αυτό το $M$ είναι σε μια έλλειψη με εστίες  $F1,F2$ και κύριο άξονα $2a$.

Αναλυτική απάντηση~:
Στο καρτεσιανό σύστημα συντεταγμένων κέντρου $F1$ 
και με  άξονα $x$ να έχει την ίδια διεύθυνση 
με το διάνυσμα $F1F2$, οι συντεταγμένες του $A$ είναι :
\[  A= (2a\cos(\theta);2a\sin(\theta)) \] 
όπου $\theta$ είναι η γωνία $(Ox,OA)$.
Τώρα επιλέξτε για παράμετρο την $t=\tan(\theta/2)$  , έτσι ώστε οι συντεταγμένες
του $A$ να είναι ρητές συναρτήσεις ως προς το $t$.
Πιο συγκεκριμένα~:
\[ A=(ax;ay)=(2a\frac{1-t^2}{1+t^2};2a\frac{2t}{1+t^2}) \]
Εάν $F1F2=2c$ και εάν $I$ είναι το μέσο σημείο του $AF2$, 
αφού οι συντεταγμένες του $F2$ είναι $F2=(2c,0)$, οι συντεταγμένες
του $I$ είναι
\[ I=(c+ax/2;ay/2)=(c+a\frac{1-t^2}{1+t^2};a\frac{2t1-t^2}{1+t^2}) \]
Η $IM$ είναι κάθετη με το $AF2$, και επομένως το σημείο $M=(x;y)$ επαληθεύει την εξίσωση
$eq1=0$ όπου
\[ eq1:=(x-ix)*(ax-2*c)+(y-iy)*ay \]
Αλλά το $M=(x;y)$ είναι επίσης επάνω στην $F1A$, γι' αυτό $M$ επαληθεύει την εξίσωση $eq2=0$, όπου
\[ eq2:=y/x-ay/ax \]
Η απαλοίφουσα ({\tt\textlatin{resultant}}) των δύο εξισώσεων κατ' ως προς $t$ 
{\en\tt resultant(eq1,eq2,t)} είναι ένα πολυώνυμο $eq3$ με
μεταβλητές $x,y$, ανεξάρτητο του $t$ και το οποίο είναι η καρτεσιανή εξίσωση
του συνόλου των σημείων $M$ όταν το $t$ μεταβάλεται.
Είσοδος :\\
{\en\tt ax:=2*a*(1-t\verb|^|2)/(1+t\verb|^|2);ay:=2*a*2*t/(1+t\verb|^|2);}\\
{\en\tt ix:=(ax+2*c)/2; iy:=(ay/2)}\\
{\en\tt eq1:=(x-ix)*(ax-2*c)+(y-iy)*ay}\\
{\en\tt eq2:=y/x-ay/ax}\\
{\en\tt factor(resultant(eq1,eq2,t))}\\
Η έξοδος δίνει σαν αποτέλσμα :\\
${\en\tt -(64\cdot(x^2+y^2)\cdot(x^2\cdot a^2-x^2\cdot c^2+-2\cdot x\cdot a^2\cdot
c+2\cdot x\cdot c^3-a^4+2\cdot a^2\cdot c^2+}$\\
${\en\tt  a^2\cdot y^2-c^4))}$\\
Ο παράγοντας ${\en\tt -64\cdot (x^2+y^2)}$ είναι πάντα διάφορος από το 0, 
γι' αυτό ο γεωμετρικός τόπος του $M$ είναι~:
\[ {\en\tt x^2a^2-x^2c^2+-2xa^2c+2xc^3-a^4+2a^2c^2+a^2y^2-c^4=0} \]
Αν η αρχή των αξόνων είναι το σημείο $O$, το μεσαίο σημείο του $F1F2$,
βρίσκουμε την καρτεσιανή εξίσωση μιας έλλειψης. 
Για να κάνουμε την αλλαγή της αρχής των αξόνων 
$\overrightarrow{F1M}=\overrightarrow{F1O}+\overrightarrow{OM}$, εισάγουμε :
\[ {\en\tt normal(subst(x^2\cdot a^2-x^2\cdot c^2+-2\cdot x\cdot a^2\cdot
c+2\cdot x\cdot c^3-a^4+2\cdot a^2\cdot c^2+} \]
\[ {\en\tt  a^2\cdot y^2-c^4,[x,y]=[c+X,Y]))} \]
Έξοδος :
\[ {\en\tt -c^2*X^2+c^2*a^2+X^2*a^2-a^4+a^2*Y^2} \]
ή εάν $b^2=a^2-c^2$, εισάγουμε :
\[ {\en\tt
  normal(subst(-c^2*X^2+c^2*a^2+X^2*a^2-a^4+a^2*Y^2,c^2=a^2-b^2))} \]
Έξοδος :
\[ {\en\tt -a^2*b^2+a^2*Y^2+b^2*X^2} \]
δηλαδή, μετά τη διαίρεση με το $a^2*b^2$, το $M$ επαληθεύει την εξίσωση :
\[ \frac{X^2}{a^2}+\frac{Y^2}{b^2}=1 \]

{\bf Άλλο παράδειγμα χρήσης της απαλοίφουσας ({\tt\textlatin{resultant}})}\\
Έστω $F1$ και $F2$ δύο σταθερά σημεία και το $A$ ένα μεταβλητό σημείο
στον κύκλο κέντρου $F1$ και ακτίνας $2a$.
Βρείτε την καρτεσιανή εξίσωση του καλύματος του $D$, της διχοτόμου του ευθύγραμμου τμήματος 
$F2A$.

Η διχοτόμος του ευθύγραμμου τμήματος του $F2A$ είναι εφαπτόμενη στην έλλειψη εστίας 
$F1,F2$ και κύριου άξονα $2a$.

Στο καρτεσιανό σύστημα συντεταγμένων κέντρου $F1$ και με τον άξονα $x$ να έχει την ίδια 
κατεύθυνση με το διάνυσμα $F1F2$, οι συντεταγμένες του $A$ είναι :
\[ A= (2a\cos(\theta);2a\sin(\theta)) \]
όπου $\theta$ είναι η γωνία $(Ox,OA)$.
Επιλέξτε για παράμετρο την $t=\tan(\theta/2)$ , έτσι ώστε οι συντεταγμένες του $A$ είναι
ρητές συναρτήσεις ως προς $t$.
Πιο συγκεκριμένα ~:
\[ A=(ax;ay)=(2a\frac{1-t^2}{1+t^2};2a\frac{2t}{1+t^2}) \]
Εάν $F1F2=2c$ και εάν $I$ είναι το μεσαίο σημείο του $AF2$:\\
\[ F2=(2c,0), \quad
I=(c+ax/2;ay/2)=(c+a\frac{1-t^2}{1+t^2};a\frac{2t1-t^2}{1+t^2}) 
\]
Αφού $D$ είναι ορθογώνιο στο $AF2$, η εξίσωση του $D$ είναι
$eq1=0$ όπου
\[ eq1:=(x-ix)*(ax-2*c)+(y-iy)*ay \]
Έτσι, το κάλυμα του $D$ είναι ο γεωμετρικός τόπος του $M$, το σημείο τομής του $D$ 
και $D'$ όπου $D'$ έχει εξίσωση $eq2:=diff(eq1,t)=0$.\\
Είσοδος :\\
{\en\tt ax:=2*a*(1-t\verb|^|2)/(1+t\verb|^|2);ay:=2*a*2*t/(1+t\verb|^|2);}\\
{\en\tt ix:=(ax+2*c)/2; iy:=(ay/2)}\\
{\en\tt eq1:=normal((x-ix)*(ax-2*c)+(y-iy)*ay)}\\
{\en\tt eq2:=normal(diff(eq1,t))}\\
{\en\tt factor(resultant(eq1,eq2,t))}\\
Η έξοδος δίνει απαλοίφουσα ({\tt\textlatin{resultant}}) :\\
${\en\tt (-(64\cdot a\verb|^|2))\cdot(x\verb|^|2+y\verb|^|2)\cdot(x\verb|^|2\cdot a\verb|^|2-x\verb|^|2\cdot c\verb|^|2+-2\cdot x\cdot a\verb|^|2\cdot c+}$\\
${\en\tt 2\cdot x\cdot c\verb|^|3-a\verb|^|4+2\cdot a\verb|^|2\cdot c\verb|^|2+a\verb|^|2\cdot y\verb|^|2-c\verb|^|4)}$\\
Ο παράγοντας ${\en\tt -64\cdot (x^2+y^2)}$ είναι πάντα διάφορος του 0, 
τότε η  εξίσωση του γεωμετρικού τόπου είναι :
\[ {\en\tt x^2a^2-x^2c^2+-2xa^2c+2xc^3-a^4+2a^2c^2+a^2y^2-c^4=0} \]
Εάν το $O$, το μέσο σημείο του $F1F2$, επειλεχθεί σαν αρχή των αξόνων,
βρίσκουμε ξανά την καρτεσιανή εξίσωση της έλλειψης~:
\[ \frac{X^2}{a^2}+\frac{Y^2}{b^2}=1 \]

\section{Ορθογώνια πολυώνυμα}
\subsection{Πολυώνυμα {\tt\textlatin{Legendre}} : {\tt\textlatin{ legendre}}}\index{legendre}
\noindent{{\en\tt legendre} παίρνει σαν όρισμα έναν ακέραιο $n$ και
προαιρετικά το όνομα μιας μεταβλητής (από προεπιλογή $x$).}\\
{\tt\textlatin{legendre}} επιστρέφει το πολυώνυμο {\tt\textlatin{Legendre}} βαθμού $n$ : δηλαδή το
 πολυώνυμο $L(n,x)$, που είναι λύση της διαφορικής εξίσωσης:
$$(x^2-1).y''-2.x.y'-n(n+1).y=0$$ 
Τα πολυώνυμα {\tt\textlatin{Legendre}} επαληθεύουν την ακόλουθη αναδρομική σχέση :
\[ L(0,x)=1, \quad 
L(1,x)=x, \quad
L(n,x)=\frac{2n-1}{n}x L(n-1,x)-\frac{n-1}{n}L(n-2,x)
\]
Αυτά τα πολυώνυμα είναι ορθογώνια ως προς το εσωτερικό γινόμενο :
\[ <f,g>=\int_{-1}^{+1}f(x)g(x)\ dx \]
Είσοδος :
\begin{center}{\en\tt legendre(4)}\end{center}
Έξοδος :
\begin{center}{\en\tt (35*x\verb|^|4+-30*x\verb|^|2+3)/8}\end{center}
Είσοδος :
\begin{center}{\en\tt legendre(4,y)}\end{center}
Έξοδος :
\begin{center}{\en\tt (35*y\verb|^|4+-30*y\verb|^|2+3)/8}\end{center}

\subsection{Πολυώνυμα {\tt\textlatin{Hermite}} : {\tt\textlatin{ hermite}}}\index{hermite}
\noindent{{\en\tt hermite} παίρνει ως όρισμα ένα ακέραιο $n$ και 
 προαιρετικά το όνομα μιας μεταβλητής (από προεπιλογή $x$).}\\
{\en\tt hermite} επιστρέφει το πολυώνυμο {\tt\textlatin{Hermite}} βαθμού $n$.\\
Εάν $H(n,x)$ δηλώνει  το πολυώνυμο {\tt\textlatin{Hermite}} βαθμού $n$,
 ισχύει η ακόλουθη αναδρομική σχέση:
\[  H(0,x)=1, \quad
H(1,x)=2x, \quad
H(n,x)=2xH(n-1,x)-2(n-1)H(n-2,x) \]
Αυτά τα πολυώνυμα είναι ορθογώνια ως προς το εσωτερικό γινόμενο :
\[ <f,g>=\int_{-\infty}^{+\infty}f(x)g(x)e^{-x^2}dx \]
Είσοδος :
\begin{center}{\en\tt hermite(6)}\end{center}
Έξοδος :
\begin{center}{\en\tt 64*x\verb|^|6+-480*x\verb|^|4+720*x\verb|^|2-120}\end{center}Input :
\begin{center}{\en\tt hermite(6,y)}\end{center}
Έξοδος :
\begin{center}{\en\tt 64*y\verb|^|6+-480*y\verb|^|4+720*y\verb|^|2-120}\end{center}

\subsection{Πολυώνυμα {\tt\textlatin{Laguerre}} : {\tt\textlatin{ laguerre}}}\index{laguerre}
\noindent{{\en\tt laguerre} παίρνει ως όρισμα έναν ακέραιο $n$ και προαιρετικά
το όνομα μιας μεταβλητής (από προεπιλογή $x$) και το όνομα μιας παραμέτρου  (από προεπιλογή $a$).}\\
{\en\tt laguerre} επιστρέφει το πολυώνυμο {\tt\textlatin{Laguerre}} βαθμού $n$ και παραμέτρου $a$.\\
Εάν $L(n,a,x)$ δηλώνει το πολυώνυμο  {\tt\textlatin{Laguerre}} βαθμού $n$ και 
παραμέτρου $a$, ισχύει η ακόλουθη αναδρομική σχέση :
\[ L(0,a,x)=1, \quad
L(1,a,x)=1+a-x, \quad
L(n,a,x)=\frac{2n+a-1-x}{n}L(n-1,a,x)-\frac{n+a-1}{n}L(n-2,a,x) \]
Αυτα τα πολυώνυμα είναι ορθογώνια ως προς το εσωτερικό γινόμενο
\[ <f,g>=\int_{0}^{+\infty}f(x)g(x)x^ae^{-x}dx \]
Είσοδος :
\begin{center}{\en\tt laguerre(2)}\end{center}
Έξοδος :
\begin{center}{\en\tt (a\verb|^|2+-2*a*x+3*a+x\verb|^|2+-4*x+2)/2}\end{center}
Είσοδος :
\begin{center}{\en\tt laguerre(2,y)}\end{center}
Έξοδος :
\begin{center}{\en\tt (a\verb|^|2+-2*a*y+3*a+y\verb|^|2+-4*y+2)/2}\end{center}
Είσοδος :
\begin{center}{\en\tt laguerre(2,y,b)}\end{center}
Έξοδος :
\begin{center}{\en\tt (b\verb|^|2+-2*b*y+3*b+y\verb|^|2+-4*y+2)/2}\end{center}

\subsection{Πολυώνυμα {\tt\textlatin{Tchebychev}} πρώτου τύπου:{\tt\textlatin{tchebyshev1}}}\index{tchebyshev1}
\noindent {{\en\tt tchebyshev1} παίρνει σαν όρισμα έναν ακέραιο $n$ και προαιρετικά το
όνομα μιας μεταβλητής (από προεπιλογή $x$).}\\
{\en\tt tchebyshev1} επιστρέφει το πολυώνυμο {\tt\textlatin{Tchebychev}} πρώτου τύπου
και βαθμού $n$. Το $T(n,x)$ ορίζεται ως
\[ T(n,x)= \cos(n.\arccos(x)) \]
και επαληθέυει την αναδρομική σχέση:
\[ T(0,x)=1, \quad 
T(1,x)=x, \quad T(n,x)=2xT(n-1,x)-T(n-2,x) \]
Τα πολυώνυμα  $T(n,x)$  είναι ορθογώνια ως προς το εσωτερικό γινόμενο
\[ <f,g>=\int_{-1}^{+1}\frac{f(x)g(x)}{\sqrt{1-x^2}}dx \]
Είσοδος :
\begin{center}{\en\tt tchebyshev1(4)}\end{center}
Έξοδος :
\begin{center}{\en\tt 8*x\verb|^|4+-8*x\verb|^|2+1}\end{center}
Είσοδος :
\begin{center}{\en\tt tchebyshev1(4,y)}\end{center}
Έξοδος :
\begin{center}{\en\tt 8*y\verb|^|4+-8*y\verb|^|2+1}\end{center}
Πράγματι
\begin{eqnarray*}
\cos( 4.x)&=&Re((\cos(x)+i.\sin(x))^4) \\
          &=&\cos(x)^4-6.\cos(x)^2.(1-\cos(x)^2)+((1-\cos(x)^2)^2 \\
          &=&T(4,\cos(x))
\end{eqnarray*}

\subsection{Πολυώνυμα {\tt\textlatin{Tchebychev}} δεύτερου τύπου:{\tt\textlatin{tchebyshev2}}}\index{tchebyshev2}
\noindent{{\en\tt tchebyshev2} παίρνει σαν όρισμα ένα ακέραιο $n$ και προαιρετικά  
το όνομα μιας μεταβλητής (από προεπιλογή $x$).}\\
{\en\tt tchebyshev2} επιστρέφει το πολυώνυμο {\tt\textlatin{Tchebychev}} δευτέρου τύπου 
βαθμού $n$.\\
Το  πολυώνυμο {\tt\textlatin{Tchebychev}} δευτέρου τύπου $U(n,x)$ ορίζεται ως:
$$U(n,x)=\frac{\sin((n+1).\arccos(x))}{\sin(\arccos(x))}$$
ή αντίστοιχα:
$$\sin((n+1)x)=\sin(x)*U(n,\cos(x))$$ 
Η $U(n,x)$ επαληθεύει την αναδρομική σχέση :
\[ U(0,x)=1, \quad
U(1,x)=2x, \quad
U(n,x)=2xU(n-1,x)-U(n-2,x) \]
Τα πολυώνυμα  $U(n,x)$ είναι ορθόγων ως προς το εσωτερικό γινόμενο
\[ <f,g>=\int_{-1}^{+1}f(x)g(x)\sqrt{1-x^2}dx \]
Είσοδος :
\begin{center}{\en\tt tchebyshev2(3)}\end{center}
Έξοδος :
\begin{center}{\en\tt 8*x\verb|^|3+-4*x}\end{center}
Είσοδος :
\begin{center}{\en\tt tchebyshev2(3,y)}\end{center}
Έξοδος :
\begin{center}{\en\tt 8*y\verb|^|3+-4*y}\end{center}
Πράγματι:
\[ \sin(4.x)=\sin(x)*(8*\cos(x)^3-4.\cos(x))=\sin(x)*U(3,\cos(x)) \]




\section{Βάση {\tt\textlatin{Gr\"obner}} και αναγωγή {\tt\textlatin{Gr\"obner}}}
 \subsection{Βάση \tt\textlatin{Gr\"obner}: {\tt\textlatin{ gbasis}}}\index{gbasis}
\label{sec:gbasis}
\noindent{{\en\tt gbasis} παίρνει τουλάχιστον δύο ορίσματα
\begin{itemize}
\item ένα διάνυσμα πολυωνύμων με πολλές μεταβλητές 
\item ένα διάνυσμα με τα ονόματα μεταβλητών,
\end{itemize}
Προαιρετικά ορίσματα μπορεί να χρησιμοποιηθούν για να ορίσουμε τη διάταξη και τους αλγορίθμους. 
Από προεπιλογή, η διάταξη είναι λεξικογραφική (αναφορικά με την λίστα
των διατεταγμένων ονομάτων των μεταβλητών) και τα πολυώνυμα γράφονται σε φθίνουσα σειρά βαθμών αναφορικά με την διάταξη
Για παράδειγμα, η έξοδος θα είναι 
$...+x^2 y^4 z^3+x^2 y^3 z^4+...$ εάν το δεύτερο όρισμα είναι $[x,y,z]$ επειδή
$(2,4,3)>(2,3,4)$ αλλά η έξοδος θα είναι 
$...+x^2 y^3z^4+x^2 y^4 z^3+...$  εάν το δεύτερο όρισμα είναι $[x,z,y]$.}\\
{\en\tt gbasis} επιστρέφει μια  βάση { \tt\textlatin{Gr\"obner}}
 του πολυωνυμικού ιδεώδους που παράγεται 
από αυτά τα πολυώνυμα.

{\bf Ιδιότητα}\\
Εάν $I$ είναι ένα ιδανικό και εάν $(G_k)_{k \in K}$ είναι μια βάση {\tt\textlatin{Gr\"obner}} του
ιδανικού $I$ τότε, εάν $F$ είναι ένα μη μηδενικό πολυώνυμο στο $I$, το μεγαλύτερο μονώνυμο
του $F$ διαιρείται από το μεγαλύτερο μονώνυμο ενός  εκ των πολυωνύμων $G_k$ της βάσης.
Μ' άλλα λόγια, εάν κάνετε μια Ευκλέιδεια διαίρεση του $F\neq 0$ 
με το αντίστοιχο $G_k$, πάρτε το υπόλοιπο αυτής της διαίρεσης, επαναλάβετε το ίδιο και ούτω καθεξής,  σε κάποιο σημείο θα πάρετε υπόλοιπο μηδέν.
  
Είσοδος :
\begin{center}{\en\tt gbasis([2*x*y-y\verb|^|2,x\verb|^|2-2*x*y],[x,y])}
\end{center}
Έξοδος :
\begin{center}{\en\tt
 [4*x\verb|^|2+-4*y\verb|^|2,2*x*y-y\verb|^|2,-(3*y\verb|^|3)]}
\end{center}

Όπως αναφέραμε πριν, η {\en\tt gbasis} μπορεί να έχει περισσότερα από δύο ορίσματα ~:
\begin{itemize}
\item {\en\tt plex} (λεξικογραφική μόνο) είναι η διάταξη από προεπιλογή. Άλλες επιλογές για να ορίσουμε την διάταξη των μονωνύμων είναι: {\en\tt tdeg} (ολικού βαθμού --- {\tt\textlatin{total degree}} --- και μετά
λεξικογραφική διάταξη),
{\en\tt revlex} (ολικού βαθμού --- {\tt\textlatin{total degree}} --- και μετά αντίστροφη λεξικογραφική διάταξη),
\item {\en\tt with\_cocoa=true} ή {\en\tt with\_cocoa=false}, εάν θέλετε να χρησιμοποιήσετε
την βιλιοθήκη {\en\tt CoCoA} για να υπολογίσετε την βάση {\tt\textlatin{Gr\"obner}} (συνιστάται,
αλλά απαιτείται η εγκατάσταση της  {\en\tt CoCoA}).

\item {\en\tt with\_f5=true} ή {\en\tt with\_f5=false} εάν θέλετε να χρησιμοποιήσετε
 τον αλγόριθμο {\tt\textlatin{F5}} της βιβλιοθήκης {\en\tt CoCoA}.
Σε αυτή την περίπτωση η καθορισθείσα διάταξη δεν χρησιμοποιείται (τα πολυώνυμα είναι 
ομογενοποιημένα).
\end{itemize}
Είσοδος~:
\begin{center}
{\en\tt gbasis([x1+x2+x3,x1*x2+x1*x3+x2*x3,x1*x2*x3-1], [x1,x2,x3],tdeg,with\_cocoa=false)}
\end{center}
Έξοδος
\begin{center}{\en\tt
\verb|[x3^3-1,-x2^2-x2*x3-x3^2,x1+x2+x3]|}
\end{center}

\subsection{Αναγωγή {\tt\textlatin{Gr\"obner}} : {\tt\textlatin{ greduce}}}\index{greduce}
\noindent{{\en\tt greduce} έχει τρία ορίσματα : ένα πολυώνυμο
πολλών μεταβλητών, 
ένα διάνυσμα από πολυώνυμα  που πρέπει να είναι μια βάση  {\tt\textlatin{Gr\"obner}},
και ένα διάνυσα με ονόματα μεταβλητών.}\\
{\en\tt greduce} επιστρέφει την αναγωγή 
του πολυωνύμου που δίνεται ως πρώτο όρισμα
αναφορικά με τη βάση {\tt\textlatin{Gr\"obner}} που δίνεται σαν δεύτερο όρισμα.
Είναι 0 αν και μόνο αν το πολυώνυμο ανήκει στο ιδεώδες που παράγεται από την βάση.

Είσοδος :
\begin{center}{\en\tt greduce(x*y-1,[x\verb|^|2-y\verb|^|2,2*x*y-y\verb|^|2,y\verb|^|3],[x,y])}\end{center}
Έξοδος :
\begin{center}{\en\tt (1/2)*y\verb|^|2-1}\end{center}
δηλαδή  $xy-1=\frac{1}{2}(y^2-2)\ \bmod I$ όπου $I$ είναι το ιδεώδες
που παράγεται από την βάση {\tt\textlatin{Gr\"obner}}  $[x^2-y^2,2xy-y^2,y^3]$, διότι 
$ \frac{1}{2}(y^2-2)$ είναι το υπόλοοιπο της Ευκλείδειας διαίρεσης του $2(xy-1)$ με το $G_2=2x y-y^2$.\\
% {\bf Σημείωση}\\
% Η πολλαπλασιαστική σταθερά μπορεί να έιναι \tt\textlatin{fixed} παρατηρώντας πως ο σταθερός
% συντελεστής μετατρέπεται. Στο παράδειγμα, ο σταθερός συντελεστής
% {\tt -1} μετατρέπεται σε σταθερό συντελεστή {\tt -2}, έτσι
% η πολλπλασιαστική σταθερά είναι {\tt 1/2}.

Όπως και η εντολή {\en\tt gbasis} (βλέπε \ref{sec:gbasis}),
έτσι και η εντολή {\tt\textlatin{greduce}} μπορεί να έχει περισσότερα από τρία ορίσματα για να καθορίσει τον αλγόριθμο και την 
διάταξη αν διαφέρει από την προεπιλογή (λεξικογραφική διάταξη).\\
Είσοδος~:
\begin{center}
{\en\tt greduce(x1\verb|^|2*x3\verb|^|2,[x3\verb|^|3-1,-x2\verb|^|2-x2*x3-x3\verb|^|2,x1+x2+x3], [x1,x2,x3],tdeg)}
\end{center}
Έξοδος
\begin{center}{\en\tt
\verb|x2|}
\end{center}

\subsection{Δημιουργία πολυωνύμου από την αποτίμησή του :\\ {\tt\textlatin{ genpoly}}}\index{genpoly}
\noindent{{\en\tt genpoly} παίρνει τρία ορίσματα : ένα πολυώνυμο $P$ με  $n-1$ 
μεταβλητές, έναν ακέραιο $b$ και το όνομα μιας μεταβλητής {\en\tt var}.}\\ 
{\en\tt genpoly} επιστρέφει το πολυώνυμο $Q$ με $n$ μεταβλητές (τις  $n-1$ μεταβλητές 
του $P$ 
και την μεταβλητή {\en\tt var} που δίνεται ως τρίτο όρισμα), τέτοιο ώστε~:
\begin{itemize}
\item {\en\tt subst(Q,var=b)==P} 
\item οι συντελεστές του $Q$ ανήκουν στο διάστημα  $]-b/2 \ , \ b/2]$
\end{itemize}
Με άλλα λόγια,το  $P$ γράφεται ως προς την  βάση $b$ αλλά με την σύμβαση
ότι το Ευκλείδειο υπόλοιπο είναι μέσα στο διάστημα $]-b/2 \ , \ b/2]$ 
(αυτή η σύμβαση είναι επίσης γνωστή σαν αναπαράσταση {\tt\textlatin{s-mod}}).\\
Είσοδος :
\begin{center}{\en\tt genpoly(61,6,x) }\end{center}
Έξοδος :
\begin{center}{\en\tt 2*x\verb|^|2-2*x+1}\end{center}
Πράγματι το 61 διαιρεμένο με το 6 δίνει πηλίκο 10, και υπόλοιπο 1, μετά το 10 διαιρεμένο με το 6 δίνει πηλίκο 2
και υπόλοιπο -2 (αντί για το συνηθισμένο πηλίκο 1 και υπόλοιπο 4 που είναι εκτός ορίων),
\[ 61=2*6^2-2*6+1 \]
Είσοδος :
\begin{center}{\en\tt genpoly(5,6,x) }\end{center}
Έξοδος :
\begin{center}{\en\tt x-1}\end{center}
Πράγματι : $5=6-1$\\
Είσοδος :
\begin{center}{\en\tt genpoly(7,6,x) }\end{center}
Έξοδος :
\begin{center}{\en\tt x+1}\end{center}
Πράγματι : $7=6+1$\\
Είσοδος :
\begin{center}{\en\tt genpoly(7*y+5,6,x) }\end{center}
Έξοδος :
\begin{center}{\en\tt x*y+x+y-1}\end{center}
Πράγματι : $x*y+x+y-1=y(x+1)+(x-1)$\\
Είσοδος :
\begin{center}{\en\tt genpoly(7*y+5*z,6,x)}\end{center}
Έξοδος :
\begin{center}{\en\tt x*y+x*z+y-z}\end{center}
Πράγματι : $x*y+x*z+y-z=y*(x+1)+z*(x-1)$

\section{Ρητά κλάσματα}
\subsection{Αριθμητής : {\tt\textlatin{ getNum}}}\index{getNum}\label{sec:getnum}
\noindent{{\en\tt getNum} παίρνει σαν όρισμα ένα ρητό κλάσμα 
και επιστρέφει τον αριθμητή αυτού του κλάσματος. Αντίθετα με την {\en\tt numer},
η {\en\tt getNum} δεν απλοποεί το κλάσμα πριν εξάγει τον 
αριθμητή.}\\
Είσοδος :
\begin{center}{\en\tt getNum((x\verb|^|2-1)/(x-1)) }\end{center}
Έξοδος :
\begin{center}{\en\tt x\verb|^|2-1}\end{center}
Είσοδος :
\begin{center}{\en\tt getNum((x\verb|^|2+2*x+1)/(x\verb|^|2-1)) }\end{center}
Έξοδος :
\begin{center}{\en\tt x\verb|^|2+2*x+1}\end{center}

\subsection{Αριθμητής μετά από απλοποίηση : {\tt\textlatin{ numer}}}\index{numer}\label{sec:numer}
\noindent{{\en\tt numer} παίρνει σαν όρισμα ένα ρητό κλάσμα
και επιστρέφει τον αριθμητή ενός ανάγωγου αντιπροσώπου αυτού του κλάσματος
(δείτε επίσης \ref{sec:inumer}).}\\
Είσοδος :
\begin{center}{\en\tt numer((x\verb|^|2-1)/(x-1)) }\end{center}
Έξοδος :
\begin{center}{\en\tt x+1}\end{center}
Είσοδος :
\begin{center}{\en\tt numer((x\verb|^|2+2*x+1)/(x\verb|^|2-1)) }\end{center}
Έξοδος :
\begin{center}{\en\tt x+1}\end{center}
 
 \subsection{Παρονομαστής : {\tt\textlatin{getDenom}}}\index{getDenom}\label{sec:getdenom}
\noindent{{\en\tt getDenom} παίρνει σαν όρισμα ένα ρητό κλάσμα και επιστρέφει τον
παρονομαστή του κλάσματος. Αντίθετα με την {\en\tt denom},
η {\en\tt getDenom} δεν απλοποιεί το κλάσμα πριν εξάγει τον 
παρονομαστή.}\\
Είσοδος :
\begin{center}{\en\tt getDenom((x\verb|^|2-1)/(x-1)) }\end{center}
Έξοδος :
\begin{center}{\en\tt x-1}\end{center}
Είσοδος :
\begin{center}{\en\tt getDenom((x\verb|^|2+2*x+1)/(x\verb|^|2-1)) }\end{center}
Έξοδος :
\begin{center}{\en\tt x\verb|^|2-1}\end{center}

\subsection{Παρονομαστής μετά από απλοποίηση : {\tt\textlatin{ denom}}}\index{denom}\label{sec:denom}
\noindent{{\en\tt denom}  παίρνει σαν όρισμα ένα ρητό κλάσμα
και επιστρέφει τον παρονομαστή ενός ανάγωγου αντιπροσώπου αυτού του κλάσματος
(δείτε επίσης \ref{sec:idenom}).}\\
Είσοδος :
\begin{center}{\en\tt denom((x\verb|^|2-1)/(x-1)) }\end{center}
Έξοδος :
\begin{center}{\tt 1}\end{center}
Είσοδος :
\begin{center}{\en\tt denom((x\verb|^|2+2*x+1)/(x\verb|^|2-1)) }\end{center}
Έξοδος :
\begin{center}{\en\tt x-1}\end{center}

\subsection{Αριθητής και παρονομαστής : {\tt\textlatin{ f2nd fxnd}}}\index{fxnd|textbf}\index{f2nd|textbf}\label{sec:fxnd}
\noindent{{\en\tt f2nd} (ή {\en\tt fxnd}) παίρνει σαν όρισμα ένα ρητό κλάσμα
και επιστρέφει μια λίστα με τον αριθμητή και τον παρονομαστή ενός
ανάγωγου αντιπροσώπου του κλάσματος (δείτε επίσης \ref{sec:ifxnd}).}\\
Είσοδος :
\begin{center}{\en\tt f2nd((x\verb|^|2-1)/(x-1)) }\end{center}
Έξοδος :
\begin{center}{\en\tt [x+1,1]}\end{center}
Είσοδος :
\begin{center}{\en\tt f2nd((x\verb|^|2+2*x+1)/(x\verb|^|2-1)) }\end{center}
Έξοδος :
\begin{center}{\en\tt [x+1,x-1]}\end{center}

\subsection{Απλοποίηση : {\tt\textlatin{ simp2}}}\index{simp2}\label{sec:simp2}
\noindent{{\en\tt simp2} παίρνει σαν όρισμα δύο πολυώνυμα (ή δύο ακεραίους\ δείτε
\ref{sec:isimp2}).
Αυτα τα δύο πολυώνυμα θεωρούνται ο  αριθμητής και ο παρονομαστής
ενός ρητού κλάσματος.}\\ 
{\en\tt simp2} επιστρέφει μια λίστα δύο πολυωνύμων που θεωρούνται ο αριθμητής και ο παρονομαστής
ενός ανάγωγου
αντιπροσώπου του ρητού κλάσματος.\\ 
 Είσοδος :
\begin{center}{\en\tt simp2(x\verb|^|3-1,x\verb|^|2-1)}\end{center}
Έξοδος :
\begin{center}{\en\tt  [x\verb|^|2+x+1,x+1]}\end{center} 

\subsection{Κοινός παρονομαστής : {\tt\textlatin{ comDenom}}}\index{comDenom|textbf}
\noindent{{\en\tt comDenom} παίρνει σαν όρισμα ένα άθροισμα από ρητά κλάσματα.}\\
{\en\tt comDenom} αναγράφει το άθροισμα σε μοναδικό ρητό κλάσμα. Ο παρονομαστής αυτού 
του ρητού κλάσματος είναι ο κοινός παρονομαστής των ρητών κλασμάτων που δόθηκαν ως όρισμα.\\
Είσοδος :
\begin{center}{\en\tt comDenom(x-1/(x-1)-1/(x\verb|^|2-1))}\end{center}
Έξοδος:
\begin{center}{\en\tt (x\verb|^|3+-2*x-2)/(x\verb|^|2-1)}\end{center} 

\subsection{Ακέραιο και κλασματικό μέρος : {\tt\textlatin{ propfrac}}}\index{propfrac}\label{sec:propfrac}
\noindent{{\en\tt propfrac} παίρνει σαν όρισμα ένα ρητό κλάσμα.}\\
{\en\tt propfrac} αναγράφει αυτό το ρητό κλάσμα σαν το άθροισμα του
ακέραιου μέρους και του κατάλληλου κλασματικού μέρους.\\
{\en\tt propfrac(A(x)/B(x))} επιστρέφει το κλάσμα $\frac{A(x)}{B(x)}$ (μετά την
αναγωγή), σαν :
\[ Q(x)+\frac{R(x)}{B(x)} \quad  \mbox{ όπου } R(x)=0 
\mbox{ ή } 0\leq \mbox{\en\tt degree}(R(x))< \mbox{\en\tt degree}(B(x)) \]
Είσοδος :
\begin{center}{\en\tt  propfrac((5*x+3)*(x-1)/(x+2))}\end{center}
Έξοδος :
\begin{center}{\en\tt 5*x-12+21/(x+2)}\end{center}

\subsection{Μερικό κλασματικό ανάπτυγμα  : {\tt\textlatin{ partfrac}}}\index{partfrac|textbf}\label{sec:convertparf}
{\en\tt partfrac} παίρνει σαν όρισμα ένα ρητό κλάσμα.\\
{\en\tt partfrac} επιστρέφει το μερικό κλασματικό ανάπτυγμα   αυτού του ρητού 
κλάσματος.\\
Η εντολή {\en\tt partfrac} είναι ισοδύναμη με την εντολή {\en\tt convert} με επιλογή  την
{\en\tt parfrac} (ή {\en\tt partfrac} ή {\en\tt fullparfrac}) 
(δείτε επίσης \ref{sec:convert}).\\
{\bf Παράδειγμα} :\\
Βρείτε το μερικό κλασματικό ανάπτυγμα :
$$\frac{x^5-2x^3+1}{x^4-2x^3+2x^2-2x+1}$$
Είσοδος :
\begin{center}{\en\tt partfrac((x\verb|^|5-2*x\verb|^|3+1)/(x\verb|^|4-2*x\verb|^|3+2*x\verb|^|2-2*x+1))}\end{center}
Έξοδος σε τρόπο λειτουργίας για πραγματικούς :
\begin{center}{\en\tt x+2-1/(2*(x-1))+(x-3)/(2*(x\verb|^|2+1)) }\end{center}
Έξοδος σε τρόπο λειτουργίας για μιγαδικούς :
\begin{center}{\en\tt x+2+(-1+2*i)/((2-2*i)*((i)*x+1))+1/(2*(-x+1))+}\end{center}
\begin{center}{\en\tt (-1-2*i)/((2-2*i)*(x+i))}\end{center}

\section{Ακριβείς ρίζες πολυωνύμου}
\subsection{Ακριβή όρια για μιγαδικές ρίζες πολυωνύμου :\\ 
{\tt\textlatin{ complexroot}}}\index{complexroot}
\noindent{{\en\tt complexroot} παίρνει δύο ή τέσσερα ορίσματα : ένα πολυώνυμο και έναν πραγματικό
αριθμό $\epsilon$ και προαιρετικά 2 μιγαδικούς αριθμούς $\alpha,\beta$.}\\
{\en\tt complexroot} επιστρέφει μια λίστα διανυσμάτων. 
\begin{itemize}
\item Εάν η {\en\tt complexroot} έχει δύο ορίσματα, 
τα στοιχεία κάθε διανύσματος είναι :
\begin{itemize}
\item είτε ένα διάστημα (τα όρια αυτού του διαστήματος είναι αντίθετες κορυφές ενός ορθογωνίου με πλευρές
παράλληλες στους άξονες και που περιέχει μια μιγαδική ρίζα του πολυωνύμου) και την  
πολλαπλότητα της ρίζας.\\
Έστω ότι το διάστημα είναι  $[a_1+ib_1,a_2+ib_2]$ τότε $|a_1-a_2|<\epsilon$,  
$|b_1-b_2|<\epsilon$ και η ρίζα  $a+ib$ επαληθεύει
$a_1\leq a \leq a_2$ και  $b_1\leq b \leq b_2$.
\item είτε η τιμή μιας ακριβούς μιγαδικής ρίζας 
του πολυωνύμου και η πολλαπλότητα αυτής της ρίζας
\end{itemize}
\item Εάν η  {\en\tt complexroot} έχει τέσσερα ορίσματα,  η {\en\tt complexroot} επιστρέφει μια λίστα με
διανύσματα όπως παραπάνω, αλλά μόνο για τις ρίζες που βρίσκονται μεσα στο 
το ορθογώνιο με πλευρές παράλληλες προς τον άξονα και που έχει τα $\alpha,\beta$ σαν
αντίθετες κορυφές.\\
\end{itemize}
Για να βρείτε τις ρίζες του $x^3+1$, εισάγετε:
\begin{center}{\en\tt complexroot(x\verb|^|3+1,0.1)}\end{center}
Έξοδος :
\begin{center}{\en\tt [[-1,1],[[(4-7*i)/8,(8-13*i)/16],1],[[(8+13*i)/16,(4+7*i)/8],1]]}\end{center}
άρα, για  $x^3+1$ :
\begin{itemize}
\item -1 είναι μια ρίζα πολλαπλότητας  1, 
\item 1/2+i*$b$ είναι μια ρίζα πολλαπλότητας 1 με  $-7/8\leq b \leq
  -13/16$, 
\item 1/2+i*$c$ είναι μια ρίζα πολλαπότητας 1 με  $13/16\leq c \leq
  7/8$.
\end{itemize}
Για να βρείτε τις ρίζες της $x^3+1$ που βρίσκονται μέσα στο ορθογώνιο
των αντίθετων κορυφών $-1,1+2*i$, εισάγετε:
\begin{center}{\en\tt complexroot(x\verb|^|3+1,0.1,-1,1+2*i)}\end{center}
Έξοδος :
\begin{center}{\en\tt [[-1,1],[[(8+13*i)/16,(4+7*i)/8],1]]}\end{center} 

\subsection{Ακριβή όρια για πραγματικές ρίζες πολυωνύμου : \\{\tt\textlatin{ realroot}}}\index{realroot}
\noindent{{\en\tt realroot} έχει ένα, δύο ή τέσσερα ορίσματα : ένα πολυώνυμο και προαιρετικά έναν πραγματικό αριθμό
$\epsilon$ και δύο πραγματικούς αριθμούς $\alpha,\beta$.\\
{\en\tt realroot} επιστρέφει μια λίστα από διανύσματα.
\begin{itemize}
\item Εάν η {\en\tt realroot} έχει ένα ή δύο ορίσματα, τα στοιχεία του κάθε διανύσματος είναι
\begin{itemize}
\item 
είτε ένα πραγματικό διάστημα που περιέχει μια πραγματική ρίζα του πολυωνύμου 
και η πολλαπλότητα αυτής της ρίζας.
(Αν το διάστημα είναι $[a_1,a_2]$, και υπάρχει δεύτερο όρισμα, τότε $|a_1-a_2|<\epsilon$ και 
η ρίζα $a$ επαληθεύει την σχέση $a_1 < a < a_2$.)
\item είτε η τιμή μιας ακριβούς πραγματικής ρίζας του
πολυωνύμου (τα άκρα του διαστήματος ταυτίζονται, $a_1=a_2$) και η πολλαπλότητα αυτής της ρίζας. 
\end{itemize}
\item Εάν η  {\en\tt realroot} έχει τέσσερα ορίσματα, η {\en\tt realroot} επιστρέφει μια λίστα από
διανύσματα όπως παραπάνω, αλλά μόνο για τις ρίζες μέσα στο 
διάστημα $[\alpha,\beta]$.
\end{itemize}
{\bf Προσοχή: } Για την απομόνωση των πραγματικών ριζών (δηλαδή όταν καλούμε την {\en\tt realroot} με ένα ή δύο ορίσματα) από προεπιλογή  {\en\tt realroot} χρησιμοποιεί την μέθοδο {\en\tt  Vincent-Akritas-Strzebonski (VAS)} (βλέπε το άρθρο: {\en Alkiviadis G. Akritas, Adam W. Strzebonski: \textit{A Comparative Study of Two Real Root Isolation Methods}. Nonlinear Analysis: Modelling and Control, Vol. 10, No. 4, 297--304, 2005.}) Στην περίπτωση αυτή, για να χρησιμοποιήσουμε την μέθοδο {\en Sturm} γράφουμε {\en\tt realroot(sturm,}
 ορίσματα)}.\\ \\
Για να βρούμε τις πραγματικές ρίζες του $x^3-7x+7$, εισάγουμε:
\begin{center}{\en\tt realroot(x\verb|^|3-7x+7)}\end{center}
Έξοδος :
\begin{center}{\en\tt [[[-4,0],1],[[1,3/2],1],[[3/2,2],1]]] }\end{center} 
Εδώ χρησιμοποιήθηκε η μέθοδος {\en VAS}. Για να δούμε το αποτέλεσμα με την μέθοδο  {\en Sturm} και για $\epsilon = 1$ εισάγουμε:
\begin{center}{\en\tt realroot(sturm, x\verb|^|3-7x+7, 1)}\end{center}
Έξοδος :
\begin{center}{\en\tt [[[-7/2,-3],1],[[1,3/2],1],[[3/2,2],1]]]}\end{center} 
Για να βρούμε τις πραγματικές ρίζες του $x^3-7x+7$ στο διάστημα $[0,2]$ για $\epsilon = 1$, εισάγουμε:
\begin{center}{\en\tt realroot(x\verb|^|3-7x+7,1,0,2)}\end{center}
Έξοδος :
\begin{center}{\en\tt [[[1,3/2],1],[[3/2,2],1]]]}\end{center} 

\subsection{Ακριβείς τιμές των ρητών ριζών ενός πολυωνύμου :\\{ \tt\textlatin{ rationalroot}}}\index{rationalroot}
\noindent{\en\tt rationalroot} πάιρνει 1 ή 3 ορίσματα : ένα πολυώνυμο και 
προαιρετικά 2 πραγματικούς αριθμούς $\alpha,\beta$.
\begin{itemize}
\item Εάν η {\en\tt rationalroot} έχει 1 όρισμα, η {\en\tt rationalroot} επιστρέφει την λίστα
των τιμών των  ρητών ριζών του πολυωνύμου χωρίς τις πολλαπλότητες.
\item Εάν η {\en\tt rationalroot} έχει 3 ορίσματα, η {\en\tt rationalroot} επιστρέφει μόνο
τις ρητές ρίζες του πολυωνύμου οι οποίες είναι στο διάστημα
$[\alpha,\beta]$.
\end{itemize}
Για να βρείτε τις ρητές ρίζες του $2*x^3-3*x^2-8*x+12$, εισάγετε:
\begin{center}{\en\tt rationalroot(2*x\verb|^|3-3*x\verb|^|2-8*x+12)}\end{center}
Έξοδος :
\begin{center}{\en\tt [2,3/2,-2]}\end{center} 
Για να βρείτε τις ρητές ρίζες του $2*x^3-3*x^2-8*x+12$ στο διάστημα $[1,2]$, εισάγετε:
\begin{center}{\en\tt rationalroot(2*x\verb|^|3-3*x\verb|^|2-8*x+12,1,2)}\end{center}
Έξοδος :
\begin{center}{\en\tt [2,3/2]}\end{center} 
Για να βρείτε τις ρητές ρίζες του $2*x^3-3*x^2+8*x-12$, εισάγετε:
\begin{center}{\en\tt rationalroot(2*x\verb|^|3-3*x\verb|^|2+8*x-12)}\end{center}
Έξοδος :
\begin{center}{\en\tt [3/2]}\end{center} 
Για να βρείτε τις ρητές ρίζες του $2*x^3-3*x^2+8*x-12$, εισάγετε:
\begin{center}{\en\tt rationalroot(2*x\verb|^|3-3*x\verb|^|2+8*x-12)}\end{center}
Έξοδος :
\begin{center}{\en\tt [3/2]}\end{center} 
Για να βρείτε τις ρητές ρίζες του $(3*x-2)^2*(2x+1)=18*x^3-15*x^2-4*x+4$, εισάγετε:
\begin{center}{\en\tt rationalroot(18*x\verb|^|3-15*x\verb|^|2-4*x+4)}\end{center}
Έξοδος :
\begin{center}{\en\tt [(-1)/2,2/3]}\end{center} 

\subsection{Ακριβείς τιμές ρητών μιγαδικών ριζών πολυωνύμου : \\{\tt\textlatin{ crationalroot}}}\index{crationalroot}
\noindent{\en\tt crationalroot} παίρνει 1 ή 3 ορίσματα : ένα πολυώνυμο και   
προαιρετικά 2 μιγαδικούς αριθμούς $\alpha,\beta$.
\begin{itemize}
\item  Εάν η {\en\tt crationalroot} έχει 1 όρισμα, η {\en\tt crationalroot} επιστρέφει τη λίστα των 
ρητών μιγαδικών ριζών του
πολυωνύμου χωρίς πολλαπλότητα.
\item εάν η  {\en\tt crationalroot} έχει 3 ορίσματα, η {\en\tt crationalroot} επιστρέφει μόνο
τις ρητές μιγαδικές ρίζες του
πολυωνύμου που είναι στο ορθογώνιο με πλευρές παράλληλες στον άξονα
που έχει  τις $[\alpha,\beta]$ σαν αντίθετες κορυφές.
\end{itemize}
Για να βρείτε τις ρητές μιγαδικές ρίζες του
$(x^2+4)*(2x-3)=2*x^3-3*x^2+8*x-12$, εισάγετε :
\begin{center}{\en\tt crationalroot(2*x\verb|^|3-3*x\verb|^|2+8*x-12)}\end{center}
Έξοδος :
\begin{center}{\en\tt [2*i,3/2,-2*i]}\end{center} 

\section{Ακριβείς ρίζες και πόλοι}
\subsection{Ρίζες και πόλοι μιας ρητής συνάρτησης : {\tt\textlatin{ froot}}}\index{froot}
\noindent{{\en\tt froot} παίρνει μια ρητή συνάρτηση $F(x)$ σαν όρισμα.}\\
{\en\tt froot} επιστρέφει ένα διάνυσμα του οποίου τα στοιχεία είναι οι ρίζες και οι πόλοι
της $F[x]$. Κάθε ένα ακολουθείται από την πολλαπλότητά του.\\
Εάν το {\en\tt Xcas} δεν μπορεί να βρει τις ακριβείς τιμές των ριζών ή των πόλων,
δοκιμάζει να βρει προσεγγιστικές τιμές εάν η  $F(x)$ έχει αριθητικούς συντελεστές.\\
Είσοδος :
\begin{center}{\en\tt froot((x\verb|^|5-2*x\verb|^|4+x\verb|^|3)/(x-2)) }\end{center}
Έξοδος :
\begin{center}{\en\tt [1,2,0,3,2,-1]}\end{center}
Έτσι, για  $\displaystyle F(x)=\frac{x^5-2.x^4+x^3}{x-2}$ :
\begin{itemize}
\item $1$ είναι ρίζα πολλαπλότητας 2,
\item $0$ είναι ρίζα πολλαπλότητας 3,
\item $2$ είναι πόλος τάξης 1.
\end{itemize}
Είσοδος :
\begin{center}{\en\tt froot((x\verb|^|3-2*x\verb|^|2+1)/(x-2)) }\end{center}
Έξοδος :
\begin{center}{\en\tt [1,1,(1+sqrt(5))/2,1,(1-sqrt(5))/2,1,2,-1]}\end{center}
{\bf Σχόλιο} : για να πάρετε τις μιγαδικές ρίζες και τους πόλους, τσεκάρετε την  επιλογή {\en\tt "}στους μιγαδικούς{\en\tt "} στις Ρυθμίσεις 
 {\en\tt Cas}  (ή στην μπάρα ρυθμίσεων).\\
Είσοδος :
\begin{center}{\en\tt froot((x\verb|^|2+1)/(x-2)) }\end{center}
Έξοδος :
\begin{center}{\en\tt [-i,1,i,1,2,-1]}\end{center}

\subsection{Ρητή συνάρτηση που δίνεται από ρίζες και πόλους : \\{\tt\textlatin{ fcoeff}}}\index{fcoeff}
\noindent{{\en\tt fcoeff} έχει σαν όρισμα ένα διάνυσμα
 τα στοιχεία του οποίου είναι οι ρίζες και οι πόλοι μιας ρητής συνάρτησης 
$F[x]$, και κάθε ένα ακολουθείται από την πολλαπλότητά του.}\\
{\en\tt fcoeff} επιστρέφει την ρητή συνάρτηση $F(x)$.\\
Είσοδος :
\begin{center}{\en\tt fcoeff([1,2,0,3,2,-1]) }\end{center}
Έξοδος :
\begin{center}{\en\tt (x-1)\verb|^|2*x\verb|^|3/(x-2)}\end{center}

\section{Υπολογισμοί στο $\Z/p\Z$ ή στο $\Z/p\Z[x]$}\index{\%|textbf}\label{sec:modulaire}
Ο τρόπος με τον οποίο κάνουμε υπολογισμούς  στο  $\Z/p\Z$ ή στο  $\Z/p\Z[x]$ εξαρτάται 
από τον τρόπο λειτουργίας του συστήματος, ο οποίος καθορίζει το συντακτικό της γλώσσας :
\begin{itemize}
\item Στον τρόπο λειτουργίας {\en\tt Xcas}, ένα αντικείμενο $n$ στο $\Z/p\Z$ γράφεται 
$n \%  p$. Μερικά παραδείγματα εισόδου για :
\begin{itemize}
\item έναν ακέραιο {\en\tt n} στο $\Z/13\Z$\\ 
{\en\tt n:=12\%13}.
\item ένα διάνυσμα {\en\tt V} στο $\Z/13\Z$ \\
{\en\tt V:=[1,2,3]\%13} ή 
{\en\tt V:=[1\%13,2\%13,3\%13]}.
\item έναν πίνακα {\tt A} στο $\Z/13\Z$ \\
{\en\tt A:=[[1,2,3],[2,3,4]]\%13} ή \\
{\en\tt A:=[[1\%13,2\%13,3\%13],[[2\%13,3\%13,4\%13]]}.
\item
ένα πολυώνυμο {\en\tt A} στο $\Z/13\Z[x]$ σε συμβολική αναπαράσταση\\
{\en\tt A:=(2*x\verb|^|2+3*x-1)\%13} ή \\ 
{\en\tt A:=2\%13*x\verb|^|2+3\%13*x-1\%13}.
\item
ένα πολυώνυμο {\en\tt A} στο $\Z/13\Z[x]$ σε αναπαράσταση λίστας\\
{\en\tt A:=poly1[1,2,3]\%13} ή 
{\en\tt A:=poly1[1\%13,2\%13,3\%13]}.
\end{itemize} 
Για να ανακτύσουμε ένα αντικείμενο {\en\tt obj} με ακέραιους συντελεστές αντί για  
συντελεστές στο  $\Z/p\Z$ ή στο  $\Z/p\Z[x]$, εισάγουμε {\en\tt obj \% 0}. Για παράδειγμα, αν εισάγουμε {\en\tt obj := 4 \% 7} και μετά
 {\en\tt obj \% 0}, τότε η έξοδος θα είναι {\en\tt -3}.
\item
Στον τρόπο λειτουργίας {\en\tt Maple}, οι ακέραιοι στο  $\Z/p\Z$ αναπαρίστανται όπως οι
συνηθισμένοι ακέραιοι.
Για να αποφύγουμε την σύγχυση με τις κανονικές εντολές, οι  
εντολές για αριθμητική υπολοίπων γράφονται με κεφαλαίο γράμμα (αδρανής μορφή) και ακολουθούνται από τη εντολή
 {\en\tt mod} (δείτε επίσης το επόμενο τμήμα).
\end{itemize} 
{\bf Σχόλιο} 
\begin{itemize}
\item Για μερικές εντολές στο $\Z/p\Z$ ή στο $\Z/p\Z[x]$, {\tt $p$} πρέπει να είναι 
ένας πρώτος ακέραιος.
\item Η αναπαράσταση είναι η συμμετρική αναπαράσταση :\\
Το {\en\tt 11\%13} επιστρέφει{\en\tt -2\%13}.
\end{itemize}

\subsection{Ανάπτυγμα και αναγωγή : {\tt\textlatin{ normal}}}\index{normal}
\noindent{{\en\tt normal} παίρνει σαν όρισμα μια πολυωνυμική παράσταση.}\\
{\en\tt normal} αναπτύσει και αναγάγει αυτήν την παράσταση στο $\Z/p\Z[x]$.\\
Είσοδος :
\begin{center}{\en\tt normal(((2*x\verb|^|2+12)*( 5*x-4))\%13)}\end{center}
Έξοδος :
\begin{center}{\en\tt (-3\%13)*x\verb|^|3+(5\%13)*x\verb|^|2+(-5\%13)*x+4\%13}\end{center}

\subsection{Πρόσθεση στο $\Z/p\Z$ ή στο $ \Z/p\Z[x]$ : {\tt +}}\index{+}
\noindent{{\tt +} προσθέτει δύο ακεραίους στο $\Z/p\Z$, ή
δύο πολυώνυμα στο $\Z/p\Z[x]$. Για πολυωνυμικές παραστάσεις, 
χρησιμοποιήστε την εντολή {\en\tt normal} για απλοποίηση.}\\
Για ακεραίους στο $\Z/p\Z$, εισάγετε :
\begin{center}{\en\tt 3\%13+10\%13}\end{center}
Έξοδος :
\begin{center}{\en\tt 0\%13}\end{center}
Για πολυώνυμα με συντελεστές στο $\Z/p\Z$, εισάγετε :
\begin{center}{\en\tt normal((11*x+5 )\% 13+(8*x+6)\%13)}\end{center} 
ή 
\begin{center}{\en\tt normal(11\%13*x+5\%13+8\%13*x+6\%13)}\end{center} 
Έξοδος :
\begin{center}{\en\tt  (6\%13)*x+-2\%13}\end{center}

\subsection{Αφαίρεση στο $\Z/p\Z$ ή στο $ \Z/p\Z[x]$ : {\tt -}}\index{-|textbf}
\noindent{{\tt -} αφαιρεί δύο ακεραίους στο $\Z/p\Z$ ή
δύο πολυώνυμα στο $\Z/p\Z[x]$. Για πολυωνυμικές παραστάσεις, 
χρησιμοποιήστε την εντολή {\en\tt normal} για απλοποίηση \index{normal}.}\\ 
Για ακεραίους στο $\Z/p\Z$, εισάγετε :
\begin{center}{\en\tt 31\%13-10\%13}\end{center}
Έξοδος :
\begin{center}{\en\tt  -5\%13}\end{center}
Για πολυώνυμα με συντελεστές στο $\Z/p\Z$, εισάγετε :
\begin{center}{\en\tt normal((11*x+5)\%13-(8*x+6)\%13)}\end{center}
ή επίσης: 
\begin{center}{\en\tt normal(11\%13*x+5\%13-8\%13*x+6\%13)}\end{center} 
Έξοδος :
\begin{center}{\en\tt  (3\%13)*x+-1\%13}\end{center}

\subsection{Πολλαπλασιασμός στο $\Z/p\Z$ ή στο $ \Z/p\Z[x]$ : {\tt *}}\index{*}
\noindent{{\tt *} πολλαπλασιάζει δύο ακεραίους στο $\Z/p\Z$ ή
δύο πολυώνυμα στο $\Z/p\Z[x]$. Για πολυωνυμικές παραστάσεις, 
χρησιμοποιήστε τη εντολή {\en\tt normal} για απλοποίηση\index{normal}.}\\
Για ακεραίους στο $\Z/p\Z$, εισάγετε :
\begin{center}{\en\tt 31\%13*10\%13}\end{center}
Έξοδος :
\begin{center}{\en\tt  -2\%13}\end{center}
Για πολυώνυμα με συντελεστές στο $\Z/p\Z$, εισάγετε :
\begin{center}{\en\tt normal((11*x+5)\%13*(8*x+6 )\% 13)}\end{center}
ή επίσης:
\begin{center}{\en\tt normal((11\%13*x+5\%13)*(8\%13*x+6\%13))}\end{center} 
Έξοδος :
\begin{center}{\en\tt (-3\%13)*x\verb|^|2+(2\%13)*x+4\%13}\end{center}

\subsection{Ευκλείδειο πηλίκο  : {\tt\textlatin{ quo}}}\index{quo}
\noindent{{\en\tt quo} παίρνει σαν ορίσματα 
δύο πολυώνυμα $A$ και $B$ με συντελεστές στο $\Z/p\Z$, όπου
$A$ και $B$ είναι λίστες πολυωνύμων ή συμβολικά πολυώνυμα 
του $x$ ή ενός προαιρετικού τρίτου ορίσματος.}\\
{\en\tt quo} επιστρέφει το πηλίκο της Ευκλείδειας διαίρεσης
του $A$ με το $B$ στο $\Z/p\Z[x]$.\\
Είσοδος :
\begin{center}{\en\tt quo((x\verb|^|3+x\verb|^|2+1)\%13,(2*x\verb|^|2+4)\%13)}\end{center}
ή :
\begin{center}{\en\tt quo(x\verb|^|3+x\verb|^|2+1,2*x\verb|^|2+4)\%13}\end{center}
Έξοδος:
\begin{center}{\en\tt (-6\%13)*x+-6\%13}\end{center}
Πράγματι $\displaystyle x^3+x^2+1=(2x^2+4)(\frac{x+1}{2})+\frac{5x-4}{4}$
και $-3*4=-6*2=1 \ \bmod 13$.

\subsection{Ευκλείδειο υπόλοιπο : {\tt\textlatin{ rem}}}\index{rem}
\noindent{{\en\tt rem} παίρνει σαν ορίσματα  
δύο πολυώνυμα $A$ και $B$ με συντελεστές στο $\Z/p\Z$, όπου 
$A$ και $B$ είναι λίστες πολυωνύμων ή συμβολικά πολυώνυμα 
του  $x$ ή ενός προαιρετικού τρίτου ορίσματος.}\\
{\en\tt rem} επιστρέφει το υπόλοιπο της Ευκλείδειας διαίρεσης  
του  $A$ με το  $B$ στο  $\Z/p\Z[x]$.\\
Είσοδος :
\begin{center}{\en\tt rem((x\verb|^|3+x\verb|^|2+1)\%13,(2*x\verb|^|2+4)\%13)}\end{center}
Or :
\begin{center}{\en\tt rem(x\verb|^|3+x\verb|^|2+1,2*x\verb|^|2+4)\%13}\end{center}
Έξοδος:
\begin{center}{\en\tt (-2\%13)*x+-1\%13}\end{center}
Πράγματι $\displaystyle x^3+x^2+1=(2x^2+4)(\frac{x+1}{2})+\frac{5x-4}{4}$
και $-3*4=-6*2=1 \ \bmod 13$.

\subsection{Ευκλείδειο πηλίκο και Ευκλείδειο υπόλοιπο : {\tt\textlatin{ quorem}}}\index{quorem}
\noindent{{\en\tt quorem} παίρνει σαν ορίσματα δύο πολυώνυμα 
$A$ και $B$ με συντελεστές στο $\Z/p\Z$, όπου
$A$ και $B$ είναι λίστες πολυωνύμων ή συμβολικά πολυώνυμα του  $x$ ή ενός προαιρετικού τρίτου ορίσματος.\\
{\en\tt quorem} επιστρέφει την λίστα του πηλίκου και του υπολοίπου
της Ευκλείδειας διαίρεσης του $A$ με το $B$ στο $\Z/p\Z[x]$
(δείτε επίσης \ref{sec:iquorem} και  \ref{sec:quorem}).}\\
Είσοδος :
\begin{center}{\en\tt quorem((x\verb|^|3+x\verb|^|2+1)\%13,(2*x\verb|^|2+4)\%13)}\end{center} 
ή :
\begin{center}{\en\tt quorem(x\verb|^|3+x\verb|^|2+1,2*x\verb|^|2+4)\%13}\end{center}
Έξοδος:
\begin{center}{\en\tt [(-6\%13)*x+-6\%13,(-2\%13)*x+-1\%13]}\end{center}
Πράγματι
$\displaystyle x^3+x^2+1=(2x^2+4)(\frac{x+1}{2})+\frac{5x-4}{4}$\\
και $-3*4=-6*2=1 \ \bmod 13$.

\subsection{Διαίρεση στο $\Z/p\Z$ ή στο $\Z/p\Z[x]$ : {\tt /}}\index{/}
\noindent{{\tt /} διαιρεί δύο ακεραίους στο $\Z/p\Z$ ή 
δύο πολυώνυμα $A$ και $B$ στο $\Z/p\Z[x]$.\\
Για πολυώνυμα, το αποτέλεσμα είναι ένας ανάγωγος αντιπρόσωπος του κλάσματος
 $\frac{A}{B}$ στο $\Z/p\Z[x]$.}\\
Για ακεραίους στο  $\Z/p\Z$, εισάγετε :
\begin{center}{\en\tt 5\%13/(2\%13)}\end{center}
Αφού το $2$ είναι αμετάβλητο στο $Z/13\Z$, παίρνουμε την έξοδο :
\begin{center}{\en\tt -4\%13}\end{center}
Για πολυώνυμα με συντελεστές στο $\Z/p\Z$, εισάγετε :
\begin{center}{\en\tt (2*x\verb|^|2+5)\%13/(5*x\verb|^|2+2*x-3)\%13}\end{center}
Έξοδος :
\begin{center}{\en\tt ((6\%13)*x+1\%13)/((2\%13)*x+2\%13)}\end{center}

\subsection{Ύψωση σε δύναμη στο $\Z/p\Z$ και στο $\Z/p\Z[x]$ : {\tt \^\ }}\index{\^\ }
Για να υψώσουμε το {\en\tt a} στη δύναμη {\en\tt n} στο $\Z/p\Z$, χρησιμοποιούμε τον τελεστή
{\en\tt \verb|^|}. Η {\en\tt Xcas} υλοποιεί τον δυαδικό αλγόριθμο ύψωσης σε δύναμη.\\
Είσοδος :
\begin{center}{\en\tt (5\%13)\verb|^|2}\end{center}
Έξοδος :
\begin{center}{\en\tt -1\%13}\end{center}
Για να υψώσουμε το {\en\tt A} στη δύναμη {\en\tt n} στο $\Z/p\Z[x]$, χρησιμοποιούμε τον τελεστή
{\en\tt \verb|^|} και την εντολή {\en\tt normal}\index{normal}.\\
Είσοδος :
\begin{center}{\en\tt normal(((2*x+1)\%13)\verb|^|5)}\end{center}
Έξοδος :
\begin{center}{\en\tt (6\%13)*x\verb|^|5+(2\%13)*x\verb|^|4+(2\%13)*x\verb|^|3+(1\%13)*x\verb|^|2+(-3\%13)*x+1\%13}\end{center}
επειδή
 $10=-3 \ (\bmod\ 13) \ \  40=1\ (\bmod\ 13)\ \   80=2 \ (\bmod\ 13)\ \ 32=6\ (\bmod\ 13)$. 

\subsection{Υπολογισμός $a^n\ \bmod \ p$ : {\tt\textlatin{ powmod powermod}}}\index{powmod}\index{powermod}
\noindent{{\tt\textlatin {powmod} } (ή {\tt\textlatin{ powermod}}) παίρνει σαν όρισμα $a,n,p$.\\
{\en\tt powmod} (ή {\en\tt powermod}) επιστρέφει $a^n\ \bmod \ p$ στο $[0,p-1]$.\\
Είσοδος :
\begin{center}{\en\tt powmod(5,2,13)}\end{center}
Έξοδος :
\begin{center}{\en\tt 12}\end{center}
Είσοδος :
\begin{center}{\en\tt powmod(5,2,12)}\end{center}
Έξοδος :
\begin{center}{\en\tt 1}\end{center}

\subsection{Υπολογισμός αντιστρόφου στο $\Z/p\Z$ : \tt\textlatin{ inv inverse} ή {\tt /}}\index{/}\index{inv}
Για να υπολογίσουμε τον αντίστροφο ενός ακεραίου {\en\tt n} στο $\Z/p\Z$, εισάγουμε {\en\tt 1/n\%p} 
ή {\tt\textlatin{ inv(n\%p)}} ή {\tt\textlatin{ inverse(n\%p)}}.\\
Είσοδος :
\begin{center}{\en\tt inv(3\%13) }\end{center}
Έξοδος :
\begin{center}{\en\tt -4\%13}\end{center}
Πράγματι $3\times-4=-12=1\ (\bmod\ 13)$.

\subsection{Αναδημιουργία  κλάσματος από την τιμή του {\tt \textlatin{modulo}} $p$ : {\tt\textlatin{ fracmod}}}\index{fracmod}
\noindent{{\en\tt fracmod} παίρνει δύο ορίσματα, έναν ακέραιο $n$ 
(που αντιπροσωπεύει το κλάσμα) και έναν ακέραιο $p$ (το {\tt\textlatin{modulus}}).}\\
Εάν είναι εφικτό,  η {\en\tt fracmod} επιστρέφει ένα κλάσμα $a/b$ τέτοιο ώστε 
\[ -\frac{\sqrt{p}}{2} < a \leq \frac{\sqrt{p}}{2}, \quad
 0 \leq b < \frac{\sqrt{p}}{2}, \quad 
 n \times b =a \pmod p \]
Με άλλα λόγια $n=a/b\pmod p$.\\
Είσοδος  :
\begin{center}{\en\tt fracmod(3,13) }\end{center}
Έξοδος :
\begin{center}{\en\tt -1/4}\end{center}
Πράγματι : $3*-4=-12=1\ (\bmod\ 13)$, και επομένως $3=-1/4\%13$.\\
Είσοδος  :
\begin{center}{\en\tt fracmod(13,121)}\end{center}
Έξοδος :
\begin{center}{\en\tt -4/9}\end{center}
Πράγματι : $13\times-9=-117=4\ (\bmod\ 121)$ και επομένως $13=-4/9\%13$.

\subsection{Μέγιστος κοινός διαιρέτης ({\tt \textlatin{GCD}}) στο $\Z/p\Z[x]$ : {\tt\textlatin{ gcd}}}\index{gcd}\label{sec:gcdm}
\noindent{{\en\tt gcd} παίρνει για ορίσματα δύο πολυώνυμα με
συντελεστές στο $\Z/p\Z$ ($p$ πρέπει να είναι πρώτος).}\\
{\en\tt gcd} επιστρέφει τον μέγιστο κοινό διαιρέτη ({\tt\textlatin{GCD}}) αυτών των πολυωνύμων
υπολογίσμένο στο $\Z/p\Z[x]$ (δείτε επίσης 
\ref{sec:gcd} για πολυώνυμα με συντελεστές σε μη πεπερασμένα πεδία ({\tt\textlatin{non modular}}).\\
Είσοδος :
\begin{center}{\en\tt gcd((2*x\verb|^|2+5)\%13,(5*x\verb|^|2+2*x-3)\%13)}\end{center}
Έξοδος :
\begin{center}{\en\tt (-4\%13)*x+5\%13}\end{center}
Είσοδος :
\begin{center}{\en\tt gcd(x\verb|^|2+2*x+1,x\verb|^|2-1) mod 5}\end{center} 
Έξοδος :
\begin{center}{\en\tt 1\%5*x + 1\%5}\end{center}
%Σημειώστε τη διαφορά με ένα  \tt\textlatin{gcd} υπολογισμό στο $\Z[X]$ που ακολουθείται από ένα
%\tt\textlatin{reduction modulo} 5, εισάγετε:
%\begin{center}{\en\tt gcd(x\verb|^|2+2*x+1,x\verb|^|2-1) mod 5}\end{center} 
%Έξοδος :
%\begin{center}{\tt 1}\end{center}

\subsection{Παραγοντοποίηση στο $\Z/p\Z[x]$ : {\tt\textlatin{factor factoriser}}}\index{factor}\index{factoriser}
\noindent{{\en\tt factor} παίρνει σαν όρισμα ένα πολυώνυμο
με συντελεστές στο $\Z/p\Z[x]$.}\\
{\en\tt factor} παραγοντοποιεί το πολυώνυμο στο $\Z/p\Z[x]$ ($p$ πρέπει να είναι
πρώτος).\\
Είσοδος :
\begin{center}{\en\tt factor((-3*x\verb|^|3+5*x\verb|^|2-5*x+4)\%13)}\end{center}
Έξοδος :
\begin{center}{\en\tt ((1\%13)*x+-6\%13)*((-3\%13)*x\verb|^|2+-5\%13)}\end{center}

\subsection{Ορίζουσα πίνακα στο $\Z/p\Z$ : {\tt\textlatin{ det}}}\index{det}
\noindent{{\en\tt det} παίρνει σαν όρισμα ένα πίνακα $A$ με συντελεστές στο
$Z/pZ$.}\\
{\en\tt det} επιστρέφει την ορίζουσα του πίνακα $A$.\\
Οι υπολογισμοί γίνονται στο $\Z/p\Z$.\\
Είσοδος :
\begin{center}{\en\tt det([[1,2,9]\%13,[3,10,0]\%13,[3,11,1]\%13])}\end{center} 
ή:
\begin{center}{\en\tt det([[1,2,9],[3,10,0],[3,11,1]]\%13)}\end{center} 
Έξοδος :
\begin{center}{\en\tt 5\%13}\end{center} 
έτσι, στο $\Z/13\Z$, η ορίζουσα του
$A=[[1,2,9],[3,10,0],[3,11,1]]$ είναι {\en\tt 5\%13} (στο $\Z$, {\en\tt det(A)=31}).

\subsection{Αντίστροφος πίνακα με συντελεστές στο $\Z/p\Z$ : {\tt\textlatin{ inv inverse}}}\index{inv}\index{inverse}
\noindent{{\en\tt inverse} (ή {\en\tt inv}) παίρνει σαν όρισμα έναν πίνακα $A$ στο
$\Z/p\Z$.}\\
{\en\tt inverse} (ή {\en\tt inv}) επιστρέφει τον αντίστροφο πίνακα του
$A$ στο $Z/p\Z$.\\
Είσοδος :
\begin{center}{\en\tt inverse([[1,2,9]\%13,[3,10,0]\%13,[3,11,1]\%13])}\end{center} 
ή:
\begin{center}{\en\tt inv([[1,2,9]\%13,[3,10,0]\%13,[3,11,1]\%13])}\end{center} 
ή :
\begin{center}{\en\tt inverse([[1,2,9],[3,10,0],[3,11,1]]\%13)}\end{center} 
ή :
\begin{center}{\en\tt inv([[1,2,9],[3,10,0],[3,11,1]]\%13)}\end{center}
Έξοδος :
\begin{center}{\en\tt [[2\%13,-4\%13,-5\%13],[2\%13,0\%13,-5\%13], [-2\%13,-1\%13,6\%13]]}\end{center} 
είναι ο αντίστροφος του $A=[[1,2,9],[3,10,0],[3,11,1]]$ στο $\Z/13\Z$.

\subsection{Αναγωγή γραμμών σε κλιμακωτή μορφή  στο $\Z/p\Z$ : {\tt\textlatin{ rref}}}\index{rref}
\noindent{\en\tt rref} ({\tt\textlatin{row reduction to echelon form}}) αναγάγει τις γραμμές του
πίνακα, με συντελεστές στο $\Z/p\Z$, σε κλιμακωτή μορφή. 

Αυτό μπορεί να χρησιμοποιηθεί
για να λύσουμε ένα γραμμικό σύστημα εξισώσεων με συντελεστές στο  $\Z/p\Z$, 
αναγράφοντάς το σε μορφή πίνακα  :
\begin{center}{\en\tt A*X=B}\end{center}
{\en\tt rref} παίρνει σαν όρισμα τον επαυξημένο πίνακα
του συστήματος (ο πίνακας που παίρνουμε αν επαυξήσουμε τον πίνακα {\en\tt A} στα δεξιά
με το διάνυσμα στήλης {\en\tt B}).\\
{\en\tt rref} επιστρέφει έναν πίνακα {\en\tt [A1,B1]} : ο {\en\tt A1} έχει 1 στην 
κύρια διαγώνιό του, και μηδενικά έξω από αυτήν, και η λύση στο
 $\Z/p\Z$, του συστήματος :
\begin{center}{\en\tt A1*X=B1}\end{center} 
είναι ίδια, με την λύση του :
\begin{center}{\en\tt A*X=B}\end{center}
Παράδειγμα: Να λυθεί στο  $\Z/13\Z$
$$\left \{\begin{array}{lcr}\ \  x\ +\ \  2 \cdot y & = &9 \\3 \cdot x +10 \cdot y & =& 0 \end{array}\right.$$
Είσοδος:
\begin{center}{\en\tt rref([[1, 2, 9]\%13,[3,10,0]\%13])}\end{center} 
ή :
\begin{center}{\en\tt rref([[1, 2, 9],[3,10,0]])\%13}\end{center} 
Έξοδος :
\begin{center}{\en\tt [[1\%13,0\%13,3\%13],[0\%13,1\%13,3\%13]]}\end{center} 
και επομένως η λύση είναι {\en\tt x=3\%13} και {\en\tt y=3\%13}.


\subsection{Κατασκευή σώματος  \tt\textlatin{Galois} : {\tt\textlatin{ GF}}}\index{GF}
\noindent  {\en\tt GF} παίρνει σαν όρισμα έναν πρώτο ακέραιο $p$ 
και έναν ακέραιο $n>1$.\\
{\en\tt GF} επιστρέφει ένα σώμα {\tt\textlatin{Galois}} χαρακτηριστικής  $p$ με $p^n$
στοιχεία .\\
Τα στοιχεία του σώματος και το ίδιο το σώμα
παρίστανται με {\en\tt GF(...)} όπου {\en\tt ...} είναι η παρακάτω
ακολουθία:
\begin{itemize}
\item η χαρακτηριστική $p$ ($px=0$),
\item ένα ανάγωγο αρχικό , ελάχιστο πολυώνυμο που παράγει ένα
ιδεώδες $I$ στο $\Z/p\Z[X]$, όπου το σώμα {\tt\textlatin{Galois}} είναι το πηλίκο του
$\Z/p\Z[X]$ με το $I$,
\item το όνομα της μεταβλητής του πολυωνύμου, από προεπιλογή {\en\tt x},
\item ένα πολυώνυμο (ένα υπόλοιπο {\tt\textlatin{modulo}} το ελάχιστο πολυώνυμο) 
για ένα στοιχείο του σώματος
(τα στοιχεία του σώματος παρίστανται με την προσθετική αναπαράσταση )
ή {\en\tt undef} για το ίδιο το σώμα.
\end{itemize}
Θα πρέπει να δώσετε ένα όνομα σε αυτό το σώμα (για παράδειγμα {\tt\textlatin{ G:=GF(p,n)}}),
για να δημιουργήσετε στοιχεία του σώματος από ένα πολυώνυμο στο
$\Z/p\Z[X]$, για παράδειγμα {\en\tt G(x\verb|^|3+x)}. Σημειώσατε ότι το {\en\tt G(x)}
είναι ένας γεννήτορας της πολλαπλασιαστικής ομάδας {\en\tt $G^*$}.\\
Είσοδος :
\begin{center}{\en\tt G:=GF(2,8)}\end{center}
Έξοδος :
\begin{center}{\en\tt GF(2,x\verb|^|8-x\verb|^|6-x\verb|^|4-x\verb|^|3-x\verb|^|2-x-1,x,undef)}\end{center}
Το σώμα $G$ έχει $2^8=256$ στοιχεία και το 
$x$ κάνει την πολλαπλασιαστική ομάδα
του σώματος ($\{ 1,x,x^2,...x^{254} \}$).\\
Είσοδος :
\begin{center}{\en\tt G(x\verb|^|9)}\end{center}
Έξοδος :
\begin{center}{\en\tt GF(2,x\verb|^|8-x\verb|^|6-x\verb|^|4-x\verb|^|3-x\verb|^|2-x-1,x,x\verb|^|7+x\verb|^|5+x\verb|^|4+x\verb|^|3+x\verb|^|2+x)}\end{center}
πράγματι $x^8=x^6+x^4+x^3+x^2+x+1$, και επομένως $x^9=x^7+x^5+x^4+x^3+x^2+x$.\\
Είσοδος :
\begin{center}{\en\tt G(x)\verb|^|255}\end{center}
Έξοδος θα πρέπει να είναι η μονάδα, πράγματι:
\begin{center}
{\en\tt GF(2,x\verb|^|8-x\verb|^|6-x\verb|^|4-x\verb|^|3-x\verb|^|2-x-1,x,1)}\end{center}
Όπως μπορεί να δει κανείς σε αυτά τα παραδείγματα, η έξοδος περιέχει πολλές φορές τις ίδιες πληροφορίες
που  θα προτιμούσατε να μην τις βλέπετε  εάν 
δουλεύατε πολλές φορές με το ίδιο σώμα. Γι' αυτό το λόγο,
ο ορισμός του σώματος {\tt\textlatin{Galois}} μπορεί να έχει ένα προαιρετικό όρισμα,
το όνομα μιας μεταβλητής που μπορεί να χρησιοποιηθεί μετά για να παριστάνουμε στοιχεία του σώματος.
Επιπλέον, επειδή μάλλον θα θέλετε 
να αλλάξετε το όνομα της μεταβλητής, το όνομα του σώματος μαζί με το όνομα της μεταβλητής δίδονται σε μια λίστα σαν το τρίτο όρισμα
του {\en\tt GF}.
Σημειώστε πως αυτά τα δύο ονόματα των μεταβλητών πρέπει να αναφέρονται.\\
Παράδειγμα, εισάγετε :
\begin{center}{\en\tt G:=GF(2,2,[{\gr\tt '}w{\gr\tt '},{\gr\tt '}G{\gr\tt '}]):; G(w\verb|^|2)}\end{center}
Έξοδος :
\begin{center}{\en\tt Done, G(w+1)}\end{center}
Είσοδος :
\begin{center}{\en\tt G(w\verb|^|3)}\end{center}
Έξοδος :
\begin{center}{\en\tt G(1)}\end{center}
Έτσι, τα στοιχεία του {\en\tt GF(2,2)} είναι
{\en\tt G(0),G(1),G(w),G(w\verb|^|2)=G(w+1)}.

Μπορούμε επίσης να καθορίσουμε εμείς  το ανάγωγο αρχικό πολυώνυμο που επιθυμούμε να 
χρησιμοποιήσουμε, βάζοντάς το σαν δεύτερο όρισμα (αντί για $n$), 
για παράδειγμα :
\begin{center}{\en\tt \verb|G:=GF(2,w^8+w^6+w^3+w^2+1,|[{\gr\tt '}w{\gr\tt '},{\gr\tt '}G{\gr\tt '}])}\end{center}
Εάν το πολυώνυμο δεν είναι αρχικό, το {\en\tt Xcas} θα το αντικαταστήσει αυτόματα
με ένα  αρχικό πολυώνυμο, για παράδειγμα :
\begin{center}{\en\tt \verb|G:=GF(2,w^8+w^7+w^5+w+1,|[{\gr\tt '}w{\gr\tt '},{\gr\tt '}G{\gr\tt '}])}\end{center}
Έξοδος :
\begin{center}{\en\tt \verb|G:=GF(2,w^8-w^6-w^3-w^2-1,|[{\gr\tt '}w{\gr\tt '},{\gr\tt '}G{\gr\tt '}],undef)} \end{center}

\subsection{Παραγοντοποίηση πολυωνύμου με συντελεστές σε σώμα {\tt\textlatin{Galois}} : {\tt\textlatin{ factor}}}\index{factor}
\noindent{{\en\tt factor} μπορεί επίσης να παραγοντοποιήσει ένα μονομετάβλητο
πολυώνυμο με συντελεστές σε σώμα {\tt\textlatin{Galois}} .}\\
Εισάγετε για παράδειγμα :\\
\begin{center}{\en\tt G:=GF(2,2,[{\gr\tt '}w{\gr\tt '},{\gr\tt '}G{\gr\tt '}])}\end{center}
Έξοδος :
\begin{center}{\en\tt GF(2,w\verb|^|2+w+1,[w,G],undef)}\end{center}
Είσοδος για παράδειγμα :
\begin{center}{\en\tt a:=G(w)}\end{center}
\begin{center}{\en\tt factor(a\verb|^|2*x\verb|^|2+1))}\end{center}
Έξοδος :
\begin{center}{\en\tt (G(w+1))*(x+G(w+1))\verb|^|2}\end{center}
 

\section{Υπολογισμοί στο $\Z/p\Z[x]$ χρησιμοποιώντας σύνταξη του \tt\textlatin{Maple}}\index{mod}\index{\%}
\subsection{Ευκλείδειο πηλίκο : {\tt\textlatin{ Quo}}}\index{Quo}
\noindent{{\en\tt Quo} είναι η αδρανής μορφή του {\en\tt quo}.\\
{\en\tt Quo} επιστρέφει {\en\tt quo}  δύο πολυωνύμων
χωρίς αποτίμηση.\\ 
Χρησιμοποιείται σε συνδυασμό με το {\en\tt mod} στον τρόπο λειτουργίας {\tt\textlatin{Maple}} για να υπολογίσουμε το
Ευκλείδειο πηλίκο της διαίρεσης δύο
πολυωνύμων με συντελεστές στο $\Z/p\Z$.}\\
Εισάγετε σε τρόπο λειτουργίας {\en\tt Xcas}:
\begin{center}{\en\tt Quo((x\verb|^|3+x\verb|^|2+1) mod 13,(2*x\verb|^|2+4) mod 13)}\end{center}
Έξοδος :
\begin{center}{\en\tt quo((x\verb|^|3+x\verb|^|2+1)\%13,(2*x\verb|^|2+4)\%13)}\end{center}
χρειάζεται να αποτιμήσουμε την τελευταία απάντηση με {\en\tt eval(ans())} για να πάρουμε το αποτέλεσμα :
\begin{center}{\en\tt (-6\%13)*x+-6\%13}\end{center}
Είσοδος σε  τρόπο λειτουργίας {\en\tt Maple} :
\begin{center}{\en\tt Quo(x\verb|^|3+x\verb|^|2+1,2*x\verb|^|2+4) mod 13}\end{center}
Έξοδος :
\begin{center}{\en\tt (-6)*x-6}\end{center}
Είσοδος σε  τρόπο λειτουργίας {\en\tt Maple} :
\begin{center}{\en\tt Quo(x\verb|^|2+2*x,x\verb|^|2+6*x+5) mod 5}\end{center}
Έξοδος :
\begin{center}{\tt 1}\end{center}
 
\subsection{Ευκλείδειο υπόλοιπο : {\tt\textlatin{ Rem}}}\index{Rem}
\noindent{\en\tt Rem} είναι η αδρανής μορφή του {\en\tt rem}.\\
{\en\tt Rem} επιστρέφει {\en\tt rem} δύο πολυωνύμων 
χωρίς αποτίμηση. 
Χρησιμοποιείται σε συνδυασμό με το {\en\tt mod} στον τρόπο λειτουργίας {\tt\textlatin{Maple}} για να υπολογίσουμε το
Ευκλείδειο υπόλοιπο της διάιρεσης δύο
πολυωνύμων με συντελεστές στο $\Z/p\Z$.\\
Είσοδος σε τρόπο λειτουργίας {\en\tt Xcas} :
\begin{center}{\en\tt Rem((x\verb|^|3+x\verb|^|2+1) mod 13,(2*x\verb|^|2+4) mod 13)}\end{center}
Έξοδος :
\begin{center}{\en\tt rem((x\verb|^|3+x\verb|^|2+1)\%13,(2*x\verb|^|2+4)\%13)}\end{center}
χρειάζεται να αποτιμήσουμε την τελευταία απάντηση με {\en\tt eval(ans())} για να πάρουμε το αποτέλεσμα :
\begin{center}{\en\tt (-2\%13)*x+-1\%13}\end{center}
Είσοδος σε  τρόπο λειτουργίας {\en\tt Maple} :
\begin{center}{\en\tt Rem(x\verb|^|3+x\verb|^|2+1,2*x\verb|^|2+4) mod 13}\end{center}
Έξοδος :
\begin{center}{\en\tt (-2)*x-1}\end{center}
Είσοδος σε  τρόπο λειτουργίας {\en\tt Maple} :
\begin{center}{\en\tt Rem(x\verb|^|2+2*x,x\verb|^|2+6*x+5) mod 5}\end{center}
Έξοδος :
\begin{center}{\en\tt 1*x}\end{center}

\subsection{Μέγιστος κοινός διαιρέτης (ΜΚΔ ή {\tt\textlatin{GCD}}) στο $\Z/p\Z[x]$ : {\tt\textlatin{ Gcd}}}\index{Gcd}
\noindent{\en\tt Gcd}  είαι η αδρανής μορφή του  {\en\tt gcd}.\\
{\en\tt Gcd} επιστρέφει {\tt\textlatin{gcd}} δύο πολυωνύμων
(ή μιας λίστας πολυωνύμων ή μιας ακολουθίας πολυωνύμων) χωρίς
αποτίμηση.\\ 
Χρησιμοποιείται σε συνδυασμό με το {\en\tt mod} στον τρόπο λειτουργίας {\tt\textlatin{Maple}} για να υπολογίσουμε
τον μέγιστο κοινό διαρέτη ({\tt\textlatin{gcd}}) δύο πολυωνύμων με συντελεστές στο $\Z/p\Z$, όπου   $p$ πρώτος
(δείτε επίσης \ref{sec:gcd}).\\
Είσοδος σε τρόπο λειτουργίας {\en\tt Xcas} :
\begin{center}{\en\tt Gcd((2*x\verb|^|2+5,5*x\verb|^|2+2*x-3)\%13)}\end{center}
Έξοδος :
\begin{center}{\en\tt gcd((2*x\verb|^|2+5)\%13,(5*x\verb|^|2+2*x-3)\%13)}\end{center}
χρειάζεται να αποτιμήσουμε την τελευταία απάντηση με {\en\tt eval(ans())} για να πάρουμε το αποτέλεσμα :
\begin{center}{\en\tt (1\%13)*x+2\%13}\end{center}
Είσοδος σε  τρόπο λειτουργίας {\en\tt Maple} :
\begin{center}{\en\tt Gcd(2*x\verb|^|2+5,5*x\verb|^|2+2*x-3) mod 13}\end{center}
Έξοδος :
\begin{center}{\en\tt 1*x+2}\end{center}
Είσοδος σε  τρόπο λειτουργίας {\en\tt Maple} :
\begin{center}{\en\tt Gcd(x\verb|^|2+2*x,x\verb|^|2+6*x+5) mod 5}\end{center}
Έξοδος :
\begin{center}{\en\tt 1*x}\end{center}

\subsection{Παραγοντοποίηση στο $\Z/p\Z[x]$ : {\tt\textlatin{ Factor}}}\index{Factor}
\noindent{\en\tt Factor} είναι η αδρανής μορφή της {\en\tt factor}.\\
{\en\tt Factor} παίρνει σαν όρισμα ένα πολυώνυμο.\\
{\en\tt Factor} επιστρέφει {\en\tt factor} χωρίς αποτίμηση. 
Χρησιμοποιείται σε συνδυασμό με το {\en\tt mod} ({\tt \%}) στον τρόπο λειτουργίας {\tt\textlatin{Maple}} για να 
παραγοντοποιήσουμε ένα πολυώνυμο με συντελεστές στο $\Z/p\Z$
όπου $p$ πρώτος.\\
Είσοδος σε τρόπο λειτουργίας {\en\tt Xcas} :
\begin{center}{\en\tt Factor((-3*x\verb|^|3+5*x\verb|^|2-5*x+4)\%13)}\end{center}
Έξοδος:
\begin{center}{\en\tt factor(-3\%13*x\verb|^|3+5\%13*x\verb|^|2+-5\%13*x+4\%13)}\end{center}
χρειάζεται να αποτιμήσουμε την τελευταία απάντηση με {\en\tt eval(ans())} για να πάρουμε το αποτέλεσμα :
\begin{center}{\en\tt ((1\%13)*x+-6\%13)*((-3\%13)*x\verb|^|2+-5\%13)}\end{center}
Είσοδος σε  τρόπο λειτουργίας {\en\tt Maple} :
\begin{center}{\en\tt Factor(-3*x\verb|^|3+5*x\verb|^|2-5*x+4) mod 13}\end{center}
Έξοδος :
\begin{center}{\en\tt -3*(1*x-6)*(1*x\verb|^|2+6)}\end{center}

\subsection{Ορίζουσα πίνακα με συντελεστές στο $\Z/p\Z$ : {\tt\textlatin{ Det}}}\index{Det}
\noindent{{\en\tt Det} είναι η αδρανής μορφή της {\en\tt det}.\\
{\en\tt Det} παίρνει ένα όρισμα με συντελεστές στο $\Z/p\Z$.\\ 
{\en\tt Det} επιστρέφει  {\en\tt det} χωρίς αποτίμηση. 
Χρησιμοποιείται σε συνδυασμό με το {\en\tt mod}  στον τρόπο λειτουργίας {\tt\textlatin{Maple}}  για να 
βρούμε την ορίζουσα ενός πίνακα με συντελεστές στο $\Z/p\Z$.}\\
Είσοδος σε τρόπο λειτουργίας {\en\tt Xcas} :
\begin{center}{\en\tt Det([[1,2,9] mod 13,[3,10,0] mod 13,[3,11,1] mod 13])}\end{center} 
Έξοδος :
\begin{center}{\en\tt det([[1\%13,2\%13,-4\%13],[3\%13,-3\%13,0\%13], [3\%13,-2\%13,1\%13]])}\end{center}
χρειάζεται να αποτιμήσουμε την τελευταία απάντηση με {\en\tt eval(ans())} για να πάρουμε το αποτέλεσμα :
\begin{center}{\tt 5\%13}\end{center} 
Έτσι, στο $\Z/13\Z$, η ορίζουσα του
$A=[[1, 2, 9],[3,10,0],[3,11,1]]$ είναι {\tt 5\%13} (ενώ στο $\Z$ {\en\tt det(A)=31}).\\
Είσοδος σε  τρόπο λειτουργίας {\en\tt Maple} :
\begin{center}{\en\tt Det([[1,2,9],[3,10,0],[3,11,1]]) mod 13}\end{center}
Έξοδος :
\begin{center}{\tt 5}\end{center}

\subsection{Αντίστροφος πίνακα στο $\Z/p\Z$ : {\tt\textlatin{ Inverse}}}\index{Inverse}
\noindent{\en\tt Inverse}  είναι η αδρανής μορφή της {\en\tt inverse}.\\
{\en\tt Inverse} παίρνει σαν όρισμα έναν πίνακα με συντελεστές στο $\Z/p\Z$.\\
{\en\tt Inverse} επιστρέφει {\en\tt inverse} χωρίς αποτίμηση. 
Χρησιμοποιείται σε συνδυασμό με το {\en\tt mod}  στον τρόπο λειτουργίας {\tt\textlatin{Maple}}  για να 
βρούμε τον αντίστροφο πίνακα με συντελεστές στο $\Z/p\Z$.\\
Είσοδος σε τρόπο λειτουργίας {\en\tt Xcas} :
\begin{center}{\en\tt Inverse([[1,2,9] mod 13,[3,10,0] mod 13,[3,11,1] mod13])}\end{center} 
Έξοδος :
\begin{center}{\en\tt inverse([[1\%13,2\%13,9\%13],[3\%13,10\%13,0\%13], [3\%13,11\%13,1\%13]])}\end{center} 
χρειάζεται να αποτιμήσουμε την τελευταία απάντηση με {\en\tt eval(ans())} για να πάρουμε το αποτέλεσμα :
\begin{center}{\en\tt [[2\%13,-4\%13,-5\%13],[2\%13,0\%13,-5\%13], [-2\%13,-1\%13,6\%13]]}\end{center} 
που είναι ο αντίστροφος του $A=[[1,2,9],[3,10,0],[3,11,1]]$ στο $\Z/13\Z$.\\
Είσοδος σε  τρόπο λειτουργίας {\en\tt Maple} :
\begin{center}{\en\tt Inverse([[1,2,9],[3,10,0],[3,11,1]]) mod 13}\end{center}
Έξοδος :
\begin{center}{\en\tt [[2,-4,-5],[2,0,-5],[-2,-1,6]]}\end{center}

\subsection{Αναγωγή γραμμών σε κλιμακωτή μορφή στο $\Z/p\Z$ : {\tt\textlatin{ Rref}}}\index{Rref}
\noindent{{\en\tt Rref}  είναι η αδρανής μορφή της {\en\tt rref} ({\tt\en row reduction to echelon form}).\\ 
{\en\tt Rref} επιστρέφει {\en\tt rref} χωρίς αποτίμηση. 
Χρησιμοποιείται σε συνδυασμό με το {\en\tt mod}  στον τρόπο λειτουργίας {\tt\textlatin{Maple}}  
για να βρούμε την αναγωγή γραμμών 
ενός πίνακα, με συντελεστές στο $\Z/p\Z$, σε κλιμακωτή μορφή (δείτε 
επίσης\ref{sec:rref}).}\\
Παράδειγμα, λύστε στο $\Z/13\Z$
$$\left \{\begin{array}{lcr}\ \  x\ +\ \  2 \cdot y & = &9 \\3 \cdot x +10 \cdot y & =& 0 \end{array}\right.$$
Είσοδος σε τρόπο λειτουργίας {\en\tt Xcas} :
\begin{center}{\en\tt Rref([[1,2,9] mod 13,[3,10,0] mod 13])}\end{center} 
Έξοδος :
\begin{center}{\en\tt rref([[1\%13, 2\%13, -4\%13],[3\%13,10\%13,0\%13]])}\end{center}
χρειάζεται να αποτιμήσουμε την τελευταία απάντηση με {\en\tt eval(ans())} για να πάρουμε το αποτέλεσμα :
\begin{center}{\en\tt [[1\%13,0,3\%13],[0,1\%13,3\%13]]}\end{center} 
και να συμπεράνουμε ότι {\en\tt x=3\%13} και  {\en\tt y=3\%13}.\\
Είσοδος σε  τρόπο λειτουργίας {\en\tt Maple} :
\begin{center}{\en\tt Rref([[1,2,9],[3,10,0]]) mod 13}\end{center}
Έξοδος :
\begin{center}{\en\tt [[1,0,3],[0,1,3]]}\end{center}


\section{Αναπτύγματα \textlatin{Taylor} και ασυμπτωτικά αναπτύγματα}
\subsection{Διάιρεση με αυξανόμενες δυνάμεις : {\tt\textlatin{ divpc}}}\index{divpc}
\noindent{{\en\tt divpc} παίρνει 3 ορίσματα: 2 πολυωνυμικές
παραστάσεις $A,\ B$ που εξαρτώνται από το $x$,
έτσι ώστε ο σταθερός όρος του $B$ να είναι $\neq 0$, και έναν ακέραιο $n$.\\
{\en\tt divpc} επιστρέφει το πηλίκο $Q$ της διαίρεσης του $A$ με το $B$ 
με αυξανόμενες δυνάμεις, με {\en\tt degree}$(Q)\leq n$ ή $ Q=0$ (δηλαδή, $A = B Q + x^{n+1}R,\  deg(Q) \leq n$, --- ή {\en\tt division by increasing power order}). Η διαίρεση αυτή είναι όπως η συνήθης Ευκλείδεια διαίρεση, μόνο που τώρα πρώτα απαλοίφονται οι όροι του διαιρεταίου με τον μικρότερο βαθμό. 
Με άλλα λόγια, το $Q$ είναι το ανάπτυγμα {\tt\textlatin{Taylor}}  τάξης $n$ του
$\displaystyle \frac{A}{B}$ στην περιοχή του $x=0$.} \\ 
Είσοδος :
\begin{center}{\en\tt divpc(1+x\verb|^|2+x\verb|^|3,1+x\verb|^|2,5)}\end{center}
Έξοδος :
\begin{center}{\en\tt -x\verb|^|5+x\verb|^|3+1}\end{center}
Σημειώσατε ότι αυτή η εντολή δεν δουλεύει για πολυώνυμα που είναι γραμμένα σαν
λίστα συντελεστών.

\subsection{Ανάπτυγμα \tt\textlatin{Taylor} : {\tt\textlatin{ taylor}}}\index{taylor}\index{order\_size|textbf} 
\noindent{{\en\tt taylor} παίρνει από 1 μέχρι 4 ορίσματα :
\begin{itemize}
\item μια παράσταση που εξαρτάται από μια μεταβλητή (από προεπιλογή {\en\tt x}),
\item μια ισότητα {\tt\textlatin{variable=value}} (π.χ. $x=a$) για τον υπολογισμό
του αναπτύγματος {\tt\textlatin{Taylor}}, από προεπιλογή {\en\tt x=0}, 
\item έναν ακέραιο $n$, την τάξη του αναπτύγματος σε σειρά,
από προεπιλογή {\tt 5},
\item μια κατεύθυνση  {\tt -1, 1} (για ανάπτυγμα σε σειρά προς μία κατεύθυνση)
  ή {\tt 0} (για ανάπτυγμα σε σειρά σε δύο κατευθύνσεις) (από προεπιλογή {\tt
    0}).
\end{itemize}
Σημειώσατε ότι η σύνταξη {\en\tt …,$x$,$n$,$a$,...} 
(αντί για {\en\tt …,$x=a$,$n$,...}) είναι επίσης αποδεκτή.\\
{\en\tt taylor} επιστρέφει ένα πολυώνυμο ως προς {\en\tt x-a}, συν ένα υπόλοιπο
της μορφής:\\
 {\en\tt (x-a)\verb|^|n*order\_size(x-a)}\\
όπου {\en\tt order\_size} είναι μια συνάρτηση τέτοια ώστε,
{\en\tt \[ \forall r>0, \quad \lim_{x\rightarrow 0} x^r \mbox{order\_size}(x) = 0 \]}\\
Για κανονικό ανάπτυγμα σε σειρά, η {\en\tt order\_size} είναι μια φραγμένη συνάρτηση,
αλλά για μη κανονικό ανάπτυγμα σε σειρά, μπορεί να τείνει αργά στο
άπειρο, για παράδειγμα σαν μια δύναμη του $\ln(x)$.}\\
Είσοδος :
\begin{center}{\en\tt taylor(sin(x),x=1,2)}\end{center}
ή (προσέξτε την διάταξη των ορισμάτων !) :
\begin{center}{\en\tt taylor(sin(x),x,2,1)}\end{center}
Έξοδος :
\begin{center}{\en\tt sin(1)+cos(1)*(x-1)+(-(1/2*sin(1)))*(x-1)\verb|^|2+ (x-1)\verb|^|3*order\_size(x-1)}\end{center}
{\bf Σχόλιο}\\
Η τάξη που επιστρέφεται από την {\en\tt taylor} μπορεί να είναι μικρότερη από $n$ εάν γίνονται απαλοιφές μεταξύ αριθμητών και παρονομαστών, για παράδειγμα
\[ \mbox{\en taylor}(\frac{x^3+\sin(x)^3}{x-\sin(x)}) \]
Είσοδος :
\begin{center}{\en\tt taylor(x\verb|^|3+sin(x)\verb|^|3/(x-sin(x)))}\end{center}
Η έξοδος είναι ανάπτυγμα σε σειρά μόνο δεύτερης τάξης :
\begin{center}{\en\tt
    6+-27/10*x\verb|^2|+x\verb|^|3*order\_size(x)}\end{center}
Πράγματι, ο μικρότερος βαθμός του αριθμητή και του παρονομαστή είναι 3, και γι' αυτό χάνουμε 3
τάξεις. Για να πάρουμε ανάπτυγμα σε σειρά 4ης τάξης, πρέπει να ζητήσουμε $n=7$, εισάγοντας :
\begin{center}{\en\tt taylor(x\verb|^|3+sin(x)\verb|^|3/(x-sin(x)),x=0,7)}\end{center}
Έξοδος είναι ανάπτυγμα σε σειρά 4ης τάξης :
\begin{center}{\en\tt 6+-27/10*x\verb|^|2+x\verb|^|3+711/1400*x\verb|^|4+x\verb|^|5*order\_size(x)}\end{center}

\subsection{Ανάπτυγμα σε σειρά : {\tt\textlatin{ series}}}\index{series}\index{order\_size} 
\noindent{\en\tt series} παίρνει από 1 μέχρι 4 ορίσματα :
\begin{itemize}
\item μια παράσταση που εξαρτάται από μια μεταβλητή (από προεπιλογή {\en\tt x}),
\item μια ισότητα {\tt\textlatin{variable=value}} (π.χ. $x=a$) για τον υπολογισμό του αναπτύγματος σε σειρά, από προεπιλογή {\en\tt x=0}, 
\item έναν ακέραιο $n$, την τάξη του αναπτύγματος σε σειρά,
 από προεπιλογή {\tt 5},
\item μια κατεύθυνσση {\tt -1, 1} (για ανάπτυγμα σε σειρά χωρίς κατεύθυνση)
  ή {\tt 0} (για ανάπτυγμα σε σειρά με 2 κατευθύνσεις) (από προεπιλογή {\en\tt
    0}).
\end{itemize}
Σημειώσατε ότι η σύνταξη {\en\tt …,$x$,$n$,$a$,...} 
(αντί για {\en\tt …,$x=a$,$n$,...}) είναι επίσης αποδεκτή.\\
{\en\tt series} επιστρέφει το πολυώνυμο στο {\en\tt x-a}, συν ένα υπόλοιπο της μορφής:
\begin{center}
 {\en\tt (x-a)\verb|^|n*order\_size(x-a)}
\end{center}
όπου {\en\tt order\_size} είναι μια συνάρτηση τέτοια ώστε,
\[ \forall r>0, \quad \lim_{x\rightarrow 0} x^r \mbox{\en\tt order\_size}(x) = 0 \]
Η τάξη που επιστρέφεται από την {\en\tt series} μπορεί να είναι μικρότερη από $n$ εάν γίνονται απαλοιφές μεταξύ αριθμητών και παρονομαστών, για παράδειγμα

Παραδείγατα~:
\begin{itemize}
\item  ανάπτυγμα σε σειρά στην περιοχή του {\en\tt x=0}\\
 Βρείτε το ανάπτυγμα σε σειρά της παράστασης
$\displaystyle\frac{x^3+\sin(x)^3}{x-\sin(x)}$ 
στην περιοχή του {\en\tt{ x=0}}.\\
Είσοδος :
\begin{center}{\en\tt series(x\verb|^|3+sin(x)\verb|^|3/(x-sin(x)))}\end{center}
Έξοδος είναι ανάπτυγμα μόνο 2ης τάξης :
\begin{center}{\en\tt 6+-27/10*x\verb|^2|+x\verb|^|3*order\_size(x)}\end{center}
Έχουμε χάσει 3 τάξεις γιατί ο μικρότερος βαθμός του αριθμητή και του παρονομαστή
είναι 3. Για να πάρουμε ανάπτυγμα 4ης τάξης, πρέπει  να ζητήσουμε
 $n=7$, εισάγοντας:
\begin{center}{\en\tt series(x\verb|^|3+sin(x)\verb|^|3/(x-sin(x)),x=0,7)}\end{center}
ή:
\begin{center}{\en\tt series(x\verb|^|3+sin(x)\verb|^|3/(x-sin(x)),x,0,7)}\end{center}
Έξοδος είναι ανάπτυγμα 4ης τάξης :
\begin{center}{\en\tt 6+-27/10*x\verb|^|2+x\verb|^|3+711/1400*x\verb|^|4+
x\verb|^|5*order\_size(x)}\end{center}
\item  ανάπτυγμα σε σειρά στην περιοχή του {\en\tt x=a}\\
Βρείτε το ανάπτυγμα σε σειρά  4ης τάξης της $\cos(2x)^2$ στην περιοχή του
$x=\frac{\pi}{6}$. \\
Είσοδος:
\begin{center}{\en\tt series(cos(2*x)\verb|^|2,x=pi/6, 4)}\end{center}
Έξοδος :
\begin{center}{\en\tt 1/4+(-(4*sqrt(3)))/4*(x-pi/6)+(4*3-4)/4*(x-pi/6)\verb|^|2+ 32*sqrt(3)/3/4*(x-pi/6)\verb|^|3+(-16*3+16)/3/4*(x-pi/6)\verb|^|4+ (x-pi/6)\verb|^|5*order\_size(x-pi/6)}\end{center} 
\item  ανάπτυγμα σε σειρά στην περιοχή του {\en\tt x=+$\infty$} ή  {\en\tt
    x=-$\infty$}
\begin{enumerate}
\item 
Βρείτε το ανάπτυγμα σε σειρά 5ης τάξης της $\arctan(x)$ στην περιοχή του
{\en\tt x=+$\infty$}.\\
 Είσοδος :
\begin{center}{\en\tt series(atan(x),x=+infinity,5)}\end{center}
Έξοδος :
\begin{center}{\en\tt pi/2-1/x+1/3*(1/x)\verb|^|3+1/-5*(1/x)\verb|^|5+
(1/x)\verb|^|6*order\_size(1/x)}\end{center}
Σημειώστε ότι η μεταβλητή του αναπτύγματος και το όρισμα της συνάρτησης
{\en\tt order\_size} είναι
$\displaystyle h=\frac{1}{x} \rightarrow_{x\rightarrow + \infty} 0 $.
\item
Βρείτε το ανάπτυγμα σε σειρά 2ης τάξης της παράστασης  $(2x-1)e^{\frac{1}{x-1}}$ στην περιοχή του
{\en\tt x=+$\infty$}. \\
Είσοδος :
\begin{center}{\en\tt series((2*x-1)*exp(1/(x-1)),x=+infinity,3)}\end{center}
Έξοδος είναι ανάπτυγμα 1ης τάξης:
\begin{center}{\en\tt  2*x+1+2/x+(1/x)\verb|^|2*order\_size(1/x)}\end{center}
Για να πάρουμε ανάπτυγμα 2ης τάξης  $1/x$, εισάγουμε :
\begin{center}{\en\tt series((2*x-1)*exp(1/(x-1)),x=+infinity,4)}\end{center}
Έξοδος :
\begin{center}{\en\tt
    2*x+1+2/x+17/6*(1/x)\verb|^|2+(1/x)\verb|^|3*order\_size(1/x)}\end{center}
\item
Βρείτε το ανάπτυγμα σε σειρά 2ης τάξης της παράστασης $(2x-1)e^{\frac{1}{x-1}})$ στην περιοχή του {\en\tt x=-$\infty$}.\\
Είσοδος :
\begin{center}{\en\tt series((2*x-1)*exp(1/(x-1)),x=-infinity,4)}\end{center}
Έξοδος:
\begin{center}{\en\tt -2*(-x)+1-2*(-1/x)+17/6*(-1/x)\verb|^|2+\\
(-1/x)\verb|^|3*order\_size(-1/x)}\end{center}
\end{enumerate}
\item  ανάπτυγμα σε σειρά με μονή κατεύθυνση\\
Η τέταρτη παράμετρος υποδεικνύει την  κατεύθυνση :
\begin{itemize}
\item {\tt 1} για ανάπτυγμα σε σειρά στην περιοχή του $x=a$ με
$ \ x>a$,
\item{\tt -1} για ανάπτυγμα σε σειρά στην περιοχή του  $x=a$ με 
$ \ x<a$,
\item{\tt 0}  για ανάπτυγμα σε σειρά στην περιοχή του  $x=a$ με
$ \ x \neq a$.
\end{itemize}
Για παράδειγμα, 
βρείτε το ανάπτυγμα σε σειρά 2ης τάξης της παράστασης $\ \frac{(1+x)^{\frac{1}{x}}}{x^3}\ $ 
στην περιοχή του $x=0^+$. \\Είσοδος :
\begin{center}{\en\tt series((1+x)\verb|^|(1/x)/x\verb|^|3,x=0,2,1)}\end{center}
Έξοδος :
\begin{center}{\en\tt exp(1)/x\verb|^|3+(-(exp(1)))/2/x\verb|^|2+1/x*order\_size(x)}\end{center}
\end{itemize}

\subsection{Ολοκληρωτικό υπόλοιπο μιας παράστασης σε ένα σημείο : {\tt\textlatin{ residue}}}\index{residue}
{\en\tt residue} παίρνει σαν όρισμα μια παράσταση που εξαρτάται από μια μεταβλητή, 
το όνομα της μεταβλητής αυτής και έναν μιγαδικό $a$ ή μια παράσταση
που εξαρτάται από μία μεταβλητή και την ισότητα : {\en variable\_name=$a$}.\\
{\en\tt residue} επιστρέφει το ολοκληρωτικό υπόλοιπο της παράστασης στο σημείο $a$.\\
Είσοδος :
\begin{center}{\en\tt residue(cos(x)/x\verb|^|3,x,0)}\end{center}
ή :
\begin{center}{\en\tt residue(cos(x)/x\verb|^|3,x=0)}\end{center}
Έξοδος :
\begin{center}{\en\tt (-1)/2}\end{center}


\subsection{Ανάπτυγμα \tt\textlatin{Pad\'e}: {\tt\textlatin{ pade}}}\index{pade}
{\en\tt pade} παίρνει 4 ορίσματα
\begin{itemize}
\item μια παράσταση $f$, 
\item το όνομα της μεταβλητής από την οποία εξαρτάται η παράσταση,
\item έναν ακέραιο $n$ ή ένα πολυώνυμο $N$,
\item έναν ακέραιο $p$.
\end{itemize}
{\en\tt pade} επιστρέφει ένα ρητό κλάσμα  $P/Q$ τέτοιο ώστε {\en\tt
  degree(P)}$<p$ και $P/Q=f \pmod{x^{n+1}}$ ή $P/Q=f \pmod{N}$.
Στην πρώτη περίπτωση, σημαίνει ότι $P/Q$ και $f$ έχουν το ίδιο ανάπτυγμα
{\tt\textlatin{Taylor}} στο 0 μέχρι την τάξη $n$.\\ 
Είσοδος :
\begin{center}{\en\tt pade(exp(x),x,5,3)}\end{center}
ή :
\begin{center}{\en\tt pade(exp(x),x,x\verb|^|6,3)}\end{center}
Έξοδος :
\begin{center}{\en\tt (3*x\verb|^|2+24*x+60)/(-x\verb|^|3+9*x\verb|^|2-36*x+60)}\end{center}
Για να επαληθεύσετε εισάγετε :
\begin{center}{\en\tt taylor((3*x\verb|^|2+24*x+60)/(-x\verb|^|3+9*x\verb|^|2-36*x+60))}\end{center}
Έξοδος :
\begin{center}{\en\tt 1+x+1/2*x\verb|^|2+1/6*x\verb|^|3+1/24*x\verb|^|4+1/120*x\verb|^|5+x\verb|^|6*order\_size(x)}\end{center}
το οποίο είναι το ανάπτυγμα 5-ης τάξης της  {\en\tt exp(x)} στο $x=0$.\\
Είσοδος :
\begin{center}{\en\tt pade((x\verb|^|15+x+1)/(x\verb|^|12+1),x,12,3)}\end{center}
ή :
\begin{center}{\en\tt pade((x\verb|^|15+x+1)/(x\verb|^|12+1),x,x\verb|^|13,3)}\end{center}
Έξοδος :
\begin{center}{\en\tt x+1}\end{center}
Είσοδος :
\begin{center}{\en\tt pade((x\verb|^|15+x+1)/(x\verb|^|12+1),x,14,4)}\end{center}
ή :
\begin{center}{\en\tt pade((x\verb|^|15+x+1)/(x\verb|^|12+1),x,x\verb|^|15,4)}\end{center}
Έξοδος :
\begin{center}{\en\tt (-2*x\verb|^|3-1)/(-x\verb|^|11+x\verb|^|10-x\verb|^|9+x\verb|^|8-x\verb|^|7+x\verb|^|6-x\verb|^|5+x\verb|^|4- x\verb|^|3-x\verb|^|2+x-1)}\end{center}
Για να επαληθεύσετε εισάγετε :
\begin{center}{\en\tt series(ans(),x=0,15)}\end{center}
Έξοδος :
\begin{center}{\en\tt 1+x-x\verb|^|{12}-x\verb|^|{13}+2x\verb|^|{15}+x\verb|^|{16}*order\_size(x)}\end{center}
έπειτα εισάγετε :
\begin{center}{\en\tt series((x\verb|^|15+x+1)/(\verb|x^|12+1),x=0,15)}\end{center}
Έξοδος :
\begin{center}{\en\tt 1+x-x\verb|^|{12}-x\verb|^|{13}+x\verb|^|{15}+x\verb|^|{16}*order\_size(x)}\end{center}
Αυτές οι 2 παραστάσεις έχουν το ίδιο ανάπτυγμα 14ης τάξης στο $x=0$.

\section{Διαστήματα}
\subsection{Ορισμός ενός διαστήματος : \tt\textlatin{{ a1..a2}}}\index{..|textbf}
Ένα διάστημα αναπαρίσταται από 2 πργματικούς αριθμούς
που διαχωρίζονται από {\en\tt ..} , για παράδειγμα
\begin{center}{\en\tt 1..3}\\
{\en\tt 1.2..sqrt(2)}
\end{center}
Είσοδος :
\begin{center}{\en\tt A:=1..4}\end{center}
\begin{center}{\en\tt B:=1.2..sqrt(2)}\end{center}
{\bf Προσοχή!}\\
Η διάταξη των ορίων του διαστήματος είναι σημαντική. Για
παράδειγμα εάν εισάγετε
\begin{center}
{\en\tt B:=2..3; C:=3..2},
\end{center}
τότε {\en\tt B} και {\en\tt C} είναι διαφορετικά, και {\en\tt B==C} επιστρέφει {\en\tt 0}.

\subsection{Άκρα ενός διαστήματος : \tt\textlatin{ left right}}\index{[]}\index{sommet}\index{feuille}\index{op}\index{left}\index{right} 
\noindent{{\en\tt left} (αντιστ. {\en\tt right}) παίρνει σαν όρισμα ένα διάστημα.\\
{\en\tt left} (αντιστ. {\en\tt right}) επιστρέφει το αριστερό (αντιστ. δεξί) άκρο
αυτού του διαστήματος}.\\
Σημειώσατε ότι {\tt\textlatin{ ..}} είναι ένας ενθηματικός τελεστής, γι' αυτό:
\begin{itemize}
\item {\en\tt sommet(1..5)} ισούται με {\tt '..'} και {\en\tt feuille(1..5)}
 ισούται με {\en\tt (1,5)}.
\item το όνομα του διαστήματος ακολουθούμενο από
{\en\tt [0]} επιστρέφει τον τελεστή {\en\tt ..}
\item 
το όνομα του διαστήματος ακολουθούμενο από {\en\tt [1]} 
(ή από την εντολή {\en\tt left})  επιστρέφει το αριστερό άκρο.
\item
Το όνομα του διαστήματος ακολουθούμενο από {\en\tt [2]} 
(ή από την εντολή {\en\tt right}) 
επιστρέφει το δεξί άκρο.
\end{itemize}
Είσοδος :
\begin{center}{\en\tt (3..5)[0]}\end{center}
ή :
\begin{center}{\en\tt sommet(3..5)}\end{center}
Έξοδος :
\begin{center}{\tt '..'}\end{center}
Είσοδος :
\begin{center}{\en\tt left(3..5)}\end{center}
ή :
\begin{center}{\en\tt (3..5)[1]}\end{center}
ή :
\begin{center}{\en\tt feuille(3..5)[0]}\end{center}
ή :
\begin{center}{\en\tt op(3..5)[0]}\end{center}
Έξοδος :
\begin{center}{\en\tt 3}\end{center}
Είσοδος :
\begin{center}{\en\tt right(3..5)}\end{center}
ή :
\begin{center}{\en\tt (2..5)[2]}\end{center}
ή :
\begin{center}{\en\tt feuille(3..5)[1]}\end{center}
ή :
\begin{center}{\en\tt op(3..5)[1]}\end{center}
Έξοδος :
\begin{center}{\en\tt 5}\end{center}
{\bf Σχόλιο}\\
{\en\tt left} (αντιστ. {\tt\textlatin{right}}) επιστρέφει επίσης το αριστερό (αντιστ. δεξί) μέλος μιας
εξίσωσης (για παράδειγμα {\en\tt left(2*x+1=x+2)} επιστρέφει {\en\tt 2*x+1}).

\subsection{Κέντρο διαστήματος :{ \tt\textlatin{ interval2center}}}\index{interval2center}
\noindent{{\en\tt interval2center} παίρνει σαν όρισμα ένα διάστημα ή μια λίστα διαστημάτων.\\
{\en\tt interval2center} επιστρέφει το κέντρο του διαστήματος ή την λίστα των κέντρων
αυτών των διαστημάτων.}\\
Είσοδος :
\begin{center}{\en\tt interval2center(3..5)}\end{center}
Έξοδος :
\begin{center}{\en\tt 4}\end{center}
Είσοδος :
\begin{center}{\en\tt interval2center([2..4,4..6,6..10])}\end{center}
Έξοδος :
\begin{center}{\en\tt [3,5,8]}\end{center}

\subsection{Διαστήματα που ορίζονται από το κέντρο τους : \\{\tt\textlatin{ center2interval}}}\index{center2interval}
\noindent{{\en\tt center2interval} παίρνει σαν όρισμα ένα διάνυσμα {\en\tt V} πραγματικών αριθμών
και προαιρετικά έναν πραγματικό αριθμό σαν δεύτερο όρισμα 
(από προεπιλογή {\en\tt V[0]-(V[1]-V[0])/2}).\\
{\en\tt center2interval} επιστρέφει το διάνυσμα των διαστημάτων που έχουν
τις πραγματικές τιμές του πρώτου ορίσματος σαν κέντρα, και όπου η τιμή
του δευτέρου ορίσματος είναι
το αριστερό άκρο του πρώτου διαστήματος.}\\
Είσοδος:
\begin{center}{\en\tt center2interval([3,5,8])}\end{center}
ή (επειδή από προεπιλογή η  τιμή είναι 3-(5-3)/2=2) :
\begin{center}{\en\tt center2interval([3,5,8],2)}\end{center}
Έξοδος :
\begin{center}{\en\tt [2..4,4..6,6..10]}\end{center}
Είσοδος:
\begin{center}{\en\tt center2interval([3,5,8],2.5)}\end{center}
Έξοδος :
\begin{center}{\en\tt [2.5..3.5,3.5..6.5,6.5..9.5]}\end{center}

\section{Ακολουθία}
\subsection{Ορισμός : {\tt\textlatin{ seq[]  ()}}}\index{seq[]}\index{()}
Μια ακολουθία αναπαρίσταται από
μια ακολουθία στοιχείων που χωρίζονται με κόμμα, και με
οριοθέτες είτε  {\en\tt ( )} είτε {\en\tt seq[…]}, για παράδειγμα
\begin{center}
{\en\tt (1,2,3,4)}\\
{\en\tt seq[1,2,3,4]}
\end{center}
Είσοδος :
\begin{center}{\en\tt A:=(1,2,3,4)} ή {\en\tt A:=seq[1,2,3,4]}\end{center}
%\begin{center}{\en\tt B:=(5,6,3,4)} ή {\en\tt B:=seq[5,6,3,4]}\end{center}
{\bf Σχόλια}
\begin{itemize}
\item Η διάταξη των στοιχειών της ακολουθίας  είναι σημαντική.
Για παράδειγμα, εάν {\en\tt B:=(5,6,3,4)} και  {\en\tt C:=(3,4,5,6)}, τότε
{\en\tt B==C} επιστρέφει {\en\tt 0}.
\item
(δείτε επίσης \ref{sec:seq})\\
{\en\tt seq([0,2])=(0,0)} και {\en\tt seq([0,1,1,5])=[0,0,0,0,0]} αλλά\\
{\en\tt seq[0,2]=(0,2)} και {\en\tt seq[0,1,1,5]=(0,1,1,5)}
\end{itemize}

\subsection{Συνένωση δύο ακολουθιών : {\tt\textlatin{ ,}}}\index{,}
Ο ενθηματικός τελεστής {\tt ','} συνενώνει δύο ακολουθίες.\\
Είσοδος :
\begin{center}{\en\tt A:=(1,2,3,4)}\end{center}
\begin{center}{\en\tt B:=(5,6,3,4)}\end{center}
\begin{center}{\en\tt A,B}\end{center}
Έξοδος :
\begin{center}{\en\tt (1,2,3,4,5,6,3,4)}\end{center}

\subsection{Επιλογή στοιχείου μιας ακολουθίας : {\tt\textlatin{ []}}}\index{[]}
Τα στοιχεία της ακολουθίας έχουν δείκτες που ξεκινούν από 0 στον τρόπο λειτουργίας {\en\tt Xcas} ή από το 1 στους άλλους τρόπους λειτουργίας.\\
Μια ακολουθία, ή το όνομα μιας μεταβλητής που έχει ανατεθεί σε μια ακολουθία,
όταν ακολουθείται από {\en\tt [n]} επιστρέφει το στοιχείο της ακολουθίας με δείκτη {\en\tt n} .\\
Είσοδος :
\begin{center}{\en\tt (0,3,2)[1]}\end{center}
Έξοδος :
\begin{center}{\en\tt 3}\end{center}

\subsection{Υποακολουθία μιας ακολουθίας : {\tt\textlatin{ []}}}\index{[]}\index{..}
Μια ακολουθία, ή το όνομα μιας μεταβλητής που έχει ανατεθεί σε μια ακολουθία,
όταν ακολουθείται από {\en\tt [n1..n2]} επιστρέφει την υποακολουθία αυτής της ακολουθίας που
αρχίζει με τον δείκτη {\en\tt n1} και τελειώνει με τον δείκτη {\en\tt n2}.\\
Είσοδος :
\begin{center}{\en\tt (0,1,2,3,4)[1..3]}\end{center}
Έξοδος :
\begin{center}{\en\tt (1,2,3)}\end{center}

\subsection{Δημιουργία μιας ακολουθίας ή μιας λίστας : {\tt\textlatin{ seq \$}}}\index{seq|textbf}\index{\$|textbf}\label{sec:seq}
\noindent{{\en\tt seq}} παίρνει δύο, τρία, τέσσερα ή πέντε ορίσματα : το πρώτο όρισμα είναι μια παράσταση ως προς την μεταβλητή (για παράδειγμα) $j$ και τα υπόλοιπα ορίσματα περιγράφουν ποιές τιμές του $j$ θα χρησιμοποιηθούν για να παράγουν την ακολουθία.
Πιο συγκεκριμένα, το $j$ θα μεταβληθεί από $a$ έως $b$:
\begin{itemize}
\item με ένα  βήμα που από προεπιλογή είναι 1 ή -1: {\en\tt j=a..b} ή 
{\en\tt j,a..b} (σύνταξη του {\tt\textlatin{Maple}}), {\en\tt j,a,b} (σύνταξη του {\tt\textlatin{TI}})
\item ή με ένα συγκεκριμένο βήμα: 
{\en\tt j=a..b,p} (σύνταξη του {\tt\textlatin{Maple}}), {\en\tt j,a,b,p} (σύνταξη του {\tt\textlatin{TI}}).
\end{itemize}
Εάν χρησιμοποιείται η σύνταξη του {\tt\textlatin{Maple}}, η {\en\tt seq} επιστρέφει μια ακολουθία,
και εάν χρησιμοποιείται η σύνταξη του  {\tt\textlatin{TI}}, η {\en\tt seq} επιστρέφει μια λίστα.

{\en\tt \$} είναι η ενθηματική εκδοχή του {\en\tt seq}, όταν {\en\tt seq} έχει δύο ορίσματα,
και επιστρέφει πάντα μια ακολουθία.\\
{\bf Σχόλιο:} 
\begin{itemize}
\item Στον τρόπο λειτουργίας {\en\tt Xcas}, η προτεραιότητα του {\en\tt \$} δεν είναι ίδια με εκείνη,
για παράδειγμα, του  {\en\tt Maple}, γιαυτό σε περίπτωση αμφιβολίας
βάλτε τα ορίσματα του {\en\tt \$} σε παρενθέσεις.
Για παράδειγμα, το ισοδύναμο του {\en\tt seq(j\verb|^|2,j=-1..3)} είναι
{\en\tt (j\verb|^|2)\$(j=-1..3)} και 
επιστρέφει {\en\tt (1,0,1,4,9)}. 
Το ισοδύναμο του {\en\tt seq(4,3)} είναι {\en\tt 4\$3} και επιστρέφει 
{\en\tt (4,4,4)}.
\item
Με την σύνταξη του {\en\tt Maple}, το {\en\tt j,a..b,p} δεν είναι έγκυρο.
Για να ορίσετε ένα βήμα $p$, για την μεταβολή του 
$j$ από το $a$ στο $b$, χρησιμοποιήστε {\en\tt j=a..b,p} ή χρησιμοποιήστε πρώτα την σύνταξη του {\en\tt TI}
{\en\tt j,a,b,p} και πάρτε την ακολουθία από την λίστα με το {\en\tt op(...)}.
\end{itemize}
Συνοψίζοντας, οι διαφορετικοί τρόποι για να δημιουργήσετε μια ακολουθία είναι οι εξής :
\begin{itemize}
\item με σύνταξη του {\en\tt Maple}  
\begin{enumerate}
\item {\en\tt seq} έχει δύο ορίσματα : 
είτε μια παράσταση που εξαρτάται από την παράμετρο 
(για παράδειγμα) $j$ και  $j=a..b$,  όπου $a$ και $b$ είναι πραγματικοί αριθμοί, 
είτε μια σταθερή παράσταση και έναν ακέραιο $n$.\\ 
{\en\tt seq} επιστρέφει :  είτε την ακολουθία όπου $j$ αντικαθίσται στην
παράσταση από $a$, $a+1$,...,$b$ εάν $b>a$ και από $a$, $a-1$,...,$b$ εάν $b<a$,
είτε  την ακολουθία που προκύπτει αντιγράφοντας $n$ φορές την σταθερά.
\item {\en\tt seq} έχει τρία ορίσματα : μια παράσταση που εξαρτάται από την παράμετρο
(για παράδειγμα) $j$ και $j=a..b,p$ όπου $a$, $b$ και $p$ είναι 
πραγματικοί αριθμοί.\\
{\en\tt seq} επιστρέφει την ακολουθία όπου το $j$ αντικαθίσταται στην
παράσταση από τα $a$, $a+p$,...,$b$ εάν $b>a$ και από $a$, $a-p$,...,$b$ 
εάν $b<a$.\\ 
Σημειώσατε ότι $j,a..b$ είναι επίσης έγκυρο αλλά $j,a..b,p$ δεν είναι έγκυρο. 
\end{enumerate}
\item με σύνταξη του {\en\tt TI} 
\begin{enumerate}
\item {\en\tt seq} έχει τέσσερα ορίσματα : μια παράσταση που εξαρτάται από την παράμετρο (για
παράδειγμα) $j$, το όνομα της παραμέτρου (για παράδειγμα) $j$, $a$ και $b$ όπου
$a$ και  $b$ είναι πραγματικοί.\\
{\en\tt seq} επιστρέφει την λίστα όπου $j$ αντικαθίσταται στην παράσταση από
 $a$, $a+1$,...,$b$ εάν $b>a$ και από $a$, $a-1$,...,$b$ εάν $b<a$.
\item  {\en\tt seq} έχει πέντε ορίσματα : μια παράσταση που εξαρτάται από την παράμετρο (για 
παράδειγμα) $j$, το όνομα της παραμέτρου (για παράδειγμα) $j$, $a$, $b$ και $p$ 
όπου $a$, $b$ και $p$ είναι πραγματικοί αριθμοί.\\
{\en\tt seq} επιστρέφει την λίστα όπου το
$j$ αντικαθίσταται στην παράσταση
από $a$, $a+p$,...,$a+k*p$ ($a+k*p \leq b <a+(k+1)*p$ ή 
$a+k*p \geq b> a+(k+1)*p$). 
Από προεπιλογή, $p$=1 εάν $b>a$ και $p$=-1  εάν $b<a$. 
%Εάν το πρόσημο του $p$ δεν είναι σωστό, το {\tt Xcas} το διορθώνει !
\end{enumerate}
\end{itemize}
{\bf Σημειώσατε} ότι 
με σύνταξη του {\en\tt Maple}, η {\en\tt seq} δεν παίρνει παραπάνω από 3 ορίσματα και
επιστρέφει μια ακολουθία, ενώ
με σύνταξη του {\en\tt TI}, η {\en\tt seq} παίρνει τουλάχιστον 4 ορίσματα
και επιστρέφει μια λίστα.\\ \\
Είσοδος για να έχουμε μια ακολουθία με ίδια στοιχεία :
\begin{center}{\en\tt seq(t,4)}\end{center}
ή : 
\begin{center}{\en\tt seq(t,k=1..4)}\end{center}
ή :
\begin{center}{\en\tt t\$4}\end{center} 
Έξοδος :
\begin{center}{\en\tt (t,t,t,t)}\end{center}
Είσοδος για να έχουμε μια ακολουθία:
\begin{center}{\en\tt seq(j\verb|^|3,j=1..4)}\end{center}
ή : 
\begin{center}{\en\tt (j\verb|^|3)\$(j=1..4)}\end{center} 
ή:
\begin{center}{\en\tt seq(j\verb|^|3,j,1..4)}\end{center}
Έξοδος :
\begin{center}{\en\tt (1,4,9,16)}\end{center}
Είσοδος για να έχουμε μια ακολουθία :
\begin{center}{\en\tt seq(j\verb|^|3,j=-1..4,2)}\end{center}
Έξοδος :
\begin{center}{\en\tt (1,1,9)}\end{center}
Ή για να έχουμε μια λίστα,\\
Είσοδος :
\begin{center}{\en\tt seq(j\verb|^|3,j,1,4)}\end{center}
Έξοδος :
\begin{center}{\en\tt [1,4,9,16]}\end{center}
Είσοδος :
\begin{center}{\en\tt seq(j\verb|^|3,j,0,5,2)}\end{center}
Έξοδος :
\begin{center}{\en\tt [0,8,64]}\end{center}
Είσοδος :
\begin{center}{\en\tt seq(j\verb|^|3,j,5,0,-2)}\end{center}
ή
\begin{center}{\en\tt seq(j\verb|^|3,j,5,0,2)}\end{center}
Έξοδος :
\begin{center}{\en\tt [125,27,1]}\end{center}
Είσοδος :
\begin{center}{\en\tt seq(j\verb|^|3,j,1,3,0.5)}\end{center}
Έξοδος :
\begin{center}{\en\tt [1,3.375,8,15.625,27]}\end{center}
Είσοδος :
\begin{center}{\en\tt seq(j\verb|^|3,j,1,3,1/2)}\end{center}
Έξοδος :
\begin{center}{\en\tt [1,27/8,8,125/8,27]}\end{center}
{\bf Παραδείγματα}
\begin{itemize}
\item Για να βρείτε την τρίτη παράγωγο της $\ \ln(t)$, εισάγετε :
\begin{center}{\en\tt diff(log(t),t\$3)}\end{center}
Έξοδος :
\begin{center}{\en\tt -((-(2*t))/t\verb|^|4)}\end{center}
\item Είσοδος :
\begin{center}{\en\tt l:=[[2,3],[5,1],[7,2]]}\end{center}
\begin{center}{\en\tt seq((l[k][0])\$(l[k][1]),k=0 .. size(l)-1)}\end{center}
Έξοδος :
\begin{center}{\en\tt 2,2,2,seq[5],7,7}\end{center}
έπειτα η {\en\tt eval(ans())} επιστρέφει:
\begin{center}{\en\tt 2,2,2,5,7,7}\end{center}
\item  Για να μετασχηματίσετε μια συμβολοσειρά σε μια λίστα από τους χαρακτήρες της ορίστε πρώτα την συνάρτηση :
{\en \tt \begin{verbatim}
 f(chn):={
 local l;
 l:=size(chn);
 return seq(chn[j],j,0,l-1);
 }
\end{verbatim}}
και μετά εισάγετε:
\begin{center}{\en\tt f("abracadabra")}\end{center}
Έξοδος :
\begin{center}{\en\tt
    ["a","b","r","a","c","a","d","a","b","r","a"]}\end{center}
\end{itemize}


\subsection{Μετασχηματισμός ακολουθίας σε λίστα : {\tt\textlatin{ [] nop}}}\index{[]}\index{nop}
Για να μετασχηματίσουμε  μια ακολουθία σε λίστα, απλά βάζουμε αγκύλες ({\en\tt []}) γύρω από την
ακολουθία ή χρησιμοποιούμε την εντολή {\en\tt nop}.\\
Είσοδος :
\begin{center}{\en\tt [seq(j\verb|^|3,j=1..4)]}\end{center}
ή :
\begin{center}{\en\tt seq(j\verb|^|3,j,1,4)}\end{center}
ή :
\begin{center}{\en\tt [(j\verb|^|3)\$(j=1..4)]}\end{center} 
Έξοδος :
\begin{center}{\en\tt [1,4,9,16]}\end{center}
Είσοδος:
\begin{center}{\en\tt nop(1,4,9,16)}\end{center}
Έξοδος :
\begin{center}{\en\tt [1,4,9,16]}\end{center}

\subsection{Ο τελεστής {\tt\textlatin{ +}}  σε ακολουθίες}\index{+}
Ο ενθηματικός τελεστής {\en\tt +}, με δύο ακολουθίες σαν όρισμα,
επιστρέφει το συνολικό άθροισμα των στοιχείων των 2 ακολουθιών.\\
Σημειώσατε την διαφορά με τις λίστες, όπου επιστρέφονται τα 
αθροίσματα όρου με όρο των στοιχείων των 2 λιστών.\\
Είσοδος :
\begin{center}{\en\tt (1,2,3,4,5,6)+(4,3,5)}\end{center}
ή :
\begin{center}{\tt '+'\en((1,2,3,4,5,6),(4,3,5))}\end{center}
Έξοδος :
\begin{center}{\en\tt 33}\end{center}
Αλλά εάν εισάγετε :
\begin{center}{\en\tt [1,2,3,4,5,6]+[4,3,5]}\end{center}
Έξοδος :
\begin{center}{\en\tt [5,5,8,4,5,6]}\end{center}
{\bf Προσοσχή}\\
Όταν ο τελεστής {\en\tt +} είναι προθηματικός, πρέπει να γίνει να αναφέρεται ({\tt '+'}).

\section{\ttΣύνολα}
\subsection{Ορισμός: {\tt\textlatin{ set[]}}}\index{\%\{ \%\}}\index{set[]}
Για να ορίσουμε ένα σύνολο στοιχείων βάζουμε τα στοιχεία, που χωρίζονται με κόμμα, μέσα στους 
οριοθέτες {\en\tt \%\{ ... \%\}} ή {\en\tt set[ ... ]}.\\
Είσοδος :
\begin{center}
{\en\tt \%\{1,2,3,4\%\}}\\
{\en\tt set[1,2,3,4]}
\end{center}
Στις απαντήσεις του {\en\tt Xcas}, οι οριοθέτες εφανίζονται
σαν $\llbracket$ και $\rrbracket$ ώστε να μην μπερεδεύουμε τα σύνολα με τις λίστες.
Για παράδειγμα, $\llbracket$1,2,3$\rrbracket$ είναι το σύνολο {\en\tt \%\{1,2,3\%\}}, 
σε αντίθεση με το [1,2,3]  που είναι η λίστα {\en\tt [1,2,3]}.\\
Είσοδος:
\begin{center}{\en\tt A:=\%\{1,2,3,4\%\}} ή {\en\tt A:=set[1,2,3,4]}\end{center}
Έξοδος :
\begin{center}{\en\tt $\llbracket$1,2,3,4$\rrbracket$  }\end{center}
Είσοδος :
\begin{center}{\en\tt B:=\%\{5,5,6,3,4\%\}} ή {\en\tt B:=set[5,5,6,3,4]}\end{center}
Έξοδος :
\begin{center}{\en\tt  $\llbracket$5,6,3,4$\rrbracket$ }\end{center}
{\bf Σχόλιο}\\
Η διάταξη των στοιχείων σε ένα σύνολο δεν είναι σημαντική και
τα στοιχεία αυτά είναι όλα διακεκριμένα. Εάν εισάγετε
{\en\tt B:=\%\{5,5,6,3,4\%\}} και {\en\tt C:=\%\{3,4,5,3,6\%\}}, τότε 
{\en\tt B==C} θα επιστρέψει {\en\tt 1}.

\subsection{Ένωση δύο συνόλων ή δύο λιστών : {\tt\textlatin{ union}}}\index{union}
\noindent{{\en\tt union} είναι ένας ενθηματικός τελεστής.\\
{\en\tt union} παίρνει σαν όρισμα δύο σύνολα ή δύο λίστες.\\
{\en\tt union} επιστρέφει ένα σύνολο που είναι η ένωση  των ορισμάτων.}\\
Είσοδος : 
\begin{center}{\en\tt set[1,2,3,4] \tt\textlatin{union set}[5,6,3,4]}\end{center}
ή : 
\begin{center}{\en\tt \%\{1,2,3,4\%\} \tt\textlatin{union} \%\{5,6,3,4\%\}}\end{center}
Έξοδος :
\begin{center}{\en\tt $\llbracket$1,2,3,4,5,6$\rrbracket$}\end{center}
Είσοδος : 
\begin{center}{\en\tt [1,2,3] union [2,5,6]}\end{center}
Έξοδος :
\begin{center}{\en\tt $\llbracket$1,2,3,5,6$\rrbracket$}\end{center}

\subsection{Τομή δύο συνόλων ή δύο λιστών : {\tt\textlatin{ intersect}}}\index{intersect}
\noindent{{\en\tt intersect} είναι ένας ενθηματικός τελεστής.\\
{\en\tt intersect} παίρνει σαν όρισμα δύο σύνολα ή δύο λίστες.\\
{\en\tt intersect} επιστρέφει ένα σύνολο που είναι η τομή των ορισμάτων.}\\
Είσοδος :
\begin{center}{\en\tt set[1,2,3,4] intersect set[5,6,3,4]}\end{center}
ή :
\begin{center}{\en\tt \%\{1,2,3,4\%\} intersect \%\{5,6,3,4\%\}}\end{center}
Έξοδος :
\begin{center}{\en\tt $\llbracket$3,4$\rrbracket$}\end{center}
Είσοδος :
\begin{center}{\en\tt [1,2,3,4] intersect [5,6,3,4]}\end{center}
Έξοδος :
\begin{center}{\en\tt $\llbracket$3,4$\rrbracket$}\end{center}

\subsection{Διαφορά δύο συνόλων ή δύο λιστών: {\tt\textlatin{ minus}}}\index{minus}
\noindent{{\en\tt minus} είναι ένας ενθηματικός τελεστής.\\
{\en\tt minus}  παίρνει σαν όρισμα δύο σύνολα ή δύο λίστες.\\
{\en\tt minus} επιστρέφει ένα σύνολο που είναι η διαφορά των ορισμάτων (δηλαδή, για τα σύνολα $A, B$, η διαφορά  $A-B = \{x\ |\ x\in A\ \mbox{και}\ x\notin B\}$).}\\
Είσοδος :
\begin{center}{\en\tt set[1,2,3,4] \tt\textlatin{minus set}[5,6,3,4]}\end{center}
ή :
\begin{center}{\en\tt \%\{1,2,3,4\%\} \tt\textlatin{minus} \%\{5,6,3,4\%\}}\end{center}
Έξοδος :
\begin{center}{\en\tt $\llbracket$1,2$\rrbracket$}\end{center}
Είσοδος :
\begin{center}{\en\tt [1,2,3,4] minus [5,6,3,4]}\end{center}
Έξοδος :
\begin{center}{\en\tt $\llbracket$1,2$\rrbracket$}\end{center}

\section{Λίστες και διανύσματα}\index{[]|textbf}
\subsection{Ορισμός}
Μια λίστα (ή ένα διάνυσμα) οριοθετείται από {\en\tt [ ]},
και τα στοιχεία της πρέπει να χωρίζονται με κόμμα. 
Για παράδειγμα {\en\tt [1,2,5]} είναι μια λίστα 3 ακεραίων.

Οι λίστες μπορεί να περιέχουν λίστες (για παράδειγμα, ένας πίνακας είναι μια λίστα από λίστες
του ίδιου μεγέθους). Οι λίστες μπορεί να χρησιμοποιηθούν για να αντιπροσωπεύσουν διανύσματα
(λίστα συντεταγμένων), πίνακες, μονομεταβλητά πολυώνυμα
(λίστα συντελεστών με φθίνουσα σειρά). 

Οι λίστες είναι διαφορετικές από τις ακολουθίες, γιατί οι ακολουθίες είναι επίπεδες ({\tt\textlatin{flat}}) : δηλαδή, ένα στοιχείο
μιας ακολουθίας δεν μπορεί να είναι ακολουθία.

Οι λίστες είναι διαφορετικές από τα σύνολα, γιατί σε μια λίστα, η διάταξη των στοιχείων είναι
σημαντική και το ίδιο το στοιχείο μπορεί να επαναληφθεί σε μια λίστα (σε αντίθεση με
ένα σύνολο όπου κάθε στοιχείο εμφανίζεται μία μόνο φορά). 

Στις απαντήσεις του {\en\tt Xcas} :
\begin{itemize}
\item οι οριοθέτες των διανυσμάτων  (ή των λιστών)  εμφανίζονται σαν {\en\tt []}, 
\item οι οριοθέτες των πινάκων εμφανίζονται σαν {\bf []},
\item οι οριοθέτες των πολυωνύμων  εμφανίζονται σαν $\talloblong \ \talloblong$,
\item οι οριοθέτες των συνόλων εμφανίζονται σαν $\llbracket \ \rrbracket$.
\end{itemize}

Τα στοιχεία από τις λίστες εντοπίζονται/επιλέγονται με δείκτες που ξεκινούν από το 0 στην σύνταξη του {\tt\textlatin{Xcas}}
και από το 1 σε όλες τις άλλες συντάξεις.


\subsection{Επιλογή στοιχείου ή υπολίστας από μια λίστα : {\tt\textlatin{ []}}}\index{at|textbf}\label{sec:at}
\subsubsection{Επιλογή στοιχείου}
\noindent{Το $n$-στο στοιχείο της λίστας {\en\tt l} μεγέθους $s$
επιλέγεται με την εντολή {\en\tt l[n]}, όπου $n$ είναι στο $[0..s-1]$ ή $[1..s]$.
Η ισοδύναμη προθηματική συνάρτηση είναι
{\en\tt at}, η οποία παίρνει σαν όρισμα μια λίστα και έναν ακέραιο {\en\tt n}.\\
{\en\tt at} επιστρέφει το στοιχείο της λίστας με δείκτη {\en\tt n}.}\\
Είσοδος :
\begin{center}{\en\tt [0,1,2][1]}\end{center}
ή
\begin{center}{\en\tt at([0,1,2],1)}\end{center}
Έξοδος :
\begin{center}{\en\tt  1}\end{center}

\subsubsection{Εξαγωγή υπολίστας}
Εάν $l$ είναι μια λίστα μεγέθους $s$, {\en\tt l[n1..n2]} επιστρέφει την λίστα
που εξάγεται από την {\en\tt l} και περιέχει τα στοιχεία με δείκτες από τον $n_1$ μέχρι τον $n_2$,
όπου $0 \leq n_1\leq n_2 < s$ (στην σύνταξη του {\tt\textlatin{Xcas}}) ή
$0 < n_1\leq n_2 \leq s$ σε όλες τις άλλες συντάξεις .
Η ισοδύναμη προθηματική συνάρτηση είναι
{\en\tt at} με ορίσματα μια λίστα και ένα διάστημα ακεραίων ({\en\tt n1..n2}) .
{\bf Δείτε επίσης} : {\en\tt mid}, στην ενότητα \ref{sec:mid}.\\
Είσοδος :
\begin{center}{\en\tt [0,1,2,3,4][1..3]}\end{center}
ή
\begin{center}{\en\tt at([0,1,2,3,4],1..3)}\end{center}
Έξοδος :
\begin{center}{\en\tt  [1,2,3]}\end{center}
{\bf Προσοχή}\\
{\en\tt at} δεν μπορεί να χρησιμοποιηθεί για ακολουθίες. Πρέπει να χρησιμοποιηθεί συμβολισμός δεικτών, όπως στην εντολή {\en\tt (0,1,2,3,4,5)[2..3]}. 


\subsection{Εξαγωγή υπολίστας : {\tt\textlatin{ mid}}}\index{mid}\label{sec:mid}
{\bf Δείτε επίσης :}  {\en\tt at} στην ενότητα \ref{sec:at}.\\
\noindent{{\en\tt mid} χρησιμοποιείται για την  εξαγωγή υπολίστας από μια λίστα \index{mid}.\\{\en\tt mid} παίρνει σαν όρισμα μια λίστα, τον δείκτη της αρχής της
υπολίστας και το μήκος της υπολίστας.\\
{\en\tt mid} επιστρέφει την υπολίστα.}\\
Είσοδος :
\begin{center}{\en\tt mid([0,1,2,3,4,5],1,3)}\end{center}
Έξοδος :
\begin{center}{\en\tt  [1,2,3]}\end{center}
{\bf Προσοχή}\\
{\en\tt mid} δεν μπορεί να χρησιμοποιηθεί για την  εξαγωγή υπακολουθίας από μια ακολουθία,
γιατί τα ορίσματα του {\en\tt mid} θα συγχωνευθούν με στην ακολουθία.
Πρέπει να χρησιμοποιηθεί συμβολισμός δεικτών, όπως στην εντολή {\en\tt (0,1,2,3,4,5)[2..3]}. 

\subsection{Επιλογή του πρώτου στοιχείου μιας λίστας : {\tt\textlatin{ head}}}\index{head}
\noindent{{\en\tt head} παίρνει σαν όρισμα μια λίστα.\\
{\en\tt head} επιστρέφει το πρώτο στοιχείο της λίστας.}\\
Είσοδος :
\begin{center}{\en\tt head([0,1,2,3])}\end{center}
Έξοδος :
\begin{center}{\en\tt  0}\end{center}
{\en\tt a:=head([0,1,2,3])} κάνει το ίδιο πράγμα με {\en\tt a:=[0,1,2,3][0]}

\subsection{Απαλοιφή  στοιχείου από μια λίστα : {\tt\textlatin{ suppress}}}\index{suppress}
\noindent{{\en\tt suppress} παίρνει σαν όρισμα μια λίστα και έναν ακέραιο {\en\tt n}.\\
{\en\tt suppress} επιστρέφει την λίστα όπου το στοιχείο με δείκτη {\en\tt n}  έχει
απαλειφθεί.}\\
Είσοδος :
\begin{center}{\en\tt suppress([3,4,2],1)}\end{center}
Έξοδος :
\begin{center}{\en\tt  [3,2]}\end{center}

\subsection{Απαλοιφή  του πρώτου στοιχείου από μια λίστα : {\tt\textlatin{ tail}}}\index{tail}
\noindent{{\en\tt tail} παίρνει σαν όρισμα μια λίστα.\\
{\en\tt tail} επιστρέφει την λίστα χωρίς το πρώτο της στοιχείο.}\\
Είσοδος:
\begin{center}{\en\tt tail([0,1,2,3])}\end{center}
Έξοδος :
\begin{center}{\en\tt  [1,2,3]}\end{center}
{\en\tt l:=tail([0,1,2,3])} κάνει το ίδιο πράγμα με την
{\en\tt l:=suppress([0,1,2,3],0)}\\

\subsection{Αντιστροφή της διάταξης σε μια λίστα : {\tt\textlatin{ revlist}}}\index{revlist}
\noindent{{\en\tt revlist} παίρνει σαν όρισμα μια λίστα (αντιστ. ακολουθία).\\
{\en\tt revlist} επιστρέφει την λίστα (αντιστ. ακολουθία) με αντίστροφη διάταξη.}\\
Είσοδος :
\begin{center}{\en\tt revlist([0,1,2,3,4])}\end{center}
Έξοδος :
\begin{center}{\en\tt  [4,3,2,1,0]}\end{center}
Είσοδος :
\begin{center}{\en\tt revlist([0,1,2,3,4],3)}\end{center}
Έξοδος :
\begin{center}{\en\tt (3,[0,1,2,3,4])}\end{center}

\subsection{Περιστροφή  λίστας αρχίζοντας από το \textlatin{n}-οστό στοιχείο της : {\tt\textlatin{ rotate}}}\index{rotate}
\noindent{{\en\tt rotate} παίρνει σαν όρισμα μια λίστα και ένα ακέραιο {\en\tt n} (από προεπιλογή {\en\tt n=-1}).\\
{\en\tt rotate} περιστρέφει την λίστα {\en\tt n} θέσεις προς τα αριστερά αν {\en\tt n>0}
ή προς τα δεξιά αν {\en\tt n<0}. Τα στοιχεία που βγαίνουν έξω από την λίστα από την μια πλευρά
επανέρχονται από την άλλη πλευρά.
Από προεπιλογή {\en\tt n=-1} και το τελευταίο στοιχείο γίνεται πρώτο.}\\
Είσοδος :
\begin{center}{\en\tt rotate([0,1,2,3,4])}\end{center}
Έξοδος :
\begin{center}{\en\tt  [4,0,1,2,3]}\end{center}
Είσοδος :
\begin{center}{\en\tt rotate([0,1,2,3,4],2)}\end{center}
Έξοδος :
\begin{center}{\en\tt  [2,3,4,0,1]}\end{center}
Είσοδος :
\begin{center}{\en\tt rotate([0,1,2,3,4],-2)}\end{center}
Έξοδος :
\begin{center}{\en\tt  [3,4,0,1,2]}\end{center}

\subsection{Μετατόπιση (ολίσθηση) λίστας αρχίζοντας από το \textlatin{n}-οστό στοιχείο της : {\tt\textlatin{ shift}}}\index{shift}
\noindent{{\en\tt shift} παίρνει σαν όρισμα μια λίστα {\en\tt l} και έναν ακέραιο {\en\tt n}
(από προεπιλογή {\en\tt n}=-1).\\
{\en\tt shift} μετατοπίζει την λίστα {\en\tt n} θέσεις προς τα αριστερά εάν {\en\tt n>0} ή προς τα
δεξιά εάν {\en\tt n<0}. Τα στοιχεία που βγαίνουν έξω από την λίστα από την μια πλευρά
αντικαθίστανται από το {\en\tt undef} στην άλλη πλευρά.}\\
Είσοδος :
\begin{center}{\en\tt shift([0,1,2,3,4])}\end{center}
Έξοδος :
\begin{center}{\en\tt  [undef,0,1,2,3]}\end{center}
Είσοδος :
\begin{center}{\en\tt shift([0,1,2,3,4],2)}\end{center}
Έξοδος :
\begin{center}{\en\tt  [2,3,4,undef,undef]}\end{center}
Είσοδος :
\begin{center}{\en\tt shift([0,1,2,3,4],-2)}\end{center}
Έξοδος :
\begin{center}{\en\tt  [undef,undef,0,1,2]}\end{center}

\subsection{Τροποποίηση στοιχείου  μιας λίστας : {\tt\textlatin{ subsop}}}\index{subsop}
\noindent{{\en\tt subsop} τροποποιεί ένα στοιχείο σε μια λίστα.\\
%? άμεσα (δεν είναι απαραίτητο να αποθηκευσεις αυτό το στοιχείο σε μια μεταβλητή).\\
{\en\tt subsop} παίρνει σαν όρισμα μια λίστα και μια ισότητα (δείκτης = νέα τιμή) σε όλες τις συντάξεις --- με την διαφορά ότι στην σύνταξη του {\en\tt Maple}
η διάταξη των ορισμάτων αντιστρέφεται. \\ 
{\bf Σχόλιο} Εάν το δεύτερο όρισμα είναι {\tt '{\en k=NULL}'}, το στοιχείο με δείκτη
{\en\tt k} αφαιρείται από την λίστα.}\\
Είσοδος στον τρόπο λειτουργίας {\en\tt Xcas} (ο δείκτης του πρώτου στοιχείου είναι 0) :
\begin{center}{\en\tt subsop([0,1,2],1=5)}\end{center}
ή :
\begin{center}{\en\tt L:=[0,1,2];L[1]:=5}\end{center}
Έξοδος :
\begin{center}{\en\tt [0,5,2]}\end{center}
Είσοδος στον τρόπο λειτουργίας {\en\tt Xcas} (ο δείκτης του πρώτου στοιχείου είναι 0) :
\begin{center}{{\en\tt subsop([0,1,2],}'{\en 1=NULL}')}\end{center}
Έξοδος :
\begin{center}{\en\tt [0,2]}\end{center}
Είσοδος στον τρόπο λειτουργίας   {\en\tt Mupad, TI}  (ο δείκτης του πρώτου στοιχείου είναι 1) :
\begin{center}{\en\tt subsop([0,1,2],2=5)}\end{center}
ή :
\begin{center}{\en\tt L:=[0,1,2];L[2]:=5}\end{center}
Έξοδος :
\begin{center}{\en\tt [0,5,2]}\end{center}
Στον τρόπο λειτουργίας {\en\tt Maple} η διάταξη των ορισμάτων αντιστρέφεται και ο δείκτης του
πρώτου στοιχείου είναι 1.\\
Είσοδος :
\begin{center}{\en\tt subsop(2=5,[0,1,2])}\end{center}
ή :
\begin{center}{\en\tt L:=[0,1,2];L[2]:=5}\end{center}
Έξοδος :
\begin{center}{\en\tt [0,5,2]}\end{center}

\subsection{Μετασχηματισμός λίστας σε  ακολουθία : {\tt\textlatin{op makesuite}}}\index{op}\index{makesuite}
\noindent{{\en\tt op} ή {\en\tt makesuite} παίρνει σαν όρισμα μια λίστα.\\
{\en\tt op} ή {\en\tt makesuite} μετασχηματίζει αυτήν την λίστα σε μια ακολουθία.
\label{sec:makesuiteop}\\
Δείτε \ref{sec:op} για άλλες χρήσεις της {\en\tt op}.}\\
Είσοδος :
\begin{center}{\en\tt op([0,1,2])}\end{center}
ή :
\begin{center}{\en\tt makesuite([0,1,2])}\end{center}
Έξοδος :
\begin{center}{\en\tt (0,1,2)}\end{center}

\subsection{Μετασχηματισμός ακολουθίας σε λίστα : {\tt\textlatin{ makevector []}}}\index{makevector}\index{[]}
Οταν γύρω από μια ακολουθία μπουν αγκύλες ({\tt []})  η ακολουθία μετασχηματίζεται  σε μια λίστα ή 
σε ένα διάνυσμα . Η ισοδύναμη προθηματική συνάρτηση είναι η
{\en\tt makevector} η οποία παίρνει μια ακολουθία σαν όρισμα.\\
{\en\tt makevector} μετασχηματίζει την ακολουθία σε μια λίστα ή 
σε ένα διάνυσμα.\\
Είσοδος :
\begin{center}{\en\tt makevector(0,1,2)}\end{center}
Έξοδος :
\begin{center}{\en\tt [0,1,2]}\end{center}
Είσοδος :
\begin{center}{\en\tt a:=(0,1,2)}\end{center}
Είσοδος :
\begin{center}{\en\tt [a]}\end{center}
ή :
\begin{center}{\en\tt makevector(a)}\end{center}
Έξοδος :
\begin{center}{\en\tt [0,1,2]}\end{center}


\subsection{Μήκος λίστας : {\tt\textlatin{ size nops length}}}\index{size}\index{nops}\index{length}
\noindent{{\en\tt size} ή {\en\tt nops} ή {\en\tt length} παίρνει σαν όρισμα μια λίστα
(αντιστ. ακολουθία).\\
{\en\tt size} ή {\en\tt nops} ή {\en\tt length} επιστρέφει το μήκος της λίστας (αντιστ. 
 ακολουθία).}\\
Είσοδος :
\begin{center}{\en\tt nops([3,4,2])}\end{center}
ή
\begin{center}{\en\tt size([3,4,2])}\end{center}
ή
\begin{center}{\en\tt length([3,4,2])}\end{center}
Έξοδος :
\begin{center}{\en\tt  3}\end{center}

\subsection{Μήκοι μιας λίστας λιστών : {\tt\textlatin{ sizes}}}\index{sizes}
\noindent{{\en\tt sizes} παίρνει σαν όρισμα μια λίστα από λίστες.\\
{\en\tt sizes} επιστρέφει την λίστα με τα μήκη αυτών των λιστών.}\\
Είσοδος :
\begin{center}{\en\tt sizes([[3,4],[2]])}\end{center}
Έξοδος :
\begin{center}{\en\tt [2,1]}\end{center}

\subsection{Αλληλουχία δύο λιστών ή μιας λίστας και ενός στοιχείου : {\tt\textlatin{ concat augment}}}\index{concat|textbf}\index{augment|textbf}
\noindent{{\en\tt concat} (ή {\en\tt augment}) παίρνει σαν όρισμα μια λίστα και ένα
στοιχείο ή δύο λίστες.\\
{\en\tt concat} (ή{\en\tt augment}) συνενώνει την λίστα και το στοιχείο, ή συνενώνει 
αυτές τις δύο λίστες.}\\
Είσοδος :
\begin{center}{\en\tt concat([3,4,2],[1,2,4])}\end{center}
ή :
\begin{center}{\en\tt augment([3,4,2],[1,2,4])}\end{center}
Έξοδος :
\begin{center}{\en\tt [3,4,2,1,2,4]}\end{center}
Είσοδος :
\begin{center}{\en\tt concat([3,4,2],5)}\end{center}
ή :
\begin{center}{\en\tt augment([3,4,2],5)}\end{center}
Έξοδος :
\begin{center}{\en\tt [3,4,2,5]}\end{center}
{\bf Προσοχή}
Εάν εισάγετε :
\begin{center}{\en\tt concat([[3,4,2]],[[1,2,4]]}\end{center}
ή
\begin{center}{\en\tt augment([[3,4,2]],[[1,2,4]]}\end{center}
η έξοδος θα είναι:
\begin{center}{\en\tt [[3,4,2,1,2,4]]}\end{center}

\subsection{Επισύναψη στοιχείου στο τέλος μια λίστας : {\tt\textlatin{ append}}}\index{append}
\noindent{{\en\tt append} παίρνει σαν όρισμα μια λίστα και ένα στοιχείο.\\
{\en\tt append} βάζει αυτό το στοιχείο στο τέλος της λίστας.}\\
Είσοδος :
\begin{center}{\en\tt append([3,4,2],1)}\end{center}
Έξοδος :
\begin{center}{\en\tt  [3,4,2,1]}\end{center}
Είσοδος :
\begin{center}{\en\tt append([1,2],[3,4])}\end{center}
Έξοδος :
\begin{center}{\en\tt [1,2,[3,4]]}\end{center}

\subsection{Πρόταξη  στοιχείου στην αρχή της λίστας : {\tt\textlatin{ prepend}}}\index{prepend}
\noindent{{\en\tt prepend} παίρνει σαν όρισμα μια λίστα και ένα στοιχείο.\\
{\en\tt prepend} προτάσσει αυτό το στοιχείο στην αρχή της λίστας.}\\
Είσοδος :
\begin{center}{\en\tt prepend([3,4,2],1)}\end{center}
Έξοδος :
\begin{center}{\en\tt  [1,3,4,2]}\end{center}
Είσοδος :
\begin{center}{\en\tt prepend([1,2],[3,4])}\end{center}
Έξοδος :
\begin{center}{\en\tt [[3,4],1,2]}\end{center}

\subsection{Ταξινόμηση: {\tt\textlatin{ sort}}}\index{sort}
\noindent{{\en\tt sort}} παίρνει σαν όρισμα μια λίστα ή μια παράσταση.
\begin{itemize}
\item Για λίστα,\\
{\en\tt sort} επιστρέφει τη λίστα ταξινομημένη σε ανιούσα τάξη.\\
Είσοδος :
\begin{center}{\en\tt sort([3,4,2])}\end{center}
Έξοδος :
\begin{center}{\en\tt [2,3,4]}\end{center}

\item Για παράσταση,\\
 {\en\tt sort} ταξινομεί και συλλέγει όρους σε αθροίσματα και γινόμενα.\\
Είσοδος :  
\begin{center}{\en\tt sort(exp(2*ln(x))+x*y-x+y*x+2*x)}\end{center}
Έξοδος :
\begin{center}{\en\tt 2*x*y+exp(2*ln(x))+x}\end{center}  
Είσοδος :  
\begin{center}{\en\tt simplify(exp(2*ln(x))+x*y-x+y*x+2*x)}\end{center}
Έξοδος :
\begin{center}{\en\tt x\verb|^|2+2*x*y+x}\end{center}  
\end{itemize}
{\en\tt sort} δέχεται ένα προαιρετικό δεύτερο όρισμα, το οποίο είναι μια διμεταβλητή συνάρτηση που επιστρέφει 0 ή 1. Εάν παρέχεται, αυτή η συνάρτηση
θα χρησιμοποιηθεί για τη διάταξη της λίστας, για παράδειγμα
{\en\tt (x,y)->x>=y} μπορεί να χρησιμοποιηθεί σαν δεύτερο όρισμα
για να ταξινομήσουμε τη λίστα σε κατιούσα τάξη.
Αυτό μπορεί επίσης να χρησιμοποιηθεί για να ταξινομήσουμε μια λίστα από λίστες
({\en\tt sort} με ένα όρισμα δεν γνωρίζει πως να ταξινομεί).\\
Είσοδος :
\begin{center}{\en\tt sort([3,4,2],(x,y)->x>=y)}\end{center}
Έξοδος :
\begin{center}{\en\tt [4,3,2]}\end{center}

\subsection{Ταξινόμηση λίστας σε ανιούσα τάξη : {\tt\textlatin{ SortA}}}\index{SortA}
\noindent{{\en\tt SortA} παίρνει σαν όρισμα μια λίστα.\\
{\en\tt SortA} επιστρέφει την λίστα διατεταγμένη σε ανιούσα τάξη.}\\
Είσοδος :
\begin{center}{\en\tt SortA([3,4,2])}\end{center}
Έξοδος :
\begin{center}{\en\tt [2,3,4]}\end{center}
{\en\tt SortA} μπορεί να έχει ένα πίνακα σαν όρισμα και σε αυτή την περίπτωση, 
η {\en\tt SortA} τροποποιεί την διάταξη των στηλών ταξινομώντας την πρώτη γραμμή
του πίνακα σε ανιούσα τάξη.\\
Είσοδος :
\begin{center}{\en\tt SortA([[3,4,2],[6,4,5]])}\end{center}
Έξοδος :
\begin{center}{\en\tt [[2,3,4],[5,6,4]]}\end{center}

\subsection{Ταξινόμηση λίστας σε κατιούσα τάξη : {\tt\textlatin{ SortD}}}\index{SortD}
\noindent{{\en\tt SortD} παίρνει σαν όρισμα μια λίστα.\\
{\en\tt SortD} επιστρέφει την λίστα διατεταγμένη σε κατιούσα τάξη.}\\
Είσοδος :
\begin{center}{\en\tt SortD([3,4,2])}\end{center}
Έξοδος :
\begin{center}{\en\tt [2,3,4]}\end{center}
{\en\tt SortD} μπορεί να έχει ένα πίνακα σαν όρισμα και σε αυτή την περίπτωση,
η {\en\tt SortD} τροποποιεί την διάταξη των στηλών ταξινομώντας την πρώτη γραμμή
του πίνακα σε κατιούσα τάξη.\\
Είσοδος :
\begin{center}{\en\tt SortD([[3,4,2],[6,4,5]])}\end{center}
Έξοδος :
\begin{center}{\en\tt [[4,3,2],[4,6,5]]}\end{center}

\subsection{Επιλογή  στοιχείων μιας λίστας : {\tt\textlatin{ select}}}\index{select}
\noindent{\en\tt select} παίρνει σαν ορίσματα : μία λογική συνάρτηση {\en\tt f}
 και μία λίστα {\en\tt L}.\\
{\en\tt select} επιλέγει στην λίστα {\en\tt L}, τα στοιχεία {\en\tt c} τέτοια ώστε
{\en\tt f(c)==true}.\\
Είσοδος :
\begin{center}{\en\tt select(x->(x>=2),[0,1,2,3,1,5])}\end{center}
Έξοδος :
\begin{center}{\en\tt  [2,3,5]}\end{center}

\subsection{Απαλοιφή στοιχείων από μια λίστα: {\tt\textlatin{ remove}}}\index{remove}
\noindent{{\en\tt remove} παίρνει σαν όρισμα : μια λογική συνάρτηση {\en\tt f} και μια λίστα
 {\en\tt L}.\\
{\en\tt remove} απαλείφει από την λίστα {\en\tt L}, τα στοιχεία {\en\tt c} τέτοια ώστε 
{\en\tt f(c)==true}.}\\
Είσοδος :
\begin{center}{\en\tt remove(x->(x>=2),[0,1,2,3,1,5])}\end{center}
Έξοδος :
\begin{center}{\en\tt  [0,1,1]}\end{center}
{\bf Σχόλιο} Το ίδιο ισχύει και στις συμβολοσειρές, για παράδειγμα,
για να αφαιρέσουμε όλα τους χαρακτήρες {\tt\textlatin{"a"}} μιας συμβολοσειράς εισάγουμε:\\
Είσοδος :
\begin{center}{\en\tt ord("a")}\end{center}
Έξοδος :
\begin{center}{\en\tt  97}\end{center}
Είσοδος :
{\en\tt \begin{verbatim}
 f(chn):={
 local l:=length(chn)-1;
 return remove(x->(ord(x)==97),seq(chn[k],k,0,l));
 }
\end{verbatim}}
\noindent Μετά, εισάγουμε:
\begin{center}{\en\tt f("abracadabra")}\end{center}
Έξοδος :
\begin{center}{\en\tt  ["b","r","c","d","b","r"]}\end{center}
Για να πάρουμε την συμβολοσειρά που αντιστοιχεί στους παραπάνω χαρακτήρες, εισάγουμε :
\begin{center}{\en\tt  char(ord(["b","r","c","d","b","r"]))}\end{center}
Έξοδος :
\begin{center}{\en\tt "brcdbr"}\end{center}

\subsection{Έλεγχος αν μια τιμή είναι μέσα σε μια λίστα: {\tt\textlatin {member}}}\index{member|textbf}
\noindent{{\en\tt member} παίρνει σαν όρισμα μια τιμή {\en\tt c} και μια λίστα
(ή ένα σύνολο) {\en\tt L}.\\
{\en\tt member} είναι μια συνάρτηση που ελέγχει αν το {\en\tt c} είναι στοιχείο της 
λίστας {\en\tt L}.\\
{\en\tt member} επιστρέφει {\en\tt 0} εάν το {\en\tt c} δεν είναι στην {\en\tt L}, ή
έναν αυστηρά θετικό ακέραιο  ($>0$) ο οποίος είναι 
1 + τον δείκτη της πρώτης εμφάνισης του {\en\tt c} στην {\en\tt L}.\\
Σημειώσατε τη διάταξη των ορισμάτων (απαιτείται για συμβατικούς λόγους)}\\
Είσοδος :
\begin{center}{\en\tt member(2,[0,1,2,3,4,2])}\end{center}
Έξοδος :
\begin{center}{\en\tt  3}\end{center}
Είσοδος :
\begin{center}{\en\tt member(2,\%\{0,1,2,3,4,2\%\})}\end{center}
Έξοδος :
\begin{center}{\en\tt  3}\end{center}

\subsection{Έλεγχος αν μια τιμή είναι μέσα σε μια λίστα :{\tt\textlatin{contains}}}\index{contains|textbf}
\noindent{{\en\tt contains}  παίρνει σαν όρισμα μια λίστα (ή ένα σύνολο) 
{\en\tt L} και μια τιμή  {\en\tt c}.\\
{\en\tt contains} ελέγχει εάν το {\en\tt c} είναι ένα στοιχείο της λίστας {\en\tt L}.\\
{\en\tt contains} επιστρέφει  {\en\tt 0} εάν το {\en\tt c} δεν είναι στην {\en\tt L}, 
ή έναν αυστηρά θετικό ακέραιο  ($>0$) ο οποίος είναι  
1 + τον δείκτη της πρώτης εμφάνισης  του {\en\tt c} στην {\en\tt L}.}\\
Είσοδος :
\begin{center}{\en\tt contains([0,1,2,3,4,2],2)}\end{center}
Έξοδος :
\begin{center}{\en\tt  3}\end{center}
Είσοδος :
\begin{center}{\en\tt contains(\%\{0,1,2,3,4,2\%\},2)}\end{center}
Έξοδος :
\begin{center}{\en\tt  3}\end{center}

\subsection{Άθροισμα στοιχείων λίστας (ή πίνακα) 
μετασχημα\-τι\-σμέ\-νων από μια συνάρτηση: {\tt\textlatin{ count}}}\index{count|textbf}
\noindent{{\en\tt count} παίρνει σαν όρισμα : μια πραγματική συνάρτηση {\en\tt f} και μια λίστα
{\en\tt l} μήκους {\en\tt n} (ή έναν πίνακα {\en\tt A} διάστασης {\en\tt p*q}).\\
{\en\tt count} εφαρμόζει την συνάρτηση στα στοιχεία της λίστας (ή του πίνακα) και επιστρέφει
το άθροισμά τους, π.χ. :\\
{\en\tt count(f,l)} επιστρέφει {\en\tt f(l[0])+f(l[1])+...+f(l[n-1])} ή\\
{\en\tt count(f,A)} επιστρέφει {\en\tt f(A[0,0])+....+f(A[p-1,q-1])}.\\
Εάν η {\en\tt f} είναι μια λογική  συνάρτηση τότε {\en\tt count} επιστρέφει τον αριθμό των στοιχείων
της λίστας (ή του πίνακα) για τα οποία η λογική συνάρτηση είναι αληθής.}\\
Είσοδος :
\begin{center}{\en\tt count((x)->x,[2,12,45,3,7,78])}\end{center}
Έξοδος :
\begin{center}{\en\tt  147}\end{center}
επειδή : 2+12+45+3+7+78=147.\\
Είσοδος :
\begin{center}{\en\tt count((x)->x<12,[2,12,45,3,7,78])}\end{center}
Έξοδος :
\begin{center}{\en\tt  3}\end{center}
Είσοδος :
\begin{center}{\en\tt count((x)->x==12,[2,12,45,3,7,78])}\end{center}
Έξοδος :
\begin{center}{\en\tt  1}\end{center}
Είσοδος :
\begin{center}{\en\tt count((x)->x>12,[2,12,45,3,7,78])}\end{center}
Έξοδος :
\begin{center}{\en\tt  2}\end{center}
Είσοδος :
\begin{center}{\en\tt count(x->x\verb|^|2,[3,5,1])}\end{center}
Έξοδος :
\begin{center}{\en\tt 35}\end{center}
Πράγματι $9+25+1=35$.\\
Είσοδος :
\begin{center}{\en\tt count(id,[3,5,1])}\end{center}
Έξοδος :
\begin{center}{\en\tt 9}\end{center}
Πράγματι {\en\tt id} είναι η ταυτοτική συνάρτηση, και  3+5+1=9.\\
Είσοδος :
\begin{center}{\en\tt count(1,[3,5,1])}\end{center}
Έξοδος :
\begin{center}{\en\tt 3}\end{center}
Πράγματι, {\en\tt 1} είναι η σταθερή συνάρτηση ίση με 1 και 1+1+1=3.

\subsection{Αριθμός στοιχείων ίσων με μια δεδομένη τιμή :{\tt\textlatin{count\_eq}}}\index{count\_eq|textbf}
\noindent{{\en\tt count\_eq} παίρνει σαν όρισμα : έναν πραγματικό αριθμό και μια πραγματική λίστα
(ή πίνακα).\\
{\en\tt count\_eq} επιστρέφει τον αριθμό των στοιχείων της λίστας (ή του πίνακα)
που είναι ίσα με το πρώτο όρισμα.}\\
Είσοδος:
\begin{center}{\en\tt count\_eq(12,[2,12,45,3,7,78])}\end{center}
Έξοδος :
\begin{center}{\en\tt  1}\end{center}

\subsection{Αριθμός στοιχείων μικρότερων μιας δεδομένης τιμής : {\tt\textlatin{ count\_inf}}}\index{count\_inf|textbf}
\noindent{{\en\tt count\_inf} παίρνει σαν όρισμα  : ένα πραγματικό αριθμό και μια πραγματική λίστα
(ή πίνακα).\\
{\en\tt count\_inf} επιστρέφει τον αριθμό των στοιχείων της λίστας (ή του πίνακα)
που είναι αυστηρά μικρότερα από το πρώτο όρισμα.}\\
Είσοδος :
\begin{center}{\en\tt count\_inf(12,[2,12,45,3,7,78])}\end{center}
Έξοδος :
\begin{center}{\en\tt  3}\end{center}

\subsection{Αριθμός στοιχείων μεγαλύτερων μιας δεδομένης τιμής : {\tt\textlatin{ count\_sup}}}\index{count\_sup|textbf}
\noindent{{\en\tt count\_sup} παίρνει σαν όρισμα  : ένα πραγματικό αριθμό και μια πραγματική λίστα 
(ή πίνακα).\\
{\en\tt count\_sup} επιστρέφει τον αριθμό των στοιχείων της λίστας (ή του πίνακα)
που είναι αυστηρά μεγαλύτερα από το πρώτο όρισμα.}\\
Είσοδος  :
\begin{center}{\en\tt count\_sup(12,[2,12,45,3,7,78])}\end{center}
Έξοδος :
\begin{center}{\en\tt  2}\end{center}

\subsection{Άθροισμα στοιχείων λίστας : {\tt\textlatin{ sum add}}}\index{sum}\index{add}
\noindent{{\en\tt sum} ή {\en\tt add} παίρνει σαν όρισμα μια λίστα {\en\tt l} (αντιστ.
ακολουθία) πραγματικών αριθμών.\\ 
{\en\tt sum} ή {\en\tt add} επιστρέφει το άθροισμα των στοιχείων του {\en\tt l}.}\\
Είσοδος  :
\begin{center}{\en\tt sum(2,3,4,5,6)}\end{center}
Έξοδος :
\begin{center}{\en\tt 20}\end{center}

\subsection{Συσσωρευτικό άθροισμα στοιχείων μιας λίστας : {\tt\textlatin{cumSum}}}\index{cumSum|textbf}
\noindent{{\en\tt cumSum} παίρνει σαν όρισμα μια λίστα  {\en\tt l} (αντιστ.
ακολουθία) 
αριθμών ή συμβολοσειρών.}\\
{\en\tt cumSum} επιστρέφει μια λίστα (αντιστ. ακολουθία)  ιδίου μήκους με την {\en\tt l} όπου
 το $k$-στό στοιχείο είναι το  άθροισμα (ή αλληλουχία) των στοιχείων
${\tt l[0],..,l[k]}$.\\
Είσοδος:
\begin{center}{\en\tt cumSum(sqrt(2),3,4,5,6)}\end{center}
Έξοδος :
\begin{center}{\en\tt sqrt(2),3+sqrt(2),3+sqrt(2)+4,3+sqrt(2)+4+5,}\end{center}
\begin{center}{\en\tt 3+sqrt(2)+4+5+6}\end{center}
Είσοδος :
\begin{center}{\en\tt normal(cumSum(sqrt(2),3,4,5,6))}\end{center}
Έξοδος :
\begin{center}{\en\tt  sqrt(2),sqrt(2)+3,sqrt(2)+7,sqrt(2)+12,sqrt(2)+18}\end{center}Είσοδος :
\begin{center}{\en\tt cumSum(1.2,3,4.5,6)}\end{center}
Έξοδος :
\begin{center}{\en\tt  1.2,4.2,8.7,14.7}\end{center}
Είσοδος :
\begin{center}{\en\tt cumSum([0,1,2,3,4])}\end{center}
Έξοδος :
\begin{center}{\en\tt  [0,1,3,6,10]}\end{center}
Είσοδος :
\begin{center}{\en\tt cumSum("a","b","c","d")}\end{center}
Έξοδος :
\begin{center}{\en\tt  "a","ab","abc","abcd"}\end{center}
Είσοδος :
\begin{center}{\en\tt cumSum("a","ab","abc","abcd")}\end{center}
Έξοδος :
\begin{center}{\en\tt "a","aab","aababc","aababcabcd"}\end{center}

\subsection{Γινόμενο : {\tt\textlatin{ product mul}}}\index{product|textbf}\index{mul|textbf}
Δείτε επίσης \ref{sec:product}, \ref{sec:product1} και
\ref{sec:product2}).

\subsubsection{Γινόμενο τιμών μιας παράστασης : {\tt\textlatin{ product}}}\label{sec:product0}
\noindent{{\en\tt product(expr,var,a,b,p)} ή {\en\tt mul(expr,var,a,b,p)} επιστρέφει το γινόμενο
των τιμών μιας παράστασης {\en\tt expr} όταν η μεταβλητή {\en\tt var} μεταβάλλεται από
το {\en\tt a} στο {\en\tt b} με ένα βήμα {\en\tt p} (από προεπιλογή {\en\tt p=1}) : αυτή η σύνταξη είναι
για συμβατότητα με το {\tt\textlatin{Maple}}.}\\
Είσοδος :
\begin{center}{\en\tt product(x\verb|^|2+1,x,1,4)}\end{center}
ή :
\begin{center}{\en\tt mul(x\verb|^|2+1,x,1,4)}\end{center}
Έξοδος :
\begin{center}{\en\tt 1700}\end{center}
Πράγματι $2*5*10*17=1700$\\
Είσοδος :
\begin{center}{\en\tt product(x\verb|^|2+1,x,1,5,2)}\end{center}
ή :
\begin{center}{\en\tt mul(x\verb|^|2+1,x,1,5,2)}\end{center}
Έξοδος :
\begin{center}{\en\tt 520}\end{center}
Πράγματι $2*10*26=520$

\subsubsection{Γινόμενο στοιχείων μιας λίστας : {\tt\textlatin{ product}}}
\noindent{\en\tt product} ή {\en\tt mul} παίρνει σαν όρισμα μια λίστα {\en\tt l}
πραγματικών αριθμών (ή αριθμών κινητής υποδιαστολής) ή δύο λίστες του ίδιου μεγέθους (δείτε 
επίσης  \ref{sec:product0}, \ref{sec:product1} και  \ref{sec:product2}).
\begin{itemize}
\item εάν {\en\tt product} ή {\en\tt mul} έχει μια λίστα {\en\tt l}
σαν όρισμα, το {\en\tt product} ή
το {\en\tt mul} επιστρέφει το γινόμενο των στοιχείων της {\en\tt l} \label{sec:product}.\\
Είσοδος :
\begin{center}{\en\tt product([2,3,4])}\end{center}
ή :
\begin{center}{\en\tt mul([2,3,4])}\end{center}
Έξοδος :
\begin{center}{\en\tt 24}\end{center}
%Είσοδος :
%\begin{center}{\en\tt product([[2,3,4],[5,6,7]])}\end{center}
%Έξοδος :
%\begin{center}{\en\tt [10,18,28]}\end{center}
\item εάν {\en\tt product} ή {\en\tt mul} παίρνει ώς ορίσματα 
{\en\tt l1} και {\en\tt l2}
(δύο λίστες ή δύο πίνακες), το {\en\tt product}  ή το {\en\tt mul} επιστρέφει το
γινόμενο όρος-προς-όρο των στοιχείων της {\en\tt l1} και της
{\en\tt l2}.\\
Είσοδος :
\begin{center}{\en\tt product([2,3,4],[5,6,7])}\end{center}
ή :
\begin{center}{\en\tt mul([2,3,4],[5,6,7])}\end{center}
Έξοδος :
\begin{center}{\en\tt [10,18,28]}\end{center}
Είσοδος :
\begin{center}{\en\tt product([[2,3,4],[5,6,7]],[[2,3,4],[5,6,7]])}\end{center}
ή :
\begin{center}{\en\tt mul([[2,3,4],[5,6,7]],[[2,3,4],[5,6,7]])}\end{center}
Έξοδος :
\begin{center}{\en\tt [[4,9,16],[25,36,49]]}\end{center}
\end{itemize}

\subsection{Εφαρμογή μιας μονομετaβλητής συνάρτησης στα στοιχεία μιας λίστας : {\tt\textlatin{ map apply of}}}\index{map}\index{apply}\index{of}
\noindent{{\en\tt map} ή  {\en\tt apply} ή {\en\tt of} εφαρμόζει μια συνάρτηση σε μια λίστα
στοιχείων.\\
{\en\tt of} είναι η  προθηματική συνάρτηση που ισοδυναμεί με τις παρενθέσεις : 
Το {\en\tt Xcas} μεταφράζει εσωτερικά το {\en\tt f(x)} σε {\en\tt of(f,x)}. 
Είναι πιο φυσικό να καλέσουμε την {\en\tt map} 
ή την {\en\tt apply} από ότι την {\en\tt of}. Προσέξετε  την διάταξη των ορισμάτων
(απαιτείται για συβατικούς λόγους).\\
Σημειώστε ότι η {\en\tt apply} επιστρέφει μια λίστα ({\en\tt []})
ακόμα κι αν το 2ο όρισμα δεν είναι λίστα.}\\
Είσοδος :
\begin{center}{\en\tt apply(x->x\verb|^|2,[3,5,1])}\end{center}
ή
\begin{center}{\en\tt of(x->x\verb|^|2,[3,5,1])}\end{center}
ή
\begin{center}{\en\tt map([3,5,1],x->x\verb|^|2)}\end{center}
ή πρώτα ορίστε τη συνάρτηση  $h(x)=x^2$, εισάγοντας :
\begin{center}{\en\tt h(x):=x\verb|^|2}\end{center}
τότε
\begin{center}{\en\tt apply(h,[3,5,1])}\end{center}
ή
\begin{center}{\en\tt of(h,[3,5,1])}\end{center}
ή
\begin{center}{\en\tt map([3,5,1],h)}\end{center}
Έξοδος :
\begin{center}{\en\tt   [9,25,1]}\end{center}
Επόμενο παράδειγμα, ορίστε τη συνάρτηση $g(x)=[x,x^2,x^3]$, είσοδος:
\begin{center}{\en\tt g:=(x)->[x,x\verb|^|2,x\verb|^|3]}\end{center}
τότε
\begin{center}{\en\tt apply(g,[3,5,1])}\end{center}
ή
\begin{center}{\en\tt of(g,[3,5,1])}\end{center}
ή
\begin{center}{\en\tt map([3,5,1],g)}\end{center}
Έξοδος :
\begin{center}{\en\tt   [[3,9,27],[5,25,125],[1,1,1]]}\end{center}
{\bf Προσοχή!!!} εκτελέστε πρώτα την εντολή {\en\tt purge(x)} εάν το {\en\tt x} δεν είναι συμβολικό (δηλαδή, αν έχει γίνει ανάθεση τιμής στο {\en\tt x}) .\\
Σημειώστε ότι εάν {\en\tt l1,l2,l3} είναι λίστες το
{\en\tt sizes([l1,l2,l3])} είναι ισοδύναμο με {\en\tt map(size,[l1,l2,l3]}.

\subsection{Εφαρμογή μιας διμεταβλητής συνάρτησης στα στοιχεία δύο λιστών : {\tt\textlatin{ zip}}}\index{zip}
\noindent{{\en\tt zip} εφαρμόζει μια διμεταβλητή συνάρτηση στα στοιχεία δύο λιστών.}\\
Είσοδος :
\begin{center}{\en\tt zip({\gr\tt'}sum{\gr\tt'}, [a,b,c,d],[1,2,3,4])}\end{center}
Έξοδος :
\begin{center}{\en\tt   [a+1,b+2,c+3,d+4]}\end{center}
Είσοδος :
\begin{center}{\en\tt zip((x,y)->x\verb|^|2+y\verb|^|2,[4,2,1],[3,5,1])}\end{center}
ή :
\begin{center}{\en\tt f:=(x,y)->x\verb|^|2+y\verb|^|2}\end{center}
και μετά,
\begin{center}{\en\tt zip(f,[4,2,1],[3,5,1])}\end{center}
Έξοδος:
\begin{center}{\en\tt   [25,29,2]}\end{center}
Είσοδος :
\begin{center}{\en\tt f:=(x,y)->[x\verb|^|2+y\verb|^|2,x+y]}\end{center}
και μετά,
\begin{center}{\en\tt zip(f,[4,2,1],[3,5,1])}\end{center}
Έξοδος :
\begin{center}{\en\tt   [[25,7],[29,7],[2,2]]}\end{center}

\subsection{Δημιουργία λίστας με μηδενικά : {\tt\textlatin{ newList}}}\index{newList}
\noindent{{\en\tt newList(n)} δημιουργεί μια λίστα με {\en\tt n} μηδενικά.}\\
Είσοδος :
\begin{center}{\en\tt newList(3)}\end{center}
Έξοδος :
\begin{center}{\en\tt   [0,0,0]}\end{center}

\subsection{Δημιουργία  λίστας με συνάρτηση : {\tt\textlatin{ makelist}}}\index{makelist}
\noindent{{\en\tt makelist} παίρνει ως όρισμα μια συνάρτηση {\en\tt f}, 
τα όρια {\en\tt a,b} ενός δείκτη και το βήμα {\en\tt p} 
(από προεπιλογή 1 ή -1 που εξαρτάται από τη διάταξη των ορίων).\\ 
{\en\tt makelist} δημιουργεί την λίστα {\en\tt [f(a),f(a+p)...f(a+k*p)]} με  $k$ τέτοιο ώστε
~: $a<a+k*p \leq b <a+(k+1)*p$ ή $a>a+k*p \geq b >a+(k+1)*p$.}\\
Είσοδος :
\begin{center}{\en\tt makelist(x->x\verb|^|2,3,5)}\end{center}
ή
\begin{center}{\en\tt makelist(x->x\verb|^|2,3,5,1)}\end{center}
ή πρώτα ορίστε τη συνάρτηση  $h(x)=x^2$ εισάγοντας {\en\tt h(x):=x\verb|^|2}
και έπειτα εισάγετε
\begin{center}{\en\tt makelist(h,3,5,1)}\end{center}
Έξοδος :
\begin{center}{\en\tt [9,16,25]}\end{center}
Είσοδος :
\begin{center}{\en\tt makelist(x->x\verb|^|2,3,6,2)}\end{center}
Έξοδος :
\begin{center}{\en\tt [9,25]}\end{center}
{\bf Προσοχή!!!} εκτελέστε πρώτα την εντολή {\en\tt purge(x)} εάν το {\en\tt x} δεν είναι συμβολικό (δηλαδή, αν έχει γίνει ανάθεση τιμής στο {\en\tt x}) .

\subsection{Δημιουργία τυχαίου διανύσματος ή λίστας  : {\tt\textlatin{randvector}}}\index{randvector}
\label{sec:ranm4}
\noindent{{\en\tt randvector} παίρνει σαν όρισμα ένα ακέραιο $n$ και προαιρετικά ένα
δεύτερο όρισμα , που μπορεί να είναι είτε ένας ακέραιος $k$ είτε το αναφερόμενο όνομα μιας (στατιστικής) κατανομής
(δείτε επίσης \ref{sec:ranm1}, \ref{sec:ranm4} και \ref{sec:ranm3}).\\
{\en\tt randvector} επιστρέφει ένα διάνυσμα μεγέθους  $n$ που περιέχει τυχαίους ακεραίους
που κατανέμονται ομοιόμορφα μεταξύ -99 και+99 (προεπιλογή), ή μεταξύ 0 και $k-1$
ή περιέχουν
ακέραιους σύμφωνα με την κατανομή που αναφέρεται.}\\
Είσοδος :
\begin{center}{\en\tt randvector(3)}\end{center}
Έξοδος :
\begin{center}{\en\tt [-54,78,-29]}\end{center}
Είσοδος :
\begin{center}{\en\tt randvector(3,5)}\end{center}
ή εισάγετε  :
\begin{center}{\en\tt randvector(3,{\gr '}rand(5){\gr '})}\end{center}
Έξοδος :
\begin{center}{\en\tt [1,2,4]}\end{center}
Είσοδος :
\begin{center}{\en\tt randvector(3,{\gr '}randnorm(0,1){\gr '})}\end{center}
Έξοδος :
\begin{center}{\en\tt [1.39091705476,-0.136794772167,0.187312440336]}\end{center}
Είσοδος :
\begin{center}{\en\tt randvector(3,2..4)}\end{center}
Έξοδος :
\begin{center}{\en\tt [3.92450003885,3.50059241243,2.7322040787]}\end{center}

\subsection{Λίστα διαφορών διαδοχικών όρων  : {\tt\textlatin{ deltalist}}}\index{deltalist}
\noindent{{\en\tt deltalist} παίρνει σαν όρισμα μια λίστα.\\
{\en\tt deltalist} επιστρέφει την λίστα διαφορών όλων των 
ζευγαριών διαδοχικών όρων της λίστας.}\\
Είσοδος :
\begin{center}{\en\tt deltalist([5,8,1,9])}\end{center}
Έξοδος :
\begin{center}{\en\tt [3,-7,8]}\end{center}

\subsection{Δημιουργία πίνακα από λίστα : {\tt\textlatin{ list2mat}}}\index{list2mat}
\noindent{{\en\tt list2mat} παίρνει σαν όρισμα μια λίστα {\en\tt l} και έναν ακέραιο 
{\en\tt p}.\\
{\en\tt list2mat} επιστρέφει έναν πίνακα που έχει {\en\tt p} στήλες
χωρίζοντας την λίστα {\en\tt l} σε γραμμές μήκους {\en\tt p}. 
Ο πίνακας συμπληρώνεται με {\en\tt 0} εάν το μήκος της {\en\tt l} δεν είναι
πολλαπλάσιο του {\en\tt p}.}\\
Είσοδος :
\begin{center}{\en\tt list2mat([5,8,1,9,5,6],2)}\end{center}
Έξοδος :
\begin{center}{\en\tt  [[5,8],[1,9],[5,6]]}\end{center}
Είσοδος :
\begin{center}{\en\tt list2mat([5,8,1,9],3)}\end{center}
Έξοδος :
\begin{center}{\en\tt  [[5,8,1],[9,0,0]]}\end{center}
{\bf Σχόλιο} \\
Το {\en\tt Xcas} εμφανίζει  πίνακες με {\bf[} και {\bf]} και λίστες με $[$ και $]$ 
σαν οριοθέτες (η κάθετη στήλη των αγκύλων είναι πιο παχιά για πίνακες). 

\subsection{Δημιουργία λίστας  από πίνακα : {\tt\textlatin{ mat2list}}}\index{mat2list}
\noindent{{\en\tt mat2list}\index{mat2list}  παίρνει σαν όρισμα έναν πίνακα.\\
{\en\tt mat2list} επιστρέφει την λίστα των συντελεστών του πίνακα.}\\
Είσοδος :
\begin{center}{\en\tt mat2list([[5,8],[1,9]])}\end{center}
Έξοδος :
\begin{center}{\en\tt [5,8,1,9]}\end{center}

\section{Συναρτήσεις για διανύσματα}
\subsection{Νόρμες ενός διανύσματος : {\tt\textlatin{ maxnorm l1norm l2norm
norm}}}\index{norm|textbf}
Οι εντολές για να υπολογίσουμε τις διάφορες νόρμες ενός διανύσματος είναι :
\begin{itemize}
\item {\en\tt maxnorm} επιστρέφει την νόρμα ${\mathnormal{l}}^\infty$ του
διανύσματος, 
που ορίζεται σαν η μεγαλύτερη των απόλυτων τιμών 
των συντεταγένων του\index{maxnorm|textbf}\label{sec:maxnormv}.\\
Είσοδος :
\begin{center}{\en\tt maxnorm([3,-4,2])}\end{center}
Έξοδος :
\begin{center}{\en\tt 4}\end{center}
Πράγματι : {\en\tt x=3, y=-4, z=2} και {\en\tt 4 = max(|x|,|y|,|z|)}.
\item {\en\tt l1norm}  επιστρέφει την νόρμα  ${\tt {\mathnormal{l}}^1}$  του
διανύσματος  που ορίζεται σαν το άθροισμα των απόλυτων τιμών 
των συντεταγμένων του\index{l1norm}\label{sec:l1normv}.\\
Είσοδος :
\begin{center}{\en\tt l1norm([3,-4,2])}\end{center}
Έξοδος :
\begin{center}{\en\tt 9}\end{center}
Πράγματι: {\en\tt x=3, y=-4, z=2} και  {\en\tt 9 = |x|+|y|+|z|}.
\item{\en\tt norm} ή  {\en\tt l2norm}  επιστρέφει την νόρμα 
 ${\mathnormal{l}}^2$  του διανύσματος που ορίζεται σαν η τετραγωνική ρίζα
του αθροίσματος των τετραγώνων  
των συντεταγμένων του\index{l2norm}\label{sec:l2normv}.\\
Είσοδος :
\begin{center}{\en\tt norm([3,-4,2])}\end{center}
Έξοδος :
\begin{center}{\en\tt sqrt(29)}\end{center}
Πράγματι : {\en\tt x=3, y=-4, z=2} και $ 29=|x|^2+|y|^2+|z|^2$.
\end{itemize}

\subsection{Κανονικοποίηση διανύσματος : {\tt\textlatin{ normalize
unitV}}}\index{normalize|textbf}\index{unitV|textbf}
\noindent{{\en\tt normalize} ή {\en\tt unitV} παίρνει σαν όρισμα ένα διάνυσμα.\\
 {\en\tt normalize} ή {\en\tt unitV}  κανονικοποιεί αυτό το διάνυσμα ως προς την νόρμα
${\mathnormal{l}}^2$ 
(την τετραγωνική ρίζα του αθροίσματος των τετραγώνων των συντεταγμένων του).}\\
Είσοδος :
\begin{center}{\en\tt normalize([3,4,5])}\end{center}
Έξοδος :
\begin{center}{\en\tt
[3/(5*sqrt(2)),4/(5*sqrt(2)),5/(5*sqrt(2))]}\end{center}
Πράγματι: {\en\tt x=3, y=4, z=5} και  $ 50=|x|^2+|y|^2+|z|^2$.

\subsection{Όρος προς όρο άθροισμα δύο λιστών : {\tt\textlatin{ +
.+}}}\index{+|textbf}
\index{.+|textbf}
Ο ενθηματικός τελεστής {\en\tt +} ή {\en\tt .+} και ο προθηματικός τελεστής
 {\tt '+'} επιστρέφουν το άθροισμα όρο προς όρο δύο λιστών .\\
Εάν οι δύο λίστες δεν έχουν το ίδο μέγεθος, η μικρότερη λίστα 
συμπληρώνεται με
μηδενικά.\\
Σημειώσατε την διαφορά με τις ακολουθίες : εάν ο ενθηματικός τελεστής {\tt +}
ή ο
προθηματικός τελεστής {\tt '+'} παίρνει σαν όρισμα δύο ακολουθίες ,
συγχωνεύει τις ακολουθίες, και γι' αυτό επιστρέφει το
άθροισμα όλων των όρων των δύο ακολουθιών.\\
Είσοδος :
\begin{center}{\en\tt [1,2,3]+[4,3,5]}\end{center}
ή :
\begin{center}{\en\tt [1,2,3] .+[4,3,5]}\end{center}
ή :
\begin{center}{\tt '+'([1,2,3],[4,3,5])}\end{center}
ή :
\begin{center}{\tt '+'([[1,2,3],[4,3,5]])}\end{center}
Έξοδος :
\begin{center}{\en\tt [5,5,8]}\end{center}
Είσοδος  :
\begin{center}{\en\tt [1,2,3,4,5,6]+[4,3,5]}\end{center}
ή :
\begin{center}{\tt '+'([1,2,3,4,5,6],[4,3,5])}\end{center}
ή :
\begin{center}{\tt '+'([[1,2,3,4,5,6],[4,3,5]])}\end{center}
Έξοδος :
\begin{center}{\en\tt [5,5,8,4,5,6]}\end{center}
{\bf Προσοχή !}\\
Όταν ο τελεστής \en\tt +} είναι προθηματικός, θα πρέπει να αναφέρεται ({\tt
'+'}).

\subsection{Όρος προς όρο διαφορά δύο λιστών : {\tt\textlatin{ -
.-}}}\index{-|textbf}
\index{.-|textbf}
Ο ενθηματικός τελεστής {\en\tt -} ή {\en\tt .-} και ο προθηματικός τελεστής 
{\tt '-'} επιστρέφουν την διαφορά όρο προς όρο δύο λιστών.\\
Εάν οι δύο λίστες δεν έχουν το ίδιο μέγεθος, η μικρότερη λίστα 
συμπληρώνεται με
μηδενικά.\\
Είσοδος  :
\begin{center}{\en\tt [1,2,3]-[4,3,5]}\end{center}
ή :
\begin{center}{\en\tt [1,2,3] .+ [4,3,5]}\end{center}
ή :
\begin{center}{\tt '-'([1,2,3],[4,3,5])}\end{center}
ή :
\begin{center}{\tt '-'([[1,2,3],[4,3,5]])}\end{center}
Έξοδος :
\begin{center}{\en\tt [-3,-1,-2]}\end{center}
{\bf Προσοχή !}\\
Όταν ο τελεστής {\en\tt -} είναι προθηματικός, θα πρέπει να αναφέρεται  ({\tt
'-'}).

\subsection{Όρος προς όρο γινόμενο δύο λιστών : {\tt\textlatin{
.*}}}\index{.*|textbf}
Ο ενθηματικός τελεστής {\en\tt .*} επιστρέφει το γινόμενο όρο προς όρο δύο λιστών
ίδιου μεγέθους.\\
Είσοδος  :
\begin{center}{\en\tt [1,2,3] .* [4,3,5]}\end{center}
Έξοδος :
\begin{center}{\en\tt [4,6,15]}\end{center}

\subsection{Όρος προς όρο πηλίκο δύο λιστών : {\tt\textlatin{
./}}}\index{./|textbf}
Ο ενθηματικός τελεστής {\en\tt ./} επιστρέφει το πηλίκο όρο προς όρο δύο λιστών
ίδιου μεγέθους.\\
Είσοδος :
\begin{center}{\en\tt [1,2,3] ./ [4,3,5]}\end{center}
Έξοδος :
\begin{center}{\en\tt [1/4,2/3,3/5]}\end{center}

\subsection{Εσωτερικό γινόμενο:{\tt\textlatin{scalar\_product * dotprod dot dotP
scalar\_Product}}}\index{dot}\index{dotP}\index{dotprod}\index{scalar\_product}\index{*|textbf}\index{scalarProduct} 
{\en\tt dot} ή {\en\tt dotP} ή {\en\tt dotprod} ή {\en\tt scalar\_product} ή
{\en\tt scalarProduct} ή ο ενθηματικός τελεστής {\en\tt *} παίρνει σαν όρισμα
2 διανύσματα.\\
{\en\tt dot} ή {\en\tt dotP} ή {\en\tt dotprod} ή {\en\tt scalar\_product} ή 
{\en\tt scalarProduct} ή {\en\tt *} επιστρέφει το  εσωτερικό γινόμενο αυτών των
δύο 
διανυσμάτων.\\
Είσοδος :
\begin{center}{\en\tt dot([1,2,3],[4,3,5])}\end{center}
ή : 
\begin{center}{\en\tt scalar\_product([1,2,3],[4,3,5])}\end{center}
ή :
\begin{center}{\en\tt [1,2,3]*[4,3,5]}\end{center}
ή :
\begin{center}{\tt '*'([1,2,3],[4,3,5])}\end{center}
Έξοδος :
\begin{center}{\en\tt 25}\end{center}
Πράγματι, {\en\tt 25=1*4+2*3+3*5}.

 Ο τελεστής  * μπορεί να χρησιμοποιηθεί και για τον υπολογισμό του γινομένου δύο πολυωνύμων που παρίστανται σαν λίστες των συντελεστών τους. Προς αποφυγήν  παρερμηνειών οι πολυωνυμικές λίστες πρέπει να παρίστανται σαν  {\en\tt poly1[…].}

\subsection{Εξωτερικό γινόμενο : {\tt\textlatin{ cross crossP
crossproduct}}}\index{cross}\index{crossP}\index{crossproduct}
{\en\tt cross} ή {\en\tt crossP} ή {\en\tt crossproduct} παίρνει σαν όρισμα  
δύο διανύσματα.\\
{\en\tt cross} ή {\en\tt crossP} ή {\en\tt crossproduct} επιστρέφει το εξωτερικό γινόμενο
αυτών των δύο διανυσμάτων.\\
Είσοδος :
\begin{center}{\en\tt cross([1,2,3],[4,3,2])}\end{center}
Έξοδος :
\begin{center}{\en\tt [-5,10,-5]}\end{center}
Πράγματι : 
$-5=2*2-3*3$, $ 10=-1*2+4*3$, $ -5=1*3-2*4$.

\section{Στατιστικές συναρτήσεις : {\tt\textlatin{ mean, variance, stddev, \\stddevp, median, quantile, quartiles, boxwhisker}}}\index{mean} \index{stddev}\index{variance}\index{median}\index{stddevp}\index{quantile}\index{boxwhisker}\index{quartiles}\label{sec:statlist}
Οι συναρτήσεις που γράφονται εδώ μπορεί να χρησιμοποιηθούν εάν η στατιστική σειρά
περιέχεται σε μία λίστα. Δείτε επίσης το τμήμα \ref{sec:statmat} για πίνακες και το 
κεφάλαιο \ref{sec:stat} για σταθμισμένες λίστες.
\begin{itemize}
\item {\en\tt mean} υπολογίζει τον αριθμητικό μέσο μιας λίστας\\
Είσοδος :
\begin{center}{\en\tt mean([3,4,2])}\end{center}
Έξοδος :
\begin{center}{\en\tt  3}\end{center}
Είσοδος:
\begin{center}{\en\tt mean([1,0,1])}\end{center}
Έξοδος
\begin{center}{\en\tt  2/3}\end{center}
\item {\en\tt stddev} υπολογίζει την τυπική απόκλιση ενός πληθυσμού,
εάν το όρισμα είναι ο πληθυσμός\\
Είσοδος :
\begin{center}{\en\tt stddev([3,4,2])}\end{center}
Έξοδος :
\begin{center}{\en\tt sqrt(2/3)}\end{center}
\item {\en\tt stddevp} υπολογίζει μια αμερόληπτη εκτίμηση 
της τυπικής απόκλισης ενός πληθυσμού ,
εάν το όρισμα είναι  ένα δείγμα. Ισχύει η ακόλουθη
σχέση:
\begin{center}
 {\en\tt stddevp(l)\verb|^|2=size(l)*stddev(l)\verb|^|2/(size(l)-1)}.
\end{center}
Είσοδος :
\begin{center}{\en\tt stddevp([3,4,2])}\end{center}
Έξοδος :
\begin{center}{\en\tt 1}\end{center}
\item {\en\tt variance} υπολογίζει την διακύμανση (διασπορά) μιας λίστας, και είναι το 
τετράγωνο της {\en\tt stddev}\\
Είσοδος :
\begin{center}{\en\tt variance([3,4,2])}\end{center}
Έξοδος:
\begin{center}{\en\tt 2/3}\end{center}
\item {\en\tt median} υπολογίζει την διάμεση τιμή μιας λίστας.\\
Είσοδος :
\begin{center}{\en\tt median([0,1,3,4,2,5,6])}\end{center}
Έξοδος :
\begin{center}{\en\tt 3.0}\end{center}
\item {\en\tt quantile} υπολογίζει τα δεκατημόρια μιας λίστας που δίνεται σαν
πρώτο όρισμα, όπου το δεκατημόριο είναι το δεύτερο όρισμα.\\
Είσοδος :
\begin{center}{\en\tt quantile([0,1,3,4,2,5,6],0.25)}\end{center}
Έξοδος το πρώτο τεταρτημόριο :
\begin{center}{\en\tt [1.0]}\end{center}
Είσοδος :
\begin{center}{\en\tt quantile([0,1,3,4,2,5,6],0.5)}\end{center}
Έξοδος η διάμεση τιμή:
\begin{center}{\en\tt [3.0]}\end{center}
Είσοδος :
\begin{center}{\en\tt quantile([0,1,3,4,2,5,6],0.75)}\end{center}
Έξοδος το τρίτο τεταρτημόριο :
\begin{center}{\en\tt [4.0]}\end{center}
\item {\en\tt quartiles} υπολογίζει το ελάχιστο, το 1ο τεταρτημόριο, την διάμεση τιμή, το 3ο τεταρτημόριο και το μέγιστο μιας λίστας.\\
Είσοδος :
\begin{center}{\en\tt quartiles([0,1,3,4,2,5,6])}\end{center}
Έξοδος :
\begin{center}{\en\tt [[0.0],[1.0],[3.0],[4.0],[6.0]]}\end{center}
\item {\en\tt boxwhisker} σχεδιάζει το θηκόγραμμα μιας σταστικής σειράς
που είναι αποθηκευμένη σε μια λίστα.\\
Είσοδος :
\begin{center}{\en\tt  boxwhisker([0,1,3,4,2,5,6])}\end{center}
Έξοδος 
\begin{center}{\tt το θηκόγραμμα αυτής της λίστας}\end{center} 
\end{itemize}
{\bf Παράδειγμα}\\
Ορίστε τη λίστα  {\en\tt A} ως εξής:
\begin{center}
{\en\tt A:=[0,1,2,3,4,5,6,7,8,9,10,11]}
\end{center}
Έξοδοι :
\begin{enumerate}
\item {\en\tt 11/2} για {\en\tt mean(A)}
\item
{\en\tt sqrt(143/12)} για {\en\tt stddev(A)}
\item
{\en\tt 0} για {\en\tt min(A)}
\item
{\en\tt [1.0]} για {\en\tt quantile(A,0.1)}
\item
{\en\tt [2.0]} για {\en\tt quantile(A,0.25)}
\item
{\en\tt [5.0]} για {\en\tt median(A)} ή για {\en\tt quantile(A,0.5)}
\item
{\en\tt [8.0]} για {\en\tt quantile(A,0.75)}
\item
{\en\tt [9.0]} για {\en\tt quantile(A,0.9)}
\item
{\en\tt 11} για {\en\tt max(A)}
\item
{\en\tt [[0.0],[2.0],[5.0],[8.0],[11.0]]} για {\en\tt quartiles(A)}
\end{enumerate}

\section{Πίνακας με συμβολοσειρά για δείκτη: {\tt\textlatin{table}}}\index{table}
Ένας πίνακας είναι μία απεικόνιση που χρησιμοποιείται για να αποθηκεύσουμε πληροφορίες
που σχετίζονται με (αντιστοιχούν σε)  δείκτες οι οποίοι  εκτός  από ακεραίους μπορεί να είναι
συμβολοσειρές ή ακολουθίες. Ένας πίνακας μπορεί για παράδειγμα να χρησιμοποιηθεί  για να αποθηκεύσουμε
 τηλεφωνικούς αριθμούς τους οποίους βρίσκουμε με την βοήθεια ονομάτων (που παίζουν τον ρόλο των δεικτών).\\
Στο {\en\tt Xcas}, οι δείκτες  ενός πίνακα μπορεί να είναι κάθε είδους αντικείμενα του {\en\tt Xcas}. Πρόσβαση γίνεται με έναν δυαδικό αλγόριθμο αναζήτησης, όπου η
συνάρτηση διάταξης πρώτα ταξινομεί με βάση τον τύπο ({\en\tt type}) και μετά χρησιμοποιεί
μια διάταξη για κάθε τύπο (π.χ. $<$ για  αριθμητικούς τύπους, λεξικογραφική διάταξη για
συμβολοσειρές, κτλ.)\\
{\en\tt table} παίρνει σαν όρισμα μια λίστα (ή μια ακολουθία) από ισότητες όνομα\_δείκτη = τιμή\_στοιχείου ({\en\tt index\_name=element\_value}).\\
{\en\tt table} επιστρέφει αυτόν τον πίνακα.\\
Είσοδος :
\begin{center}{\en\tt T:=table(3=-10,"a"=10,"b"=20,"c"=30,"d"=40)}\end{center}
Είσοδος :
\begin{center}{\en\tt T["b"]}\end{center}
Έξοδος :
\begin{center}{\en\tt 20}\end{center}
Είσοδος :
\begin{center}{\en\tt T[3]}\end{center}
Έξοδος :
\begin{center}{\en\tt -10}\end{center}
{\bf Σχόλιο}\\
Εάν κάνετε την ανάθεση {\en\tt T[n]:= ...} όπου {\tt T} είναι όνομα μεταβλητής
και {\en\tt n} ακέραιος ισχύουν τα ακόλουθα :
\begin{itemize}
\item εάν το όνομα μεταβλητής έχει ανατεθεί σε μια λίστα ή μια ακολουθία, τότε το 
$n$-στό στοιχείο της {\en\tt T} τροποποιείται,
\item εάν το όνομα μεταβλητής δεν έχει ανατεθεί πουθενά, δημιουργείται ένας πίνακας {\en\tt T}
με μια καταχώρηση (που αντιστοιχεί στον δείκτη $n$). Σημειώσατε
ότι μετά την ανάθεση το {\en\tt T} δεν είναι λίστα, παρά το γεγονός ότι το $n$
ήταν ένας ακέραιος.
\end{itemize}

\section{Συνήθεις πίνακες}
Ένας πίνακας αναπαρίσταται από μια λίστα από λίστες, που έχουν όλες το ίδιο μέγεθος.
Στις απαντήσεις του {\en\tt Xcas}, οι οριοθέτες των πινάκων είναι παχειές αγκύλες {\bf []}.
Για παράδειγμα, {\bf [}1,2,3{\bf ]} είναι ο πίνακας [[1,2,3]] με μόνο μια γραμμή, 
ενώ [1,2,3] (με κανονικές αγκύλες) είναι η λίστα [1,2,3].\\
Σε αυτό το εγχειρίδιο, ο συμβολισμός  εισόδου στο {\en\tt Xcas} (δηλαδή [[1,2,3]]) θα χρησιμοποιηθεί τόσο για την είσοδο όσο
και για την  έξοδο.

\subsection{Ταυτοτικός πίνακας : {\tt\textlatin{ idn identity}}}\index{idn}\index{identity}
\noindent{{\en\tt idn} παίρνει σαν όρισμα έναν ακέραιο  $n$ ή έναν τετραγωνικό πίνακα.\\
{\en\tt idn} επιστρέφει τον ταυτοτικό πίνακα μεγέθους $n$ ή μεγέθους ίδιου 
με το μέγεθος του  πίνακα που ήταν όρισμα.}\\
Είσοδος :
\begin{center}{\en\tt idn(2)}\end{center}
Έξοδος :
\begin{center}{\en\tt  [[1,0],[0,1]]}\end{center}
Είσοδος :
\begin{center}{\en\tt idn(3)}\end{center}
Έξοδος :
\begin{center}{\en\tt  [[1,0,0],[0,1,0],[0,0,1]]}\end{center}

\subsection{Μηδενικός πίνακας : {\tt\textlatin{ newMat matrix}}}\index{newMat}
\noindent{{\en\tt newMat(n,p)} ή {\en\tt matrix(n,p)}
παίρνει σαν όρισμα δύο ακεραίους.\\
{\en\tt newMat(n,p)} επιστρέφει τον μηδενικό πίνακα με {\en\tt n} γραμμές και 
{\en\tt p} στήλες.}\\
Είσοδος :
\begin{center}{\en\tt newMat(4,3)}\end{center}
Έξοδος :
\begin{center}{\en\tt[[0,0,0],[0,0,0],[0,0,0],[0,0,0]]}\end{center}

\subsection{Τυχαίος πίνακας : {\tt\textlatin{ ranm randMat randmatrix}}}\index{ranm}\index{randMat}\index{randmatrix}\label{sec:ranm2}
\noindent{{\en\tt ranm} ή {\en\tt randMat} ή {\en\tt randmatrix} παίρνει σαν όρισμα έναν
ακέραιο  $n$ ή δύο ακεραίους $n, m$ και προαιρετικά ένα τρίτο όρισμα, είτε έναν
ακέραιο $k$ είτε το αναφερμένο όνομα μιας τυχαίας κατανομής
(δείτε επίσης \ref{sec:ranm1}, \ref{sec:ranm4} και \ref{sec:ranm3}).\\
{\en\tt ranm} επιστρέφει ένα διάνυσμα μεγέθους $n$ ή έναν πίνακα μεγέθους $n\times m$
που περιέχει τυχαίους ακεραίους ομοιόμορφα κατανεμημένους μεταξύ -99 και +99 
(προεπιλογή), ή μεταξύ 0 και $k-1$ ή έναν πίνακα μεγέθους $n\times m$
που περιέχει τυχαίους ακεραίους σύμφωνα με την κατανομή, το όνομα της οποίας έχει αναφερθεί.}\\
Είσοδος :
\begin{center}{\en\tt ranm(3)}\end{center}
Έξοδος :
\begin{center}{\en\tt [-54,78,-29]}\end{center}
Είσοδος :
\begin{center}{\en\tt ranm(2,4)}\end{center}
Έξοδος :
\begin{center}{\en\tt [[27,-29,37,-66],[-11,76,65,-33]]}\end{center}
Είσοδος :
\begin{center}{\en\tt ranm(2,4,3)}\end{center}
ή  :
\begin{center}{\en\tt ranm(2,4,}{\tt'}{\en\tt rand(3)}{\tt'})\end{center}
Έξοδος :
\begin{center}{\en\tt [[0,1,1,0],[0,1,2,0]]}\end{center}
Είσοδος :
\begin{center}{\en\tt ranm(2,4,}{\tt'}{\en\tt randnorm(0,1)}{\tt'})\end{center}
Έξοδος :
\begin{center}{\en\tt [[1.83785427742,0.793007112053,-0.978388964902,-1.88602023857], [-1.50900874199,-0.241173369698,0.311373795585,-0.532752431454]]}\end{center}
Είσοδος :
\begin{center}{\en\tt ranm(2,4,2..4)}\end{center}
Έξοδος :
\begin{center}{\en\tt [[2.00549363438,3.03381264955,2.06539073586,2.04844321217],
 [3.88383254968,3.28664474655,3.76909781061,2.39113253355]]}\end{center}


\subsection{Διαγώνιος ενός πίνακα ή πίνακας με δεδομένη διαγώνιο : {\tt\textlatin{ BlockDiagonal diag}}}\index{diag}\index{BlockDiagonal}
\noindent{{\en\tt diag} ή {\en\tt BlockDiagonal} παίρνει σαν όρισμα ένα πίνακα  $A$ ή 
μια λίστα  $l$.\\
{\en\tt diag} επιστρέφει την διαγώνιο του $A$ ή τον διαγώνιο πίνακα με τη λίστα
$l$ στη διαγώνιο (και 0 στις υπόλοιπες θέσεις).}\\
Είσοδος :
\begin{center}{\en\tt diag([[1,2],[3,4]])}\end{center}
Έξοδος :
\begin{center}{\en\tt  [1,4]}\end{center}
Είσοδος :
\begin{center}{\en\tt diag([1,4])}\end{center}
Έξοδος :
\begin{center}{\en\tt  [[1,0],[0,4]]}\end{center}

\subsection{Mplok (ή υποπίνακας) {\tt\textlatin{Jordan}} : {\tt\textlatin{ JordanBlock}}}\index{JordanBlock}
\noindent{{\en\tt JordanBlock} παίρνει σαν όρισμα μια παράσταση $a$ και έναν ακέραιο 
$n$.\\
{\en\tt JordanBlock} επιστρέφει έναν τετραγωνικό πίνακα μεγέθους  $n$ με  $a$
στην κύρια διαγώνιο, 1 πάνω από αυτή τη διαγώνιο και 0 στις υπόλοιπες θέσεις.}\\
Είσοδος :
\begin{center}{\en\tt JordanBlock(7,3)}\end{center}
Έξοδος :
\begin{center}{\en\tt [[7,1,0],[0,7,1],[0,0,7]]}\end{center}

\subsection{Πίνακας {\tt\textlatin{Hilbert}} : {\tt \textlatin{ hilbert}}}\index{hilbert}
\noindent{{\en\tt hilbert} παίρνει σαν όρισμα έναν ακέραιο $n$.\\
{\en\tt hilbert} επιστρέφει τον πίνακα {\tt\textlatin{Hilbert}}.}\\
 Ο πίνακας {\tt\textlatin{Hilbert}} είναι ένας τετραγωνικός πίνακας μεγέθους $n$ του οποίου τα στοιχεία
$a_{j,k}$ είναι :
\[ a_{j,k}=\frac{1}{j+k+1}, \quad 0\leq j, \quad 0 \leq k \]
Είσοδος :
\begin{center}{\en\tt hilbert(4)}\end{center}
Έξοδος :
\begin{center}{\en\tt [[1,1/2,1/3,1/4],[1/2,1/3,1/4,1/5],[1/3,1/4,1/5,1/6], [1/4,1/5,1/6,1/7]]}\end{center}

\subsection{Πίνακας \tt\textlatin{Vandermonde} : {\tt\textlatin{ vandermonde}}}\index{vandermonde}
\noindent{{\en\tt vandermonde} παίρνει σαν όρισμα ένα διάνυσμα του οποίου τα στοιχεία παριστάνονται με $x_j$ για $j=0..n-1$.\\
{\en\tt vandermonde} επιστρέφει τον αντίστοιχο πίνακα {\tt\textlatin{Vandermonde}}
(η $k$-στή γραμμή του πίνακα είναι το διάνυσμα, τα στοιχεία του οποίου είναι
$x_i^{k}$ για $i=0..n-1$ και $k=0..n-1$).\\
{\bf Προσοχή !}\\ 
Οι δείκτες των  γραμών και των στηλών αρχίζουν με το 0 στο {\en\tt Xcas}.}\\
Είσοδος :
\begin{center}{\en\tt vandermonde([a,2,3])}\end{center}
Έξοδος (εάν {\en\tt a} είναι συμβολικό,  αλλιώς χρησιμοποιείστε την εντολή {\tt\textlatin{purge(a)}}) :
\begin{center}{\en\tt  [[1,1,1],[a,2,3],[a*a,4,9]]}\end{center}

\section{Αριθητική πινάκων}
\subsection{Αποτίμηση πίνακα : {\tt\textlatin{ evalm}}}\index{evalm}
\noindent{{\en\tt evalm} χρησιμοποιείται στο {\en\tt Maple} για να αποτιμήσει έναν πίνακα.  
Στο {\en\tt Xcas}, οι πίνακες αποτιμούνται από προεπιλογή. Η εντολή 
{\en\tt evalm} είναι διαθέσιμη μόνο για συμβατότητα, και είναι ισοδύναμη με την {\en\tt eval}.}

\subsection{Πρόσθεση και αφαίρεση δύο πινάκων : {\tt\textlatin{ + - .+ .-}}}\index{+}\index{-}\index{.+}\index{.-}  
\noindent{Ο ενθηματικός τελεστής {\en\tt +} ή {\en\tt .+} (αντίστ. {\en\tt -} ή {\en\tt .-})
χρησιοποιείται για την πρόσθεση (αντίστ. αφαίρεση) δύο πινάκων.}\\
Είσοδος :
\begin{center}{\en\tt [[1,2],[3,4]] + [[5,6],[7,8]]}\end{center}
Έξοδος :
\begin{center}{\en\tt [[6,8],[10,12]]}\end{center}
Είσοδος :
\begin{center}{\en\tt [[1,2],[3,4]] - [[5,6],[7,8]]}\end{center}
Έξοδος :
\begin{center}{\en\tt [[-4,-4],[-4,-4]]}\end{center}
{\bf Σχόλιο}\\
{\en\tt +} μπορεί να χρησιμοποιηθεί σαν ένας προθηματικός τελεστής , αλλά στην περίπτωση
 αυτή το {\tt +} πρέπει να αναφερθεί ({\tt '+'}).\\
Είσοδος :
\begin{center}{\tt '+'([[1,2],[3,4]],[[5,6],[7,8]],[[2,2],[3,3]])}\end{center}
Έξοδος :
\begin{center}{\en\tt [[8,10],[13,15]]}\end{center}

\subsection{Πολλαπλασιασμός δύο πινάκων : {\tt\textlatin{ * \&*}}}\index{*}\index{\&*}
\noindent{ Ο ενθηματικός τελεστής {\en\tt *} (ή {\en\tt \&*}) χρησιμοποιείται για τον 
πολλαπλασιασμό δύο πινάκων.}\\
Είσοδος :
\begin{center}{\en\tt [[1,2],[3,4]] * [[5,6],[7,8]]}\end{center}
ή :
\begin{center}{\en\tt [[1,2],[3,4]] \&* [[5,6],[7,8]]}\end{center}
Έξοδος :
\begin{center}{\en\tt [[19,22],[43,50]]}\end{center}

\subsection{Πρόσθεση στοιχείων μιας στήλης πίνακα : {\tt\textlatin{ sum}}}\index{sum} 
\noindent{{\en\tt sum} παίρνει σαν όρισμα έναν πίνακα $A$.\\
{\en\tt sum} επιστρέφει την λίστα τα στοιχεία της οποίας  είναι το άθροισμα των στοιχείων κάθε 
στήλης του πίνακα $A$.}\\
Είσοδος :
\begin{center}{\en\tt sum([[1,2],[3,4]])}\end{center}
Έξοδος :
\begin{center}{\en\tt [4,6]}\end{center}

\subsection{Συσσωρευτικό άθροισμα στοιχείων κάθε στήλης πίνακα : {\tt\textlatin{ cumSum}}}\index{cumSum} 
\noindent{{\en\tt cumSum} παίρνει σαν όρισμα έναν πίνακα $A$.\\
{\en\tt cumSum} επιστρέφει τον πίνακα  οι στήλες του οποίου είναι το συσσωρευτικό άθροισμα των στοιχείων
της αντίστοιχης στήλης του πίνακα $A$.}\\
Είσοδος :
\begin{center}{\en\tt cumSum([[1,2],[3,4],[5,6]])}\end{center}
Έξοδος :
\begin{center}{\en\tt [[1,2],[4,6],[9,12]]}\end{center}
αφού τα συσσωρευτικά αθροίσματα είναι : 1, 1+3=4, 1+3+5=9 και 2, 2+4=6, 2+4+6=12.

\subsection{Πολλαπλασιασμός στοιχείων κάθε στήλης πίνακα :\\ {\tt\textlatin{ product}}}\index{product}\label{sec:product1}
\noindent{{\en\tt product} παίρνει σαν όρισμα έναν πίνακα $A$.\\
{\en\tt product} επιστρέφει την λίστα τα στοιχεία της οποίας είναι το γινόμενο των στοιχείων
κάθε στήλης του πίνακα $A$ (δείτε επίσης \ref{sec:product} και  
\ref{sec:product2}).}\\
Είσοδος :
\begin{center}{\en\tt product([[1,2],[3,4]])}\end{center}
Έξοδος :
\begin{center}{\en\tt [3,8]}\end{center}

\subsection{Δύναμη πίνακα :\ \^\  \ \&\^\ }\index{\^\ |textbf}\index{\&\^\ }
Ο ενθηματικός τελεστής {\en\tt \verb|^|} (ή {\tt \&\verb|^|}) χρησιμοποιείται για να υψώσουμε
έναν πίνακα σε μια ακέραια δύναμη.\\
Είσοδος :
\begin{center}{\en\tt [[1,2],[3,4]] \verb|^| 5}\end{center}
ή :
\begin{center}{\en\tt [[1,2],[3,4]] \&\verb|^| 5}\end{center}
Έξοδος :
\begin{center}{\en\tt [[1069,1558],[2337,3406]]}\end{center}

\subsection{Γινόμενο \tt\textlatin{Hadamard} : {\tt\textlatin{ hadamard product}}}\index{hadamard}\index{product}\label{sec:product2}
\noindent{{\en\tt hadamard} (ή {\en\tt product}) παίρνει σαν ορίσματα δύο πίνακες $A$ 
και $B$ του ίδιου μεγέθους.\\
{\en\tt hadamard} (ή {\en\tt product}) επιστρέφει τον πίνακα όπου κάθε όρος είναι το γινόμενο
όρος πρός όρο των $A$ και $B$.}\\ 
Είσοδος :
\begin{center}{\en\tt hadamard([[1, 2],[3,4]],[[5, 6],[7, 8]])}\end{center}
Έξοδος :
\begin{center}{\en\tt [[5,12],[21,32]]}\end{center}
Δείτε επίσης \ref{sec:product} και \ref{sec:product1} για {\en\tt product}.

\subsection {Γινόμενο {\tt\textlatin{Hadamard}} (ενθηματική μορφή): {\tt\textlatin{ .*}}}\index{.*}
\noindent{{\en\tt .*} παίρνει σαν ορίσματα δύο πίνακες ή δύο λίστες $A$ και $B$ του 
ίδιου μεγέθους.\\
{\en\tt  .*} είναι ένας ενθηματικός τελεστής που επιστρέφει τον πίνακα ή την λίστα 
όπου κάθε όρος είναι το γινόμενο όρος προς όρο (των  αντίστοιχων όρων) των $A$ και $B$.}\\ 
Είσοδος :
\begin{center}{\en\tt [[1, 2],[3,4]] .* [[5, 6],[7, 8]]}\end{center}
Έξοδος :
\begin{center}{\en\tt [[5,12],[21,32]]}\end{center}
Είσοδος :
\begin{center}{\en\tt [1,2,3,4] .* [5,6,7,8]}\end{center}
Έξοδος :
\begin{center}{\en\tt [5,12,21,32]}\end{center}

\subsection{Διαίρεση {\tt\textlatin{Hadamard}} (ενθηματική μορφή): {\tt\textlatin{ ./}}}\index{./}
\noindent{{\en\tt ./} παίρνει σαν ορίσματα δύο πίνακες ή δύο λίστες $A$ και 
$B$ του ίδιου μεγέθους.\\
{\en\tt  ./} είναι ένας ενθηματικός τελεστής που επιστρέφει τον πίνακα ή την λίστα 
όπου κάθε όρος είναι το πηλίκο όρος προς όρο  (των  αντίστοιχων όρων) των  $A$ και $B$.}\\ 
Είσοδος :
\begin{center}{\en\tt [[1, 2],[3,4]] ./ [[5, 6],[7, 8]]}\end{center}
Έξοδος :
\begin{center}{\en\tt [[1/5,1/3],[3/7,1/2]]}\end{center}

\subsection{Δύναμη {\tt\textlatin{Hadamard}}  (ενθηματική μορφή): {\tt\textlatin{ .\^\ }}}\index{.\^\ }
\noindent{{\en\tt .\verb|^|} παίρνει σαν όρισμα έναν πίνακα ή μια λίστα
$A$ και έναν πραγματικό αριθμό $b$.\\
{\en\tt  .\verb|^|}  είναι ένας ενθηματικός τελεστής που επιστρέφει τον πίνακα
ή την λίστα όπου κάθε όρος είναι ο αντίστοιχος όρος του
 $A$ υψωμένος στη δύναμη $b$.}\\ 
Είσοδος :
\begin{center}{\en\tt [[1, 2],[3,4]] .\verb|^| 2}\end{center}
Έξοδος :
\begin{center}{\en\tt [[1,4],[9,16]]}\end{center}

\subsection{Επιλογή στοιχείου(-ων) πίνακα : {\tt\textlatin{ [] at}}}\index{at}
Θυμηθείτε ότι ένας πίνακας είναι μια λίστα από λίστες του ίδιου μεγέθους.\\
Είσοδος : 
\begin{center}{\en\tt A:=[[3,4,5],[1,2,6]]}\end{center}
Έξοδος :
\begin{center}{\en\tt [[3,4,5],[1,2,6]]}\end{center} 
Η ενθηματική συνάρτηση {\en\tt at} ή ο συμβολισμός δεικτών {\en\tt [..]} χρησιμοποιείται για να έχουμε πρόσβαση
σε ένα στοιχείο ή σε μια γραμμή ή σε μια στήλη ενός πίνακα:
\begin{itemize}
\item Για να επιλέξετε ένα στοιχείο , γράψτε τον πίνακα, και μετά γράψτε μέσα σε αγκύλες τον δείκτη της γραμμής του στοιχείου, ένα κόμμα, και τον δείκτη της στήλης του στοιχείου.
Στον τρόπο λειτουργίας {\en\tt Xcas} ο πρώτος δείκτης είναι 0, ενώ στους άλλους τρόπους λειτουργίας ο πρώτος δείκτης
είναι  1.\\
Είσοδος :
\begin{center}{\en\tt [[3,4,5],[1,2,6]][0,1]}\end{center}
ή
\begin{center}{\en\tt A[0,1]}\end{center}
ή
\begin{center}{\en\tt A[0][1]}\end{center}
ή
\begin{center}{\en\tt at(A,[0,1])}\end{center}
Έξοδος :
\begin{center}{\en\tt 4}\end{center}

\item Για να επιλέξετε μια γραμμή του πίνακα {\en\tt A}, 
γράψτε τον πίνακα, και μετά γράψτε μέσα σε αγκύλες τον δείκτη της γραμμής.\\ Είσοδος :
\begin{center}{\en\tt  [[3,4,5],[1,2,6]][0]}\end{center}
ή
\begin{center}{\en\tt A[0]}\end{center}
ή
\begin{center}{\en\tt at(A,0)}\end{center}
Έξοδος :
\begin{center}{\en\tt [3,4,5]}\end{center}

\item Για να επιλέξετε μέρος μιας γραμμής βάλτε δύο ορίσματα 
μέσα στις αγκύλες  : 
τον δείκτη της γραμμής και ένα διάστημα δεικτών που αντιστοιχεί στις επιλεγμένες στήλες.\\
Είσοδος :
\begin{center}{\en\tt A[1,0..2]}\end{center}
Έξοδος :
\begin{center}{\en\tt [1,2,6]}\end{center}
Είσοδος :
\begin{center}{\en\tt A[1,1..2]}\end{center}
Έξοδος :
\begin{center}{\en\tt [2,6]}\end{center}

\item Για να επιλέξετε μια στήλη του πίνακα {\en\tt A}, πρώτα αναστρέψτε τον 
{\en\tt A} ({\en\tt tran(A)}) και μετά επιλέξτε την γραμμή όπως παραπάνω.\\
Είσοδος :
\begin{center}{\en\tt tran(A)[1]}\end{center}
ή
\begin{center}{\en\tt at(tran(A),1)}\end{center}
Έξοδος :
\begin{center}{\en\tt [4,2]}\end{center}

\item  Για να επιλέξετε μέρος μιας στήλης του πίνακα {\en\tt A} 
σαν λίστα, βάλτε δύο ορίσματα
μέσα στις αγκύλες : ένα διάστημα  δεικτών που αντιστοιχεί στις επιλεγμένες γραμμές και τον δείκτη της στήλης.\\
Είσοδος :
\begin{center}{\en\tt A[0..0,1]}\end{center}
Έξοδος :
\begin{center}{\en\tt [4]}\end{center}

Αυτό μπορεί να χρησιμοποιηθεί για να επιλέξετε μια ολόκληρη στήλη, ορίζοντας ένα διάστημα  δεικτών που αντιστοιχεί σε όλες τις γραμμές.\\
Είσοδος :
\begin{center}{\en\tt A[0..1,1]}\end{center}
Έξοδος :
\begin{center}{\en\tt [4,2]}\end{center}

\item
Για να επιλέξετε έναν υποπίνακα ενός πίνακα, βάλτε μέσα σε αγκύλες δύο 
διαστήματα : ένα διάστημα για τις επιλεγμένες γραμμές και ένα διάστημα για τις 
επιλεγμένες στήλες .\\
Για  τον πίνακα {\en\tt A} :
\begin{center}{\en\tt A:=[[3,4,5],[1,2,6]]}\end{center}
Είσοδος :
\begin{center}{\en\tt A[0..1,1..2]}\end{center}
Έξοδος :
\begin{center}{\en\tt [[4,5],[2,6]]}\end{center}
Είσοδος :
\begin{center}{\en\tt A[0..1,1..1]}\end{center}
Έξοδος :
\begin{center}{\en\tt [[4],[2]]}\end{center}
{\bf Σχόλιο}
Εάν το δεύτερο διάστημα παραλείπεται, ο υποπίνακας φτιάχνεται με τις διαδοχικές 
γραμμές που αντιστοιχούν στο πρώτο διάστημα.\\ 
Είσοδος :
\begin{center}{\en\tt A[1..1]}\end{center}
Έξοδος :
\begin{center}{\en\tt [[1,2,6]]}\end{center}
\end{itemize}

Μπορείτε επίσης να κάνετε ανάθεση τιμής σε ένα στοιχείο ενός πίνακα χρησιμοποιώντας τον συμβολισμό των δεικτών.
Εάν κάνετε ανάθεση τιμής με  {\en\tt :=} δημιουργείται ένα νέο αντίγραφο του πίνακα
και το στοιχείο τροποποιείται, ενώ εάν κάνετε ανάθεση τιμής με {\en\tt =<},
ο πίνακας τροποποιείται στη θέση του στοιχείου.


\subsection{Τροποποίηση  στοιχείου ή  γραμμής πίνακα: {\tt\textlatin{ subsop}}}\index{subsop|textbf}
\noindent{{\en\tt subsop} τροποποιεί ένα στοιχείο ή μια γραμμή πίνακα.
Χρησιμοποιείται κυρίως για λόγους συμβατότητας με το {\en\tt Maple} και το {\en\tt MuPAD}.
Αντίθετα με το {\en\tt :=} ή το {\en\tt =<},
δεν απαιτείται να αποθηκεύσουμε τον πίνακα σε μια μεταβλητή.}\\
{\en\tt subsop} παίρνει δύο ή τρία ορίσματα,
{\bf αυτά τα ορίσματα μετατίθενται} στον τρόπο λειτουργίας {\tt\textlatin{Maple}}.
\begin{enumerate}
\item Τροποποίηση  στοιχείου
\begin{itemize}
\item Στον τρόπο λειτουργίας {\en\tt Xcas} ο πρώτος δείκτης είναι  0.\\
{\en\tt subsop} έχει δύο (αντιστοιχα τρία) ορίσματα: έναν πίνακα {\en\tt A} και μια ισότητα
 {\en\tt [r,c]=v} (αντιστ. έναν πίνακα {\en\tt A}, μια λίστα με δείκτες {\en\tt [r,c]},
και μια τιμή {\en\tt v}).\\ 
{\en\tt subsop} αντικαθιστά το στοιχείο {\en\tt A[r,c]} με το {\en\tt v}.\\
Είσοδος στον τρόπο λειτουργίας {\tt\textlatin{ Xcas}} :
\begin{center}{\en\tt subsop([[4,5],[2,6]],[1,0]=3)}\end{center}
ή :
\begin{center}{\en\tt subsop([[4,5],[2,6]],[1,0],3)}\end{center}
Έξοδος :
\begin{center}{\en\tt [[4,5],[3,6]]}\end{center}
{\bf Σχόλιο}\\
Εάν ο πίνακας αποθηκεύεται σε μια μεταβλητή, για παράδειγμα 
{\en\tt A:=[[4,5],[2,6]]}  τότε είναι ευκολότερο να εισάγουμε 
{\en\tt A[1,0]:=3} για να 
τροποποιήσουμε τον  πίνακα {\en\tt A} σε\\ {\en\tt [[4,5],[3,6]]}.

\item Στους τρόπους λειτουργίας {\tt\textlatin{ Mupad, TI}}, ο πρώτος δείκτης είναι 1.\\
{\en\tt subsop} έχει δύο (αντιστοιχα τρία) ορίσματα : έναν πίνακα  {\en\tt A} και μια ισότητα
{\en\tt [r,c]=v} (αντίστ. έναν πίνακα {\en\tt A}, μια λίστα  με δείκτες {\en\tt [r,c]},
και μια τιμή {\en\tt v}).\\ 
{\en\tt subsop} αντικαθιστά το στοιχείο {\en\tt A[r,c]} με το {\en\tt v}.\\
Εισάγετε στους τρόπους λειτουργίας {\tt\textlatin{ Mupad, TI}} :
\begin{center}{\en\tt subsop([[4,5],[2,6]],[2,1]=3)}\end{center}
ή :
\begin{center}{\en\tt subsop([[4,5],[2,6]],[2,1],3)}\end{center}
Έξοδος :
\begin{center}{\en\tt [[4,5],[3,6]]}\end{center}
{\bf Σχόλιο}\\
Εάν ο πίνακας αποθηκεύεται σε μια μεταβλητή, για παράδειγμα  
{\en\tt A:=[[4,5],[2,6]]}, τότε είναι ευκολότερο να εισάγουμε {\en\tt A[2,1]:=3} 
για να 
τροποποιήσουμε τον  πίνακα  {\en\tt A} σε \\{\en\tt [[4,5],[3,6]]}.

\item Στον τρόπο λειτουργίας {\tt\textlatin{ Maple}}, 
τα ορίσματα μετατίθενται και ο πρώτος δείκτης είναι 1.\\
{\en\tt subsop} έχει δύο ορίσματα: μια ισότητα {\en\tt [r,c]=v} και έναν πίνακα
{\en\tt A}.\\
{\en\tt subsop} αντικαθιστά το στοιχείο {\en\tt A[r,c]} με {\en\tt v}.\\
Εισάγετε στον τρόπο λειτουργίας {\tt\textlatin{ Maple}}
\begin{center}{\en\tt subsop([2,1]=3,[[4,5],[2,6]])}\end{center}
Έξοδος :
\begin{center}{\en\tt [[4,5],[3,6]]}\end{center}
{\bf Σχόλιο}\\
Εάν ο πίνακας αποθηκεύεται σε μια μεταβλητή, για παράδειγμα 
{\en\tt A:=[[4,5],[2,6]]}, τότε είναι ευκολότερο να εισάγουμε {\en\tt A[2,1]:=3} 
για να 
τροποποιήσουμε τον  πίνακα  {\en\tt A} σε\\ {\en\tt [[4,5],[3,6]]}.
\end{itemize}

\item Τροποποίηση γραμμής
\begin{itemize}
\item στον τρόπο λειτουργίας {\tt\textlatin{ Xcas}}, ο πρώτος δείκτης  είναι 0.\\ 
{\en\tt subsop} έχει δύο ορίσματα : έναν πίνακα και μια
ισότητα (τον δείκτη της γραμμής προς αλλαγή, το σύμβολο {\en\tt =} και την νέα
τιμή της γραμμής).\\ 
{\en\tt subsop} αντικαθιστά μία γραμμή του  πίνακα  {\en\tt A} με μία νέα γραμμή.
\\ 
Εισάγετε στον τρόπο λειτουργίας {\tt\textlatin{ Xcas}}  :
\begin{center}{\en\tt subsop([[4,5],[2,6]],1=[3,3])}\end{center}
Έξοδος :
\begin{center}{\en\tt [[4,5],[3,3]]}\end{center}
{\bf Σχόλιο}\\
Εάν ο πίνακας αποθηκεύεται σε μια μεταβλητή, για παράδειγμα 
{\en\tt A:=[[4,5],[2,6]]}, τότε είναι ευκολότερο να εισάγουμε {\en\tt A[1]:=[3,3]}
για να 
τροποποιήσουμε τον  πίνακα  {\en\tt A} σε\\ {\en\tt [[4,5],[3,3]]}.
\end{itemize}

\begin{itemize}
\item Στους τρόπους λειτουργίας {\tt\textlatin{ Mupad, TI}}, ο πρώτος δείκτης είναι 1. \\
{\en\tt subsop} έχει δύο ορίσματα : έναν πίνακα και μια
ισότητα (τον δείκτη της γραμμής προς αλλαγή, το σύμβολο {\en\tt =} και την νέα
τιμή της γραμμής.)\\ 
{\en\tt subsop} αντικαθιστά μία γραμμή του  πίνακα  {\en\tt A} με μία νέα γραμμή.
\\ 
Εισάγεται στους τρόπους λειτουργίας {\tt\textlatin{ Mupad, TI}} :
\begin{center}{\en\tt subsop([[4,5],[2,6]],2=[3,3])}\end{center}
Έξοδος :
\begin{center}{\en\tt [[4,5],[3,3]]}\end{center}
{\bf Σχόλιο}\\
Εάν ο πίνακας αποθηκεύεται σε μια μεταβλητή, για παράδειγμα  
{\en\tt A:=[[4,5],[2,6]]}, τότε είναι ευκολότερο να εισάγουμε {\en\tt A[2]:=[3,3]} για να 
τροποποιήσουμε τον  πίνακα  {\en\tt A} σε\\ {\en\tt [[4,5],[3,3]]}.
\end{itemize}
\begin{itemize}
\item  Στον τρόπο λειτουργίας {\tt\textlatin{ Maple}}, 
τα ορίσματα μετατίθενται και ο πρώτος δείκτης είναι 1. \\ 
{\en\tt subsop} έχει δύο ορίσματα : μια 
ισότητα (τον δείκτη της γραμμής προς αλλαγή, το σύμβολο {\en\tt =} και την νέα
τιμή της γραμμής) και έναν πίνακα.\\ 
Εισάγετε στον τρόπο λειτουργίας {\tt\textlatin{ Maple}} :
\begin{center}{\en\tt subsop(2=[3,3],[[4,5],[2,6]])}\end{center}
Έξοδος :
\begin{center}{\en\tt [[4,5],[3,3]]}\end{center}
{\bf Σχόλιο}\\
Εάν ο πίνακας αποθηκεύεται σε μια μεταβλητή, για παράδειγμα   
{\en\tt A:=[[4,5],[2,6]]}, τότε είναι ευκολότερο να εισάγουμε {\en\tt A[2]:=[3,3]} για να 
τροποποιήσουμε τον  πίνακα  {\en\tt A} σε\\ {\en\tt [[4,5],[3,3]]}.
\end{itemize}
\end{enumerate}
{\bf Σχόλιο}\\
Σημειώσατε επίσης ότι η εντολή {\en\tt subsop} με ένα όρισμα {\tt '{\en n=NULL}'}  
διαγράφει την γραμμή με αριθμό {\en\tt n}.
Στον τρόπο λειτουργίας {\tt\textlatin{Xcas}} εισάγετε :
\begin{center}{\tt {\en subsop([[4,5],[2,6]]},'{\en 1=NULL}')}\end{center}
Έξοδος :
\begin{center}{\en\tt [[4,5]]}\end{center}

\subsection{Επιλογή γραμμών ή στηλών  πίνακα (συμβατότητα με το {\tt\textlatin{Maple}}) : {\tt\textlatin{ row col}}}\index{row}\index{col}
\noindent{{\en\tt row} (αντίστ. {\en\tt col}) επιλέγει μία ή περισσότερες γραμμές (αντίστ. στήλες) ενός
πίνακα.\\
{\en\tt row} (αντίστ. {\en\tt col}) παίρνει δύο ορίσματα : έναν πίνακα $A$, και έναν ακέραιο $n$
ή ένα διάστημα $n_1..n_2$.\\
{\en\tt row} (αντίστ. {\en\tt col}) επιστρέφει την γραμμή (αντίστ. την στήλη)  του $A$ με δείκτη $n$, 
ή την ακολουθία των γραμμών (αντίστ. στηλών) του $A$ με δείκτη από  $n_1$ μέχρι $n_2$ .}\\
Είσοδος :
\begin{center}{\en\tt row([[1,2,3],[4,5,6],[7,8,9]],1)}\end{center}
Έξοδος :
\begin{center}{\en\tt [4,5,6]}\end{center}
Είσοδος :
\begin{center}{\en\tt row([[1,2,3],[4,5,6],[7,8,9]],0..1)}\end{center}
Έξοδος :
\begin{center}{\en\tt ([1,2,3],[4,5,6])}\end{center}
Είσοδος :
\begin{center}{\en\tt  col([[1,2,3],[4,5,6],[7,8,9]],1)}\end{center}
Έξοδος :
\begin{center}{\en\tt [2,5,8]}\end{center}
Είσοδος :
\begin{center}{\en\tt  col([[1,2,3],[4,5,6],[7,8,9]],0..1)}\end{center}
Έξοδος :
\begin{center}{\en\tt ([1,4,7,[2,5,8])}\end{center}

\subsection{Διαγραφή γραμμών ή στηλών πίνακα :{\tt\textlatin{delrows\\ delcols}}}\index{delrows}\index{delcols}
\noindent{{\en\tt delrows} (αντίστ. {\en\tt delcols}) διαγράφει μία ή περισσότερες γραμμές (αντίστ.
στήλες) του πίνακα.\\
{\en\tt delrows} (αντίστ. {\en\tt delcols}) παίρνει δύο ορίσματα : έναν πίνακα $A$, και 
ένα διάστημα  $n_1..n_2$.\\
{\en\tt delrows} (αντίστ. {\en\tt delcols}) επιστρέφει τον πίνακα όπου οι γραμμές 
(αντίστ. στήλες) του $A$ με δείκτη από $n_1$ μέχρι $n_2$  έχουν διαγραφεί.}\\
Είσοδος :
\begin{center}{\en\tt delrows([[1,2,3],[4,5,6],[7,8,9]],1..1)}\end{center}
Έξοδος :
\begin{center}{\en\tt [[1,2,3],[7,8,9]]}\end{center}
Είσοδος :
\begin{center}{\en\tt delrows([[1,2,3],[4,5,6],[7,8,9]],0..1)}\end{center}
Έξοδος :
\begin{center}{\en\tt [[7,8,9]]}\end{center}
Είσοδος :
\begin{center}{\en\tt delcols([[1,2,3],[4,5,6],[7,8,9]],1..1)}\end{center}
Έξοδος :
\begin{center}{\en\tt [[1,3],[4,6],[7,9]]}\end{center}
Είσοδος :
\begin{center}{\en\tt delcols([[1,2,3],[4,5,6],[7,8,9]],0..1)}\end{center}
Έξοδος :
\begin{center}{\en\tt [[3],[6],[9]]}\end{center}

\subsection{Επιλογή υποπίνακα από πίνακα (συμβατότητα με {\tt \textlatin{TI}}) : {\tt\textlatin{ subMat}}}\index{subMat}
\noindent{{\en\tt subMat} παίρνει πέντε ορίσματα : έναν πίνακα $A$, και τέσσερεις ακεραίους
$nl1,nc1,nl2,nc2$, όπου
$nl1$ είναι ο δείκτης για την πρώτη γραμμή, $nc1$ είναι
ο δείκτης για την πρώτη στήλη, $nl2$ ο δείκτης για την 
τελευταία γραμμή και  $nc2$ είναι ο δείκτης για την τελευταία στήλη.\\    
{\en\tt subMat(A,nl1,nc1,nl2,nc2)} επιλέγει από τον πίνακα {\en\tt A}
έναν υποπίνακα 
με πρώτο στοιχείο  {\en\tt A[nl1,nc1]} και τελευταίο στοιχείο
{\en\tt A[nl2,nc2]}.}\\
Ορίστε τον πίνακα {\en\tt A} :
\begin{center}{\en\tt A:=[[3,4,5],[1,2,6]]}\end{center}
Είσοδος :
\begin{center}{\en\tt subMat(A,0,1,1,2)}\end{center}
Έξοδος :
\begin{center}{\en\tt [[4,5],[2,6]]}\end{center}
Είσοδος :
\begin{center}{\en\tt subMat(A,0,1,1,1]}\end{center}
Έξοδος :
\begin{center}{\en\tt [[4],[2]]}\end{center}
Από προεπιλογή $nl1=0$, $nc1=0$, $nl2$={\en\tt nrows(A)}-1 και 
$nc2$={\en\tt ncols(A)}-1\\
Είσοδος :
\begin{center}{\en\tt  subMat(A,1)}\end{center}
ή :
\begin{center}{\en\tt  subMat(A,1,0)}\end{center}
ή :
\begin{center}{\en\tt  subMat(A,1,0,1)}\end{center}
ή :
\begin{center}{\en\tt  subMat(A,1,0,1,2)}\end{center}
Έξοδος :
\begin{center}{\en\tt [[1,2,6]]}\end{center}

\subsection{Πρόσθεση γραμμής σε μια άλλη γραμμή : {\tt\textlatin{ rowAdd}}}\index{rowAdd}
\noindent{{\en\tt rowAdd} παίρνει τρία ορίσματα : έναν πίνακα $A$ και δύο ακεραίους
$n1$ και $n2$.\\
{\en\tt rowAdd} επιστρέφει τον πίνακα που παίρνουμε αντικαθιστώντας στον $A$ την γραμμή με δείκτη
$n2$ με το άθροισμα των γραμμών με δείκτες $n1$ και $n2$.}\\
Είσοδος :
\begin{center}{\en\tt rowAdd([[1,2],[3,4]],0,1)}\end{center}
Έξοδος :
\begin{center}{\en\tt  [[1,2],[4,6]]}\end{center}

\subsection{Πολλαπλασιασμός γραμμής με μια παράσταση: {\tt\textlatin{ mRow}}}\index{mRow}
\noindent{{\en\tt mRow} παίρνει τρία ορίσματα: μια παράσταση, έναν πίνακα  $A$ και έναν 
ακέραιο $n$.\\
{\en\tt mRow} επιστρέφει τον πίνακα που παίρνουμε αντικαθιστώντας στον πίνακα $A$, την γραμμή με δείκτη
$n$ με το γινόμενο της γραμμής με δείκτη $n$ επί την παράσταση.}\\
Είσοδος :
\begin{center}{\en\tt mRow(12,[[1,2],[3,4]],1)}\end{center}
Έξοδος :
\begin{center}{\en\tt [[1,2],[36,48]]}\end{center}

\subsection{Πρόσθεση σε μια γραμμή του πολλαπλάσιου, επί $k$, μιας άλλης γραμμής : {\tt\textlatin{ mRowAdd}}}\index{mRowAdd}
\noindent{{\en\tt mRowAdd} παίρνει τέσσερα ορίσματα: έναν πραγματικό αριθμό $k$, έναν πίνακα $A$ και δύο ακεραίους
 $n1$ και $n2$.\\
{\en\tt mRowAdd} επιστρέφει τον πίνακα που παίρνουμε αν στον $A$, αντικαταστήσουμε την  
γραμμή με δείκτη $n2$ με το άθροισμα της γραμμής με δείκτη $n2$ με $k$ επί την γραμμή 
με δείκτη $n1$.}\\
Είσοδος :
\begin{center}{\en\tt mRowAdd(1.1,[[5,7],[3,4],[1,2]],1,2)}\end{center}
Έξοδος :
\begin{center}{\en\tt [[5,7],[3,4],[4.3,6.4]]}\end{center}

\subsection{Ανταλλαγή δύο  γραμμών : {\tt\textlatin{ rowSwap}}}\index{rowSwap}
\noindent{{\en\tt rowSwap} παίρνει τρία ορίσματα : έναν πίνακα $A$ και δύο ακεραίους
$n1$ και  $n2$.\\
{\en\tt rowSwap} επιστρέφει τον πίνακα που παίρνουμε αν στον $A$, ανταλλάξουμε την γραμμή με δείκτη 
 $n1$ με τη γραμμή με δείκτη $n2$.}\\
Είσοδος :
\begin{center}{\en\tt rowSwap([[1,2],[3,4]],0,1)}\end{center}
Έξοδος :
\begin{center}{\en\tt  [[3,4],[1,2]]}\end{center}

\subsection{Δημιουργία πίνακα από λίστα πινάκων : {\tt\textlatin{ blockmatrix}}}\index{blockmatrix}
\noindent{{\en\tt blockmatrix} παίρνει ως όρισμα δύο ακεραίους $n,m$ και μια λίστα 
(μεγέθους  $n*m$) πινάκων ίδιας  διαστάσης $p \times q$ 
(ή γενικά τέτοιας ώστε οι πρώτοι $m$ πίνακες
έχουν τον ίδιο αριθμό γραμμών και $c$ στήλες, οι  
 επόμενοι $m$ πίνακες έχουν τον ίδιο αριθμό γραμμών και $c$ στήλες, κοκ.).
Και στις δύο περιπτώσεις, έχουμε $n$ μπλόκα από $c$ στήλες.\\
{\en\tt  blockmatrix} βάζει αυτά τα $n$ μπλόκα το ένα κάτω από το άλλο (κάθετη προσκόλληση)  και επιστρέφει έναν πίνακα που έχει $c$ στήλες. 
Εάν τα ορίσματα της λίστας
έχουν την ίδια διάσταση $p \times q$, η απάντηση είναι ένας πίνακας
διάστασης $p*n \times q*m$.}\\
Είσοδος :
\begin{center}{\en\tt blockmatrix(2,3,[idn(2),idn(2),idn(2), idn(2),idn(2),idn(2)])}\end{center}
Έξοδος :
\begin{center}{\en\tt [[1,0,1,0,1,0],[0,1,0,1,0,1], [1,0,1,0,1,0],[0,1,0,1,0,1]]}\end{center}
Είσοδος :
\begin{center}{\en\tt blockmatrix(3,2,[idn(2),idn(2), idn(2),idn(2), idn(2),idn(2)])}\end{center}
Έξοδος :
\begin{center}{\en\tt [[1,0,1,0],[0,1,0,1], [1,0,1,0],[0,1,0,1],[1,0,1,0],[0,1,0,1]]}\end{center}
Είσοδος :
\begin{center}{\en\tt blockmatrix(2,2,[idn(2),newMat(2,3), newMat(3,2),idn(3)])}\end{center}
Έξοδος :
\begin{center}{\en\tt [[1,0,0,0,0],[0,1,0,0,0],[0,0,1,0,0], [0,0,0,1,0],[0,0,0,0,1]] }\end{center}
Είσοδος :
\begin{center}{\en\tt blockmatrix(3,2,[idn(1),newMat(1,4), newMat(2,3),idn(2),newMat(1,2),[[1,1,1]]])}\end{center}
Έξοδος :
\begin{center}{\en\tt [[1,0,0,0,0],[0,0,0,1,0],[0,0,0,0,1],[0,0,1,1,1]]}\end{center}
Είσοδος  :
\begin{center}{\en\tt A:=[[1,1],[1,1]];B:=[[1],[1]]}\end{center}
τότε :
\begin{center}{\en\tt blockmatrix(2,3,[2*A,3*A,4*A,5*B,newMat(2,4),6*B])}\end{center}
Έξοδος :
\begin{center}{\en\tt [[2,2,3,3,4,4],[2,2,3,3,4,4], [5,0,0,0,0,6],[5,0,0,0,0,6]]}\end{center}

\subsection{Δημιουργία  πίνακα από δύο πίνακες : {\tt\textlatin{ semi\_augment}}}\index{semi\_augment|textbf}
\noindent{{\en\tt semi\_augment} παραθέτει  δύο πίνακες με τον 
ίδιο αριθμό στηλών (κάθετη προσκόλληση).}\\
Είσοδος :
\begin{center}{\en\tt  semi\_augment([[3,4],[2,1],[0,1]],[[1,2],[4,5]])}\end{center}
Έξοδος :
\begin{center}{\en\tt [[3,4],[2,1],[0,1],[1,2],[4,5]]}\end{center}
Είσοδος  :
\begin{center}{\en\tt semi\_augment([[3,4,2]],[[1,2,4]])}\end{center}
Έξοδος :
\begin{center}{\en\tt [[3,4,2],[1,2,4]]}\end{center}
Σημειώσατε την διαφορά με την {\en\tt concat}.\\
Είσοδος  :
\begin{center}{\en\tt concat([[3,4,2]],[[1,2,4]])}\end{center}
Έξοδος :
\begin{center}{\en\tt [[3,4,2,1,2,4]]}\end{center}
Πράγματι, όταν οι δύο πίνακες $A$ και  $B$ έχουν την ίδια διάσταση, η  {\en\tt concat} 
φτιάχνει έναν πίνακα με τον ίδιο αριθμό γραμμών όπως οι $A$ και $B$ 
προσκολλώντας τους δίπλα-δίπλα (οριζόντια προσκόλληση).\\
Είσοδος  :
\begin{center}{\en\tt concat([[3,4],[2,1],[0,1]],[[1,2],[4,5]])}\end{center}
Έξοδος :
\begin{center}{\en\tt [[3,4],[2,1],[0,1],[1,2],[4,5]]}\end{center}
αλλά,
Είσοδος :
\begin{center}{\en\tt concat([[3,4],[2,1]],[[1,2],[4,5]]}\end{center}
Έξοδος :
\begin{center}{\en\tt [[3,4,1,2],[2,1,4,5]]}\end{center}

\subsection{Δημιουργία  πίνακα από δύο πίνακες :{\tt\textlatin{augment concat}}}\index{augment}\index{concat}
\noindent{{\en\tt augment} ή {\en\tt concat} παραθέτει δύο πίνακες $A$ και $B$ 
που έχουν τον ίδιο αριθμό γραμμών, ή που έχουν τον ίδιο αριθμό στηλών. 
Στην πρώτη περίπτωση, επιστρέφει έναν πίνακα που έχει τον ίδιο αριθμό γραμμών
όπως οι  $A$ και $B$, με οριζόντια προσκόλληση, ενώ στην δεύτερη περίπτωση
επιστρέφει έναν πίνακα που έχει τον ίδιο αριθμό στηλών όπως οι  $A$ και $B$, με
κάθετη προσκόλληση.}\\
Είσοδος :
\begin{center}{\en\tt  augment([[3,4,5],[2,1,0]],[[1,2],[4,5]])}\end{center}
Έξοδος :
\begin{center}{\en\tt [[3,4,5,1,2],[2,1,0,4,5]]}\end{center}
Είσοδος :
\begin{center}{\en\tt  augment([[3,4],[2,1],[0,1]],[[1,2],[4,5]])}\end{center}
Έξοδος :
\begin{center}{\en\tt [[3,4],[2,1],[0,1],[1,2],[4,5]]}\end{center}
Είσοδος :
\begin{center}{\en\tt augment([[3,4,2]],[[1,2,4]]}\end{center}
Έξοδος :
\begin{center}{\en\tt [[3,4,2,1,2,4]]}\end{center}
Σημειώστε ότι εάν $A$ και $B$ έχουν την  ίδια διάσταση, η {\en\tt augment} 
φτιάχνει ένα πίνακα με τον ίδιο αριθμό γραμμών όπως οι $A$ και  $B$ 
με οριζόντια προσκόλληση. Σε αυτήν την περίπτωση, για κάθετη προσκόλληση 
πρέπει να πρέπει να χρησιμοποιήσετε την {\en\tt semi\_augment}.\\
Είσοδος :
\begin{center}{\en\tt  augment([[3,4],[2,1]],[[1,2],[4,5]])}\end{center}
Έξοδος :
\begin{center}{\en\tt [[3,4,1,2],[2,1,4,5]]]}\end{center}

\subsection{Δημιουργία  πίνακα με συνάρτηση : {\tt\textlatin{ makemat}}}\index{makemat}
\noindent{{\en\tt makemat} παίρνει τρία ορίσματα : 
\begin{itemize}
\item μια συνάρτηση δύο μεταβλητών {\en\tt j} και {\en\tt k} η οποία 
θα  επιστρέφει την τιμή του $a_{j,k}$, στοιχείου του πίνακα προς σύνθεση, με
δείκτη γραμμής {\en\tt j} και δείκτη στήλης {\en\tt k} .
\item δύο ακεραίους $n$ και $p$.
\end{itemize}
{\en\tt makemat} επιστρέφει έναν πίνακα $A=(a_{j,k})$ 
($j=0..n-1$ και $k=0..p-1$) διάστασης $n \times p$.}\\
Είσοδος :
\begin{center}{\en\tt makemat((j,k)->j+k,4,3)}\end{center}
ή πρώτα ορίστε την συνάρτηση $h$ :
\begin{center}{\en\tt h(j,k):=j+k}\end{center}
και μετά , εισάγετε :
\begin{center}{\en\tt makemat(h,4,3)}\end{center}
Έξοδος :
\begin{center}{\en\tt [[0,1,2],[1,2,3],[2,3,4],[3,4,5]]}\end{center}
Προσέξτε ότι οι δείκτες αρχικοποιούνται με 0.

\subsection{Ορισμός πίνακα : {\tt\textlatin{ matrix}}}\index{matrix}
\noindent{{\en\tt matrix} παίρνει τρία ορίσματα:
\begin{itemize}
\item  δύο ακεραίους $n$ και $p$.  
\item  μια συνάρτηση δύο μεταβλητών {\en\tt j} και {\en\tt k}, η οποία  
θα  επιστρέφει την τιμή του $a_{j,k}$, στοιχείου του πίνακα προς σύνθεση, με
δείκτη γραμμής {\en\tt j} και δείκτη στήλης {\en\tt k}.
\end{itemize}
{\en\tt matrix} επιστρέφει τον πίνακα $A=(a_{j,k})$ ($j=1..n$ και $k=1..p$)  
διάστασης $n \times p$.}\\
Είσοδος :
\begin{center}{\en\tt matrix(4,3,(j,k)->j+k)}\end{center}
ή πρώτα ορίστε την  συνάρτηση $h$ :
\begin{center} {\en\tt h(j,k):=j+k}\end{center}
και μετά εισάγετε :
\begin{center}{\en\tt matrix(4,3,h)}\end{center}
Έξοδος :
\begin{center}{\en\tt [[2,3,4],[3,4,5],[4,5,6],[5,6,7]]}\end{center}
Προσέξτε την σειρά των ορισμάτων και το γεγονός ότι οι δείκτες αρχικοποιούνται με 1. Εάν το τελευταίο όρισμα (η συνάρτηση) δεν παρέχεται, είναι από προεπιλογή το 0.

\subsection{Επισύναψη στήλης στο τέλος ενός πίνακα : {\tt\textlatin{ border}}}\index{border}
\noindent{{\en\tt border} παίρνει σαν όρισμα έναν πίνακα {\en\tt A} διάστασης $p\times q$
 και μια λίστα {\en\tt b} μεγέθους $p$ (δηλαδή {\en\tt nrows(A)=size(b)}).}\\
{\en\tt border} επιστρέφει τον πίνακα που προκύπτει από τον {\en\tt A} αν του επισυνάψουμε σαν τελευταία στήλη  την  ανεστραμένη λίστα
{\en\tt tran(b)}, έτσι:
\begin{center}
{\en\tt border(A,b)=tran([op(tran(A)),b])=tran(append(tran(A),b))}
\end{center}
Είσοδος :
\begin{center}{\en\tt border([[1,2,4],[3,4,5]],[6,7])}\end{center}
Έξοδος  :
\begin{center}{\en\tt  [[1,2,4,6],[3,4,5,7]]}\end{center}
Είσοδος :
\begin{center}{\en\tt border([[1,2,3,4],[4,5,6,8],[7,8,9,10]],[1,3,5])}\end{center}
Έξοδος  :
\begin{center}{\en\tt  [[1,2,3,4,1],[4,5,6,8,3],[7,8,9,10,5]]}\end{center}

\subsection{Καταμέτρηση των στοιχείων ενός πίνακα που ικανοποι\-ούν μια συγκεκριμένη ιδιότητα : {\tt\textlatin{ count}}}\index{count}
\noindent{{\en\tt count} παίρνει σαν ορίσματα : μια πραγματική συνάρτηση {\en\tt f} και 
έναν πραγματικό πίνακα {\en\tt A} διάστασης {\en\tt $p\times q$} (αντιστ. μια λίστα {\en\tt l} μεγέθους
{\en\tt n}).\\
{\en\tt count} επιστρέφει το {\en\tt f(A[0,0])+..f(A[p-1,q-1])} (αντιστ.  
{\en\tt f(l[0])+..f(l[n-1])})\\
Επομένως, εάν {\en\tt f} είναι μια λογική συνάρτηση, η {\en\tt count} επιστρέφει τον αριθμό  
των στοιχείων του πίνακα {\en\tt A} (αντιστ. της λίστας {\en\tt l}) που ικανοποιούν την 
ιδιότητα {\en\tt f}.}\\
Είσοδος :
\begin{center}{\en\tt count(x->x,[[2,12],[45,3],[7,78]])}\end{center}
Έξοδος :
\begin{center}{\en\tt  147}\end{center}
Πράγματι: 2+12+45+3+7+78=147.\\
Είσοδος :
\begin{center}{\en\tt count(x->x<10,[[2,12],[45,3],[7,78]])}\end{center}
Έξοδος :
\begin{center}{\en\tt  3}\end{center}

\subsection{Καταμέτρηση των στοιχείων ενός πίνακα που εί\-ναι ίσα με μια δοθείσα τιμή : {\tt\textlatin{ count\_eq}}}\index{count\_eq}
\noindent{{\en\tt count\_eq} παίρνει σαν ορίσματα: έναν πραγματικό αριθμό και μια λίστα πραγματικών  
αριθμών ή έναν πραγματικό πίνακα.\\
{\en\tt count\_eq} επιστρέφει τον αριθμό των στοιχείων της λίστας ή του πίνακα
που ισούνται με το πρώτο όρισμα.}\\
Είσοδος :
\begin{center}{\en\tt count\_eq(12,[[2,12,45],[3,7,78]])}\end{center}
Έξοδος :
\begin{center}{\en\tt  1}\end{center}

\subsection{Καταμέτρηση των στοιχείων ενός πίνακα που εί\-ναι μικρότερα μιας δοθείσας τιμής : {\tt\textlatin{ count\_inf}}}\index{count\_inf}
\noindent{{\en\tt count\_inf} παίρνει ως ορίσματα: έναν πραγματικό αριθμό και μια λίστα πραγματικών αριθμών
ή έναν πραγματικό πίνακα.\\
{\en\tt count\_inf} επιστρέφει τον αριθμό των στοιχείων της λίστας ή του πίνακα
που είναι αυστηρά μικρότερα του πρώτου ορίσματος.}\\
Είσοδος :
\begin{center}{\en\tt count\_inf(12,[2,12,45,3,7,78])}\end{center}
Έξοδος :
\begin{center}{\en\tt  3}\end{center}

\subsection{Καταμέτρηση των στοιχείων ενός πίνακα που εί\-ναι μεγαλύτερa μιας δοθείσας τιμής : {\tt\textlatin{ count\_sup}}}\index{count\_sup}
\noindent{{\en\tt count\_sup} παίρνει ως ορίσματα: έναν πραγματικό αριθμό και μια λίστα πραγματικών αριθμών
 ή έναν πραγματικό πίνακα.\\
{\en\tt count\_sup} επιστρέφει τον αριθμό των στοιχείων της λίστας ή του πίνακα
που είναι αυστηρά μεγαλύτερα του πρώτου ορίσματος.}\\
Είσοδος :
\begin{center}{\en\tt count\_sup(12,[[2,12,45],[3,7,78]])}\end{center}
Έξοδος :
\begin{center}{\en\tt  2}\end{center}

\subsection{Στατιστικές συναρτήσεις που δρουν σε στήλες πινάκων : {\tt\textlatin{ mean}}, {\tt\textlatin{ stddev}}, {\tt\textlatin{ variance}}, {\tt\textlatin{ median}}, {\tt\textlatin{ quantile}}, {\tt\textlatin{ quartiles}}, {\tt\textlatin{ boxwhisker}}}\label{sec:statmat}
\index{mean} \index{stddev}\index{variance}\index{median}\index{quartiles}\index{quantile}\index{boxwhisker}
Οι ακόλουθες συναρτήσεις έχουν για όρισμα πίνακες, και δρουν σε κάθε μία στήλη χωριστά:
\begin{itemize}
\item {\en\tt mean} υπολογίζει τους αριθμητικούς μέσους  των στατιστικών σειρών
που είναι αποθηκευμένες στις στήλες ενός πίνακα.\\ 
Είσοδος :
\begin{center}{\en\tt mean([[3,4,2],[1,2,6]])}\end{center}
Έξοδος είναι το διάνυσμα των αριθμητικών μέσων κάθε στήλης :
\begin{center}{\en\tt  [2,3,4]}\end{center}
Είσοδος :
\begin{center}{\en\tt mean([[1,0,0],[0,1,0],[0,0,1]])}\end{center}
Έξοδος 
\begin{center}{\en\tt [1/3,1/3,1/3]}\end{center}

\item {\en\tt stddev} υπολογίζει την τυπική απόκλιση του πληθυσμού
των στατιστικών σειρών που είναι αποθηκευμένες στις στήλες ενός πίνακα.\\  
Είσοδος :
\begin{center}{\en\tt stddev([[3,4,2],[1,2,6]])}\end{center}
Έξοδος είναι το διάνυσμα των τυπικών αποκλίσεων κάθε στήλης :
\begin{center}{\en\tt [1,1,2]}\end{center}
\item{\en\tt variance} υπολογίζει το τετράγωνο των τυπικών αποκλίσεων των στατιστικών σειρών
που είναι αποθηκευμένες στις στήλες ενός πίνακα.\\ 
Είσοδος :
\begin{center}{\en\tt variance([[3,4,2],[1,2,6]])}\end{center}
Έξοδος είναι το διάνυσμα των τετραγώνων της τυπικής απόκλισης κάθε στήλης :
\begin{center}{\en\tt [1,1,4]}\end{center}

\item {\en\tt median} υπολογίζει τις διάμεσες τιμές των στατιστικών σειρών
που είναι αποθηκευμένες στις στήλες ενός πίνακα.\\ 
Είσοδος :
\begin{center}{\en\tt median([[6,0,1,3,4,2,5],[0,1,3,4,2,5,6],[1,3,4,2,5,6,0], [3,4,2,5,6,0,1],[4,2,5,6,0,1,3],[2,5,6,0,1,3,4]])}\end{center}
Έξοδος είναι το διάνυσμα των διάμεσων τιμών κάθε στήλης  :
\begin{center}{\en\tt [3,3,4,4,4,3,4]}\end{center}

\item {\en\tt quantile} υπολογίζει τα δεκατημόρια (που ορίζονται από το δεύτερο
όρισμα) των στατιστικών σειρών που είναι αποθηκευμένες στις στήλες ενός πίνακα.\\ 
Είσοδος :\begin{center}{\en\tt quantile([[6,0,1,3,4,2,5],[0,1,3,4,2,5,6],[1,3,4,2,5,6,0], [3,4,2,5,6,0,1],[4,2,5,6,0,1,3],[2,5,6,0,1,3,4]],0.25)}\end{center}
Έξοδος είναι το διάνυσμα των πρώτων τεταρτημόριων κάθε στήλης  :
\begin{center}{\en\tt [1,1,2,2,1,1,1]}\end{center}
Είσοδος :
\begin{center}{\en\tt quantile([[6,0,1,3,4,2,5],[0,1,3,4,2,5,6],[1,3,4,2,5,6,0], [3,4,2,5,6,0,1],[4,2,5,6,0,1,3],[2,5,6,0,1,3,4]],0.75)}\end{center}
Έξοδος είναι το διάνυσμα των τρίτων τεταρτημόριων κάθε στήλης  :
\begin{center}{\en\tt [3,3,4,4,4,3,4]}\end{center}

\item {\en\tt quartiles} υπολογίζει το ελάχιστο, το πρώτο τεταρτημόριο, την διάμεση τιμή, το τρίτο τεταρτημόριο και το μέγιστο των στατιστικών σειρών που είναι
αποθηκευμένες στις στήλες ενός πίνακα.\\ 
Είσοδος :
\begin{center}{\en\tt quartiles([[6,0,1,3,4,2,5],[0,1,3,4,2,5,6],[1,3,4,2,5,6,0], [3,4,2,5,6,0,1], [4,2,5,6,0,1,3], [2,5,6,0,1,3,4]])}\end{center}
Έξοδος είναι ένας πίνακας, η πρώτη στήλη του οποίου είναι το ελάχιστο κάθε στήλης,
η δεύτερη στήλη του είναι το πρώτο τεταρτημόριο κάθε στήλης, 
η τρίτη του στήλη είναι η διάμεση τιμή 
κάθε στήλης, η τέταρτη στήλη του, το τρίτο τεταρτημόριο
κάθε στήλης και η τελευταία στήλη του είναι  είναι το μέγιστο κάθε στήλης:
\begin{center}{\en\tt [[0,0,1,0,0,0,0],[1,1,2,2,1,1,1], [2,2,3,3,2,2,3],}\end{center}
\begin{center}{\en\tt [3,3,4,4,4,3,4],[6,5,6,6,6,6,6]]}\end{center}

\item{\en\tt boxwhisker} σχεδιάζει τα θηκογράμματα των στατιστικών σειρών
που είναι αποθηκευμένες στις στήλες ενός πίνακα .\\
Είσοδος :
\begin{center}{\en\tt boxwhisker([[6,0,1,3,4,2,5],[0,1,3,4,2,5,6], [1,3,4,2,5,6,0],[3,4,2,5,6,0,1], [4,2,5,6,0,1,3],[2,5,6,0,1,3,4]])}\end{center}
Έξοδος :
\begin{center}{\tt το σχέδιο των θηκογραμμάτων των στατιστικών σειρών κάθε στήλης του πίνακα που δόθηκε σαν όρισμα}\end{center} 
\end{itemize}

\subsection{Διάσταση πίνακα : {\tt\textlatin{ dim}}}\index{dim}
\noindent{\en\tt dim} παίρνει ως όρισμα έναν πίνακα $A$.\\
{\en\tt dim} επιστρέφει την λίστα του αριθμού των γραμμών και των στηλών
του πίνακα $A$.\\
Είσοδος :
\begin{center}{\en\tt dim([[1,2,3],[3,4,5]])}\end{center}
Έξοδος :
\begin{center}{\en\tt  [2,3]}\end{center}

\subsection{Αριθμός γραμμών : {\tt\textlatin{ rowdim rowDim nrows}}}\index{rowdim}\index{rowDim}\index{nrows}
\noindent{{\en\tt rowdim} (ή {\en\tt rowDim} ή {\en\tt nrows}) παίρνει ως όρισμα έναν 
πίνακα $A$.\\
{\en\tt rowdim} (ή {\en\tt rowDim} ή {\en\tt nrows}) επιστρέφει τον αριθμό των γραμμών του 
πίνακα $A$.}\\
Είσοδος :
\begin{center}{\en\tt rowdim([[1,2,3],[3,4,5]])}\end{center}
ή
\begin{center}{\en\tt nrows([[1,2,3],[3,4,5]])}\end{center}
Έξοδος :
\begin{center}{\en\tt  2}\end{center}

\subsection{Αριθμός στηλών : {\tt\textlatin{ coldim colDim ncols}}}\index{coldim}\index{colDim}\index{ncols}
\noindent{{\en\tt coldim} (ή {\en\tt colDim} ή {\en\tt ncols}) παίρνει ως όρισμα έναν 
πίνακα $A$.\\
{\en\tt coldim} (ή {\en\tt colDim} ή {\en\tt ncols}) επιστρέφει τον αριθμό των στηλών του 
πίνακα $A$.}\\
Είσοδος :
\begin{center}{\en\tt coldim([[1,2,3],[3,4,5]])}\end{center}
ή
\begin{center}{\en\tt ncols([[1,2,3],[3,4,5]])}\end{center}
Έξοδος :
\begin{center}{\en\tt  3}\end{center}


\section{Γραμμική Άλεβρα }
\subsection{Αναστροφή πίνακα : {\tt\textlatin{ tran transpose}}}\index{tran}\index{transpose}
\noindent{{\en\tt tran} ή {\en\tt transpose} παίρνει ως όρισμα έναν πίνακα $A$.\\
{\en\tt tran} ή {\en\tt transpose} επιστρέφει τον ανεστραμένο πίνακα του $A$.}\\
Είσοδος :
\begin{center}{\en\tt tran([[1,2],[3,4]])}\end{center}
Έξοδος :
\begin{center}{\en\tt [[1,3],[2,4]]}\end{center}

\subsection{Αντίστροφος  πίνακα : {\tt\textlatin{ inv /}}}\index{inv|textbf}\index{/|textbf}
\noindent{{\en\tt inv} παίρνει ως όρισμα έναν τετραγωνικό πίνακα $A$.\\
{\en\tt inv} επιστρέφει τον αντίστροφο πίνακα του $A$.}\\
Είσοδος :
\begin{center}{\en\tt inv([[1,2],[3,4]])}\end{center}
ή
\begin{center}{\en\tt 1/[[1,2],[3,4]])}\end{center}
ή
\begin{center}{\en\tt A:=[[1,2],[3,4]];1/A}\end{center}
Έξοδος :
\begin{center}{\en\tt  [[-2,1],[3/2,1/-2]]}\end{center}

\subsection{Ίχνος πίνακα : {\tt\textlatin{ trace}}}\index{trace}
\noindent{{\en\tt trace} παίρνει ως όρισμα έναν πίνακα $A$.\\
{\en\tt trace} επιστρέφει το ίχνος του πίνακα $A$, που είναι 
το άθροισμα των διαγώνιων στοιχείων.}\\
Είσοδος :
\begin{center}{\en\tt trace([[1,2],[3,4]])}\end{center}
Έξοδος :
\begin{center}{\en\tt  5}\end{center}

\subsection{Ορίζουσα πίνακα : {\tt\textlatin{ det}}}\index{det|textbf}
\noindent{{\en\tt det} παίρνει ως όρισμα έναν πίνακα $A$.\\
{\en\tt det} επιστρέφει την ορίζουσα του πίνακα $A$.}\\
Είσοδος :
\begin{center}{\en\tt det([[1,2],[3,4]])}\end{center}
Έξοδος :
\begin{center}{\en\tt -2}\end{center}
Είσοδος :
\begin{center}{\en\tt det(idn(3))}\end{center}
Έξοδος :
\begin{center}{\en\tt 1}\end{center}

\subsection{Ορίζουσα αραιού πίνακα : {\tt\textlatin{ det\_minor}}}\index{det\_minor}
\noindent{{\en\tt det\_minor} παίρνει ως όρισμα έναν πίνακα $A$.\\
{\en\tt det\_minor} επιστρέφει την ορίζουσα του πίνακα $A$ που υπολογίζεται  
αναπτύσσοντας την ορίζουσα με την μέθοδο του {\tt\textlatin{Laplace}}.}\\
Είσοδος :
\begin{center}{\en\tt det\_minor([[1,2],[3,4]])}\end{center}
Έξοδος :
\begin{center}{\en\tt -2}\end{center}
Είσοδος :
\begin{center}{\en\tt det\_minor(idn(3))}\end{center}
Έξοδος :
\begin{center}{\en\tt 1}\end{center}

\subsection{Τάξη πίνακα : {\tt\textlatin{ rank}}}\index{rank}
\noindent{{\en\tt rank} παίρνει ως όρισμα έναν πίνακα $A$.\\
{\en\tt rank} επιστρέφει την τάξη του πίνακα $A$.}\\
Είσοδος :
\begin{center}{\en\tt rank([[1,2],[3,4]])}\end{center}
Έξοδος :
\begin{center}{\en\tt 2}\end{center}
Είσοδος :
\begin{center}{\en\tt rank([[1,2],[2,4]])}\end{center}
Έξοδος :
\begin{center}{\en\tt 1}\end{center}

\subsection{Αναστροφοσυζυγής πίνακα : {\tt\textlatin{ trn}}}\index{trn}
\noindent{{\en\tt trn} παίρνει ως όρισμα έναν πίνακα $A$.\\
{\en\tt trn} επιστρέφει τον αναστροφοσυζυγή πίνακα του $A$  (δηλαδή τον συζυγή του ανεστραμένου πίνακα του $A$).}\\
Είσοδος :
\begin{center}{\en\tt trn([[i, 1+i],[1, 1-i]])}\end{center}
Έξοδος κατόπιν απλοποίησης:
\begin{center}{\en\tt [[-i,1],[1-i,1+i]]}\end{center}


\subsection{Ισοδύναμος πίνακας : {\tt\textlatin{ changebase}}}\index{changebase}
\noindent{{\en\tt changebase} παίρνει ως όρισμα έναν πίνακα $A$ και έναν 
πίνακα αλλαγής βάσης $P$.\\
{\en\tt changebase} επιστρέφει τον πίνακα $B$ τέτοιο ώστε $B=P^{-1}AP$.}\\
Είσοδος :
\begin{center}{\en\tt changebase([[1,2],[3,4]],[[1,0],[0,1]])}\end{center}
Έξοδος :
\begin{center}{\en\tt [[1,2],[3,4]]}\end{center}
Είσοδος :
\begin{center}{\en\tt changebase([[1,1],[0,1]],[[1,2],[3,4]])}\end{center}
Έξοδος :
\begin{center}{\en\tt [[-5,-8],[9/2,7]]}\end{center}
Πράγματι :
 $${\left[\begin{array}{rr} 1 & 2\\3&4\end{array}\right]}^{-1}*\left[\begin{array}{rr}1 & 1\\0&1\end{array}\right]*\left[\begin{array}{rr}1 & 2\\3&4\end{array}\right]=\left[\begin{array}{rr}-5 & -8\\\frac{9}{2}&7\end{array}\right]$$.

\subsection{Βάση γραμμικού υποχώρου  : {\tt\textlatin{ basis}}}\index{basis}
\noindent{{\en\tt basis} παίρνει ως όρισμα μια λίστα διανυσμάτων που παράγουν 
έναν γραμμικό υποχώρο του $\mathbb R^n$.\\
{\en\tt basis} επιστρέφει μια λίστα διανυσμάτων, που είναι η βάση αυτού του γραμμικού
υποχώρου.}\\
Είσοδος :
\begin{center}{\en\tt basis([[1,2,3],[1,1,1],[2,3,4]])}\end{center}
Έξοδος :
\begin{center}{\en\tt [[1,0,-1],\ [0,1,2]]}\end{center}

\subsection{Βάση της τομής δύο υποχώρων : {\tt\textlatin{ ibasis}}}\index{ibasis}
\noindent{{\en\tt ibasis} παίρνει ως όριμσα δύο λίστες διανυσμάτων που παράγουν
δύο υποχώρους του $\mathbb R^n$.\\ 
{\en\tt ibasis} επιστρέφει μια λίστα διανυσμάτων που είναι η βάση της 
τομής αυτών των δύο υποχώρων.}\\
Είσοδος :
\begin{center}{\en\tt ibasis([[1,2]],[[2,4]])}\end{center}
Έξοδος :
\begin{center}{\en\tt [[1,2]]}\end{center}

\subsection{Εικόνα γραμμικού μετασχηματισμού : {\tt\textlatin{ image}}}\index{image}
\noindent{{\en\tt image} παίρνει ως όρισμα τον πίνακα ενός γραμμικού μετασχηματισμού   $f$ ως προς την κανονική βάση.\\
{\en\tt image} επιστρέφει την λίστα των διανυσμάτων που είναι μια βάση της εικόνας (πεδίου τιμών)
του $f$.}\\
Είσοδος :
\begin{center}{\en\tt image([[1,1,2],[2,1,3],[3,1,4]])}\end{center}
Έξοδος :
\begin{center}{\en\tt   [[-1,0,1],[0,-1,-2]]}\end{center}

\subsection{Πυρήνας (ή Μηδενόχωρος) γραμμικού μετασχηματισμού : {\tt\textlatin{ kernel nullspace ker}}}\index{ker}\index{kernel}\index{nullspace}
\noindent{{\en\tt ker} (ή {\en\tt kernel} ή {\en\tt nullspace}) παίρνει ως όρισμα τον πίνακα ενός
γραμμικού μετασχηματισμού $f$ ως προς την κανονική βάση.\\
{\en\tt ker} (ή {\en\tt kernel} ή  {\en\tt nullspace}) επιστρέφει μια λίστα
διανυσμάτων που είναι μια βάση του πυρήνα του $f$.}\\
Είσοδος :
\begin{center}{\en\tt ker([[1,1,2],[2,1,3],[3,1,4]])}\end{center}
Έξοδος :
\begin{center}{\en\tt [[1,1,-1]]}\end{center}
Ο πυρήνας παράγεται από το διάνυσμα {\en\tt [1,1,-1]}.

\subsection{Πυρήνας (ή Μηδενόχωρος) γραμμικού μετασχηματισμού : {\tt\textlatin{ Nullspace}}}\index{Nullspace}
\noindent{{\bf Προσοχή} Αυτή η συνάρτηση είναι χρήσιμη μόνο στον τρόπο λειτουργίας {\tt\textlatin{Maple}} 
(για να τον επιλέξετε πατήστε την μπάρα Ρυθμίσεων, έπειτα {\tt Στυλ Προγραμ}, 
έπειτα επιλέξτε {\tt\textlatin{Maple}} και {\tt Εφαρμογή} ή από το μενού Ρυθμίσεις $->$ Λειτουργία $->$ {\tt\textlatin{maple}} ).\\
{\en\tt Nullspace} είναι η αδρανής μορφή του {\en\tt nullspace}.\\
{\en\tt Nullspace} παίρνει ως όρισμα έναν ακέραιο πίνακα ενός
γραμμικού μετασχηματισμού $f$ ως προς την κανονική βάση.}\\
{\en\tt Nullspace} ακολουθούμενο από {\en\tt mod p} επιστρέφει μια λίστα διανυσμάτων 
που είναι μια βάση του πυρήνα του $f$ υπολογισμένη στο $\mathbb Z/p\mathbb Z$.\\
Είσοδος :
\begin{center}{\en\tt Nullspace([[1,1,2],[2,1,3],[3,1,4]])}\end{center}
Έξοδος :
\begin{center}{\en\tt nullspace([[1,1,2],[2,1,3],[3,1,4]])}\end{center}
Είσοδος (στον τρόπο λειτουργίας {\tt\textlatin{Maple}}):
\begin{center}{\en\tt Nullspace([[1,2],[3,1]]) mod 5}\end{center}
Έξοδος :
\begin{center}{\en\tt [2,-1]}\end{center}
Σστον τρόπο λειτουργίας { \tt\textlatin{Xcas}}, η ισοδύναμη είσοδος είναι :
\begin{center}{\en\tt nullspace([[1,2],[3,1]] \% 5)}\end{center}
Έξοδος :
\begin{center}{\en\tt [2\% 5,-1\% 5]}\end{center}

\subsection{Υπόχωρος που παράγεται από τις στήλες ενός πίνακα : {\tt\textlatin{ colspace}}}\index{colspace}
\noindent{{\en\tt colspace} παίρνει ως όρισμα τον πίνακα $A$ ενός
γραμμικού μετασχηματισμού $f$ ως προς την κανονική βάση.\\
{\en\tt colspace} επιστρέφει έναν πίνακα. Οι στήλες αυτού του πίνακα είναι μια βάση του
υποχώρου που παράγεται από τις στήλες του $A$.\\
{\en\tt colspace} μπορεί να έχει το όνομα μιας μεταβλητής ως δεύτερο όρισμα, όπου το 
{\en\tt  Xcas}
θα αποθηκεύσει την διάσταση του υποχώρου που παράγεται από τις στήλες του $A$.}\\
Είσοδος :
\begin{center}{\en\tt colspace([[1,1,2],[2,1,3],[3,1,4]])}\end{center}
Έξοδος :
\begin{center}{\en\tt  [[-1,0],[0,-1],[1,-2]]}\end{center}
Είσοδος :
\begin{center}{\en\tt colspace([[1,1,2],[2,1,3],[3,1,4]],dimension)}\end{center}
Έξοδος :
\begin{center}{\en\tt  [[-1,0],[0,-1],[1,-2]]}\end{center}
Μετά εισάγετε:
\begin{center}{\en\tt dimension}\end{center}
Έξοδος :
\begin{center}{\en\tt  2}\end{center}

\subsection{Υπόχωρος που παράγεται από τις γραμμές ενός πίνακα : {\tt\textlatin{ rowspace}}}\index{rowspace}
\noindent{{\en\tt rowspace} παίρνει ως όρισμα τον πίνακα $A$ ενός
γραμμικού μετασχηματισμού $f$ ως προς την κανονική βάση.\\
{\en\tt rowspace} επιστρέφει την λίστα των διανυσμάτων που είναι μια βάση του
υποχώρου που παράγεται από τις γραμμές του $A$.\\
{\en\tt rowspace} μπορεί να έχει το όνομα μιας μεταβλητής ως δεύτερο όρισμα, όπου το {\en\tt Xcas}
θα αποθηκεύσει τη διάσταση του υποχώρου που παράγεται από τις γραμμές του $A$.}\\
Είσοδος :
\begin{center}{\en\tt rowspace([[1,1,2],[2,1,3],[3,1,4]])}\end{center}
Έξοδος :
\begin{center}{\en\tt  [[-1,0,-1],[0,-1,-1]]}\end{center}
Είσοδος :
\begin{center}{\en\tt rowspace([[1,1,2],[2,1,3],[3,1,4]],dimension)}\end{center}
Έξοδος :
\begin{center}{\en\tt  [[-1,0,-1],[0,-1,-1]]}\end{center}
Μετά εισάγετε:
\begin{center}{\en\tt dimension}\end{center}
Έξοδος :
\begin{center}{\en\tt  2}\end{center}


\section{Γραμμικός προγραματισμός}\index{simplex\_reduce|textbf}
Τα προβλήματα  του Γραμμικά Προγραμματισμού είναι προβλήματα μεγιστο\-ποίησης μιας γραμμικής
συνάρτησης κάτω από περιορισμούς που εκφράζονται με γραμμικές ισότητες ή ανισότητες.
Η πιο απλή περίπτωση μπορεί να επιλυθεί άμεσα από τον αποκαλούμενο αλγόριθμο {\en\tt simplex}.
Οι περισσότερες περιπτώσεις απαιτούν την επίλυση ενός βοηθητικού  προβλήματος γραμμικού
προγραμματισμού για την εύρεση μια αρχικής κορυφής για τον αλγόριθμο
 {\en\tt simplex}.

\subsection{Αλγόριθμος {\tt\textlatin{Simplex}} : {\tt\textlatin {simplex\_reduce}}}
{\bf Η απλή περίπτωση}\\
Η συνάρτηση {\en\tt simplex\_reduce} κάνει την αναγωγή 
με τον αλγόριθμο {\tt\textlatin{simplex}} για την εύρεση : 
\[ \mbox{{\en max}}(c.x), \quad  A.x \leq b,\ x \geq 0,\ b\geq 0 \]
όπου $c,x$ είναι διανύσματα του $\mathbb R^n$, $b\geq 0$ είναι ένα διάνυσμα του 
$\mathbb R^p$ και $A$ είναι ένας πίνακας με $p$ γραμμές και $n$ στήλες.\\
{\en\tt simplex\_reduce} παίρνει ως όρισμα {\en\tt A,b,c} και
επιστρέφει το  {\en\tt max(c.x)}, την επαυξημένη λύση του {\en\tt x}
(επαυξημένη επειδή ο αλγόριθμος δουλεύει προσθέτοντας {\en\tt nrows($A$)} βοηθητικές
μεταβλητές) και τον ανηγμένο πίνακα.\\ 
{\bf Παράδειγμα}\\
Βρείτε το \[ \mbox{{\en max}}(X+2Y)  \mbox{ όπου }
\left\{
\begin{array}{rcl}
(X,Y) & \geq & 0 \\
-3X +2Y  & \leq & 3\\
X +Y  & \leq & 4
\end{array} 
\right.
\]
Είσοδος :
\begin{center}{\en\tt simplex\_reduce([[-3,2],[1,1]],[3,4],[1,2])}\end{center}
Έξοδος :
\begin{center}{\en\tt (7,[1,3,0,0],[[0,1,1/5,3/5,3],[1,0,(-1)/5,2/5,1], [0,0,1/5,8/5,7]])}\end{center}
Το οποίο σημαίνει ότι το μέγιστο του {\en\tt X+2Y} κάτω από αυτές τις συνθήκες είναι  {\en\tt 7}, που προκύπτει  θέτοντας {\en\tt X=1,Y=3} 
επειδή το {\en\tt [1,3,0,0]} είναι η επαυξημένη λύση και ο ανηγμένος πίνακας είναι ο  :\\
{\en\tt [[0,1,1/5,3/5,3],[1,0,(-1)/5,2/5,1], [0,0,1/5,8/5,7]]}.

{\bf Μια πιο περίπλοκη περίπτωση που ανάγεται στην απλή περίπτωση}\\
Για να καλέσουμε την  {\en\tt simplex\_reduce}, πρέπει το πρόβλημα να έχει μετασχηματισθεί στην κανονική του μορφή (για περισσότερες λεπτομέρειες δείτε και {\en http://en.wikipedia.org/wiki/Simp\-lex\_algorithm}). Δηλαδή πρέπει να έχουν γίνουν τα ακόλουθα :
\begin{itemize}
\item οι περιορισμοί να έχουν ξαναγραφεί στη μορφή $x_k \geq 0$,
\item οι μεταβλητές χωρίς περιορισμούς  να έχουν απαλειφεί,
\item να έχουν προστεθεί (χαλαρές) μεταβλητές ({\en slack variables}) ώστε όλοι οι περιορισμοί να εκφράζονται με ισότητες.
\end{itemize}
Για παράδειγμα, βρείτε :
\[ \mbox{{\en min}}(2x+y-z+4)  \quad \mbox{ όπου } \quad
\left\{
\begin{array}{rcl}
x & \leq & 1 \\
y & \geq & 2 \\
x+3y-z & = & 2 \\
2x-y+z & \leq & 8\\
-x+y & \leq & 5
\end{array} 
\right.
\]
Αν θέσουμε $x=1-X$, $y=Y+2$, $z=5-X+3Y$
τότε το παραπάνω πρόβλημα είναι ισοδύναμο με την εύρεση του ελαχίστου της παράστασης
$(-2X+Y-(5-X+3Y)+8)$ 
όπου :
\[ 
\left\{
\begin{array}{rcl}
X & \geq & 0 \\
Y & \geq & 0 \\
2(1-X)-(Y+2)+ 5-X+3Y & \leq & 8\\
-(1-X) +(Y+2)  & \leq & 5
\end{array} 
\right.
\]
το οποίο είναι ισοδύναμο με την εύρεση του ελαχίστου της παράστασης~:
\[ (-X-2Y+3) \quad \mbox{ όπου} \quad
\left\{
\begin{array}{rcl}
X & \geq & 0 \\
Y & \geq & 0 \\
-3X+2Y & \leq & 3\\
X +Y  & \leq & 4
\end{array} 
\right.
\]
Δηλαδή το πιο σύνθετο πρόβλημα ανάγεται στο να βρούμε το μέγιστο της παράστασης $-(-X-2Y+3)=X+2Y-3$
κάτω από  συνθήκες, που είναι ίδιες  με εκείνες της 
εύρεσης του μεγίστου της παράστασης $X+2Y$ (απλή περίπτωση). Στην απλή περίπτωση η απάντηση ήταν {\tt 7}, 
συνεπώς, το αποτέλεσμα εδώ είναι το {\tt 7-3=4}.

{\bf Η γενική περίπτωση}\\
Ένα πρόβλημα γραμμικού προγραμματισμού  μπορεί, γενικά, να μην ανάγεται άμεσα στην
απλή περίπτωση όπως παραπάνω. Ο λόγος είναι ότι, πριν την εφαρμογή του αλγορίθμου {\en simplex},
 πρέπει να βρεθεί μια αρχική κορυφή (αρχικό σημείο). Επομένως,
{\en\tt simplex\_reduce} μπορεί να κληθεί συγκεκριμενοποιώντας αυτήν την αρχική
κορυφή. Στην περίπτωση αυτή, όλα τα ορίσματα, συμπεριλαμβανομένης και της αρχικής
κορυφής, ομαδοποιούνται σε ένα μόνο πίνακα. 

Επεξηγούμε αρχικά αυτό το είδος της
κλήσης στην {\bf απλή περίπτωση} όπου για το αρχικό σημείο δεν απαιτείται
η επίλυση βοηθητικού προβλήματος.
Εάν ο {\en\tt A} έχει $p$ γραμμές και $n$ στήλες και αν ορίσουμε :
\begin{center}
{\en\tt B:=augment(A,idn(p));} {\en\tt C:=border(B,b);} \\
{\en\tt d:=append(-c,0\$(p+1));} {\en\tt D:=augment(C,[d]);}
\end{center}
τότε η εντολή {\en\tt simplex\_reduce} μπορεί να κληθεί με το {\en\tt D} σαν το μόνο όρισμα.\\
Για το προηγούμενο παράδειγμα, εισάγετε :
\begin{center}{\en\tt A:=[[-3,2],[1,1]];B:=augment(A,idn(2)); C:=border(B,[3,4]);
D:=augment(C,[[-1,-2,0,0,0]])}\end{center}
Εδώ το 
{\en\tt C=[[-3,2,1,0,3],[1,1,0,1,4]]}\\
και το {\en\tt D=[[-3,2,1,0,3],[1,1,0,1,4],[-1,-2,0,0,0]]}\\
Είσοδος :
\begin{center}{\en\tt simplex\_reduce(D)}\end{center}
Έξοδος είναι το ίδιο αποτέλεσμα όπως πριν.

{\bf Επανερχόμαστε στην γενική περίπτωση.}\\
Η κανονική μορφή ενός προβλήματος  γραμμικού προγραμματισμού είναι ίδια με
την απλή περίπτωση παραπάνω, αλλά με ισότητα $Ax=b$ (αντί $Ax\leq b$)
και  $x\geq 0$. Μπορούμε επιπλέον  να υποθέσουμε ότι $b\geq 0$
(εάν όχι,  μπορούμε να αλλάξουμε το πρόσημο της αντίστοιχης γραμμής).
\begin{itemize}
\item Το πρώτο πρόβλημα είναι να βρούμε ένα $x$ που να ικανοποιεί $Ax=b, \quad x\geq 0$ .
Έστω $m$ ο αριθμός των γραμμών του $A$. Προσθέτουμε τις χαλαρές μεταβλητές  
$y_1,...,y_m$ και μεγιστοποιούμε το άθροισμα
$-\sum y_i$ κάτω από τις συνθήκες $Ax=b, x \geq 0, y \geq 0$ 
ξεκινώντας με την αρχική τιμή  $x=0$
και $y=b$
(για να λυθεί αυτό με το {\en\tt Xcas}, καλούμε {\en \tt \verb|simplex_reduce|} με
 μονaδικό όρισμα τον πίνακα που προκύπτει επαυξάνοντας τον $A$ με τον
ταυτοτικό πίνακα, την στήλη $b$ αμετάβλητη και μία επαυξημένη γραμμή  
$c$ με  0 κάτω από τον $A$ και 1 κάτω από τον ταυτοτικό).
Εάν το μέγιστο υπάρχει και είναι 0, μας δίνεται και μια λύση $x$ , όπου όλες οι χαλαρές 
μεταβλητές  είναι 0.
\item Στην συνέχεια κάνουμε μια δεύτερη κλήση στο {\en\tt \verb|simplex_reduce|}
με το αρχικό $c$ και τον ανηγμένο πίνακα που βρίσκουμε λύνοντας το πρώτο πρόβλημα παραπάνω.
\item
Παράδειγμα~: βρείτε το ελάχιστο του $2x+3y-z+t$ με 
$x,y,z,t\geq 0$ και~:
\[ \left\{ \begin{array}{rcl}
-x-y+t&=&1\\
y-z+t&=&3
\end{array}
\right. \]
Αυτό είναι ισοδύναμο με το να βρούμε το μέγιστο του αντίθετου $-(2x+3y-z+t)$.
Προσθέτουμε δύο χαλαρές μεταβλητές $y_1$ και $y_2$, και εισάγουμε
{\en \begin{verbatim}
simplex_reduce([[-1,-1,0,1,1,0,1], 
                [ 0,1,-1,1,0,1,3], 
                [ 0,0, 0,0,1,1,0]])
\end{verbatim}}
Έξοδος: βέλτιστο=0, λύση $x=(0,1,0,2,0,0)$ όπου οι χαλαρές μεταβλητές=0 (τα δύο μηδενικά στο τέλος του διανύσματος $x$), και ο ανηγμένος πίνακας
\[
\left(\begin{array}{ccccccc}
-1/2 & 0 & -1/2 & 1 & 1/2 & 1/2 & 2 \\
1/2 & 1 & -1/2 & 0 & -1/2 & 1/2 & 1 \\
0 & 0 & 0 & 0 & 1 & 1 & 0
\end{array}\right) 
\]
Επομένως, $x=(0,1,0,2)$ είναι ένα αρχικό σημείο που ικανοποιεί τις συνθήκες.
Για να βρούμε την λύση του αρχικού προβλήματος,  αντικαθιστούμε τις 
γραμμές του $Ax=b$ με τις πρώτες δύο γραμμές του παραπάνω ανηγμένου πίνακα,
αφαιρώντας τις τελευταίες στήλες που αντιστοιχούν στις χαλαρές μεταβλητές.
Προσθέτουμε το $c$ ως τελευταία γραμμή.
{\en \begin{verbatim}
simplex_reduce([[-1/2,0,-1/2,1,2],
                [ 1/2,1,-1/2,0,1],
                [  2, 3, -1, 1,0]])
\end{verbatim}}
Έξοδος: {\en\tt maximum=-5}, γι' αυτό το ελάχιστο του αντίθετου είναι 5,
και προκύπτει από την λύση $x=(0,1,0,2)$, μετά την αντικατάσταση 
$x=0$, $y=1$, $z=0$ και $t=2$.
\end{itemize}

Για περισσότερες λεπτομέρειες, ψάξτε στο {\en\tt google} τον {\en\tt \verb|simplex algorithm|}.

\section{Διάφορες νόρμες πινάκων}
\subsection{Νόρμα $l^2$ πινάκων : {\tt\textlatin{ nomm l2norm}}}\index{norm}\index{l2norm|textbf}\label{sec:l2normm}
\noindent{{\en\tt norm} (ή {\en\tt l2norm}) παίρνει ως όρισμα έναν πίνακα $A=(a_{j,k})$ 
(δείτε επίσης \ref{sec:l2normv}).}\\
{\en\tt norm} (ή {\en\tt l2norm}) επιστρέφει  
$\displaystyle \sqrt{\sum_{j,k} a_{j,k}^2}$.\\
Είσοδος :
\begin{center}{\en\tt norm([[1,2],[3,-4]])}\end{center}
ή
\begin{center}{\en\tt l2norm([[1,2],[3,-4]])}\end{center}
Έξοδος :
\begin{center}{\en\tt sqrt(30)}\end{center}

\subsection{Νόρμα $l^\infty$ πινάκων : {\tt\textlatin{ maxnorm}}}\index{maxnorm}\label{sec:maxnormm}
\noindent{\en\tt maxnorm} παίρνει ως όρισμα έναν πίνακα $A=(a_{j,k})$ (δείτε επίσης \ref{sec:maxnormv}).\\
{\en\tt maxnorm} επιστρέφει $ \max(|a_{j,k}|)$.\\
Είσοδος :
\begin{center}{\en\tt maxnorm([[1,2],[3,-4]])}\end{center}
Έξοδος :
\begin{center}{\en\tt 4}\end{center}

\subsection{Νόρμα γραμμής πινάκων : {\tt\textlatin{ rownorm rowNorm}}}\index{rowNorm}\index{rownorm}
\noindent{{\en\tt rownorm} (ή {\en\tt rowNorm}) παίρνει ως όρισμα έναν πίνακα $A=(a_{j,k})$.\\
{\en\tt rownorm} (ή {\en\tt rowNorm}) επιστρέφει $\max_k(\sum_j |a_{j,k}|)$.}\\
Είσοδος :
\begin{center}{\en\tt rownorm([[1,2],[3,-4]])}\end{center}
ή
\begin{center}{\en\tt rowNorm([[1,2],[3,-4]])}\end{center}
Έξοδος :
\begin{center}{\en\tt 7}\end{center}
Πράγματι : $\max(1+2,3+4)=7$

\subsection{Νόρμα στήλης πινάκων : {\tt\textlatin{ colnorm colNorm}}}\index{colNorm}\index{colnorm}
\noindent{{\en\tt colnorm} (ή {\en\tt colNorm}) παίρνει ως όρισμα έναν πίνακα 
$A=(a_{j,k})$.\\
{\en\tt colnorm} (ή {\en\tt colNorm}) επιστρέφει $\max_j(\sum_k(|a_{j,k}|))$.}\\
Είσοδος :
\begin{center}{\en\tt colnorm([[1,2],[3,-4]])}\end{center}
ή
\begin{center}{\en\tt colNorm([[1,2],[3,-4]])}\end{center}
Έξοδος :
\begin{center}{\en\tt 6}\end{center}
Πράγματι : $\max(1+3,2+4)=6$

\section{Αναγωγή πίνακα}
\subsection{Ιδιοτιμές : {\tt\textlatin{ eigenvals}}}\index{eigenvals}
\noindent{{\en\tt eigenvals} παίρνει ως όρισμα έναν τετραγωνικό
πίνακα $A$ μεγέθους $n$.\\
{\en\tt eigenvals} επιστρέφει την ακολουθία των $n$ ιδιοτιμών του $A$.\\
{\bf Σχόλιο} : Εάν ο πίνακας $A$ είναι ακριβής, το {\en\tt Xcas} μπορεί να μην είναι σε θέση 
να βρει τις ακριβείς ρίζες του χαρακτηριστικού πολυωνύμου. 
{\en\tt eigenvals} θα επιστρέψει τις προσεγγιστικές  τιμές των ιδιοτιμών του $A$  εάν οι συντελεστές του χαρακτηριστικού πολυωνύμου
είναι αριθμητικοί (πραγματικοί αριθμοί) ή ένα υποσύνολο των ιδιοτιμών εάν οι συντελεστές είναι
συμβολικοί.}\\
Είσοδος :
\begin{center}{\en\tt eigenvals([[4,1,-2],[1,2,-1],[2,1,0]])}\end{center}
Έξοδος :
\begin{center}{\en\tt (2,2,2) }\end{center}
Είσοδος :
\begin{center}{\en\tt eigenvals([[4,1,0],[1,2,-1],[2,1,0]])}\end{center}
Έξοδος :
\begin{center}{\en\tt (0.324869129433,4.21431974338,1.46081112719)}\end{center}

\subsection{Ιδιοτιμές : {\tt\textlatin{ egvl eigenvalues eigVl}}}\index{egvl}\index{eigVl}\index{eigenvalues}
\noindent{{\en\tt egvl} (ή {\en\tt eigenvalues eigVl}) παίρνει ως όρισμα έναν 
τετραγωνικό πίνακα $A$ μεγέθους $n$.\\
{\en\tt egvl} (ή {\en\tt eigenvalues eigVl}) επιστρέφει την κανονική μορφή {\en Jordan} 
του $A$.\\
{\bf Σχόλιο} : Εάν ο πίνακας $A$ είναι ακριβής, το {\en\tt Xcas} μπορεί να μην είναι σε θέση 
να βρει τις ακριβείς ρίζες του χαρακτηριστικού πολυωνύμου. 
{\en\tt eigenvalues} θα επιστρέψει μια προσεγγιστική διαγωνοποίηση του $A$ εάν οι
συντελεστές είναι αριθμητικοί (πραγματικοί αριθμοί).}\\
Είσοδος :
\begin{center}{\en\tt egvl([[4,1,-2],[1,2,-1],[2,1,0]])}\end{center}
Έξοδος :
\begin{center}{\en\tt [[2,1,0],[0,2,1],[0,0,2]] }\end{center}
Είσοδος :
\begin{center}{\en\tt egvl([[4,1,0],[1,2,-1],[2,1,0]])}\end{center}
Έξοδος :
\begin{center}{\en\tt [[0.324869129433,0,0],[0,4.21431974338,0],[0,0,1.46081112719]]}\end{center}

\subsection{Ιδιοδιανύσματα : {\tt\textlatin{ egv eigenvectors eigenvects \\
eigVc}}}\index{egv}\index{eigenvectors}\index{eigenvects}\index{eigVc}
\noindent{{\en\tt egv} (ή {\en\tt eigenvectors eigenvects eigVc}) παίρνει ως όρισμα 
έναν τετραγωνικό πίνακα $A$ μεγέθους $n$.\\
Εάν ο $A$ είναι ένας διαγωνοποιήσιμος πίνακας, {\en\tt egv} (ή 
{\en\tt eigenvectors eigenvects eigVc}) επιστρέφει έναν πίνακα του οποίου οι στήλες είναι τα  
ιδιοδιανύσματα του $A$. Διαφορετικά, θα αποτύχει (δείτε επίσης
{\en\tt jordan} για τα χαρακτηριστικά διανύσματα).}\\ 
Είσοδος :
\begin{center}{\en\tt egv([[1,1,3],[1,3,1],[3,1,1]])}\end{center}
Έξοδος :
\begin{center}{\en\tt [[-1,1,1],[2,1,0],[-1,1,-1]] }\end{center}
Είσοδος :
\begin{center}{\en\tt egv([[4,1,-2],[1,2,-1],[2,1,0]])}\end{center}
Έξοδος :
\begin{center}{\en\tt "Not diagonalizable at eigenvalue 2"}\end{center}
Στον τρόπο λειτουργίας για μιγαδικούς , εισάγετε :
\begin{center}{\en\tt egv([[2,0,0],[0,2,-1],[2,1,2]])}\end{center}
Έξοδος :
\begin{center}{\en\tt [0,1,0],[-1,-2,-1],[i,0,-i]]}\end{center}

\subsection{Ρητός πίνακας \tt\textlatin{Jordan}  : {\tt\textlatin{ rat\_jordan}}}\index{rat\_jordan}
\noindent{{\en\tt rat\_jordan} παίρνει ως όρισμα έναn τετραγωνικό πίνακα
 $A$ μεγέθους $n$ με ακριβείς συντελεστές.}\\
{\en\tt rat\_jordan} επιστρέφει :
\begin{itemize}
\item στους τρόπους λειτουργίας {\en\tt Xcas}, {\en\tt Mupad} και {\en\tt TI} \\ 
μια ακολουθία δυο πινάκων : έναν πίνακα $P$ (οι στήλες του $P$ είναι
τα ιδιοδιανύσματα εάν ο $A$ είναι διαγωνοποιήσιμος στο σώμα των συντελεστών του)
και τον ρητό πίνακα {\en\tt Jordan} $J$ του $A$, που είναι ο πιο ανηγμένος
πίνακας στο σώμα των συντελεστών του $A$ (ή στο μιγαδικοποιημένο
σώμα στον τρόπο λειτουργίας για μιγαδικούς), όπου 
\[ J=P^{-1}AP \]
\item στον τρόπο λειτουργίας {\en\tt Maple} \
τον πίνακα {\en\tt Jordan}  $J$ του $A$. Μπορούμε επίσης να αποθηκεύσουμε σε μια μεταβλητή τον πίνακα $P$ που ικανοποιεί την σχέση
$J=P^{-1}AP$ 
περνώντας  ως δεύτερο όρισμα την μεταβλητή αυτή, για παράδειγμα 
\begin{center} {\tt {\en rat\_jordan}([[1,0,0],[1,2,-1],[0,0,1]],'{\en P}')}
\end{center}
\end{itemize}
{\bf Σχόλιο}
\begin{itemize}
\item η σύνταξη του {\en\tt Maple} είναι επίσης έγκυρη σε άλλους τρόπους λειτουργίας, για παράδειγμα, στον τρόπο λειτουργίας {\en\tt Xcas} εισάγετε 
\begin{center} {\tt {\en rat\_jordan}([[4,1,1],[1,4,1],[1,1,4]],'{\en P}')}
\end{center}
Έξοδος :
\begin{center} {\en\tt [[6,0,0],[0,3,0],[0,0,3]]}
\end{center}
μετά ο {\tt {\en P}} επιστρέφει
\begin{center} {\en\tt [[1,2,-1],[1,0,2],[1,-2,-1]]}
\end{center}
\item οι συντελεστές (τα στοιχεία) του $P$ και του $J$ ανήκουν στο σώμα των
συντελεστών του $A$.\\
Για παράδειγμα, στον τρόπο λειτουργίας {\en\tt Xcas}, εισάγετε :
\begin{center} {\en\tt rat\_jordan([[1,0,1],[0,2,-1],[1,-1,1]])}
\end{center}
Έξοδος :
\begin{center} {\en\tt [[1,1,2],[0,0,-1],[0,1,2]],[[0,0,-1],[1,0,-3],[0,1,4]]}\end{center}
Εισάγετε  (δείτε και \ref{sec:compagne}) :
\begin{center} {\en\tt companion(pcar([[1,0,1],[0,2,-1],[1,-1,1]],x),x)}\end{center}
Έξοδος :
\begin{center} {\en\tt [[0,0,-1],[1,0,-3],[0,1,4]]}\end{center}
Είσοδος :
\begin{center} {\en\tt rat\_jordan([[1,0,0],[0,1,1],[1,1,-1]])}\end{center}
Έξοδος :
\begin{center} {\en\tt [[-1,0,0],[1,1,1],[0,0,1]],[[1,0,0],[0,0,2],[0,1,0]]}\end{center}
Είσοδος :
\begin{center} {\en\tt factor(pcar([[1,0,0],[0,1,1],[1,1,-1]],x))}\end{center}
Έξοδος :
\begin{center} {\en\tt -(x-1)*(x\verb|^|2-2)}\end{center}
Είσοδος :
\begin{center} {\en\tt companion((x\verb|^|2-2),x)}\end{center}
Έξοδος :
\begin{center} {\en\tt [[0,2],[1,0]]}\end{center}



\item Όταν ο $A$ είναι συμμετρικός και έχει ιδιοτιμές πολλαπλής τάξης,
το {\en\tt Xcas} επιστρέφει ορθογώνια ιδιοδιανύσματα (όχι πάντα με νόρμα ίση με 1)
π.χ. ο {\en\tt tran(P)*P} είναι ένας διαγώνιος πίνακας όπου η διαγώνιος είναι το τετράγωνο της
 νόρμας ({\en\tt norm} ή  {\en\tt l2norm})
των ιδιοδιανυσμάτων, για παράδειγμα :
\begin{center} {\en\tt rat\_jordan([[4,1,1],[1,4,1],[1,1,4]])}
\end{center}
επιστρέφει :
\begin{center} {\en\tt [[1,2,-1],[1,0,2],[1,-2,-1]],[[6,0,0],[0,3,0],[0,0,3]]}
\end{center}
\end{itemize} 
Εισάγετε στον τρόπο λειτουργίας {\en\tt Xcas}, {\en\tt Mupad} ή {\en\tt TI} :
\begin{center}{\en\tt rat\_jordan([[1,0,0],[1,2,-1],[0,0,1]])}\end{center}
Έξοδος :
\begin{center}{\en\tt [[0,1,0],[1,0,1],[0,1,1]],[[2,0,0],[0,1,0],[0,0,1]]}\end{center}
Εισάγετε στον τρόπο λειτουργίας {\en\tt Xcas}, {\en\tt Mupad} ή {\en\tt TI} :
\begin{center}{\en\tt rat\_jordan([[4,1,-2],[1,2,-1],[2,1,0]])}\end{center}
Έξοδος :
\begin{center}{\en\tt  [[[1,2,1],[0,1,0],[1,2,0]],[[2,1,0],[0,2,1],[0,0,2]]]}\end{center}
Στον τρόπο λειτουργίας για μιγαδικούς και  {\en\tt Xcas}, {\en\tt Mupad} ή {\en\tt TI} , εισάγετε :
\begin{center}{\en\tt rat\_jordan([[2,0,0],[0,2,-1],[2,1,2]])}\end{center}
Έξοδος :
\begin{center}{\en\tt [[1,0,0],[-2,-1,-1],[0,-i,i]],[[2,0,0],[0,2+i,0],[0,0,2-i]]}\end{center}
Εισάγετε στον τρόπο λειτουργίας  {\en\tt Maple}  :
\begin{center}{\tt {\en rat\_jordan}([[1,0,0],[1,2,-1],[0,0,1]],'{\en P}')}\end{center}
Έξοδος :
\begin{center}{\en\tt [[2,0,0],[0,1,0],[0,0,1]]}\end{center}
μετά εισάγετε : 
\begin{center}{\en\tt P}\end{center}
Έξοδος :
\begin{center}{\en\tt [[0,1,0],[1,0,1],[0,1,1]]}\end{center}



\subsection{Κανονική μορφή \tt\textlatin{Jordan}  : {\tt\textlatin{ jordan}}}\index{jordan}
\noindent{{\en\tt jordan} παίρνει ως όρισμα έναν τετραγωνικό
πίνακα $A$ μεγέθους $n$.}\\
{\en\tt jordan} επιστρέφει :
\begin{itemize}
\item στον τρόπο λειτουργίας {\en\tt Xcas}, {\en\tt Mupad} ή {\en\tt TI} \\ 
μια ακολουθία δύο πινάκων : έναν πίνακα $P$ του οποίου οι στήλες είναι
τα ιδιοδιανύσματα (ή τα χαρακτηριστικά διανύσματα) 
του πίνακα $A$ και τον πίνακα {\en\tt Jordan} $J$ του $A$ που ικανοποιεί την σχέση $J=P^{-1}AP$,
\item στον τρόπο λειτουργίας {\en\tt Maple} \\
τον πίνακα {\en\tt Jordan} $J$ του $A$. Μπορούμε επίσης να αποθηκεύσουμε σε μια μεταβλητή τον πίνακα $P$, που ικανοποιεί την σχέση
$J=P^{-1}AP$,  
περνώντας ως δεύτερο όρισμα την μεταβλητή αυτή, για παράδειγμα 
\begin{center} {\tt {\en jordan}([[1,0,0],[0,1,1],[1,1,-1]],'{\en P}')}
\end{center}
\end{itemize}
{\bf Σχόλια}
\begin{itemize}
\item η σύνταξη του {\en\tt Maple} ισχύει επίσης και για άλλους τρόπους λειτουργίας, για παράδειγμα, στον
τρόπο λειτουργίας {\en\tt Xcas} εισάγετε :
\begin{center} {\tt {\en jordan}([[4,1,1],[1,4,1],[1,1,4]],'{\en P}')}
\end{center}
Έξοδος :
\begin{center} {\en\tt [[6,0,0],[0,3,0],[0,0,3]]}
\end{center}
και μετά ο  {\en\tt P} επιστρέφει
\begin{center} {\en\tt [[1,2,-1],[1,0,2],[1,-2,-1]]}
\end{center}
\item Όταν ο $A$ είναι συμμετρικός και έχει ιδιοτιμές πολλαπλής τάξης,
το {\en\tt Xcas} επιστρέφει ορθογώνια ιδιοδιανύσματα (όχι πάντα με νόρμα ίση με 1)
π.χ. ο {\en\tt tran(P)*P} είναι ένας διαγώνιος πίνακας όπου η διαγώνιος είναι το τετράγωνο της νόρμας ({\en\tt norm} ή {\en\tt l2norm})
των ιδιοδιανυσμάτων, για παράδειγμα :
\begin{center} {\en\tt jordan([[4,1,1],[1,4,1],[1,1,4]])}
\end{center}
επιστρέφει :
\begin{center} {\en\tt [[1,2,-1],[1,0,2],[1,-2,-1]],[[6,0,0],[0,3,0],[0,0,3]]}
\end{center}
\end{itemize} 
 Εισάγετε στον τρόπο λειτουργίας {\en\tt Xcas}, {\en\tt Mupad} ή
 {\en\tt TI} :
\begin{center}{\en\tt jordan([[1,0,0],[0,1,1],[1,1,-1]])}\end{center}
Έξοδος :
\begin{center} {\en\tt \begin{verbatim}
 [[-1,0,0],[1,1,1],[0,sqrt(2)-1,-sqrt(2)-1]],
    [[1,0,0],[0,sqrt(2),0],[0,0,-sqrt(2)]]
\end{verbatim}}\end{center}
Είσοδος στον τρόπο λειτουργίας  {\en\tt Maple} :
\begin{center}{\en\tt jordan([[1,0,0],[0,1,1],[1,1,-1]],{\gr '}P{\gr '})}\end{center}
Έξοδος :
\begin{center}{\en\tt [[1,0,0],[0,sqrt(2),0],[0,0,-sqrt(2)]]}\end{center}
και μετά εισάγετε : 
\begin{center}{\en\tt P}\end{center}
Έξοδος :
\begin{center}{\en\tt [[-1,0,0],[1,1,1],[0,sqrt(2)-1,-sqrt(2)-1]]}\end{center}
Εισάγετε στον τρόπο λειτουργίας  {\en\tt Xcas}, {\en\tt Mupad} ή  {\en\tt TI} :
\begin{center}{\en\tt jordan([[4,1,-2],[1,2,-1],[2,1,0]])}\end{center}
Έξοδος :
\begin{center}{\en\tt  [[[1,2,1],[0,1,0],[1,2,0]],[[2,1,0],[0,2,1],[0,0,2]]]}\end{center}
Στον τρόπο λειτουργίας για μιγαδικούς και  {\en\tt Xcas}, {\en\tt Mupad} ή
 {\en\tt TI} , εισάγετε :
\begin{center}{\en\tt jordan([[2,0,0],[0,2,-1],[2,1,2]])}\end{center}
Έξοδος :
\begin{center}{\en\tt [[1,0,0],[-2,-1,-1],[0,-i,i]],[[2,0,0],[0,2+i,0],[0,0,2-i]]}\end{center}

\subsection{Χαρακτηριστικό πολυώνυμο : {\tt\textlatin{ charpoly}}}\index{pcar}\index{charpoly}
\noindent{{\en\tt charpoly} (ή {\en\tt pcar}) παίρνει ένα ή δύο ορίσματα,
 έναν τετραγωνικό πίνακα $A$ μεγέθους $n$ και προαιρετικά
το όνομα μιας συμβολικής μεταβλητής.\\
{\en\tt charpoly} (ή {\en\tt pcar}) επιστρέφει το χαρακτηριστικό πολυώνυμο 
$P$ του $A$ γραμμένο σαν λίστα 
των συντελεστών του, εάν κανένα όνομα μεταβλητής δεν παρέχεται,
ή γραμμένο σαν παράσταση ως προς το όνομα της μεταβλητής
που παρέχεται σαν δεύτερο όρισμα.}\\
Το χαρακτηριστικό πολυώνυμο $P$ του $A$ ορίζεται ως
\[ P(x)=\det(x.I-A) \]
Είσοδος :
\begin{center}{\en\tt charpoly([[4,1,-2],[1,2,-1],[2,1,0]])}\end{center}
Έξοδος :
\begin{center}{\en\tt[1,-6,12,-8]}\end{center}
Γι' αυτό, το χαρακτηριστικό πολυώνυμο αυτού του πίνακα είναι
$x^3-6x^2+12x-8$ (εισάγετε {\en\tt normal(poly2symb([1,-6,12,-8],x))} για να πάρετε
τη συμβολική του αναπαράσταση).\\
Είσοδος :
\begin{center}{\en\tt purge(x):;
charpoly([[4,1,-2],[1,2,-1],[2,1,0]],x)}\end{center}
Έξοδος :
\begin{center}{\en\tt x\verb|^|3-6*x\verb|^|2+12*x-8}\end{center}

\subsection{Χαρακτηριστικό πολυώνυμο με την χρήση του αλγορίθμου \tt\textlatin{Hessenberg} :
 {\tt\textlatin{ pcar\_hessenberg}}}\index{pcar\_hessenberg}
\noindent{\en\tt pcar\_hessenberg} παίρνει ως όρισμα έναν τετραγωνικό
πίνακα $A$ μεγέθους $n$ και προαιρετικά το όνομα μιας συμβολικής μεταβλητής.\\
{\en\tt pcar\_hessenberg} επιστρέφει το χρακτηριστικό πολυώνυμο $P$ του $A$ γραμμένο σαν λίστα 
των συντελεστών του, εάν κανένα όνομα μεταβλητής δεν παρέχεται,
ή γραμμένο σαν παράσταση ως προς το όνομα της μεταβλητής
που παρέχεται σαν δεύτερο όρισμα, όπου
\[ P(x)=\det(xI-A) \]
Το χαρακτηριστικό πολυώνυμο υπολογίζεται χρησιμοποιώντας τον αλγόριθμο {\en\tt Hessenberg} 
(πληροφορίες στο διαδίκτυο) το οποίο είναι πιο αποτελεσματικό ($O(n^3)$ προσ\-διο\-ριστικός χρόνος) εάν 
οι συντελεστές του $A$ είναι σε ένα πεπερασμένο σώμα ή χρησιμοποιούν μια πεπερασμένη 
αναπαράσταση όπως οι προσεγγιστικοί αριθμητικοί συντελεστές. Ωστόσο, αξίζει να σημειωθεί πως
αυτός ο αλγόριθμος συμπεριφέρεται κακώς εάν οι συντελεστές είναι για παράδειγμα στο $\mathbb Q$.\\
Είσοδος :
\begin{center}{\en\tt pcar\_hessenberg([[4,1,-2],[1,2,-1],[2,1,0]] \% 37)}\end{center}
Έξοδος :
\begin{center}{\en\tt[1 ,-6\% 37,12 \% 37,-8 \% 37]}\end{center}
Είσοδος :
\begin{center}{\en\tt pcar\_hessenberg([[4,1,-2],[1,2,-1],[2,1,0]] \% 37,x)}\end{center}
Έξοδος :
\begin{center}{\en\tt x\verb|^|3-6 \%37 *x\verb|^|2+12 \% 37 *x-8 \% 37}\end{center}
Επομένως, το χαρακτηριστικό πολυώνυμο του [[4,1,-2],[1,2,-1],[2,1,0]] στο
$\mathbb Z/37 \mathbb Z$ είναι
\[ x^3-6x^2+12x-8 \]

\subsection{Ελάχιστο πολυώνυμο : {\tt\textlatin{ pmin}}}\index{pmin}
\noindent{{\en\tt pmin}  παίρνει ένα (αντίστοιχα δύο) όρισμα(-τα):
έναν τετραγωνικό πίνακα $A$ μεγέθους $n$ και προαιρετικά
το όνομα μιας συμβολικής μεταβλητής.\\
{\en\tt pmin} επιστρέφει το ελάχιστο πολυώνυμο του $A$ γραμμένο σαν λίστα 
των συντελεστών του, εάν κανένα όνομα μεταβλητής δεν παρέχεται,
ή γραμμένο σαν παράσταση ως προς το όνομα της μεταβλητής
που παρέχεται σαν δεύτερο όρισμα.
Το ελάχιστο πολυώνυμο του $A$ είναι το πολυώνυμο $P$ 
με τον  ελάχιστο βαθμό τέτοιο ώστε $P(A)=0$.}\\
Είσοδος :
\begin{center}{\en\tt pmin([[1,0],[0,1]])}\end{center}
Έξοδος :
\begin{center}{\en\tt [1,-1]}\end{center}
Είσοδος :
\begin{center}{\en\tt pmin([[1,0],[0,1]],x)}\end{center}
Έξοδος :
\begin{center}{\en\tt x-1}\end{center}
Επομένως, το ελάχιστο πολυώνυμο του [[1,0],[0,1]] είναι {\en\tt x-1}.\\
Είσοδος :
\begin{center}{\en\tt pmin([[2,1,0],[0,2,0],[0,0,2]])}\end{center}
Έξοδος :
\begin{center}{\en\tt [1,-4,4]}\end{center}
Είσοδος :
\begin{center}{\en\tt pmin([[2,1,0],[0,2,0],[0,0,2]],x)}\end{center}
Έξοδος :
\begin{center}{\en\tt x\verb|^|2-4*x+4}\end{center}
Επομένως, το ελάχιστο πολυώνυμο του [[2,1,0],[0,2,0],[0,0,2]] είναι $x^2-4x+4$.

\subsection{Προσαρτημένος πίνακας : {\tt\textlatin{ adjoint\_matrix}}}\index{adjoint\_matrix}
\noindent{{\en\tt adjoint\_matrix } παίρνει ως όρισμα έναν τετραγωνικό πίνακα 
$A$ μεγέθους $n$.\\
{\en\tt adjoint\_matrix } επιστρέφει την λίστα των συντελεστών του $P$ 
(χαρακτηριστικό πολυώνυμο του $A$), και
μία λίστα πινάκων που είναι οι συντελεστές του (γενικού) προσαρτημένου πίνακα μεγέθους $n$, δηλαδή του πολυωνύμου  $Q(x)\ =\ I\times  x^{n-1}+\dotsm+B_0 $ βαθμού $n-1$. {\bf Προσοχή!} Το {\en\tt Xcas} επιστρέφει την απόλυτη τιμή του $Q(x)$. O προσαρτημένος πινακας του $A$ είναι $(-1)^{n-1}Q(x)$ και επομένως, ο απλός προσαρτημένος πινακας του $A$ 
είναι $(-1)^{n-1}B_0\ =\ (-1)^{n-1}Q(0)$.} 

Ο απλός προσαρτημένος πινακας ενός τετραγωνικού πίνακα $A$ μεγέθους $n$ είναι ο πίνακας $B$ 
μεγέθους $n$ του οποίου το στοιχείο στην θέση $(i,j)$ είναι $(-1)^{i+j}$ επί την ορίζουσα του πίνακα που προκύπτει από τον $A$ αν διαγράψουμε την σειρά $j$ και την στήλη $i$ (προσέξτε την αναστροφή!). Ο (γενικός) προσαρτημένος πίνακας του $A$ είναι ο απλός προσαρτημένος πινακας του { \tt $xI-A$}. 
Ισχύει \[ A\times B = B\times A = \det(A)\times I \] 
καθώς επίσης και :
\[ P(x)\times I=\det(xI-A)I=(xI-A)Q(x)\]
Εφόσον το πολυώνυμο $P(x)\times I-P(A)$ 
μπορεί επίσης να διαιρεθεί από $x\times I-A$ (από αλγεβρικές ταυτότητες), 
αυτό αποδεικνύει ότι $P(A)=0$.\\
Είσοδος :
\begin{center}{\en\tt adjoint\_matrix([[4,1,-2],[1,2,-1],[2,1,0]])}\end{center}
Έξοδος :
\begin{center}
{\en\tt [
  {\bf [}1,-6,12,-8{\bf ]},\\
{\bf [} [[1,0,0],[0,1,0],[0,0,1]],
  [[-2,1,-2], [1,-4,-1],[2,1,-6]],
  [[1,-2,3],[-2,4,2],[-3,-2,7]] {\bf ]}
] }\end{center}
Επομένως, το χαρακτηριστικό πολυώνυμο  $P$ είναι :
\[ P(x)=x^3-6*x^2+12*x-8 \]
Η ορίζουσα του $A$ ισούται με $-P(0)=8$.
Ο απλός προσαρτημένος πίνακας του $A$ ισούται με :
\[ B=Q(0)=[[1,-2,3],[-2,4,2],[-3,-2,7]] \]
Επομένως, ο αντίστροφος του $A$ ισούται με :
\[ 1/8*[[1,-2,3],[-2,4,2],[-3,-2,7]] \]
Ο γενικός προσαρτημένος πίνακας του $A$ είναι :
\[ [[x^2-2x+1,x-2,-2x+3],[x-2,x^2-4x+4,-x+2],[2x-3,x-2,x^2-6x+7]] \]
Είσοδος :
\begin{center}{\en\tt adjoint\_matrix([[4,1],[1,2]])}\end{center}
Έξοδος :
\begin{center}{\en\tt[[1,-6,7],[[[1,0],[0,1]],[[-2,1],[1,-4]]]]}\end{center}
Επομένως, το χαρακτηριστικό πολυώνυμο $P$ είναι :
\[ P(x)=x^2-6*x+7 \]
Η ορίζουσα του $A$ ισούται με $+P(0)=7$.
Ο απλός προσαρτημένος πίνακας του $A$ ισούται με 
\[ Q(0)= -[[-2,1],[1,-4]] \]
Επομένως, ο αντίστροφος του $A$ ισούται με :
\[ -1/7*[[-2,1],[1,-4]] \]
Ο γενικός προσαρτημένος πίνακας του $A$ είναι :
\[ -[[x-2,1],[1,x-4]] \]

\subsection{Συνοδεύων πίνακας πολυωνύμου : {\tt\textlatin{ companion}}}\index{companion|textbf}\label{sec:compagne}
\noindent{{\en\tt companion} παίρνει ως όρισμα ένα πολυώνυμο $P$ με μοναδιαίο κύριο συντελεστή και το όνομα 
της μεταβλητής του.\\
{\en\tt companion} επιστρέφει τον πίνακα του οποίου το χαρακτηριστικό πολυώνυμο είναι $P$.\\
Εάν $P(x)=x^n+a_{n-1}x^{n-1}+...+a_{-1}x+a_0$,
αυτός ο πίνακας ισούται με τον μοναδιαίο πίνακα μεγέθους $n-1$ στον οποίο έχουν προστεθεί
$[0,0..,0,-a_0]$ σαν πρώτη γραμμή, και 
$[-a_0,-a_1,....,-a_{n-1}]$ σαν τελευαταία στήλη.}\\
Είσοδος :
\begin{center}{\en\tt companion(x\verb|^|2+5x-7,x)}\end{center}
Έξοδος :
\begin{center}{\en\tt  [[0,7],[1,-5]]}\end{center}
Είσοδος :
\begin{center}{\en\tt companion(x\verb|^|4+3x\verb|^|3+2x\verb|^|2+4x-1,x)}\end{center}
Έξοδος :
\begin{center}{\en\tt  [[0,0,0,1],[1,0,0,-4],[0,1,0,-2],[0,0,1,-3]]}\end{center}

\subsection{Αναγωγή σε πίνακα {\tt\textlatin{Hessenberg}}  : {\tt\textlatin{ hessenberg}}}\index{hessenberg}
\noindent{{\en\tt hessenberg} παίρνει ως όρισμα έναν πίνακα $A$.\\
{\en\tt hessenberg} επιστρέφει μία λίστα δύο πινάκων $P$ και  $H$ έτσι ώστε 
$A=PHP^{-1}$. Ο $H$ είναι πίνακας {\en\tt Hessenberg}} ισοδύναμος με τον $A$ και οι
συντελεστές του κάτω από την υποδιαγώνιο είναι μηδενικοί. 
\\
Είσοδος :
\begin{center}{\en\tt hessenberg([[3,2,2,2,2],[2,1,2,-1,-1],[2,2,1,-1,1], [2,-1,-1,3,1],[2,-1,1,1,2]])}\end{center}
Έξοδος :
\begin{center}{\en\tt  [[1,0,0,0,0],[0,1,0,0,0],[0,1,1,0,0],[0,1,1/2,1/4,1],
[0,1,1,1,0]],[[3,8,5,5/2,2],[2,1,1/2,-5/4,-1],[0,2,1,2,0],
[0,0,2,3/2,2],[0,0,0,13/8,7/2]]}\end{center}
Είσοδος 
\begin{center}
{\en\tt A:=[[3,2,2,2,2],[2,1,2,-1,-1],[2,2,1,-1,1],}\\
{\en\tt \ [2,-1,-1,3,1],[2,-1,1,1,2]] :;}\\
{\en\tt [P,H]:= hessenberg(A):; pcar(A,x)==pcar(H,x)
}
\end{center}
Έξοδος: 1.\\
Επομένως, οι πίνακες  $A$ και  $H$ έχουν το ίδιο χαρακτηριστικό πολυώνυμο.

\subsection{Ερμιτιανή κανονική μορφή : {\tt\textlatin{ ihermite}}}\index{ihermite}
\noindent{{\en\tt ihermite} παίρνει ως όρισμα έναν πίνακα {\en\tt A} με συντελεστές 
στο $\mathbb Z$.\\
{\en\tt ihermite} επιστρέφει δύο πίνακες {\en\tt U} και {\en\tt B} τέτοιους ώστε 
{\en\tt B=U*A}, όπου ο {\en\tt U} είναι αντιστρέψιμος στο  $\mathbb Z$ ({\en det}$(U) = \pm 1$)
και ο {\en\tt B} είναι άνω τριγωνικός. Εκτός αυτού,
η απόλυτη τιμή των συντελεστών πάνω από τη διαγώνιο του {\en\tt B} είναι μικρότερη 
από τον οδηγό της στήλης διαιρεμένο δια 2.

Η απάντηση υπολογίζεται από έναν αλγόριθμο σαν τον αλγόριθμο αναγωγής {\en\tt Gauss}
κάνοντας μόνο πράξεις γραμμών με ακέραιους συντελεστές.}\\
Είσοδος :
\begin{center}{\en\tt A:=[[9,-36,30],[-36,192,-180],[30,-180,180]]; [U,B]:=ihermite(A)}\end{center}
΄Έξοδος :
\begin{center}{\en\tt [[9,-36,30],[-36,192,-180],[30,-180,180]], [[13,9,7],[6,4,3],[20,15,12]],[[3,0,30],[0,12,0],[0,0,60]]}\end{center}

{\bf Εφαρμογή: Υπολογίστε μια $\mathbb Z$-βάση  του πυρήνα ενός πίνακα
με ακέραιους συντελεστές}\\
Έστω {\en\tt M} ένας πίνακας με ακέραιους συντελεστές.
Είσοδος :
\begin{center}
{\en\tt (U,A):=ihermite(transpose(M))}.
\end{center}
Αυτό επιστρέφει τους πίνακες  $U$ και τον $A$ τέτοιους ώστε {\en\tt A=U*transpose(M)} και επομένως \\
{\en\tt transpose(A)=M*transpose(U)}.\\
Οι μηδενικές στήλες του {\en\tt transpose(A)}  (στα δεξιά του πίνακα,
προερχόμενες από τις μηδενικές γραμμές του $A$  στο κάτω μέρος του πίνακα)
αντιστοιχούν στις στήλες του {\en\tt transpose(U)} οι οποίες σχηματίζουν μια βάση 
του {\en\tt Ker(M)}. Με άλλα λόγια, οι μηδενικές γραμμές του {\en\tt A}
 αντιστοιχούν στις γραμμές του {\en\tt U} 
οι οποίες σχηματίζουν μια βάση του {\en\tt Ker(M)}.\\ 
{\bf Παράδειγμα}\\
Έστω {\en\tt  M:=[[1,4,7],[2,5,8],[3,6,9]]}. Είσοδος 
\begin{center}{\en\tt  [U,A]:=ihermite(tran(M))}\end{center}
Έξοδος:
\begin{center}
{\en\tt  U:=[[-3,1,0],[4,-1,0],[-1,2,-1]] {\gr και} A:=[[1,-1,-3],[0,3,6],[0,0,0]]}
\end{center}
Αφού {\en\tt A[2]=[0,0,0]}, μια $\mathbb Z$-βάση του {\en\tt Ker(M)} είναι
{\en\tt U[2]=[-1,2,-1]}.\\
Επαλήθευση {\en\tt  M*U[2]=[0,0,0]}.

\subsection{Κανονική μορφή \tt\textlatin{Smith} : {\tt\textlatin{ ismith}}}\index{ismith}
\noindent{{\en\tt ismith} παίρνει ως όρισμα έναν πίνακα με συντελεστές στο
$\mathbb Z$.\\
{\en\tt ismith} επιστρέφει τους τρεις πίνακες {\en\tt U, B, V} όπου  {\en\tt B=U*A*V}, και οι {\en\tt U} και {\en\tt V} είναι αντιστρέψιμοι στο  $\mathbb Z$. 
Ο {\en\tt B} είναι διαγώνιος, και το {\en\tt B[i,i]} διαιρεί το {\en\tt B[i+1,i+1]}.
Οι συντελεστές {\en\tt B[i,i]} ονομάζονται 
αναλλοίωτοι παράγοντες, και χρησιμοποιούνται για να περιγράψουν
την δομή πεπερασμένων αβελιανών (αντιμεταθετικών) ομάδων.}\\
Είσοδος :
\begin{center}
{\en\tt A:=[[9,-36,30],[-36,192,-180],[30,-180,180]]; 
[U,B,V]:=ismith(A)}
\end{center}
Έξοδος :
\begin{center}{\en\tt
[[-3,0,1],[6,4,3],[20,15,12]],
[[3,0,0],[0,12,0],[0,0,60]], 
[[1,24,-30],[0,1,0],[0,0,1]] }
\end{center}
Οι αναλλοίωτοι παράγοντες είναι οι 3, 12 και 60.

\section{Ισομετρίες}
\subsection{Αναγνώριση ισομετρίας : {\tt\textlatin isom}}\index{isom}
\noindent{\en\tt isom} παίρνει ως όρισμα τον πίνακα ενός γραμμικού μετασχηματισμού 
σε  2 ή 3 διαστάσεις.\\
{\en\tt isom} επιστρέφει :
\begin{itemize}
\item  
εάν ο γραμμικός μετασχηματισμός  είναι άμεση ισομετρία,\\
την λίστα των χαρακτηριστικών στοιχείων αυτής της ισομετρίας και {\en\tt +1},
\item εάν ο γραμμικός μετασχηματισμός είναι έμμεση ισομετρία,\\
την λίστα των χαρακτηριστικών στοιχείων αυτής της ισομετρίας και {\en\tt -1} 
\item εάν ο γραμμικός μετασχηματισμός δεν είναι ισομετρία,\\
{\en\tt [0]}.
\end{itemize}
Είσοδος :
\begin{center}{\en\tt isom([[0,0,1],[0,1,0],[1,0,0]])}\end{center}
Έξοδος :
\begin{center}{\en\tt  [[1,0,-1],-1]}\end{center}
το οποίο σημαίνει ότι αυτή η ισομετρία είναι μια {\en\tt 3-d} συμμετρία σχετικά με το επίπεδο 
$x\ -\ z\ =\ 0$.\\ 
Είσοδος :
\begin{center}{\en\tt isom(sqrt(2)/2*[[1,-1],[1,1]])}\end{center}
Έξοδος :
\begin{center}{\en\tt [pi/4,1]}\end{center}
Επομένως, αυτή η ισομετρία είναι μια {\en\tt 2-d} περιστροφή κατά 
$\displaystyle \frac{\pi}{4}$.\\
Είσοδος :
\begin{center}{\en\tt isom([[0,0,1],[0,1,0],[0,0,1]])}\end{center}
Έξοδος :
\begin{center}{\en\tt [0]}\end{center}
Επομένως, αυτός ο μετασχηματισμός δεν είναι ισομετρία.

\subsection{Εύρεση του πίνακα ισομετρίας : {\tt\textlatin{ mkisom}}}\index{mkisom}
{\en\tt mkisom} παίρνει ως όρισμα :
\begin{itemize}
\item  Στις 3 διαστάσεις, την λίστα των χαρακτηριστικών στοιχείων 
(κατεύθυνση άξονα, την γωνία για περιστοφή ή την κάθετο στο επίπεδο για
μια συμμετρία) και {\en\tt +1} για μια άμεση ισομετρία ή 
{\en\tt -1} μια έμμεση ισομετρία.
\item Στις 2 διαστάσεις, ένα χαρακτηριστικό στοιχείο (μια γωνία ή ένα διάνυσμα) και 
{\en\tt +1} για άμεση ισομετρία (περιστροφή) ή {\en\tt -1} για 
έμμεση ισομετρία (συμμετρία).
\end{itemize}
{\en\tt mkisom} επιστρέφει τον πίνακα της αντίστοιχης ισομμετρίας.\\ 
Είσοδος :
\begin{center}{\en\tt mkisom([[-1,2,-1],pi],1)}\end{center}
Έξοδος, ο πίνακας περιστροφής του άξονα $[-1,2,-1]$ κατά $\pi$:
\begin{center}{\en\tt [[-2/3,-2/3,1/3],[-2/3,1/3,-2/3],[1/3,-2/3,-2/3]]}\end{center}
Είσοδος :
\begin{center}{\en\tt  mkisom([pi],-1)}\end{center}
Έξοδος, ο πίνακας συμμετρίας ως προς  το $O$ :
\begin{center}{\en\tt [[-1,0,0],[0,-1,0],[0,0,-1]]}\end{center}
Είσοδος :
\begin{center}{\en\tt  mkisom([1,1,1],-1)}\end{center}
Έξοδος, ο πίνακας συμμετρίαςως προς  το επίπεδο $x+y+z=0$ :
\begin{center}{\en\tt [[1/3,-2/3,-2/3],[-2/3,1/3,-2/3],[-2/3,-2/3,1/3]]}\end{center}
Είσοδος :
\begin{center}{\en\tt mkisom([[1,1,1],pi/3],-1)}\end{center}
Έξοδος,ο πίνακας του γινομένου μιας περιστροφής του άξονα $[1,1,1]$ κατά 
$\frac{\pi}{3}$ και μιας συμμετρίας ως προς το $x+y+z=0$:
\begin{center}{\en\tt  [[0,-1,0],[0,0,-1],[-1,0,0]]}\end{center}
Είσοδος :
\begin{center}{\en\tt mkisom(pi/2,1)}\end{center}
Έξοδος, ο πίνακας της επίπεδης περιστροφής κατά $\frac{\pi}{2}$ :
\begin{center}{\en\tt [[0,-1],[1,0]]}\end{center}
Είσοδος :
\begin{center}{\en\tt mkisom([1,2],-1)}\end{center}
Έξοδος, ο πίνακας της επίπεδης συμμετρίας ως προς την ευθεία της εξίσωσης
$x+2y=0$:
\begin{center}{\en\tt [[3/5,-4/5],[-4/5,-3/5]]}\end{center}

\section{Παραγοντοποιήσεις πινάκων}\label{sec:factormatrice}
Σημειώστε ότι οι περισσότεροι αλγόριθμοι παραγοντοποίησης πινάκων υλοποιούνται αριθμητικά,
και ότι μόνο λίγοι απ' αυτούς θα δουλέψουν συμβολικά.

\subsection{Παραγοντοποίηση {\tt\textlatin{Cholesky}} : {\tt\textlatin{ cholesky}}}\index{cholesky}
\noindent{{\en\tt cholesky} παίρνει ως όρισμα έναν τετραγωνικό, συμμετρικό, και
θετικά ορισμένο πίνακα {\en\tt A} μεγέθους $n$.}\\
{\en\tt cholesky} επιστρέφει έναν συμβολικό ή αριθμητικό πίνακα {\en\tt P}. Ο {\en\tt P} είναι ένας
κάτω τριγωνικός πίνακας τέτοιος ώστε :
\begin{center}
{\en\tt P*tran(P)=A}
\end{center}
Είσοδος :
\begin{center}{\en\tt cholesky([[1,1],[1,5]])}\end{center}
Έξοδος :
\begin{center}{\en\tt [[1,0],[1,2]]}\end{center}
Είσοδος :
\begin{center}{\en\tt cholesky([[3,1],[1,4]])}\end{center}
Έξοδος :
\begin{center}{\en\tt [[sqrt(3),0],[(sqrt(3))/3,(sqrt(33))/3]]}\end{center}
Είσοδος :
\begin{center}{\en\tt cholesky([[1,1],[1,4]])}\end{center}
Έξοδος :
\begin{center}{\en\tt [[1,0],[1,sqrt(3)]]}\end{center}
{\bf Προσοχή} Εάν το όρισμα, δηλαδή ο πίνακας $A$, δεν είναι ένας συμμετρικός πίνακας,
{\en\tt cholesky} δεν επιστρέφει λάθος, αντ' αυτού  {\en\tt cholesky} θα
χρησιμοποιήσει τον συμμετρικό πίνακα $B$ της τετραγωνικής μορφής  $q$ 
που αντιστοιχεί στη (μη συμμετρική) διγραμμική μορφή του πίνακα $A$.\\
Είσοδος :
\begin{center}{\en\tt cholesky([[1,-1],[-1,4]])}\end{center}
ή :
\begin{center}{\en\tt cholesky([[1,-3],[1,4]])}\end{center}
Έξοδος :
\begin{center}{\en\tt [[1,0],[-1,sqrt(3)]]}\end{center}

\subsection{Παραγοντοποίηση {\tt\textlatin{QR}} : {\tt\textlatin{ qr}}}\index{qr}
\noindent{\en\tt qr} παίρνει ως όρισμα έναν
τετραγωνικό πίνακα $A$ μεγέθους $n$.\\
{\en\tt qr} παραγοντοποιεί αριθμητικά 
%(συμβολικά όταν είναι πιθανόν) 
αυτόν τον πίνακα σε $Q*R$ όπου
$Q$ είναι ένας ορθογώνιος πίνακας (${\tt\en tran}(Q)*Q=I$) και ο $R$ είναι ένας άνω τριγωνικός 
πίνακας. 
\\
Είσοδος :
\begin{center}{\en\tt qr([[3,5],[4,5]])}\end{center}
Έξοδος :
\begin{center}{\en\tt [[-0.6,-0.8],[-0.8,0.6]], [[-5.0,-7.0],[0,-1.0]]}\end{center}

\subsection{Παραγοντοποίηση {\tt\textlatin{LU}} : {\tt\textlatin{ lu}}}\index{lu}
\noindent{{\en\tt lu} παίρνει ως όρισμα έναν τετραγωνικό πίνακα $A$ μεγέθους $n$ (αριθμητικό ή
συμβολικό).\\
{\en\tt lu} επιστρέφει μια μετάθεση (διάταξη) $p$ των στοιχείων 0..$n-1$, 
έναν κάτω τριγωνικό πίνακα $L$, με $1$ στη διαγώνιο, 
και έναν άνω τριγωνικό πίνακα $U$, έτσι ώστε : 
\begin{itemize}
\item $P*A=L*U$, όπου $P$ είναι ο πίνακας μεταθέσεων που
παράγεται από την $p$ (υπολογίζεται με {\en\tt P:=permu2mat(p)}),
\item η εξίσωση $A*x=B$ είναι ισοδύναμη με :
\[ L*U*x=P*B=p(B) \mbox{ όπου } p(B)=[b_{p(0)},b_{p(1)}..b_{p(n-1)}],
\quad  B=[b_0,b_1..b_{n-1}] \]
\end{itemize}
Ο  πίνακας μεταθέσεων $P$ ορίζεται από την $p$ ως :
\[ P[i, p(i)]=1, \quad P[i, j]=0 \mbox{ εάν } j \ \neq\ p(i) \]
Με άλλα λόγια, είναι ο ταυτοτικός πίνακας όπου οι γραμμές μετατίθενται 
σύμφωνα με την μετάθεση $p$. 
Η συνάρτηση {\en\tt permu2mat}\index{permu2mat} χρησιμοποιείται για τον υπολογισμό του $P$
({\en\tt permu2mat(p)} επιστρέφει τον πίνακα $P$).}\\ 
Είσοδος :
\begin{center}{\en\tt [p,L,U]:=lu([[3.,5.],[4.,5.]])}\end{center}
Έξοδος :
\begin{center}{\en\tt [1,0],[[1,0],[0.75,1]],[[4.0,5.0],[0,1.25]]}\end{center}
Εδώ $n=2$, και επομένως :
\[ P[0,p(0)]=P[0,1]=1, \quad  P[1,p(1)]=P[1,0]=1, \quad
P=[[0,1],[1,0]] \]
Επαλήθευση :\\
Είσοδος (όπου ${\tt A:=[[3.,5.],[4.,5.]]}$) :
\begin{center}{\en\tt permu2mat(p)*A; L*U}\end{center}
Έξοδος:
\begin{center}{\en\tt [[4.0,5.0],[3.0,5.0]],[[4.0,5.0],[3.0,5.0]]}\end{center}
Να σημειωθεί ότι η μετάθεση είναι διαφορετική για ακριβή είσοδο (ως 
οδηγός επιλέγεται ο απλούστερος αντί για τον μεγαλύτερο σε απόλυτη τιμή).\\
Είσοδος :
\begin{center}{\en\tt lu([[1,2],[3,4]])}\end{center}
Έξοδος :
\begin{center}{\en\tt [1,0],[[1,0],[3,1]],[[1,2],[0,-2]]}\end{center}
Είσοδος :
\begin{center}{\en\tt lu([[1.0,2],[3,4]])}\end{center}
Έξοδος :
\begin{center}{\en\tt [1,0],[[1,0],[0.333333333333,1]],[[3,4], [0,0.666666666667]]}\end{center}

\subsection{Παραγοντοποίηση \tt\textlatin{svd} : {\tt\textlatin{ svd}}}\index{svd}
\noindent{{\en\tt svd} ({\tt\textlatin{singular value decomposition}}) παίρνει ως όρισμα έναν αριθμητικό
τετραγωνικό πίνακα μεγέθους $n$.}\\
{\en\tt svd} επιστρέφει έναν ορθογώνιο πίνακα $U$, την διαγώνιο $s$ ενός διαγώνιου 
πίνακα $S$ και έναν ορθογώνιο πίνακα $Q$ (${\en\tt tran}(Q)*Q=I$) τέτοιο ώστε :
\[ A=U.S.Q \]
Είσοδος :
\begin{center}{\en\tt svd([[1,2],[3,4]])}\end{center}
Έξοδος :
\begin{center}{\en\tt [[-0.404553584834,-0.914514295677],[-0.914514295677, 0.404553584834]], [5.46498570422,0.365966190626], [[-0.576048436766,-0.81741556047],[0.81741556047, -0.576048436766]]}\end{center}
Είσοδος :
\begin{center}{\en\tt [U,s,Q]:=svd([[3,5],[4,5]])}\end{center}
Έξοδος :
\begin{center}{\en\tt [[-0.672988041811,-0.739653361771],[-0.739653361771, 0.672988041811]],[8.6409011028,0.578643354497], [[-0.576048436766,-0.81741556047],[0.81741556047, -0.576048436766]]}\end{center}
Επαλήθευση :\\
Είσοδος :
\begin{center}{\en\tt U*diag(s)*Q}\end{center}
Έξοδος :
\begin{center}{\en\tt [[3.0,5.0],[4.0,5.0]]}\end{center}


\subsection{Βραχεία (σχεδόν ορθόγωνη) βάση πλέγματος : {\tt\textlatin{lll}}}\index{lll}
\noindent{{\en\tt lll} παίρνει ως όρισμα έναν αντιστρέψιμο πίνακα $M$ με
ακέραιους συντελεστές.}\\
{\en\tt lll} επιστρέφει $(S,A,L,O)$ έτσι ώστε:
\begin{itemize}
\item οι γραμμές του $S$ είναι η βραχεία βάση του $\mathbb Z$-προτύπου, η 
παραγόμενη από τις γραμμές του $M$,
\item $A$ είναι ο πίνακας-αλλαγής-βάσης από την βραχεία βάση στην βάση 
που ορίζεται από τις γραμμές του $M$ ($A*M=S$),
\item $L$ είναι ένας κάτω τριγωνικός πίνακας, στον οποίο το μέτρο των μη διαγώνιων 
συντελεστών του είναι μικρότερο από 1/2,
\item $O$ είναι ένας πίνακας με ορθογώνιες γραμμές έτσι ώστε $L*O=S$.
\end{itemize}
% Εάν στη διάσταση 2, τα $[a,b]$ είναι συντεταγμένες ενός συστήματος διανυσμάτων σε βάση που ορίζεται από 
%  το $M$ και εάν οι  $[a1,b1]$ ειναι οι συντεταγμένες στη short βάση 
% που ορίζεται από το $S$ π.χ. εάν $[a,b]*M=[a1,b1]*S$, τότε:\\ 
% $[a,b]=[a1,b1]*A$\\
% $[a1,b1]*S=[a1,b1]*A*M=[a,b]*M$ et\\
% $[a,b]*M=[a,b]*A^{-1}*S=[a1,b1]*S$\\
Είσοδος :
\begin{center}{\en\tt [S,A,L,O]:=lll(M:=[[2,1],[1,2]])}\end{center}
Έξοδος :
\begin{center}{\en\tt [[-1,1],[2,1]], [[-1,1],[1,0]], [[1,0],[1/-2,1]], [[-1,1],[3/2,3/2]]}\end{center}
Επομένως :\\
{\en\tt S=[[-1,1],[2,1]]}\\
{\en\tt A=[[-1,1],[1,0]]}\\
{\en\tt L=[[1,0],[1/-2,1]]}\\
{\en\tt O=[[-1,1],[3/2,3/2]]}\\
Επομένως, η αρχική βάση είναι η {\en\tt v1=[2,1], v2=[1,2]}\\
και η βραχεία βάση είναι η {\en\tt w1=[-1,1], w2=[2,1]}.\\
Αφού {\en\tt w1=-v1+v2} και {\en\tt w2=v1} τότε :\\
{\en\tt A:=[[-1,1],[1,0]]}, {\en\tt A*M==S} και {\en\tt L*O==S}.\\
Είσοδος :
\begin{center}{\en\tt (S,A,L,O):=lll([[3,2,1],[1,2,3],[2,3,1]])}\end{center}
Έξοδος :
\begin{center}{\en\tt S=[[-1,1,0],[-1,-1,2],[3,2,1]] }\end{center}
\begin{center}{\en\tt A= [[-1,0,1],[0,1,-1],[1,0,0]]}\end{center}
\begin{center}{\en\tt L= [[1,0,0],[0,1,0],[(-1)/2,(-1)/2,1]]}\end{center}
\begin{center}{\en\tt O= [[-1,1,0],[-1,-1,2],[2,2,2]]}\end{center}
Είσοδος :\\
{\en\tt M:=[[3,2,1],[1,2,3],[2,3,1]]}\\
Ιδιότητες :\\
{\en\tt A*M==S} και {\en\tt L*O==S}

\section{Τετραγωνικές μορφές}
\subsection{Πίνακας μιας τετραγωνικής μορφής : {\tt\textlatin{ q2a}}}\index{q2a} 
\noindent{{\en\tt q2a} παίρνει δύο ορίσματα : την συμβολική παράσταση μιας
τετραγωνικής μορφής $q$ και ένα
διάνυσμα με ονόματα μεταβλητών.\\ 
{\en\tt q2a} επιστρέφει τον πίνακα $A$ του $q$.}\\ 
Είσοδος :
\begin{center}{\en\tt q2a(2*x*y,[x,y])}\end{center}
Έξοδος :
\begin{center}{\en\tt  [[0,1],[1,0]]}\end{center}

\subsection{Μετατροπή πίνακα σε τετραγωνική μορφή : {\tt\textlatin{ a2q}}}\index{a2q}
\noindent{{\en\tt a2q} παίρνει δύο ορίσματα : τον συμμετρικό πίνακα $A$ 
μιας τετραγωνικής 
μορφής $q$ και ένα διάνυσμα από ονόματα μεταβλητών του ίδιου μεγέθους.\\
{\en\tt a2q} επιστρέφει την συμβολική παράσταση της τετραγωνικής μορφής $q$.}\\   
Είσοδος :
\begin{center}{\en\tt a2q([[0,1],[1,0]],[x,y])}\end{center}
Έξοδος :
\begin{center}{\en\tt 2*x*y}\end{center}
Είσοδος :
\begin{center}{\en\tt a2q([[1,2],[2,4]],[x,y]) }\end{center}
Έξοδος :
\begin{center}{\en\tt x\verb|^|2+4*x*y+4*y\verb|^|2}\end{center}

\subsection{Αναγωγή μιας τετραγωνικής μορφής : {\tt\textlatin{ gauss}}}\index{gauss}
\noindent{{\en\tt gauss} παίρνει δύο ορίσματα : μια συμβολική παράσταση
που αντιπροσωπεύει μια τετραγωνική μορφή $q$ και ένα διάνυσμα 
από ονόματα μεταβλητών.\\ 
{\en\tt gauss} επιστρέφει το  $q$ γραμμένο σαν ένα άθροισμα ή μια διαφορά τετραγώνων
χρησιμοποιώντας τον αλγόριθμο του {\en\tt Gauss}.}\\      
Είσοδος :
\begin{center}{\en\tt gauss(2*x*y,[x,y])}\end{center}
Έξοδος :
\begin{center}{\en\tt (y+x)\verb|^|2/2+(-(y-x)\verb|^|2)/2}\end{center}

\subsection{Ορθοκανονικοποίηση \tt\textlatin{Gramschmidt}  : {\tt\textlatin{ gramschmidt}}}\index{gramschmidt}
\noindent{{\en\tt gramschmidt} παίρνει ένα ή δύο ορίσματα : 
\begin{itemize}
\item έναν πίνακα που αναπαρίσταται σαν μια λίστα διανυσμάτων (γραμμών), όπου από προεπιλογή
ως εσωτερικό γινόμενο θεωρείται το κανονικό εσωτερικό γινόμενο, ή
\item μια λίστα στοιχείων
που είναι η βάση ενός διανυσματικού υποχώρου, και μια συνάρτηση που ορίζει ένα εσωτερικό 
γινόμενο σε αυτόν τον διανυσματικό χώρο.
\end{itemize}
{\en\tt gramschmidt} επιστρέφει την ορθοκανονική βάση ως προς αυτό το εσωτερικό γινόμενο.}\\ 
Είσοδος :
\begin{center}{\en\tt normal(gramschmidt([[1,1,1],[0,0,1],[0,1,0]]))}\end{center}
ή εισάγετε :
\begin{center}{\en\tt normal(gramschmidt([[1,1,1],[0,0,1],[0,1,0]],dot))}\end{center}
Έξοδος :
\begin{center}{\en\tt [[(sqrt(3))/3,(sqrt(3))/3,(sqrt(3))/3]},\end{center} 
\begin{center}{\en\tt [(-(sqrt(6)))/6,(-(sqrt(6)))/6,(sqrt(6))/3]},\end{center} 
\begin{center}{\en\tt [(-(sqrt(2)))/2,(sqrt(2))/2,0]]}\end{center}
{\bf Παράδειγμα}\\
Ορίζουμε ένα εσωτερικό γινόμενο στον διανυσματικό χώρο 
των πολυωνύμων με: 
$$P.Q=\int_{-1}^1P(x).Q(x)dx $$
Είσοδος :
 \begin{center}{\en\tt gramschmidt([1,1+x],(p,q)->integrate(p*q,x,-1,1))}\end{center}
ή ορίζουμε την συνάρτηση {\en\tt p\_scal}, και εισάγουμε :\\
{\en\tt p\_scal(p,q):=integrate(p*q,x,-1,1)}\\
και μετά εισάγουμε :
\begin{center}{\en\tt gramschmidt([1,1+x],p\_scal)}\end{center}
Έξοδος :
\begin{center}{\en\tt [1/(sqrt(2)),(1+x-1)/sqrt(2/3)]}\end{center}

\subsection{Γράφος μιας κωνικής καμπύλης : {\tt\textlatin{ conic}}}\index{conic}
\noindent{{\en\tt conic} παίρνει σαν όρισμα την εξίσωση μιας κωνικής καμπύλης ως προς  $x,y$. Μπορείτε επίσης να ορίσετε τα ονόματα των μεταβλητών σαν
δεύτερο και τρίτο όρισμα ή με διάνυσμα σαν δεύτερο όρισμα.\\ 
{\en\tt conic} σχεδιάζει την  κωνική καμπύλη.}\\
Είσοδος :
\begin{center}{\en\tt conic(2*x\verb|^|2+2*x*y+2*y\verb|^|2+6*x)}\end{center}
Έξοδος :
\begin{center}{ Ο γράφος έλλειψης, κέντρου {\en\tt -2+i} ή (-2, 1) και εξίσωσης {\en\tt 2*x\verb|^|2+2*x*y+2*y\verb|^|2+6*x=0}}\end{center}

\subsection{Γράφος μιας δευτεροβάθμιας επιφάνειας : {\tt\textlatin{ quadric}}}\index{quadric}
\noindent{{\en\tt quadric} παίρνει σαν όρισμα την παράσταση μιας δευτεροβάθμιας επιφάνειας ως προς $x,y,z$. Μπορείτε επίσης να ορίσετε τις μεταβλητές
σαν ένα διάνυσμα (δεύτερο όρισμα) ή σαν δεύτερο, τρίτο και τέταρτο όρισμα.\\ 
{\en\tt quadric} σχεδιάζει την δευτεροβάθμια επιφάνεια.}\\
Είσοδος :
\begin{center}{\en\tt quadric(7*x\verb|^|2+4*y\verb|^|2+4*z\verb|^|2+4*x*y- 4*x*z-2*y*z-4*x+5*y+4*z-18)}\end{center}
Έξοδος :
\begin{center}{ η σχεδίαση του ελλειψοειδούς της εξίσωσης {\en\tt 7*x\verb|^|2+4*y\verb|^|2+4*z\verb|^|2+4*x*y-4*x*z-2*y*z-4*x+5*y+4*z-18=0}}\end{center}

\section{Πολυμεταβλητός Λογισμός}
\subsection{Ανάδελτα (\tt\textlatin{gradient}) : {\tt\textlatin{ derive deriver diff \\ grad}}}\index{derive}\index{diff}\index{grad}\index{deriver}
\noindent{{\en\tt derive} (ή {\en\tt diff} ή {\en\tt grad}) παίρνει δύο ορίσματα : μια 
παράσταση $F$ από $n$ πραγματικές μεταβλητές και ένα διάνυσμα με τα ονόματα αυτών των μεταβλητών.}\\
{\en\tt derive} (ή {\en\tt diff} ή {\en\tt grad}) επιστρέφει την κλίση (ανάδελτα) της $F$,
όπου το ανάδελτα είναι το διάνυσμα όλων των μερικών παραγώγων,
για παράδειγμα, στις τρεις διαστάσεις ($n=3$):
\[ \overrightarrow{\mbox{\en\tt grad}}(F)= [\frac{\partial F}{\partial x},\frac{\partial F}{\partial y},\frac{\partial F}{\partial z}] \] 
{\bf Παράδειγμα} \\
Βρείτε την κλίση της $F(x,y,z)=2x^2y-xz^3$.\\
Είσοδος :
\begin{center}{\en\tt derive(2*x\verb|^|2*y-x*z\verb|^|3,[x,y,z])}\end{center}
ή :
\begin{center}{\en\tt diff(2*x\verb|^|2*y-x*z\verb|^|3,[x,y,z])}\end{center}
ή :
\begin{center}{\en\tt grad(2*x\verb|^|2*y-x*z\verb|^|3,[x,y,z])}\end{center}
Έξοδος :
\begin{center}{\en\tt [2*2*x*y-z\verb|^|3,2*x\verb|^|2,-(x*3*z\verb|^|2)]}\end{center}
Έξοδος μετά από απλοποίηση με την {\en\tt normal(ans())} :
\begin{center}{\en\tt [4*x*y-z\verb|^|3,2*x\verb|^|2,-(3*x*z\verb|^|2)]}\end{center}
Για να βρείτε τα κρίσιμα σημεία της 
$F(x,y,z)=2x^2y-xz^3$, εισάγετε :
\begin{center}{\en\tt solve(derive(2*x\verb|^|2*y-x*z\verb|^|3,[x,y,z]),[x,y,z])}\end{center} 
Έξοδος :
\begin{center}{\en\tt [[0,y,0]]}\end{center} 

\subsection{Λαπλασιανή : {\tt\textlatin{ laplacian}}}\index{laplacian}
\noindent{{\en\tt laplacian} παίρνει δύο ορίσματα : μια 
παράσταση $F$ από $n$ πραγματικές μεταβλητές και ένα διάνυσμα με τα ονόματα αυτών των μεταβλητών.}\\
{\en\tt laplacian} επιστρέφει την λαπλασιανή της $F$, που είναι το άθροισμα όλων των δεύτερων
μερικών παραγώγων, για παράδειγμα, στις τρεις διαστάσεις ($n=3$):
\[ \nabla^2(F)=\frac{\partial^2 F}{\partial x^2}+\frac{\partial^2 F}{\partial y^2}+\frac{\partial^2 F}{\partial z^2} \]
{\bf Παράδειγμα}\\
Βρείτε την λαπλασιανή της $F(x,y,z)=2x^2y-xz^3$.\\
Είσοδος :
\begin{center}{\en\tt laplacian(2*x\verb|^|2*y-x*z\verb|^|3,[x,y,z])}\end{center}
Έξοδος :
\begin{center}{\en\tt 4*y+-6*x*z}\end{center}

\subsection{Πίνακας του {\tt\textlatin{Hesse}} : {\tt\textlatin{ hessian}}}\index{hessian}
\noindent{{\en\tt hessian}  παίρνει δύο ορίσματα : μια 
παράσταση $F$ από $n$ πραγματικές μεταβλητές και ένα διάνυσμα με τα ονόματα αυτών των μεταβλητών.\\
{\en\tt hessian} επιστρέφει τον πίνακα του {\tt\textlatin{Hesse}} της $F$, που είναι ο πίνακας 
των παραγώγων 2ης τάξης, δηλαδή $H(f)_{ij}(x) = D_i D_j f(x)$, όπου $x = (x_1, x_2, ..., x_n$).}\\
{\bf Παράδειγμα}\\
Βρείτε τον πίνακα του {\tt\textlatin{Hesse}} της $F(x,y,z)=2x^2y-xz^3$.\\
Είσοδος :
\begin{center}{\en\tt hessian(2*x\verb|^|2*y-x*z\verb|^|3 , [x,y,z])}\end{center}
Έξοδος :
\begin{center}{\en\tt[[4*y,4*x,-(3*z\verb|^|2)],[2*2*x,0,0],[-(3*z\verb|^|2),0,x*3*2*z]]}\end{center}
Για να πάρετε τον πίνακα του {\tt\textlatin{Hesse}} στα κρίσιμα σημεία, αρχικά εισάγετε :
\begin{center}{\en\tt solve(derive(2*x\verb|^|2*y-x*z\verb|^|3,[x,y,z]),[x,y,z])}\end{center} 
Έξοδος είναι τα κρίσιμα σημεία : 
\begin{center}{\en\tt [[0,y,0]]}\end{center}
Έπειτα, για να πάρετε τον πίνακα του {\tt\textlatin{Hesse}} σε αυτά τα σημεία, εισάγετε : 
\begin{center}{\en\tt subst([[4*y,4*x,-(3*z\verb|^|2)],[2*2*x,0,0], [-(3*z\verb|^|2),0,6*x*z]],[x,y,z],[0,y,0])}\end{center}
Έξοδος :
\begin{center}{\en\tt [[4*y,4*0,-(3*0\verb|^|2)],[4*0,0,0],[-(3*0\verb|^|2),0,6*0*0]]}\end{center}
και μετά από απλοποίηση :
\begin{center}{\en\tt [[4*y,0,0],[0,0,0],[0,0,0]]}\end{center}

\subsection{Απόκλιση : {\tt\textlatin{ divergence}}}\index{divergence}
\noindent{{\en\tt divergence} παίρνει δύο ορίσματα : ένα διανυσματικό
πεδίο διάστασης $n$, και το διάνυσμα με τις  $n$ πραγματικές μεταβλητές του πεδίου.}\\
{\en\tt divergence} επιστρέφει την απόκλιση της $F$ που είναι το άθροισμα 
των παραγώγων του $k$-στού στοιχείου ως προς
την $k$-στή μεταβλητή. Για παράδειγμα, στις τρεις  διαστάσεις ($n=3$):
\begin{center}
  {\en\tt divergence([A,B,C],[x,y,z])}=$\displaystyle\frac{\partial A}{\partial x}+\frac{\partial B}{\partial y}+\frac{\partial C}{\partial z}$
\end{center}
Είσοδος :
\begin{center}{\en\tt divergence([x*z,-y\verb|^|2,2*x\verb|^|y],[x,y,z])}\end{center}
Έξοδος :
\begin{center}{\en\tt z+-2*y}\end{center}

\subsection{Στροβιλισμός : {\tt\textlatin{ curl}}}\index{curl}
\noindent{{\en\tt curl}  παίρνει δύο ορίσματα : ένα τρισδιάστατο διανυσματικό
πεδίο, και το διάνυσμα με τις  $3$ πραγματικές μεταβλητές του πεδίου.}\\
{\en\tt curl} επιστρέφει τον  στροβιλισμό του διανύσματος, που ορίζεται από :
\begin{center}
{\en\tt curl([A,B,C],[x,y,z])}=$\displaystyle [\frac{\partial C}{\partial y}-\frac{\partial B}{\partial z},\ \frac{\partial A}{\partial z}-\frac{\partial C}{\partial x},\ \frac{\partial B}{\partial x}-\frac{\partial A}{\partial y}]$
\end{center}
Να σημειωθεί πως ο αριθμός $n$ των διαστάσεων {\bf πρέπει να ισούται με 3}.\\
Είσοδος :
\begin{center}{\en\tt curl([x*z,-y\verb|^|2,2*x\verb|^|y],[x,y,z])}\end{center}
Έξοδος :
\begin{center}{\en\tt [2*x\verb|^|y*log(x),x-2*y*x\verb|^|(y-1),0]}\end{center}

\subsection{Δυναμικό : {\tt\textlatin{ potential}}}\index{potential}
\noindent{{\en\tt potential} παίρνει δύο ορίσματα : ένα διανυσματικό πεδίο
$\overrightarrow V$ στον $R^n$  με $n$ πραγματικές μεταβλητές
και το διάνυσμα με τα ονόματα αυτών των μετβλητών.\\
{\en\tt  potential} επιστρέφει, εάν είναι δυνατόν, μια συνάρτηση $U$ τέτοια ώστε 
$\overrightarrow{\mbox{\en \tt grad}}(U)\\ = \overrightarrow V$. Όταν είναι εφικτό,
λέμε ότι $\overrightarrow V$ απορρέει από το δυναμικό $U$, και ο ορισμός του
$U$ περιλαμβάνει και  μία  σταθερά.\\
{\en\tt  potential} είναι η αντίστροφη συνάρτηση του {\en\tt derive}.}\\
Είσοδος :
\begin{center}{\en\tt potential([2*x*y+3,x\verb|^|2-4*z,-4*y],[x,y,z])}\end{center}
Έξοδος :
\begin{center}{\en\tt 2*y*x\verb|^|2/
2+3*x+(x\verb|^|2-4*z-2*x\verb|^|2/2)*y}\end{center}
Σημειώστε ότι, στο $\R^3$, 
ένα διάνυσμα $\overrightarrow V$  είναι κλίση εάν και μόνο εάν ο στροβιλισμός του είναι 0, δηλαδή εάν {\en\tt curl(V)=0}.
Στον χρονικά-ανεξάρτητο ηλεκτρομαγνητισμό, 
$\overrightarrow V$=$\overrightarrow E$ είναι το
ηλεκτρικό πεδίο και  $U$ είναι το ηλεκτρικό δυναμικό.

\subsection{Συντηρητικό πεδίο ροής : {\tt\textlatin{ vpotential}}}\index{vpotential}
\noindent{{\en\tt  vpotential} παίρνει δύο ορίσματα : ένα διανυσματικό πεδίο
$\overrightarrow V$ 
στον $R^n$  με  $n$ πραγματικές μεταβλητές 
και το διάνυσμα με τα ονόματα αυτών των μεταβλητών.\\
{\en\tt  vpotential} επιστρέφει, εάν είναι δυνατόν, ένα διάνυσμα $\overrightarrow U$ τέτοιο
ώστε $\overrightarrow{\mbox{\en\tt curl}}(\overrightarrow U)\\=\overrightarrow V$.
Όταν είναι εφικτό, λέμε ότι το $\overrightarrow V$ είναι ένα συντηρητικό  
πεδίο ροής ή ένα σωληνοειδές πεδίο.
Η γενική λύση είναι το άθροισμα μιας συγκεκριμένης λύσης και  της κλίσης 
 μιας αυθαίρετης συνάρτησης. Το {\en\tt Xcas} επιστρέφει μια συγκεκριμένη
λύση με 0 σαν πρώτο στοιχείο.\\ 
{\en\tt  vpotential} είναι η  αντίστροφη συνάρτηση της {\en\tt curl}.}\\
Είσοδος :
\begin{center}{\en\tt vpotential([2*x*y+3,x\verb|^|2-4*z,-2*y*z],[x,y,z]) }\end{center}
Έξοδος:~
\begin{center}{\en\tt [0,(-(2*y))*z*x,-x\verb|^|3/3-(-(4*z))*x+3*y]}\end{center}
Στον $\R^3$, ένα διανυσματικό πεδίο $\overrightarrow V$ έχει στροβιλισμό   
εάν και μόνο εάν η απόκλισή του είναι 0 ({\en\tt divergence(V,[x,y,z])=0}).
Στον χρονικά-ανεξάρτητο ηλεκτρομαγνητισμό, το
$\overrightarrow V$= $\overrightarrow B$ είναι το μαγνητικό πεδίο και το
$\overrightarrow U$= $\overrightarrow A$ είναι το διάνυσμα δυναμικού.

\section{Εξισώσεις}
\subsection{Ορισμός μιας εξίσωσης : {\tt\textlatin{ equal}}}\index{equal}
\noindent{{\en\tt equal} παίρνει σαν όρισμα τα δύο μέλη μιας εξίσωσης.}\\
{\en\tt equal} επιστρέφει αυτήn την εξίσωση. Αυτή είναι η προθηματική έκδοση του {\en\tt =}\\
Είσοδος :
\begin{center}{\en\tt equal(2x-1,3)}\end{center}
Έξοδος :
\begin{center}{\en\tt (2*x-1)=3}\end{center}
Μπορούμε επίσης να γράψουμε απ'ευθείας {\en\tt (2*x-1)=3}.

\subsection{Μετασχηματισμός εξίσωσης σε διαφορά : \\{\tt\textlatin{ equal2diff}}}\index{equal2diff}
\noindent{{\en\tt equal2diff} παίρνει σαν όρισμα μια εξίσωση.\\
{\en\tt equal2diff} επιστρέφει την διαφορά των δύο μελών της εξίσωσης.}\\
Είσοδος :
\begin{center}{\en\tt equal2diff(2x-1=3)}\end{center}
Έξοδος :
\begin{center}{\en\tt 2*x-1-3}\end{center}

\subsection{Μετασχηματισμός εξίσωσης σε λίστα : \\{\tt\textlatin{ equal2list}}}\index{equal2list}
\noindent{{\en\tt equal2list} παίρνει σαν όρισμα μια εξίσωση.\\
{\en\tt equal2list} επιστρέφει την λίστα των δύο μελών της εξίσωσης.}\\
Είσοδος:
\begin{center}{\en\tt equal2list(2x-1=3)}\end{center}
Έξοδος :
\begin{center}{\en\tt [2*x-1,3]}\end{center}

\subsection{Το αριστερό μέλος μιας εξίσωσης : {\tt\textlatin{left  gauche lhs}}}\index{left|textbf}\index{lhs|textbf}\index{gauche|textbf}
\noindent{{\en\tt left} ή {\en\tt lhs} ή {\en\tt gauche} παίρνει σαν όρισμα μια εξίσωση ή 
ένα διάστημα.\\
{\en\tt left}  ή {\en\tt lhs} ή {\en\tt gauche} επιστρέφει το αριστερό μέλος της εξίσωσης ή 
το αριστερό άκρο του διαστήματος.}\\
Είσοδος :
\begin{center}{\en\tt left(2x-1=3)}\end{center}
ή :
\begin{center}{\en\tt lhs(2x-1=3)}\end{center}
Έξοδος :
\begin{center}{\en\tt 2*x-1}\end{center}
Είσοδος :
\begin{center}{\en\tt left(1..3)}\end{center}
ή :
\begin{center}{\en\tt lhs(1..3)}\end{center}
Έξοδος :
\begin{center}{\en\tt 1}\end{center}

\subsection{Το δεξί μέλος μιας εξίσωσης : {\tt\textlatin{ right  droit rhs}}}\index{right|textbf}\index{rhs|textbf} \index{droit|textbf}
\noindent{{\en\tt right} ή {\en\tt rhs} ή {\en\tt droit} παίρνει σαν όρισμα μία εξίσωση ή ένα διάστημα.\\
{\en\tt right}  ή  {\en\tt rhs} ή {\en\tt droit} επιστρέφει το δεξί μέλος της εξίσωσης ή το δεξί
άκρο του διαστήματος.}\\
Είσοδος :
\begin{center}{\en\tt right(2x-1=3)}\end{center}
ή :
\begin{center}{\en\tt rhs(2x-1=3)}\end{center}
Έξοδος :
\begin{center}{\en\tt 3}\end{center}
Είσοδος :
\begin{center}{\en\tt right(1..3)}\end{center}
ή :
\begin{center}{\en\tt rhs(1..3)}\end{center}
Έξοδος :
\begin{center}{\en\tt 3}\end{center}

\subsection{Επίλυση  εξισώσεων: {\tt\textlatin{ solve}}}\index{solve|textbf}
\noindent{{\en\tt solve}} λύνει μια εξίσωση ή ένα σύστημα πολυωνυμικών
εξισώσεων. Παίρνει 2 ορίσματα:
\begin{itemize}
\item Επίλυση εξίσωσης\\
{\en\tt solve} παίρνει σαν ορίσματα   δύο παραστάσεις αριστερά και δεξιά του {\tt$=$} ή μια
παράσταση (το {\tt $=0$} παραλείπεται), και το όνομα μιας μεταβλητής (από προεπιλογή {\en\tt x}).\\
{\en\tt solve}  λύνει αυτήν την εξίσωση.
\item Επίλυση συστήματος πολυωνυμικών εξισώσεων\\
{\en\tt solve} παίρνει σαν ορίσματα δύο διανύσματα : 
ένα διάνυσμα πολυωνυμικών εξισώσεων και ένα 
διάνυσμα με τα ονόματα των μεταβλητών. \\ 
{\en\tt solve} λύνει αυτό το πολυωνυμικό σύστημα εξισώσεων.
\end{itemize}
{\bf Σχόλια}:
\begin{itemize}
\item Στον τρόπο λειτουργίας για πραγματικούς αριθμούς, {\en\tt solve} επιστρέφει μόνο πραγματικές λύσεις. Για να πάρουμε τις 
μιγαδικές λύσεις, αλλάξτε τον τρόπο λειτουργίας για μιγαδικούς αριθμούς, επιλέγοντας
{\en\tt"}{\ttστους μιγαδικούς}{\en\tt"} στις Ρυθμίσεις {\en\tt cas}, ή χρησιμοποιήστε την εντολή {\en\tt cSolve}.
\item
Για τριγωνομετρικές εξισώσεις,  η {\en\tt solve} επιστρέφει από προεπιλογή τις βασικές 
λύσεις. Για να πάρετε όλες τις λύσεις , επιλέξτε {\en\tt"}{\ttόλες\_τριγ\_λύ\-σεις}{\en\tt"} στις Ρυθμίσεις {\en\tt cas}.
\end{itemize}
{\bf Παραδείγματα} :
\begin{itemize}
\item Λύστε την $x^4-1=3$\\
 Είσοδος :
\begin{center}{\en\tt  solve(x\verb|^|4-1=3)}\end{center}
Έξοδος στον τρόπο λειτουργίας για πραγματικούς αριθμούς :
\begin{center}{\en\tt [sqrt(2),-(sqrt(2))]}\end{center}
Έξοδος στον τρόπο λειτουργίας για μιγαδικούς αριθμούς :
\begin{center}{\en\tt [sqrt(2),-(sqrt(2)),(i)*sqrt(2),-((i)*sqrt(2))]}\end{center}
Είσοδος :
\begin{center}{\en\tt  solve(exp(x)=2)}\end{center}
Έξοδος στον τρόπο λειτουργίας για πραγματικούς αριθμούς :
\begin{center}{\en\tt [log(2)]}\end{center}
\item Βρείτε  $x,y$ τέτοια ώστε $x+y=1,x-y=0$\\
 Είσοδος :
\begin{center}{\en\tt  solve([x+y=1,x-y],[x,y])}\end{center}
Έξοδος :
\begin{center}{\en\tt [[1/2,1/2]] }\end{center}
\item Βρείτε $x,y$ τέτοια ώστε $x^2+y=2,x+y^2=2$\\
Είσοδος :
\begin{center}{\en\tt  solve([x\verb|^|2+y=2,x+y\verb|^|2=2],[x,y])}\end{center}
Έξοδος :
\begin{center}{\en\tt [[-2,-2],[1,1],[(-sqrt(5)+1)/2,(1+sqrt(5))/2],}\end{center}
\begin{center}{\en\tt [(sqrt(5)+1)/2,(1-sqrt(5))/2]] }\end{center}
\item Βρείτε $x,y,z$ τέτοια ώστε $x^2-y^2=0,x^2-z^2=0$\\
Είσοδος :
\begin{center}{\en\tt  solve([x\verb|^|2-y\verb|^|2=0,x\verb|^|2-z\verb|^|2=0],[x,y,z])}\end{center}
Έξοδος :
\begin{center}{\en\tt [[x,x,x],[x,-x,-x],[x,-x,x],[x,x,-x]]}\end{center}
\item Λύστε την  $\cos(2*x)=1/2$\\
Είσοδος :
\begin{center}{\en\tt  solve(cos(2*x)=1/2)}\end{center}
'Eξοδος :
\begin{center}{\en\tt [pi/6,(-pi)/6]}\end{center}
Έξοδος έχοντας επιλέξει {\en\tt"}{\ttόλες\_τριγ\_λύ\-σεις}{\en\tt"} στις Ρυθμίσεις {\en\tt cas} :
\begin{center}{\en\tt [(6*pi*n\_0+pi)/6,(6*pi*n\_0-pi)/6]}\end{center}
\item
Βρείτε την τομή μιας ευθείας γραμμής 
(που δίνεται από μια λίστα εξισώσεων) και ενός επιπέδου.\\Για παράδειγμα,
έστω $D$ η ευθεία γραμμή με καρτεσιανές εξισώσεις 
$[y-z=0,z-x=0]$ και έστω $P$ το επίπεδο με εξίσωση $x-1+y+z=0$.
Βρείτε την τομή των $D$ και $P$.\\
Είσοδος :
\begin{center}{\en\tt solve([[y-z=0,z-x=0],x-1+y+z=0],[x,y,z])}\end{center}
Έξοδος :
\begin{center}{\en\tt [[1/3,1/3,1/3]]}\end{center}
\end{itemize}

\subsection{Επίλυση εξισώσεων στο $\mathbb C$ : {\tt\textlatin{ cSolve}}}\index{cSolve}
\noindent{Η {\en\tt cSolve} παίρνει δύο ορίσματα και λύνει μια εξίσωση ή ένα σύστημα 
πολυωνυμικών εξισώσεων.}
\begin{itemize}
\item Επίλυση εξίσωσης\\
{\en\tt cSolve} παίρνει σαν ορίσματα   δύο παραστάσεις αριστερά και δεξιά του {\tt$=$} ή μια
παράσταση (το {\tt $=0$} παραλείπεται), και το όνομα μιας μεταβλητής (από προεπιλογή {\en\tt x}).\\
{\en\tt cSolve} λύνει την εξίσωση στο $\mathbb C$ ακόμα κι αν είστε στον τρόπο λειτουργίας για πραγματικούς αριθμούς.
\item Επίλυση συστήματος πολυωνυμικών εξισώσεων\\
{\en\tt cSolve} παίρνει ως ορίσματα δύο διανύσματα : ένα διάνυσμα πολυωνυμικών εξισώσεων
και ένα διάνυσμα με τα ονόματα μεταβλητών. \\
{\en\tt cSolve} λύνει αυτό το σύστημα εξισώσεων στο $\mathbb C$  ακόμα κι αν είστε στον τρόπο λειτουργίας για πραγματικούς αριθμούς.
\end{itemize}
Είσοδος :
\begin{center}{\en\tt  cSolve(x\verb|^|4-1=3)}\end{center}
Έξοδος :
\begin{center}{\en\tt [sqrt(2),-(sqrt(2)),(i)*sqrt(2),-((i)*sqrt(2))]}\end{center}
Είσοδος :
\begin{center}{\en\tt  cSolve([-x\verb|^|2+y=2,x\verb|^|2+y],[x,y])}\end{center}
Έξοδος :
\begin{center}{\en\tt [[i,1],[-i,1]]}\end{center}



\section{Γραμμικά συστήματα}
Σε αυτή την ενότητα, αποκαλούμε {\en\tt"}επαυξημένο πίνακα{\en\tt"} του συστήματος
$A \cdot X\\=B$ (ή πίνακα που {\en\tt"}αντιπροσωπεύει{\en\tt"} το σύστημα $A \cdot X=B$),
τον πίνακας που παίρνουμε προσκολλώντας το στηλο-διάνυσμα $B$ ή $-B$
στα δεξιά του πίνακα $A$, όπως με την εντολή {\en\tt border(A,tran(B))}.
   
\subsection{Πίνακας συστήματος: {\tt\textlatin{ syst2mat}}}\index{syst2mat}
\noindent{{\en\tt syst2mat} παίρνει δύο διανύσματα σαν όρισμα. Τα στοιχεία του 
πρώτου διανύσατος είναι οι εξισώσεις ενός γραμμικού συστήματος και τα στοιχεία του δεύτερου διανύσματος
είναι  ονόματα μεταβλητών.\\
{\en\tt syst2mat} επιστρέφει τον επαυξημένο πίνακα του συστήματος  $AX=B$,
που παίρνουμε προσκολλώντας το στηλο-διάνυσμα $-B$
στα δεξιά του πίνακα $A$.}\\
Είσοδος :
\begin{center}{\en\tt syst2mat([x+y,x-y-2],[x,y])}\end{center}
Έξοδος :
\begin{center}{\en\tt [[1,1,0],[1,-1,-2]]}\end{center}
Είσοδος :
\begin{center}{\en\tt syst2mat([x+y=0,x-y=2],[x,y])}\end{center}
έξοδος :
\begin{center}{\en\tt [[1,1,0],[1,-1,-2]]}\end{center}
{\bf Προσοσχή !!!}\\
Στις μεταβλητές (εδώ {\en\tt x} και {\en\tt y}) δεν πρέπει να έχει γίνει απόδοση τιμών. Σε αντίθετη περίπτωση χρησιμοποιείστε την εντολή {\en\tt purge}.

\subsection{Αναγωγή   {\tt\textlatin{Gauss}} : {\tt\textlatin{ ref}}}\index{ref}\label{ref} \label{sec:ref}
\noindent{{\en\tt ref} χρησιμοποιείται για να λύσουμε ένα γραμμικό σύστημα εξισώσεων που γράφεται 
σε μορφή πίνακα:
 \begin{center}{\en\tt A*X=B}\end{center}
Το όρισμα της {\en\tt ref} είαι ο επαυξημένος πίνακας του συστήματος
(ο πίνακας που προκύπτει, αυξάνοντας τον πίνακα {\en\tt A} στα δεξιά με το
στήλο-διάνυσμα {\en\tt B}).}\\
Το αποτέλεσμα είναι ένας πίνακας {\en\tt [A1,B1]} όπου ο {\en\tt A1} έχει 0
κάτω από την κύρια διαγώνιο, και οι λύσεις του συστήματος :
\begin{center}{\en\tt A1*X=B1}\end{center} 
είναι ίδιες με τις λύσεις του συστήματος :
\begin{center}{\en\tt A*X=B}\end{center}

Για παράδειγμα, λύστε το σύστημα :
\[ \left \{
\begin{array}{lcr} 3x + y & = &-2 \\3x +2y & =& 2 \end{array}\right.
\] 
Είσοδος  :
\begin{center}{\en\tt ref([[3,1,-2],[3,2,2]])}\end{center}
Έξοδος :
\begin{center}{\en\tt [[1,1/3,-2/3],[0,1,4]]}\end{center}
Επομένως, η λύση είναι $y=4$ (τελευταία γραμμή) και $x=-2$ (αντικαθιστούμε το $y$ 
στην 1η γραμμή).

\subsection{Αναγωγή {\tt\textlatin{Gauss-Jordan}} : {\tt\textlatin{ rref gaussjord}}}\index{rref|textbf}\index{gaussjord|textbf}\label{sec:rref}
\noindent{\en\tt rref} λύνει ένα γραμμικό σύστημα εξισώσεων γραμμένων σε 
μορφή πίνακα (δείτε επίσης \ref{sec:rrefm}) :
 \begin{center}{\en\tt A*X=B}\end{center}
{\en\tt rref} παίρνει ένα ή 2 ορίσματα.
\begin{itemize}
\item
Εάν {\en\tt rref}  έχει μόνο ένα όρισμα, αυτό το όρισμα είναι ο επαυξημένος πίνακας
του συστήματος (ο πίνακας που παίρνουμε αυξάνοντας τον πίνακα {\en\tt A} στα  
δεξιά με το στήλο-διάνυσμα {\en\tt B}).\\
Το αποτέλεσμα είναι ένας πίνακας {\en\tt [A1,B1]} : Ο {\en\tt A1} έχει μηδενικά και πάνω και κάτω από την 
κύρια διαγώνιο και έχει 1 στην κύρια διαγώνιο, και οι λύσεις 
του συστήματος :
\begin{center}{\en\tt A1*X=B1}\end{center} 
είναι ίδιες με τις λύσεις του συστήματος :
\begin{center}{\en\tt A*X=B}\end{center}
Για παράδειγμα, για να λύσουμε το σύστημα:
\[
\left \{
\begin{array}{lcr} 3x + y & = &-2 \\3x +2y & =& 2 \end{array}\right.
\] 
Είσοδος :
\begin{center}{\en\tt rref([[3,1,-2],[3,2,2]])}\end{center}
Έξοδος :
\begin{center}{\en\tt [[1,0,-2],[0,1,4]]}\end{center}
έτσι,  $x=-2$ και $y=4$ είναι η λύση του συστήματος.

\noindent{{\en\tt rref} μπορεί επίσης να λύσει διάφορα γραμικά συστήματα
εξισώσεων που έχουν το ίδιο πρώτο μέρος, (δηλαδή τον ίδιο πίνακα $A$).
Γράφουμε τα δεύτερα μέλη (δηλαδή τις στήλες $B$) σαν έναν στηλο-πίνακα.  Για παράδειγμα :}\\ 
Είσοδος  :
\begin{center}{\en\tt rref([[3,1,-2,1],[3,2,2,2]])}\end{center}
Έξοδος  :
\begin{center}{\en\tt [[1,0,-2,0],[0,1,4,1]]}\end{center}
Το οποίο σημαίνει ότι $x=-2$ και $y=4$ είναι η λύση του συστήματος
$$\left \{
\begin{array}{lcr} 3x + y & = &-2 \\3x +2y & =& 2 \end{array}\right.$$
ενώ $x=0$ και $y=1$ είναι η λύση του συστήματος
$$\left \{
\begin{array}{lcr} 3x + y & = &1 \\3x +2y & =& 2 \end{array}\right.$$
\item
Εάν η {\en\tt rref}  έχει δύο παραμέτρους, η δεύτερη παράμετρος πρέπει να είναι ένας ακέραιος
$k$, και η αναγωγή {\en\tt Gauss-Jordan} θα γίνει στις 
πρώτες $k$ στήλες (το πολύ).\\
Είσοδος  :
\begin{center}{\en\tt rref([[3,1,-2,1],[3,2,2,2]],1)}\end{center}
Έξοδος  :
\begin{center}{\en\tt [[3,1,-2,1],[0,1,4,1]]}\end{center}
\end{itemize}

\subsection{Επίλυση συστημάτος (-των) {\tt\textlatin{A*X=B}} : {\tt\textlatin{ simult}}}\index{simult}
\noindent{\en\tt simult} χρησιμοποιείται για την επίλυση ενός γραμμικού συστήματος εξισώσεων (αντιστ. 
διαφόρων γραμμικών συστημάτων εξισώσεων με τον ίδιο πίνακα {\en\tt A}) γραμμένο σε
μορφή πίνακα (δείτε επίσης \ref{sec:rrefm}) :
\begin{center}{\en\tt A*X=b  (\mbox{\gr αντιστ.} A*X=B)}\end{center}
{\en\tt simult} παίρνει σαν ορίσματα τον πίνακα {\en\tt A} του συστήματος και το
στηλο-διάνυσμα (δηλαδή τον πίνακα μιας στήλης) {\en\tt b} του δεύτερου
μέλους του συστήματος (αντιστ.
τον πίνακα  {\en\tt B} του οποίου οι στήλες είναι τα
διανύσματα {\en\tt b} των δεύτερων μελών των διαφόρων συστημάτων).\\
Το αποτέλεσμα είναι ένα στηλο-διάνυσμα, λύση του συστήματος (αντιστ. ένας πίνακας
του οποίου οι στήλες είναι οι λύσεις των διαφόρων συστημάτων).\\
Για παράδειγμα για να λύσετε το σύστημα :
$$\left \{
\begin{array}{lcr} 3x + y & = &-2 \\3x +2y & =& 2 \end{array}\right.$$ 
εισάγετε  :
\begin{center}{\en\tt simult([[3,1],[3,2]],[[-2],[2]])}\end{center}
Έξοδος  :
\begin{center}{\en\tt [[-2],[4]]}\end{center}
Έτσι,  $x=-2$ και $y=4$ είναι η λύση.\\
Είσοδος  :
\begin{center}{\en\tt simult([[3,1],[3,2]],[[-2,1],[2,2]])}\end{center}
Έξοδος :
\begin{center}{\en\tt [[-2,0],[4,1]]}\end{center}
Επομένως, $x=-2$ και $y=4$ είναι η λύση του συστήματος :
$$\left \{
\begin{array}{lcr} 3x + y & = &-2 \\3x +2y & =& 2 \end{array}\right.$$
ενώ $x=0$ και  $y=1$ είναι η λύση του συστήματος :
$$\left \{
\begin{array}{lcr} 3x + y & = &1 \\3x +2y & =& 2 \end{array}\right.$$

\subsection{Βήμα-βήμα αναγωγή {\tt\textlatin{Gauss-Jordan}}  : {\tt\textlatin{ pivot}}}\index{pivot}\label{sec:pivot}
\noindent{{\en\tt pivot} παίρνει τρία ορίσματα : έναν πίνακα με $n$ γραμμές και $p$ 
στήλες και δύο ακέραιους $r$ και $c$ τέτοιους ώστε $0\leq r<n$, $0\leq c<p$
και $A_{r,c}\neq 0$.\\
{\en\tt pivot(A,r,c)} εκτελεί ένα βήμα της μεθόδου {\en\tt Gauss-Jordan} 
χρησιμοποιώντας για οδηγό το στοιχείο  {\en\tt A[r,c]}  και επιστρέφει έναν ισοδύναμο πίνακα
με μηδενικά στη στήλη {\en\tt c} του {\en\tt A} (εκτός από τη γραμμή $r$).}\\
Είσοδος  :
\begin{center}{\en\tt pivot([[1,2],[3,4],[5,6]],1,1)}\end{center}
Έξοδος  :
\begin{center}{\en\tt [[-2,0],[3,4],[2,0]]}\end{center}
Είσοδος :
\begin{center}{\en\tt pivot([[1,2],[3,4],[5,6]],0,1)}\end{center}
έξοδος  :
\begin{center}{\en\tt [[1,2],[2,0],[4,0]]}\end{center}

\subsection{Επίλυση  γραμμικού  συστήματος : {\tt\textlatin{ linsolve}}}\index{linsolve}
\noindent{{\en\tt linsolve} χρησιμοποιείται για την επίλυση συστήματος γραμμικών εξισώσεων.\\
{\en\tt linsolve} έχει 2 ορίσματα: μια λίστα εξισώσεων ή 
εκφράσεων (μία εξίσωση
είναι : $\mbox{\gr\tt παράσταση = 0}$), και μια λίστα με ονόματα μεταβλητών.\\
{\en\tt linsolve} επιστρέφει την λύση του συστήματος σε μια λίστα.}\\
Είσοδος :
\begin{center}{\en\tt linsolve([2*x+y+z=1,x+y+2*z=1,x+2*y+z=4],[x,y,z])}\end{center}
Έξοδος  :
\begin{center}{\en\tt  [1/-2,\quad 5/2,\quad 1/-2]}\end{center} 
Το οποίο σημαίνει ότι
\[ x=-\frac{1}{2}, \quad y=\frac{5}{2}, \quad z=-\frac{1}{2} \]
είναι η λύση του συστήματος :
$$\left\{
\begin{array}{rl}
2x+y+z &=1\\
x+y+2z &=1\\
x+2y+z &=4
\end{array}
\right.$$ 

\subsection{Εύρεση γραμμικών αναδρομών : {\tt\textlatin{ reverse\_rsolve}}}\index{reverse\_rsolve}
\noindent{\en\tt reverse\_rsolve} παίρνει σαν όρισμα ένα διάνυσμα 
$v=[v_0,\dots, v_{2n-1}]$  των πρώτων  $2n$ όρων μιας ακολουθίας $(v_n)$
που υποτίθεται ότι ικανοποιεί μια γραμμική αναδρομική σχέση
βαθμού το πολύ $n$
\[ x_n*v_{n+k}+\cdots+x_0*v_k=0 \]
όπου τα $x_j$ είναι  $n+1$ άγνωστα.\\
{\en\tt reverse\_rsolve} επιστρέφει την λίστα $x=[x_n,\dots,x_0]$
των $x_j$ συντελεστών (εάν το $x_n\neq 0$ ανάγεται στο 1).

Με άλλα λόγια {\en\tt reverse\_rsolve} λύνει το γραμμικό σύστημα των 
 $n$ εξι\-σώ\-σεων:
\begin{eqnarray*}
x_n*v_{n}+\cdots+x_0*v_0 &=&0 \\
&\vdots&\\
x_n*v_{n+k}+\cdots+x_0*v_k &=&0 \\
&\vdots&\\
x_n*v_{2*n-1}+\cdots+x_0*v_{n-1}&=&0
\end{eqnarray*}
Ο πίνακας $A$ του συστήματος έχει $n$ γραμμές και $n+1$ στήλες :
\[ A=[[v_0,v_1,\dots,v_n],[v_1,v_2,\dots,v_{n-1}],\dots,[v_{n-1},v_n,\dots,v_{2n-1}]] \]
{\en\tt reverse\_rsolve} επιστρέφει την λίστα $x=[x_n,\dots,x_1,x_0]$ με $x_n=1$
και  $x$  λύση του συστήματος $A*{\en\tt revlist}(x)$, δηλαδή $A$ επί την ανεστραμένη λίστα $x$.

{\bf Παραδείγματα}
\begin{itemize}
\item Βρείτε μια ακολουθία που ικανοποιεί μια γραμμική αναδρομή βαθμού το πολύ
 2 και της οποίας τα πρώτα στοιχεία είναι 1, -1, 3, 3.\\
Είσοδος :
\begin{center}{\en\tt reverse\_rsolve([1,-1,3,3])}\end{center}
Έξοδος  :
\begin{center}{\en\tt  [1,-3,-6]}\end{center} 
Έτσι, $x_0=-6$, $x_1=-3$, $x_2=1$ και η αναδρομική σχέση είναι
 \[ v_{k+2} -3v_{k+1} -6 v_k =0\]
Χωρίς την εντολή {\en\tt reverse\_rsolve}, θα γράφαμε τον πίνακα του συστήματος:
{\en\tt [[1,-1,3],[-1,3,3]]} και θα χρησιμοποιούσαμε την εντολή {\en\tt rref}:\\
{\en\tt rref([[1,-1,3],[-1,3,3]]).}\\
Η έξοδος είναι ο πίνακας {\en\tt [[1,0,6],[0,1,3]]} και επομένως  $x_0=-6$ και  $x_1=-3$ 
(επειδή $x_2=1$).

\item Βρείτε μια ακολουθία που επαληθεύει μια γραμμική αναδρομή βαθμού το πολύ
 3 και της οποίας τα πρώτα στοιχεία είναι 1, -1, 3, 3,-1, 1.\\
Είσοδος :
\begin{center}{\en\tt reverse\_rsolve([1,-1,3,3,-1,1])}\end{center}
Έξοδος  :
\begin{center}{\en\tt [1,(-1)/2,1/2,-1]}\end{center} 
Επομένως, $x_0=-1$, $x_1=1/2$, $x_2=-1/2$, $x_3=1$, και η  αναδρομική
σχέση είναι
\[ v_{k+3} -\frac{1}{2} v_{k+2} +\frac{1}{2} v_{k+1} -v_k =0 \]
Χωρίς την {\en\tt reverse\_rsolve}, θα γράφαμε τον πίνακα του συστήματος :\\
{\tt [[1,-1,3,3],[-1,3,3,-1],[3,3,-1,1]]} και θα χρησιμοποιούσαμε την εντολή {\en\tt rref}  :\\
{\en\tt rref([[1,-1,3,3],[-1,3,3,-1],[3,3,-1,1]])}\\
Η έξοδος είναι ο πίνακας {\en\tt [1,0,0,1],[0,1,0,1/-2],[0,0,1,1/2]]}
και επομένως $x_0=-1$, $x_1=1/2$ και $x_2=-1/2$ (επειδή $x_3=1$).
\end{itemize}




\section{Διαφορικές εξιώσεις}
Η ενότητα αυτή περιορίζεται στις συμβολικές (ή ακριβείς) λύσεις των
διαφορικών εξισώσεων.
Για αριθμητικές  λύσεις διαφορικών εξισώσεων, δείτε την {\en\tt odesolve}.
Για γραφική αναπαράσταση των λύσεων των διαφορικών εξισώσεων, 
δείτε  {\en\tt plotfield}, {\en\tt plotode} και {\en\tt interactive\_plotode}. 

\subsection{Επίλυση διαφορικών εξισώσεων : {\tt\textlatin{ desolve deSolve \\
dsolve}}}\index{desolve}\index{deSolve}\index{dsolve}
{\en\tt desolve} (ή {\en\tt deSolve}) μπορεί να λύσει:
\begin{itemize}
\item γραμμικές διαφορικές εξισώσεις με σταθερούς συντελεστές,
\item γραμμικές διαφορικές εξισώσεις πρώτης τάξης,
\item πρώτης τάξης διαφορικές εξισώσεις χωρίς $y$,
\item πρώτης τάξης διαφορικές εξισώσεις χωρίς $x$,
\item πρώτης τάξης διαφορικές εξισώσεις με χωρισμένες μεταβλητές,
\item πρώτης τάξης ομογενείς διαφορικές εξισώσεις ($y'=F(y/x)$),
\item πρώτης τάξης διαφορικές εξισώσεις με ολοκληρωτικό παράγοντα,
\item πρώτης τάξης διαφορικές εξισώσεις {\en\tt Bernoulli} ($a(x)y'+b(x)y=c(x)y^n$),
\item πρώτης τάξης διαφορικές εξισώσεις {\en\tt Clairaut} ($y=x*y'+f(y')$).
\end{itemize}
{\en\tt desolve} παίρνει σαν ορίσματα : 
\begin{itemize}
\item  εάν η ανεξάρτητη μεταβλητή είναι η τρέχουσα μεταβλητή (εδώ υποτίθεται 
ότι είναι το $x$), 
\begin{itemize}
\item την διαφορική εξίσωση (ή την λίστα της
διαφορικής εξίσωσης και των αρχικών συνθηκών) 
\item την άγνωστο (συνήθως {\en\tt y}).
\end{itemize}
Στη διαφορική εξίσωση, η συνάρτηση $y$ παρίσταται ως  $y$, 
η πρώτη της παράγωγος  $y'$  παρίσταται ως  
${\en\tt y'}$, και η δεύτερη παράγωγός της $y'{'}$ γράφεται ως
${\en\tt y''}$.\\
Για παράδειγμα  {\en\tt desolve(y$''$+2*y$'$+y,y)} ή \\
{\en\tt desolve([y$''$+2*y$'$+y,y(0)=1,y$'$(0)=0],y)}.
\item εάν η ανεξάρτητη μεταβλητή δεν είναι η τρέχουσα μεταβλητή, 
για παράδειγμα $t$ αντί για $x$, 
\begin{itemize}
\item την διαφορική εξίσωση (ή την λίστα της 
διαφορικής εξίσωσης και των αρχικών συνθηκών), 
\item  την μεταβλητή, π.χ. {\en\tt t} 
\item την άγνωστο σαν μεταβλητή {\en\tt y} ή σαν συνάρτηση {\en\tt y(t)}.
\end{itemize}
Στη διαφορική εξίσωση, η συνάρτηση $y$ παρίσταται ως  $y(t)$,
η πρώτη της παράγωγος $y'$  παρίσταται ως {\en\tt diff(y(t),t)}, και η δεύτερη παράγωγός της
$y'{'}$  γράφεται ως {\en\tt diff(y(t),t\$2)}.\\ 
Για παράδειγμα : \\
{\en\tt desolve(diff(y(t),t\$2)+2*diff(y(t),t)+y(t),y(t))}, ή\\
{\en\tt desolve(diff(y(t),t\$2)+2*diff(y(t),t)+y(t),t,y)}
και \\

{\en\tt \begin{verbatim}
desolve([diff(y(t),t$2)+2*diff(y(t),t)+y(t),
         y(0)=1,y'(0)=0],y(t)), or
desolve([diff(y(t),t$2)+2*diff(y(t),t)+y(t), 
         y(0)=1,y'(0)=0],t,y)
\end{verbatim}}
\end{itemize}
Εάν δεν υπάρχουν αρχικές συνθήκες (ή μια αρχική συνθήκη για μια εξίσωση
δευτέρας τάξης), {\en\tt desolve} επιστρέφει την γενική λύση ως προς τις  
σταθερές ολοκλήρωσης
{\en\tt c\_0, c\_1}, όπου {\en\tt y(0)=c\_0} και {\en\tt y$'$(0)=c\_1},
ή μια λίστα λύσεων.\\
{\bf Παραδείγματα}
\begin{itemize}
\item Γραμμικές διαφορικές εξισώσεις δευτέρας τάξης με σταθερούς 
συντελεστές.
\begin{enumerate}
\item 
Λύστε :
$$y''+y=\cos (x) $$
Εισάγετε (πληκτρολογώντας δύο φορές $'$ για  {\en\tt y$''$}): 
\begin{center}{\en\tt desolve(y$''$+y=cos(x),y)}\end{center}
ή εισάγετε :
\begin{center}{\en\tt desolve((diff( diff(y))+y)=(cos(x)),y)}\end{center}
Έξοδος :
\begin{center}{\en\tt  c\_0*cos(x)+(x+2*c\_1)*sin(x)/2}\end{center}
{\en\tt c\_0, c\_1} είναι οι σταθερές της ολοκλήρωσης: {\en\tt y(0)=c\_0} και
{\en\tt y$'$(0)=c\_1}.\\ 
Εάν η μεταβλητή δεν είναι {\en\tt x} αλλά {\en\tt t}, εισάγετε : 
\begin{center}
{\en\tt desolve(derive(derive(y(t),t),t)+y(t)=cos(t),t,y)}
\end{center}
Έξοδος :
\begin{center}{\en\tt  c\_0*cos(t)+(t+2*c\_1)/2*sin(t)}\end{center}
{\en\tt c\_0, c\_1} είναι οι σταθερές της ολοκλήρωσης : {\en\tt y(0)=c\_0} και
{\en\tt y$'$(0)=c\_1}.
\item
Λύστε :
$$y''+y=\cos (x), \; \; y(0)=1 $$
Είσοδος :
\begin{center}{\en\tt desolve([y$''$+y=cos(x),y(0)=1],y)}\end{center}
Έξοδος   :
\begin{center}{\en\tt [cos(x)+(x+2*c\_1)/2*sin(x)]}\end{center}
τα στοιχεία του διανύσματος είναι λύσεις (εδώ υπάρχει μόνο ένα στοιχείο, 
έτσι έχουμε ακριβώς μία λύση που εξαρτάται από το {\en\tt c\_1}).
\item
Λύστε :
$$y''+y=\cos (x) \; \; (y(0))^2=1 $$
Είσοδος :
\begin{center}{\en\tt desolve([y$''$+y=cos(x),y(0)\verb|^|2=1],y)}\end{center}
Έξοδος:
\begin{center}{\en\tt [-cos(x)+(x+2*c\_1)/2*sin(x),cos(x)+(x+2*c\_1)/2*sin(x)]}\end{center}
κάθε στοιχείο αυτής της λίστας είναι μια λύση, 
έχουμε δύο λύσεις που εξαρτώνται από
τη σταθερά {\en\tt c\_1} ($y'(0)=c_1$)
και που αντιστοιχούν σε $y(0)=1$ και σε $y(0)=-1$.
\item
Λύστε :
$$y''+y=\cos (x), \; \; (y(0))^2=1 \; \; y'(0)=1$$
Είσοδος:
\begin{center}{\en\tt desolve([y$''$+y=cos(x),y(0)\verb|^|2=1,y'(0)=1],y)}
\end{center}
Έξοδος :
\begin{center}{\en\tt [-cos(x)+(x+2)/2*sin(x),cos(x)+(x+2)/2*sin(x)]}\end{center}
κάθε στοιχείο αυτής της λίστας είναι μια λύση (έχουμε δύο λύσεις).
\item
Λύστε :
$$y''+2y'+y=0$$
Είσοδος :
\begin{center}{\en\tt desolve(y$''$+2*y$'$+y=0,y)}\end{center}
Έξοδος :
\begin{center}{\en\tt (x*c\_0+x*c\_1+c\_0)*exp(-x)}\end{center}
Η λύση εξαρτάται από δύο σταθερές ολοκλήρωσης : 
{\en\tt c\_0, c\_1} ({\en\tt y(0)=c\_0} και {\en\tt y$'$(0)=c\_1}).
\item
Λύστε:
$$y''-6y'+9y=xe^{3x}$$
Είσοδος:
\begin{center}{\en\tt desolve(y$''$-6*y$'$+9*y=(x*exp(3*x),y)}\end{center}
Έξοδος :
\begin{center}{\en\tt (x\verb|^|3+(-(18*x))*c\_0+6*x*c\_1+6*c\_0)*1/6*exp(3*x)}\end{center}
η λύση εξαρτάται από δύο σταθερές ολοκλήρωσης : 
{\en\tt c\_0, c\_1} ({\en\tt y(0)=c\_0} και {\en\tt y$'$(0)=c\_1}).
\end{enumerate}
\item Γραμμικές διαφορικές εξισώσεις πρώτης τάξης.
\begin{enumerate}
\item 
Λύστε :
$$xy'+y-3x^2=0$$
Είσοδος :
{\en\tt \begin{center}{ desolve(x*y$'$+y-3*x\verb|^|2,y)}\end{center}}
Έξοδος :
\begin{center}{\en\tt(3*1/3*x\verb|^|3+c\_0)/x }\end{center}
\item
Λύστε :
$$y'+x*y=0, y(0)=1$$
Είσοδος :
\begin{center}{\en\tt desolve([y$'$+x*y=0, y(0)=1]),y)}\end{center}
ή :
\begin{center}{\en\tt desolve((y$'$+x*y=0) \&\& (y(0)=1),y)}\end{center}
Έξοδος  :
\begin{center}{\en\tt [1/(exp(1/2*x\verb|^|2))]}\end{center} 
\item
Λύστε :
$$x(x^2-1)y'+2y=0$$
Είσοδος :
\begin{center}{\en\tt desolve(x*(x\verb|^|2-1)*y$'$+2*y=0,y)}\end{center}
Έξοδος  :
\begin{center}{\en\tt (c\_0)/((x\verb|^|2-1)/(x\verb|^|2))}\end{center}
\item
Λύστε :
$$x(x^2-1)y'+2y=x^2$$
Είσοδος:
\begin{center}{\en\tt desolve(x*(x\verb|^|2-1)*y$'$+2*y=x\verb|^|2,y)}\end{center}
Έξοδος  :
\begin{center}{\en\tt (ln(x)+c\_0)/((x\verb|^|2-1)/(x\verb|^|2))}\end{center}
\item
Εάν η μεταβλητή είναι $t$ αντί για $x$, για παράδειγμα  :
$$t(t^2-1)y'(t)+2y(t)=t^2$$
Είσοδος :
\begin{center}{\en\tt desolve(t*(t\verb|^|2-1)*diff(y(t),t)+2*y(t)=(t\verb|^|2),y(t))}\end{center}
Έξοδος :
\begin{center}{\en\tt (ln(t)+c\_0)/((t\verb|^|2-1)/(t\verb|^|2))}\end{center}
\item
Λύστε :
$$x(x^2-1)y'+2y=x^2,y(2)=0$$
Είσοδος :
\begin{center}{\en\tt desolve([x*(x\verb|^|2-1)*y$'$+2*y=x\verb|^|2,y(0)=1],y)}\end{center}
Έξοδος  :
\begin{center}{\en\tt [(ln(x)-ln(2))*1/(x\verb|^|2-1)*x\verb|^|2]}\end{center}
\item
Λύστε  :
$$\sqrt{1+x^2}y'-x-y=\sqrt{1+x^2}$$
Είσοδος :
\begin{center}{\en\tt desolve(y$'$*sqrt(1+x\verb|^|2)-x-y-sqrt(1+x\verb|^|2),y)}\end{center}
Έξοδος  :
\begin{center}{\en\tt (-c\_0+ln(sqrt(x\verb|^|2+1)-x))/(x-sqrt(x\verb|^|2+1))}\end{center}
\end{enumerate}

\item  Διαφορικές εξισώσεις πρώτης τάξης  με χωρισμένες μεταβλητές
\begin{enumerate}
\item Λύστε  :
$$y'=2\sqrt{y}$$
Είσοδος :
\begin{center}{\en\tt desolve(y$'$=2*sqrt(y),y)}\end{center}
Έξοδος  :
\begin{center}{\en\tt [x\verb|^|2+-2*x*c\_0+c\_0\verb|^|2]}\end{center}
\item
Λύστε :
$$xy'\ln(x)-y(3\ln(x)+1)=0$$
Είσοδος :
\begin{center}{\en\tt desolve(x*y$'$*ln(x)-(3*ln(x)+1)*y,y)}\end{center}
Έξοδος :
\begin{center}{\en\tt c\_0*x\verb|^|3*ln(x)}\end{center}
\end{enumerate}

\item Διαφορικές εξισώσεις {\en\tt Bernoulli}
$a(x)y'+b(x)y=c(x)y^n$ όπου $n$ είναι μια πραγματική σταθερά.\\
Η μέθοδος που χρησιμοποιείται είναι να διαιρέσουμε την εξίσωση με $y^n$, 
έτσι ώστε να γίνει μια διαφορική εξίσωση πρώτης τάξης
ως προς $u=1/y^{n-1}$.
\begin{enumerate}
\item 
Λύστε :
$$xy'+2y+xy^2=0$$
Είσοδος :
\begin{center}{\en\tt desolve(x*y$'$+2*y+x*y\verb|^|2,y)}\end{center}
Έξοδος  :
\begin{center}{\en\tt [1/(exp(2*ln(x))*(-1/x+c\_0))]}\end{center}
\item
Λύστε :
$$xy'-2y=xy^3$$
Είσοδος :
\begin{center}{\en\tt desolve(x*y$'$-2*y-x*y\verb|^|3,y)}\end{center}
Έξοδος  :
\begin{center}{\en\tt [((-2*1/5*x\verb|^|5+c\_0)*exp(-(4*log(x))))\verb|^|(1/-2),}\end{center}
\begin{center}{\en\tt -((-2*1/5*x\verb|^|5+c\_0)*exp(-(4*log(x))))\verb|^|(1/-2)]}\end{center}
\item 
Λύστε :
$$x^2y'-2y=xe^{4/x}y^3$$
Είσοδος :
\begin{center}{\en\tt desolve(x*y$'$-2*y-x*exp(4/x)*y\verb|^|3,y)}\end{center}
Έξοδος   :
\begin{center}{\en\tt [((-2*ln(x)+c\_0)*exp(-(4*(-(1/x)))))\verb|^|(1/-2),}\end{center}
\begin{center}{\en\tt -(((-2*ln(x)+c\_0)*exp(-(4*(-(1/x)))))\verb|^|(1/-2))]}\end{center}
\end{enumerate}

\item Ομογενείς διαφορικές εξισώσεεις πρώτης τάξης ($y'=F(y/x)$,
η μέθοδος της ολοκλήρωσης συνίσταται στο να ψάξουμε για $t=y/x$ αντί για $y$).
\begin{enumerate}
\item 
Λύστε :
$$(3x^3y'=y(3x^2-y^2)$$
Είσοδος :
\begin{center}{\en\tt desolve(3*x\verb|^|3*diff(y)=((3*x\verb|^|2-y\verb|^|2)*y),y)}\end{center}
Έξοδος  :
\begin{center}{\en\tt [0,[c\_0*exp((3*1/2)/(`t`\verb|^|2)),\\`t`*c\_0*exp((3*1/2)/(`t`\verb|^|2))]]}\end{center}
έτσι, οι λύσεις είναι $y=0$ και η οικογένεια των καμπυλών 
των παραμετρικών εξισώσεων  $x=c_0\exp(3/(2t^2)), y=t*c_0\exp(3/(2t^2))$ 
(η παράμετρος δηλώνεται με {\en\tt `t`} στην απάντηση).
\item
Λύστε :
$$xy'=y+\sqrt{x^2+y^2}$$
Είσοδος :
\begin{center}{\en\tt desolve(x*y$'$=y+sqrt(x\verb|^|2+y\verb|^|2),y)}\end{center}
Έξοδος  :
\begin{center}{\en\tt [(-i)*x,(i)*x,[c\_0/(sqrt(`t`\verb|^|2+1)-`t`),\\(`t`*c\_0)/(sqrt(`t`\verb|^|2+1)-`t`)]]}\end{center}
άρα οι λύσεις είναι  :
$$y=ix,y=-ix$$
 και η οικογένεια των καμπυλών των παραμετρικών εξισώσεων
$$x=c_0/(\sqrt{t^2+1}-t), y=t*c_0/(\sqrt{t^2+1}-t)$$ 
(η παράμετρος δηλώνεται με {\en\tt `t`} στην απάντηση).
\end{enumerate}


\item Διαφορικές εξισώσεις πρώτης τάξης με ολοκληρωτικό παράγοντα. Πολλαπλασιάζοντας την εξίσωση με μια συνάρτηση του $x,y$,
γίνεται μια κλειστή διαφορική μορφή.
\begin{enumerate}
\item 
Λύστε :
$$yy'+x$$
Είσοδος :
\begin{center}{\en\tt desolve(y*y$'$+x,y)}\end{center}
Έξοδος  :
\begin{center}{\en\tt [sqrt(-2*c\_0-x\verb|^|2),-(sqrt(-2*c\_0-x\verb|^|2))]}\end{center}
Σε αυτό το παράδειγμα, το $xdx+ydy$ είναι κλειστό, ο ολοκληρωτικός παράγοντας ήταν 1.
\item
Λύστε :
$$2xyy'+x^2-y^2+a^2=0$$
Είσοδος :
\begin{center}{\en\tt desolve(2*x*y*y$'$+x\verb|^|2-y\verb|^|2+a\verb|^|2,y)}\end{center}
Έξοδος  :
\begin{center}{\en\tt [sqrt(a\verb|^|2-x\verb|^|2-c\_1*x),-(sqrt(a\verb|^|2-x\verb|^|2-c\_1*x))]}\end{center}
Σε αυτό το παράδειγμα, ο ολοκληρωτικός παράγοντας ήταν $1/x^2$.
\end{enumerate}

\item Διαφορικές εξισώσεις πρώτης τάξης χωρίς το $x$.\\
Λύστε :
$$(y+y')^4+y'+3y=0$$
Αυτό το είδος εξισώσεων δεν μπορεί να λυθεί άμεσα με το {\en\tt Xcas}, εξηγούμε όμως πως
λύνονται με τη βοήθειά του. 
Η ιδέα είναι να βρούμε μια παραμετρική αναπαράσταση του 
$F(u,v)=0$ όπου η εξίσωση είναι $F(y,y')=0$, 
Έστω $u=f(t),v=g(t)$ μια τέτοια παραμετροποίηση του $F=0$, τότε 
$y=f(t)$ και $dy/dx=y'=g(t)$. Άρα,
\[ dy/dt=f'(t)=y'*dx/dt=g(t)*dx/dt \]
Η λύση είναι η καμπύλη των παραμετρικών εξισώσεων
$x(t), y(t)=f(t)$, όπου $x(t)$ είναι λύση της διαφορικής εξίσωσης 
 $g(t)dx=f'(t)dt$.\\
Πίσω στο παράδειγμα, θέτουμε $y+y'=t$, οπότε:
\[ y=-t-8*t^4, \quad y'=dy/dx=3*t+8*t^4 \quad dy/dt=-1-32*t^3
\] 
και επομένως
\[ (3*t+8*t^4)*dx=(-1-32*t^3)dt \]
Είσοδος  :
\begin{center}{\en\tt desolve((3*t+8*t\verb|^|4)*diff(x(t),t)=(-1-32*t\verb|^|3),x(t))}\end{center}
Έξοδος :
\begin{center}{\en\tt -11*1/9*ln(8*t\verb|^|3+3)+1/-9*ln(t\verb|^|3)+c\_0}\end{center}
Η λύση είναι η καμπύλη της παραμετρικής εξίσωσης:
\[ x(t)=-11*1/9*\ln(8*t^3+3)+1/-9*\ln(t^3)+c_0,
\quad y(t)=-t-8*t^4 \]

\item Διαφορικές εξισώσεις
{\en\tt Clairaut} πρώτης τάξης ($y=x*y'+f(y')$).\\
Οι λύσεις είναι οι γραμμές $D_m$ της εξίσωσης $y=mx+f(m)$ όπου
 $m$ είναι μια πραγματική σταθερά.
\begin{enumerate}
\item Λύστε :
$$xy'+y'^3-y)=0$$
Είσοδος  :
\begin{center}{\en\tt desolve(x*y$'$+y$'$\verb|^|3-y),y)}\end{center}
Έξοδος  :
\begin{center}{\en\tt c\_0*x+c\_0\verb|^|3}\end{center}
\item 
Λύστε :
$$y-xy'=\sqrt{a^2+b^2*y'^2}=0$$
Είσοδος  :
\begin{center}{\en\tt desolve((y-x*y$'$-sqrt(a\verb|^|2+b\verb|^|2*y$'$\verb|^|2),y)}\end{center}
Έξοδος  :
\begin{center}{\en\tt c\_0*x+sqrt(a\verb|^|2+b\verb|^|2*c\_0\verb|^|2)}\end{center}
\end{enumerate}
\end{itemize}

\subsection{Μετασχηματισμός {\tt\textlatin{Laplace}} και αντίστροφος μετα\-σχη\-μα\-τι\-σμός {\tt\textlatin{Laplace}} : {\tt\textlatin{ laplace ilaplace}}}\index{laplace}\index{ilaplace}\label{sec:lap}
{\en\tt laplace} (ή {\en\tt ilaplace}) παίρνει ένα, δύο ή τρία ορίσματα :
 μια παράσταση και προαιρετικά το (τα) όνομα (ονόματα) της (των) μεταβλητής (μεταβλητών).\\
Η παράσταση είναι ως προς την τρέχουσα μεταβλητή (όπου $x$) ή ως προς την μεταβλητή που δίνεται σαν δεύτερο όρισμα.\\
{\en\tt laplace} επιστρέφει τον μετασχηματισμό  {\en\tt Laplace}  της παράστασης που δίνεται ως όρισμα
και η {\en\tt ilaplace} τον αντίστροφο μετασχηματισμό {\en\tt Laplace} της παράστασης που δίνεται
ως όρισμα. Το αποτέλσμα της {\en\tt laplace} και της {\en\tt ilaplace} είναι 
ως προς την μεταβλητή που δίνεται  σαν τρίτο όρισμα, εάν αυτό παρέχεται,
ή ως προς το δεύτερο όρισμα εάν αυτό παρέχεται ή  διαφορετικά ως προς $x$.

Ο μετασχηματισμός {\en\tt Laplace} ({\en\tt laplace}) και ο αντίστροφος μετασχηματισμός {\en\tt Laplace} 
({\en\tt ilaplace}) χρησιμεύουν στην επίλυση γραμμικών διαφορικών εξισώσεων
με σταθερούς συντελεστές. Για παράδειγμα :
$$y'' +p. y \prime+q. y \ =\ f(x)$$ $$ y(0)=a, \ y'(0)=b$$
Συμβολίζοντας με ${\mathcal{L}}$ τον μετασχηματισμό {\en\tt Laplace},
οι ακόλουθες σχέσεις ισχύουν :
\begin{eqnarray*}
{\mathcal{L}}(y)(x)&=&\int_0^{+\infty}e^{-x.u}y(u)du \\
{\mathcal{L}}^{-1}(g)(x)&=&\frac{1}{2i\pi}\int_C e^{z.x}g(z)dz
\end{eqnarray*}
όπου $C$ είναι μία κλειστή καμπύλη που περιλαμβάνει τους πόλους του {\en\tt g}.\\
Είσοδος :
\begin{center}{\en\tt laplace(sin(x))}\end{center}
Η παράσταση (εδώ $\sin(x)$) είναι ως προς την τρέχουσα μεταβλητή 
(εδώ $x$) και η απάντηση θα είναι επίσης μια παράσταση ως προς την τρέχουσα μεταβλητή 
$x$.\\
Έξοδος :
\begin{center}{\en\tt 1/((-x)\verb|^|2+1)}\end{center}
Ή εισάγετε :
\begin{center}{\en\tt laplace(sin(t),t)}\end{center}
εδώ το όνομα μεταβλητής είναι $t$ και αυτό το όνομα χρησιμοποιείται επίσης στην απάντηση.\\
Έξοδος :
\begin{center}{\en\tt 1/((-t)\verb|^|2+1)}\end{center}
Ή εισάγετε :
\begin{center}{\en\tt laplace(sin(t),t,s)}\end{center}
εδώ το όνομα της μεταβλητής είναι $t$ και το όνομα της μεταβλητής στην απάντηση είναι $s$.\\
Έξοδος:
\begin{center}{\en\tt 1/((-s)\verb|^|2+1)}\end{center}
Οι ακόλουθες ιδιότητες ισχύουν :
\begin{eqnarray*}
{\mathcal{L}}(y')(x) &=&-y(0)+x.{\mathcal{L}}(y)(x) \\
{\mathcal{L}}(y'')(x) &=&-y'(0)+x.{\mathcal{L}}(y')(x) \\
 &=& -y'(0)-x.y(0)+x^2.{\mathcal{L}}(y)(x)
\end{eqnarray*}
Εάν $y''(x) +p. y'(x)+q. y(x) \ =\ f(x)$ τότε :
\begin{eqnarray*}
{\mathcal{L}}(f)(x) &=&{\mathcal{L}}(y''+p.y'+q.y)(x) \\
&=& -y'(0)-x.y(0)+x^2.{\mathcal{L}}(y)(x)-p.y(0)+p.x.{\mathcal{L}}(y)(x))+q.{\mathcal{L}}(y)(x) \\
&=& (x^2+p.x+q).{\mathcal{L}}(y)(x)-y'(0)-(x+p).y(0)
\end{eqnarray*}
Επομένως, εάν $a=y(0)$ και $b=y'(0)$, έχουμε
$${\mathcal{L}}(f)(x)=(x^2+p.x+q).{\mathcal{L}}(y)(x)-(x+p).a-b$$
και η λύση της διαφορικής εξίσωσης είναι :
\[ y(x)=
{\mathcal{L}}^{-1}(({\mathcal{L}}(f)(x)+(x+p).a +b)/(x^2+p.x+q))
\]
Παράδειγμα :\\
Λύστε :
\[ y'' -6. y'+9. y \ =\ x. e^{3. x},
\quad  y(0)=c\_0, \quad y'(0)=c\_1
\]
Εδώ, $p=-6,\ q=9$.\\
Εισάγετε :
\begin{center}{\en\tt laplace(x*exp(3*x))}\end{center}
Έξοδος :
\begin{center}{\en\tt 1/(x\verb|^| 2-6*x+9)}\end{center}
Είσοδος:
\begin{center}{\en\tt ilaplace((1/(x\verb|^|2-6*x+9)+(x-6)*c\_0+c\_1)/(x\verb|^|2-6*x+9))}\end{center}
Έξοδος :
\begin{center}{\en\tt (216*x\verb|^|3-3888*x*c\_0+1296*x*c\_1+1296*c\_0)*exp(3*x)/1296}\end{center}
Μετά από απλοποίηση και παραγοντοποίηση (με την εντολή {\en\tt factor}) 
η λύση $y$ είναι :
\begin{center}{\en\tt (-18*c\_0*x+6*c\_0+x\verb|^|3+6*x*c\_1)*exp(3*x)/6}\end{center}
Σημειώστε ότι αυτή η εξίσωση θα μπορούσε να λυθεί άμεσα, εισάγοντας :
\begin{center}{\en\tt desolve(y$''$-6*y$'$+9*y=x*exp(3*x),y)}\end{center}
Έξοδος :
\begin{center}{\en\tt exp(3*x)*(-18*c\_0*x+6*c\_0+x\verb|^|3+6*x*c\_1)/6}\end{center}

\section{Άλλες συναρτήσεις}
\subsection{Αντικατάσταση μικρών τιμών με  0 : {\tt\textlatin{epsilon2zero}}}
\index{epsilon2zero} \label{sec:epsilon2zero}
\noindent{{\en\tt epsilon2zero} παίρνει σαν όρισμα μια παράσταση του {\en\tt x}.\\
{\en\tt epsilon2zero} επιστρέφει την παράσταση όπου οι τιμές με μέτρο
μικρότερο του  {\en\tt epsilon} αντικαθίστανται με 0. Η παράσταση
 δεν αποτιμάται.}\\
Η τιμή {\en\tt epsilon}\index{epsilon} ορίζεται στις Ρυθμίσεις {\en\tt cas} (από προεπιλογή {\en\tt epsilon=1e-10}).\\
Είσοδος :
\begin{center}{\en\tt epsilon2zero(1e-13+x) }\end{center}
Έξοδος (με {\en\tt epsilon=1e-10}) :
\begin{center}{\en\tt 0+x}\end{center}
Είσοδος :
\begin{center}{\en\tt epsilon2zero((1e-13+x)*100000) }\end{center}
Έξοδος (με {\en\tt epsilon=1e-10}) :
\begin{center}{\en\tt (0+x)*100000}\end{center}
Είσοδος :
\begin{center}{\en\tt epsilon2zero(0.001+x) }\end{center}
Έξοδος (με {\en\tt epsilon=0.0001}) :
\begin{center}{\en\tt 0.001+x}\end{center}

\subsection{Λίστα μεταβλητών : {\tt\textlatin{ lname indets}}}\index{lname}\index{indets}
\noindent{{\en\tt lname} (ή {\en\tt indets}) παίρνει σαν όρισμα μια παράσταση.\\
{\en\tt lname} (ή {\en\tt indets}) επιστρέφει την λίστα με τα συμβολικά ονόματα των μεταβλητών
που χρησιμοποιούνται σε αυτή την παράσταση.}\\
Είσοδος :
\begin{center}{\en\tt lname(x*y*sin(x))}\end{center}
Έξοδος :
\begin{center}{\en\tt  [x,y]}\end{center}
Είσοδος :
\begin{center}{\en\tt a:=2;assume(b>0);assume(c=3);}\end{center}
\begin{center}{\en\tt lname(a*x\verb|^|2+b*x+c)}\end{center}
Έξοδος :
\begin{center}{\en\tt  [x,b,c]}\end{center}

\subsection{Λίστα μεταβλητών και παραστάσεων : {\tt\textlatin{ lvar}}}\index{lvar}\label{sec:lvar}
\noindent{\en\tt lvar} παίρνει σαν όρισμα μια παράσταση.\\
{\tt lvar}  επιστρέφει μια λίστα με ονόματα μεταβλητών και  μη-ρητών 
παραστάσεων των οποίων το όρισμά τους  είναι ένα ρητό κλάσμα
ως προς τις μεταβλητές και τις παραστάσεις της λίστας.\\
Είσοδος :
\begin{center}{\en\tt lvar(x*y*sin(x)\verb|^|2)}\end{center}
Έξοδος :
\begin{center}{\en\tt [x,y,sin(x)]}\end{center}
Είσοδος :
\begin{center}{\en\tt lvar(x*y*sin(x)\verb|^|2+ln(x)*cos(y))}\end{center}
Έξοδος :
\begin{center}{\en\tt [x,y,sin(x),ln(x),cos(y)]}\end{center}
Είσοδος :
\begin{center}{\en\tt lvar(y+x*sqrt(z)+y*sin(x))}\end{center}
Έξοδος :
\begin{center}{\en\tt [x,y,sqrt(z),sin(x)]}\end{center}

\subsection{Λίστα μεταβλητών αλγεβρικών παραστάσεων : \\{\tt\textlatin{ algvar}}}\index{algvar}
\noindent{{\en\tt algvar} παίρνει σαν όρισμα μια παράσταση.\\ 
{\en\tt algvar} επιστρέφει την λίστα των ονομάτων των συμβολικών μεταβλητών 
που χρησιμοποιούνται στην παράσταση. Η λίστα είναι ταξινομημένη ως προς
τις αλγεβρικές επεκτάσεις που απαιτούνται για την δημιουργία της αρχικής παράστασης.}\\
Είσοδος :
\begin{center}{\en\tt algvar(y+x*sqrt(z))}\end{center}
Έξοδος :
\begin{center}{\en\tt  [[y,x],[z]]}\end{center}
Είσοδος :
\begin{center}{\en\tt algvar(y*sqrt(x)*sqrt(z))}\end{center}
Έξοδος :
\begin{center}{\en\tt  [[y],[z],[x]]}\end{center}
Είσοδος :
\begin{center}{\en\tt algvar(y*sqrt(x*z))}\end{center}
Έξοδος :
\begin{center}{\en\tt  [[y],[x,z]]}\end{center}
Είσοδος :
\begin{center}{\en\tt algvar(y+x*sqrt(z)+y*sin(x))}\end{center}
Έξοδος :
\begin{center}{\en\tt [[x,y,sin(x)],[z]]}\end{center}

\subsection{Έλεγχος για το αν μια μεταβλητή είναι σε μία \\παράσταση : {\tt\textlatin{ has}}}\index{has|textbf}
\noindent{{\en\tt has} παίρνει σαν όρισμα μια παράσταση και το όνομα μιας 
μεταβλητής.\\
{\en\tt has} επιστρέφει {\en\tt 1} εάν αυτή η μεταβλητή είναι στην παράσταση, και διαφορετικά επιστρέφει
{\en\tt 0}.}\\
Είσοδος :
\begin{center}{\en\tt has(x*y*sin(x),y)}\end{center}
Έξοδος :
\begin{center}{\en\tt  1}\end{center}
Είσοδος :
\begin{center}{\en\tt has(x*y*sin(x),z)}\end{center}
Έξοδος :
\begin{center}{\en\tt  0}\end{center}

\subsection{Αριθμητική αποτίμηση : {\tt\textlatin{ evalf}}}\index{evalf}
\noindent{{\en\tt evalf} παίρνει σαν όρισμα μια παράσταση ή έναν πίνακα.\\
{\en\tt evalf} επιστρέφει την αριθμητική τιμή της παράστασης ή του πίνακα.}\\
Είσοδος :
\begin{center}{\en\tt evalf(sqrt(2))}\end{center} 
Έξοδος :
\begin{center}{\en\tt 1.41421356237}\end{center}
Είσοδος :
\begin{center}{\en\tt evalf([[1,sqrt(2)],[0,1]])}\end{center} 
Έξοδος :
\begin{center}{\en\tt [[1.0,1.41421356237],[0.0,1.0]]}\end{center}

\subsection{Ρητή προσέγγιση : {\tt\textlatin{ float2rational exact}}}\index{float2rational}\index{exact}
\noindent{{\en\tt float2rational} (ή {\en\tt exact}) 
παίρνει σαν όρισμα μια παράσταση.}\\
{\en\tt float2rational} επιστρέφει μια ρητή προσέγγιση  
όλων των αριθμών κινητής υποδιαστολής $r$ που περιέχονται στην παράσταση, όπου η ρητή 
προσέγγιση ικανοποιεί την σχέση  $|r-\mbox{\en\tt float2rational}(r)|<\epsilon$. Το 
$\epsilon$  ορίζεται με το {\en\tt epsilon} στις Ρυθμίσεις  {\en\tt cas} 
(στο μενού {\tt Ρυθμίσεις}, ή στην μπάρα ρυθμίσεων ή με την εντολή  {\en\tt cas\_setup}).\\
Είσοδος :
\begin{center}{\en\tt float2rational(1.5)}\end{center}
Έξοδος :
\begin{center}{\en\tt 3/2}\end{center}
Είσοδος :
\begin{center}{\en\tt float2rational(1.414)}\end{center}
Έξοδος :
\begin{center}{\en\tt 707/500}\end{center}
Είσοδος :
\begin{center}{\en\tt float2rational(0.156381102937*2)}\end{center}
Έξοδος :
\begin{center}{\en\tt 5144/16447}\end{center}
Είσοδος :
\begin{center}{\en\tt float2rational(1.41421356237)}\end{center}
Έξοδος :
\begin{center}{\en\tt 114243/80782}\end{center}
Είσοδος :
\begin{center}{\en\tt float2rational(1.41421356237\verb|^|2)}\end{center}
Έξοδος :
\begin{center}{\en\tt 2}\end{center}

\chapter{Γράφοι}\label{sec:plot}
Οι περισσότερες εντολές γράφων παίρνουν παραστάσεις σαν ορίσματα. Λίγες εξαιρέσεις
(κυρίως εντολές συμβατότητας με το {\en\tt maple}) επίσης δέχονται
συναρτήσεις. 
Μερικά προαιρετικά ορίσματα, όπως {\tt χρώμα, πάχος γραμμής}, μπορούν να χησιμοποιηθούν σαν
προαιρετικά ορίσματα σε όλες τις γραφικές εντολές. Περιγράφονται παρακάτω.

\section{Χαρακτηριστικά  γράφων και  γεωμετρικών αντικειμένων}
Υπάρχουν δύο είδη χαρακτηριστικών: καθολικά χαρακτηριστικά ενός γραφικού
περιβάλλοντος και ατομικά χαρακτηριστικά. 

\subsection{Ατομικά χαρακτηριστικά}\index{color}\index{display}
\index{red@{\it red}|textbf}\index{blue@{\it blue}|textbf}\index{yellow@{\it yellow}|textbf}\index{magenta@{\it magenta}|textbf}\index{green@{\it green}|textbf}\index{cyan@{\it cyan}|textbf}\index{white@{\it white}|textbf}\index{black@{\it black}|textbf}\index{filled@{\it filled}}
Τα γραφικά χαρακτηριστικά είναι προαιρετικά ορίσματα της μορφής
 {\en\tt display = value}, και πρέπει να δίνονται σαν το
τελευταίο όρισμα μιας γραφικής εντολής. Τα χαρακτηριστικά είναι
ταξινομημένα σε διάφορες κατηγορίες: χρώμα, σχήμα σημείου, πλάτος σημείου,
στυλ γραμμής, πάχος γραμμής,  τιμή της λεζάντας, θέση  και παρουσίαση. 
Επιπρόσθετα, οι επιφάνειες μπορεί να καλύπτονται ({\en filled}) ή όχι, οι τρισδιάστατες επιφάνειες
μπορεί να καλύπτονται με μία υφή, τα τρισδιάστατα αντικείμενα μπορεί να έχουν επίσης ιδιότητες 
ως προς το φως. 
Χαρακτηριστικά από διαφορετικές κατηγορίες μπορεί
να προστεθούν, π.χ. \\
{\en\tt plotfunc($x^2+y^2$,[x,y],display=red+line\_width\_3+filled)}
\begin{itemize}
\item Χρώματα {\en\tt display=} ή {\en\tt color=}
\begin{itemize}
\item {\en\tt black}, {\en\tt white}, {\en\tt red}, {\en\tt blue}, {\en\tt green}, 
{\en\tt magenta}, {\en\tt cyan}, {\en\tt yellow},
\item μια αριθητική τιμή μεταξύ 0 και 255,
\item μια αριθητική τιμή μεταξύ 256 και 256+7*16+14 για ένα χρώμα του ουράνιου τόξου ,
\item οποιαδήποτε άλλη τιμή μικρότερη του 65535. Δεν υπάρχει εγγύηση πως η απόδοση είναι φορητή.
\end{itemize}
\item Σχήματα σημείων {\en\tt display=} μια απ' τις ακόλουθες τιμές
{\en\tt rhombus\_point plus\_point  square\_point cross\_point 
triangle\_point \\star\_point point\_point invisible\_point}
\item Πλάτος σημείου: {\en\tt display=} μια απ' τις ακόλουθες τιμές
{\en\tt point\_width\_n} όπου {\en\tt n} είναι ένας
ακέραιος μεταξύ 1 και 7
\item Πάχος γραμμής: {\en\tt thickness=n}
ή {\en\tt display=line\_width\_n} όπου {\en\tt n} είναι ένας 
ακέραιος μεταξύ 1 και 7 ή 
\item Σχήμα γραμμής: {\en\tt display=} μια από τις ακόλουθες τιμές
{\en\tt dash\_line \\solid\_line dashdot\_line dashdotdot\_line
  cap\_flat\_line \\cap\_square\_line cap\_round\_line }
\item Τιμή λεζάντας: {\en\tt legend="legendname"}.
 Θέση: {\en\tt display=} μια από τις ακόλουθες τιμές 
{\en\tt quandrant1 quadrant2 quadrant3 quadrant4}
που αντιστοιχεί στη θέση της λεζάντας του αντικειμένου 
(χρησιμοποιώντας  την τριγωνομετρική σύμβαση αρίθμησης).
Η λεζάντα δεν εμφανίζεται εάν προστεθεί το όρισμα 
{\en\tt display=hidden\_name}.
\item {\en\tt display=filled} ορίζει ότι οι επιφάνειες θα καλυφθούν (γεμίσουν),
\item {\en\tt gl\_texture="picture\_filename"} χρησιμοποιείται για να καλύψουμε (γεμίσουμε)
μια επιφάνεια με μία υφή.  
Βλέπε το εγχειρίδιο διεπαφής για μια  πιο ολοκληρωμένη 
περιγραφή και για {\en\tt gl\_material=} επιλογές.
\end{itemize}
{\bf Παραδείγματα}
Είσοδος:
\begin{center}{\en\tt polygon(-1,-i,1,2*i,legend="P")}\end{center}
Είσοδος:
\begin{center}{\en\tt point(1+i,legend="hello")}\end{center}
Είσοδος:
\begin{center}{\en\tt A:=point(1+i);B:=point(-1);display(D:=line(A,B),hidden\_name)}\end{center}
Είσοδος:
\begin{center}{\en\tt color(segment(0,1+i),red)}\end{center}
Έξοδος:
\begin{center}{\en\tt segment(0,1+i,color=red)}\end{center}

\subsection{Καθολικά χαρακτηριστικά}
Αυτά τα χαρακτηριστικά είναι κοινά για όλα τα αντικέιμενα του ίδιου γραφικού
περιβάλλοντος
\begin{itemize}
\item {\en\tt title="titlename"} ορίζει τον τίτλο 
\item {\en\tt labels=["xname","yname","zname"]}: ονόματα των αξόνων $x,y,z$ 

\item {\en\tt gl\_x\_axis\_name="xname"}, {\en\tt gl\_y\_axis\_name="yname"},
και \\{\en\tt gl\_z\_axis\_name="zname"}: ατομικός ορισμός
των ονομάτων των αξόνων $x,y,z$  
\item {\en\tt legend=["xunit","yunit","zunit"]}: ορισμός μονάδων των αξόνων
$x,y,z$ 
\item {\en\tt gl\_x\_axis\_unit="xunit"}, {\en\tt  gl\_y\_axis\_unit="yunit"},
και \\{\en\tt gl\_z\_axis\_unit="zunit"}: ατομικός ορισμός των μονάδων των αξόνων 
 $x,y,z$ 
\item {\en\tt axes=true} ή {\en\tt false} δείχνει ή κρύβει τους άξονες
\item {\en\tt gl\_texture="filename"}: εικόνα φόντου
\item {\en\tt gl\_x=xmin..xmax}, {\en\tt gl\_y=ymin..ymax},
{\en\tt gl\_z=zmin..zmax}: καθορίζει το πεδίο γραφικών  
(να μην χρησιμοποιείτε σε αλληλεπιδραστικά περιβάλλοντα)
\item {\en\tt gl\_xtick=}, {\en\tt gl\_ytick=}, {\en\tt gl\_ztick=}:
καθορίζει τις υποδιαιρέσεις για τoυς άξονες 
\item {\en\tt gl\_shownames=true} ή {\en\tt false}: δείχνει ή κρύβει τα ονόματα των αντικειμένων
\item {\en\tt gl\_rotation=[x,y,z]}: καθορίζει τον άξονα περιστροφής
για το εφέ περιστροφής σε τρισδιάστατο περιβάλλον
\item {\en\tt gl\_quaternion=[x,y,z,t]}: καθορίζει την τετράδα 
για την οπτικοποίηση σε τρισδιάστατο περιβάλλον (να μην χρησιμοποιείτε τα αλληλεπιδραστικά 
περιβάλλοντα)
\item Ακόμα λίγες επιλογές {\en\tt OpenGL} για ρυθμίσεις φωτός είναι διαθέσιμες
αλλά δεν περιγράφονται εδώ.
\end{itemize}
{\bf Παραδείγματα}
Είσοδος:
\begin{center}{\en\tt legend=["mn","kg"]}\end{center}
Είσοδος:
\begin{center}{\en\tt title="median\_line";triangle(-1-i,1,1+i);\\median\_line(-1-i,1,1+i);median\_line(1,-1-i,1+i);\\median\_line(1+i,1,-1-i)}\end{center}
Είσοδος:
\begin{center}{\en\tt labels=["u","v"];plotfunc(u+1,u)}\end{center}



\section{Γράφος μιας συνάρτησης : {\tt\textlatin{ plotfunc \\funcplot DrawFunc Graph}}}\index{plotfunc|textbf}\index{funcplot|textbf}\index{DrawFunc|textbf}\index{Graph|textbf}\index{xstep@{\sl xstep}}\index{ystep@{\sl ystep}}\index{zstep@{\sl zstep}}\index{nstep@{\sl nstep}}

\subsection{Διδιάστατος γράφος}\label{sec:plotfunc}
\noindent{{\en\tt plotfunc(f(x),x)} σχεδιάζει τον γράφο της $y=f(x)$ για  $x$ στο προεπιλεγμένο
  διάστημα, ενώ
{\en\tt plotfunc(f(x),x=a..b)} σχεδιάζει τον γράφο της $y=f(x)$ στο διάστημα $a\leq x\leq b$.
{\en\tt plotfunc} δέχεται ένα προαιρετικό όρισμα, {\en\tt \verb|xstep=…|}, που καθορίζει  
 το βήμα του $x$ στην  διακριτοποίηση.}\\
Είσοδος :
\begin{center}{\en\tt  plotfunc(x\verb|^|2-2)}\end{center}
ή
\begin{center}{\en\tt  plotfunc(a\verb|^|2-2,a=-1..2)}\end{center}
Έξοδος :
\begin{center}{\tt ο γράφος της \en y=x\verb|^|2-2}\end{center}
Είσοδος :
\begin{center}{\en\tt  plotfunc(x\verb|^|2-2,x,xstep=1)}\end{center}
Έξοδος :
\begin{center}{\tt μια πολυγωνική γραμμή η οποία είναι μια κακή αναπαράσταση της {\en\tt y=x\verb|^|2-2} }\end{center}
Μπορούμε επίσης να καθορίσουμε τον αριθμό των σημείων  που χρησιμοποιούνται για την αναπαράσταση
της συνάρτησης με το όρισμα {\en\tt \verb|nstep=|} αντί για {\en\tt \verb|xstep=|}.
Για παράδειγμα, εισάγετε~:
\begin{center}{\en\tt  plotfunc(x\verb|^|2-2,x=-2..3,nstep=30)}\end{center}

\subsection{Τρισδιάστατος γράφος}\label{sec:plotfunc3}
\noindent{\en\tt plotfunc} παίρνει δύο κύρια ορίσματα : μια παράσταση δύο μεταβλητών
ή μια λίστα διαφόρων παραστάσεων δύο μεταβλητών και την λίστα αυτών των δύο
μεταλητών, όπου κάθε μεταβλητή μπορεί να αντικατασταθεί από
μια ισότητα $μεταβλητή=διάστημα$ για να καθορίσουμε το πεδίο τιμών  αυτής της μεταβλητής
(αν δεν ορίσουμε εμείς το πεδίο τιμών, αυτό προεπιλέγεται από τα χαρακτηριστικά του γράφου).
{\en\tt plotfunc} δέχεται δύο προαιρετικά ορίσματα, {\en\tt xstep=...} και {\en\tt ystep=…}, για να καθορίσει 
τα βήματα των $x$ και $y$ στην διακριτοποίηση.
Εναλλακτικά κάποιος μπορεί να ορίσει τον αριθμό των σημείων που χρησιμοποιούνται
για την αναπαράσταση της συνάρτησης με {\en\tt \verb|nstep=|} (αντί για  {\en\tt \verb|xstep|} και
{\en\tt ystep}).\\
{\en\tt plotfunc} σχεδιάζει την επιφάνεια(-ες) που ορίζεται(-ονται) από $z=$ το πρώτο όρισμα.\\
Είσοδος :
\begin{center}{\en\tt plotfunc( x\verb|^|2+y\verb|^|2,[x,y])}\end{center}
Έξοδος :
\begin{center}{\tt Τρισδιάστατος γράφος της \en z=x\verb|^|2+y\verb|^|2}\end{center}
Είσοδος :
\begin{center}{\en\tt plotfunc(x*y,[x,y]) }\end{center}
Έξοδος  :
\begin{center}{\tt Η επιφάνεια {\en z=x*y}, στα προεπιλεγμένα πεδία τιμών}\end{center}
Είσοδος :
\begin{center}{\en\tt plotfunc([x*y-10,x*y,x*y+10],[x,y]) }\end{center}
Έξοδος  :
\begin{center}{\tt Οι επιφάνειες {\en\tt z=x*y-10, z=x*y} και {\en\tt z=x*y+10}}\end{center}
Είσοδος :
\begin{center}{\en\tt plotfunc(x*sin(y),[x=0..2,y=-pi..pi]) }\end{center}
Έξοδος  :
\begin{center}{\tt Η επιφάνεια $z=x*y$ για συγκεκριμένα πεδία τιμών}\end{center}
Τώρα ένα παράδειγμα όπου θα ορίσουμε τα βήματα των $x$ και $y$ στην διακριτικοποίηση
με {\en\tt \verb|xstep|} και {\en\tt  \verb|ystep|}. Είσοδος :
\begin{center}
{\en\tt plotfunc(x*sin(y),[x=0..2,y=-pi..pi],xstep=1,ystep=0.5) }\end{center}
Έξοδος  :
\begin{center}{\tt Ένα τμήμα της επιφάνειας $z=x*y$}\end{center}
Εναλλακτικά μπορούμε να ορίσουμε
τον αριθμό των σημείων που χρησιμοποιούνται για την αναπαράσταση 
της συνάρτησης με {\en\tt \verb|nstep|} αντί {\en\tt \verb|xstep|} και {\en\tt\verb|ystep|}, Είσοδος~:
\begin{center}{\en\tt plotfunc(x*sin(y),[x=0..2,y=-pi..pi],nstep=300)}\end{center}
Έξοδος  :
\begin{center}{\tt Ένα τμήμα της επιφάνειας $z=x*y$}\end{center}
{\bf Σχόλια}
\begin{itemize}
\item
Όπως σε κάθε  τρισδιάστατο περιβάλλον, η οπτική γωνία μπορεί να αλλάξει με περιστροφή  γύρω από 
τον άξονα των {\en\tt x}, γύρω από τον άξονα των {\en\tt y} ή γύρω από τον άξονα των {\en\tt z} , είτε σέρνοντας το ποντίκι μέσα στο παράθυρο γραφικών
(σπρώξτε το ποντίκι έξω από το παραλληλεπίπεδο που χρησιμοποιείται για 
την αναπαράσταση), ή με συντομεύσεις, πατώντας τα πλήκτρα
{\en\tt x}, {\en\tt X}, {\en\tt y}, {\en\tt Y}, {\en\tt z} και {\en\tt Z}.
\item
Εάν θέλετε να τυπώσετε έναν γράφο ή να πάρετε την    μετάφραση \LaTeX\, χρησιμοποιήστε το μενού γραφικών {\en\tt M} (δεξιά από το παράθυρο γραφικών)\\
{\en\tt M$\blacktriangleright$\mbox{\gr\tt Εξαγωγή-Εκτύπωση}$\blacktriangleright$\mbox{\gr\tt Εκτύπωση (με}
  Latex)}
\end{itemize}

\subsection{Τρισδιάστατος γράφος με χρώματα ουράνιου τόξου }\label{sec:plotfunc3d}
\noindent{{\en\tt plotfunc} αναπαριστά μια καθαρά φανταστική παράσταση {\en\tt i*E}
δύο μεταβλητών με ένα χρώμα ουράνιου τόξου πoυ εξαρτάται 
από την μεταβλητή  {\en\tt z=E}. Αυτό δίνει ένα εύκολο τρόπο 
για να βρούμε σημεία που έχουν την ίδια τρίτη συντεταγμένη.}
Το πρώτο όρισμα της {\en\tt plotfunc} πρέπει να είναι  {\en\tt i*E} αντί για {\en\tt E}, ενώ τα υπόλοιπα ορίσματα είναι ίδια με  εκείνα του
 πραγματικού τρισδιάστατου γράφου (βλέπε \ref{sec:plotfunc3})\\
Είσοδος :
\begin{center}{\en\tt plotfunc(i*x*sin(y),[x=0..2,y=-pi..pi]) }\end{center}
Έξοδος  :
\begin{center}{\tt Ένα τμήμα της επιφάνειας $z=x*\sin(y)$ με χρώματα ουράνιου τόξου }\end{center}
{\bf Σχόλιο}\\
Εάν θέλετε να τυπώσετε έναν γράφο ή να πάρετε την    μετάφραση \LaTeX\, χρησιμοποιήστε το μενού γραφικών {\en\tt M} (δεξιά από το παράθυρο γραφικών)\\
{\en\tt M$\blacktriangleright$\mbox{\gr\tt Εξαγωγή-Εκτύπωση}$\blacktriangleright$\mbox{\gr\tt Εκτύπωση (με}
  Latex)}. 

\subsection{Τετραδιάστατος γράφος.}\label{sec:plotfunc4}
\noindent{{\en\tt plotfunc}  αναπαριστά μια μιγαδική παράσταση {\en\tt E} 
(τέτοια ώστε η {\en\tt re(E)} δεν ταυτίζεται με το 0 στο πλέγμα διακριτοποίησης)
με την επιφάνεια {\en\tt z=abs(E)}, όπου {\en\tt arg(E)} ορίζει το χρώμα 
του ουράνιου τόξου. Αυτό δίνει έναν εύκολο τρόπο να
βλέπουμε τα σημεία που έχουν το ίδιο όρισμα ({\en\tt arg(z)}).
Σημειώσατε ότι εάν {\en\tt re(E)==0} στο πλέγμα διακριτοποίησης, τότε
είναι η επιφάνεια {\en\tt z=E/i} που αναπαρίσταται με τα χρώματα του ουράνιου τόξου 
(βλέπε \ref{sec:plotfunc3d}).}\\
Το πρώτο όρισμα της {\en\tt plotfunc} είναι {\en\tt E}, 
ενώ τα υπόλοιπα ορίσματα είναι ίδια με  εκείνα του
 πραγματικού τρισδιάστατου γράφου (βλέπε \ref{sec:plotfunc3}).\\
Είσοδος :
\begin{center}{\en\tt plotfunc((x+i*y)\verb|^|2,[x,y])}\end{center}
Έξοδος :
\begin{center}{\tt Ένας {\en\tt 3D} γράφος της {\en\tt z=abs((x+i*y)\verb|^|2} με το ίδιο χρώμα για τα σημεία
που έχουν το ίδιο όρισμα}\end{center}
Είσοδος :
\begin{center}{\en\tt plotfunc((x+i*y)\verb|^|2x,[x,y], display=filled)}\end{center}
Έξοδος :
\begin{center}{\tt Η ίδια επιφάνεια αλλά γεμισμένη}\end{center}
Μπορούμε να ορίσουμε το πεδίο τιμών του $x$ και του $y$ καθώς επίσης και τον αριθμό
των σημείων διακριτοποίησης.\\
Είσοδος :
\begin{center}{\en\tt plotfunc((x+i*y)\verb|^|2,[x=-1..1,y=-2..2], nstep=900,display=filled)}\end{center}
Έξοδος  :
\begin{center}{\tt Το καθορισμένο μέρος της επιφάνειας, με $x$ μεταξύ -1 και 1, \\με $y$ μεταξύ -2 και 2 και με 900 σημεία}\end{center}
 
\section{Διδιάστατος γράφος για συμβατότητα με το {\tt\textlatin{Maple}} : {\tt\textlatin{ plot}}}
\index{plot} \label{sec:plot2d}
\noindent{{\en\tt plot(f(x),x)} σχεδιάζει τον γράφο της $y=f(x)$. 
Το δεύτερο όρισμα καθορίζει το πεδίο τιμών {\en\tt
  x=xmin..xmax}. Mπορούμε επίσης να σχεδιάσουμε μια συνάρτηση αντί για μια παράσταση
χρησιμοποιώντας την  ακόλουθη σύνταξη {\en\tt plot(f,xmin..xmax)}.\\
{\en\tt plot} δέχεται ένα προαιρετικό όρισμα για να ορίσει 
το βήμα που χρησιμοποιεί το  $x$  στην διακριτοποίηση   με  
{\en\tt \verb|xstep=|} ή τον αριθμό των σημείων για την διακριτοποίηση
με {\en\tt \verb|nstep=|}.}\\
Είσοδος :
\begin{center}{\en\tt  plot(x\verb|^|2-2,x)}\end{center}
Έξοδος :
\begin{center}{\tt ο γράφος της \en y=x\verb|^|2-2}\end{center}
Είσοδος :
\begin{center}{\en\tt  plot(x\verb|^|2-2,xstep=1)}\end{center}
ή 
\begin{center}{\en\tt  plot(x\verb|^|2-2,x,xstep=1)}\end{center}
Έξοδος :
\begin{center}{\tt μια πολυγωνική γραμμή η
     οποία είναι μια κακή αναπαράσταση της \en y=x\verb|^|2-2 }\end{center}
Είσοδος  :
\begin{center}{\en\tt  plot(x\verb|^|2-2,x=-2..3,nstep=30)}\end{center}
Έξοδος :
\begin{center}{\tt ο γράφος της {\en y=x\verb|^|2-2} στο διάστημα [-2, 3] και με 30 σημεία}\end{center}


\section{Τρισδιάστατες  επιφάνειες για συμβατότητα με το {\tt\textlatin{Maple}} :\\ {\tt\textlatin{ plot3d}}}\index{plot3d}
\noindent{{\en\tt plot3d} παίρνει τρία ορίσματα : μια συνάρτηση δύο μεταβλητών ή
μια παράσταση δύο μεταβλητών  ή μια λίστα τριών συναρτήσεων με δύο μεταβλητές 
ή μια λίστα τριών παραστάσεων με δύο μεταβλητές και τα ονόματα αυτών των δύο  
μεταβλητών με ένα προαιρετικό πεδίο τιμών  (για παραστάσεις) ή τα πεδία τιμών 
(για συναρτήσεις).}\\
{\en\tt plot3d(f(x,y),x,y)} (αντιστ. {\en\tt plot3d([f(u,v),g(u,v),h(u,v)],u,v)}) σχεδιάζει την 
επιφάνεια $z=f(x,y)$ (αντιστ. $x=f(u,v),y=g(u,v),z=h(u,v)$).
{\en\tt plot3d(f(x,y),x=x0..x1,y=y0..y1)} ή 
{\en\tt plot3d(f,x0..x1,y0..y1)} καθορίζει πιο τμήμα της επιφάνειας
θα σχεδιαστεί (διαφορετικά από προεπιλογή τα πεδία τιμών ορίζονται από τα χαρακτηριστικά του γράφου).\\ 
Είσοδος :
\begin{center}{\en\tt plot3d(x*y,x,y)}\end{center}
Έξοδος :
\begin{center}{\tt Η επιφάνεια $z=x*y$}\end{center}
Είσοδος  :
\begin{center}{\en\tt plot3d([v*cos(u),v*sin(u),v],u,v) }\end{center}
Έξοδος :
\begin{center}{\tt Ο κώνος $x=v*\cos(u),y=v*\sin(u),z=v$}\end{center}
Είσοδος  :
\begin{center}{\en\tt plot3d([v*cos(u),v*sin(u),v],u=0..pi,v=0..3)}\end{center}
Έξοδος :
\begin{center}{\tt Ένα τμήμα του κώνου $x=v*\cos(u),y=v*\sin(u),z=v$}\end{center}

\section{Γράφος ευθείας και εφαπτομένη σε γράφο}
\subsection{Σχεδιασμός ευθείας : {\tt\textlatin{ line}}}\index{line}\label{sec:doite}
\noindent{{\en\tt line}} παίρνει σαν όρισμα καρτεσιανή(-ές) εξίσωση(-ώσεις) :
\begin{itemize}
\item {\en\tt 2D}: εξίσωση μιας ευθείας,
\item {\en\tt 3D}: δύο εξισώσεις που ορίζουν ένα επίπεδο.
\end{itemize}
{\en\tt line} ορίζει και σχεδιάζει την αντίστοιχη ευθείας.\\
Είσοδος  :
\begin{center}{\en\tt line(2*y+x-1=0)}\end{center}
Έξοδος :
\begin{center}{\tt Η ευθεία \en 2*y+x-1=0}\end{center}
Είσοδος  :
\begin{center}{\en\tt line(y=1)}\end{center}
Έξοδος :
\begin{center}{\tt Η οριζόντια γραμμή {\en\tt y=1}}\end{center}
Είσοδος  :
\begin{center}{\en\tt line(x=1)}\end{center}
Έξοδος :
\begin{center}{\tt Η κάθετη γραμμή {\en\tt x=1}}\end{center}
Είσοδος  :
\begin{center}{\en\tt line(x+2*y+z-1=0,z=2)}\end{center}
Έξοδος :
\begin{center}{\tt Η ευθεία  {\en\tt x+2*y+1=0} στο επίπεδο {\en\tt z=2}}\end{center}
Είσοδος  :
\begin{center}{\en\tt line(y=1,x=1)}\end{center}
Έξοδος :
\begin{center}{\tt Η κάθετη γραμμή που περνά από το σημείο (1,1,0)}\end{center}
{\bf Σχόλιο}\\
{\en\tt line} ορίζει μια προσανατολισμένη ευθεία :
\begin{itemize}
\item όταν η διδιάστατη ({\en\tt 2D}) ευθεία δίνεται από μια εξίσωση, ξαναγράφεται σαν
$"$αριστερό\_μέ\-λος-δεξί\_μέλος={\en\tt ax+by+c=0}$"$. Αυτό καθορίζει
το κάθετο διάνυσμα {\en\tt [a,b]} και η κατεύθυνση της ευθείας δίνεται από το διάνυσμα
{\en\tt [b,-a]}) (ή η κατεύθυνση της ευθείας ορίζεται από το τρισδιάστατο εξωτερικό γινόμενο
του κάθετου διανύσματος (με τρίτη συντεταγμένη 0) επί το διάνυσμα [0,0,1]).\\
Για παράδειγμα {\en\tt line(y=2*x)} ορίζει την γραμμή {\en\tt -2x+y=0} με κατεύθυνση το διάνυσμα {\en\tt [1,2]} (ή {\en\tt cross([-2,1,0],[0,0,1])}={\en\tt [1,2,0]}).
\item όταν η τρισδιάστατη ({\en\tt 3D}) ευθεία δίνεται από δύο (επίπεδες)  εξισώσεις, η κατεύθυνσή του 
ορίζεται από το εξωτερικό γινόμενο των καθέτων στα επίπεδα 
(όπου η εξίσωση επιπέδου ξαναγράφεται σαν 
$"$αριστερό\_μέ\-λος-δεξί\_μέλος={\en\tt ax+by+cz+d=0}$"$, και η κάθετος
 είναι {\en\tt [a,b,c]}).\\
Για παράδειγμα,  {\en\tt line(x=y,y=z)} είναι η ευθεία {\en\tt x-y=0,y-z=0} και η 
κατεύ\-θυν\-σή της είναι :\\
{\en\tt cross([1,-1,0],[0,1,-1])}={\tt [1,1,1]}.
\end{itemize}

\subsection{Σχεδιασμός διδιάστατης ({\tt\textlatin{2D}}) οριζόντιας ευθείας : \\{\tt\textlatin{LineHorz}}}\index{LineHorz}
\noindent{{\en\tt LineHorz} παίρνει σαν όρισμα μια παράσταση $a$.\\
{\en\tt LineHorz} σχεδιάζει την οριζόντια ευθεία $y=a$.}\\
Είσοδος :
\begin{center}{\en\tt LineHorz(1)}\end{center}
Έξοδος :
\begin{center}{\tt Η ευθεία \en y=1}\end{center}

\subsection{Σχεδιασμός διδιάστατης ({\tt\textlatin{2D}}) κάθετης ευθείας : \\{\tt\textlatin{ LineVert}}}\index{LineVert}
\noindent{{\en\tt LineVert} παίρνει σαν όρισμα μία παράσταση $a$.\\
{\en\tt LineVert} σχεδιάζει την κάθετη ευθεία $x=a$.}\\
Είσοδος:
\begin{center}{\en\tt LineVert(1)}\end{center}
Έξοδος :
\begin{center}{\tt Η ευθεία {\en x=1}}\end{center}

\subsection{Εφαπτόμενη σε έναν διδιάστατο ({\tt\textlatin{2D}}) γράφο: \\{\tt\textlatin{ LineTan}}}\index{LineTan}
\noindent{\en\tt LineTan} παίρνει δύο ορίσματα: μια παράσταση $E_x$ ως προς την
μεταβλητή $x$ και μια τιμή $x0$ του $x$.\\
{\en\tt LineTan} σχεδιάζει την εφαπτομένη στο $x=x0$ στον γράφο της $y=E_x$.\\
Είσοδος :
\begin{center}{\en\tt LineTan(ln(x),1)}\end{center}
Έξοδος :
\begin{center}{\tt Η ευθεία {\en\tt y=x-1}}\end{center}
Είσοδος:
\begin{center}{\en\tt equation(LineTan(ln(x),1))}\end{center}
Έξοδος :
\begin{center}{\en\tt y=(x-1)}\end{center}

\subsection{Εφαπτομένη σε έναν διδιάστατο ({\tt\textlatin{2D}}) γράφο : \\{\tt\textlatin{ tangent}}}\index{tangent|textbf}\label{sec:tangente}
\noindent{{\en\tt tangent} παίρνει δύο ορίσματα : ένα γεωμετρικό αντικείμενο και ένα σημείο 
{\en\tt A}.\\
{\en\tt tangent} σχεδιάζει  την  εφαπτομένη (ή τις εφαπτομένες) σε αυτό το γεωμετρικό αντικείμενο που περνάει (ή περνάνε) από το
{\en\tt A}. Εάν το γεωμετρικό αντικείμενο είναι ο γράφος {\en\tt G} μιας διδιάστατης ({\en\tt 2D}) συνάρτησης, 
το δεύτερο όρισμα είναι είτε  ένας πραγματικός αριθμός {\en\tt x0}, ή ένα
σημείο {\en\tt A} στον {\en\tt G}. Σε αυτή την περίπτωση  {\en\tt tangent} σχεδιάζει μια εφαπτομένη σε αυτόν τον
γράφο {\en\tt G} που περνάει από το σημείο {\en\tt A} ή απο το σημείο 
με τετμημένη {\en\tt x0}.}\\
Για παράδειγμα, ορίστε την συνάρτηση {\en\tt g}
\begin{center}{\en\tt \verb|g(x):=x^2|}\end{center}
και μετά τον γράφο {\en\tt G=\{(x,y)$\in \R^2$, y=g(x)\}}
της $g$ και ένα σημείο $A$ στον γράφο $G$:
\begin{center}
{\en\tt G:=plotfunc(g(x),x);}\\
{\en\tt A:=point(1.2,g(1.2));}
\end{center}
Εάν θέλουμε να σχεδιάσουμε την εφαπτομένη στο σημείο {\en\tt A} του γράφου {\en\tt
  G}, θα εισάγουμε:
\begin{center}
{\en\tt T:=tangent(G, A)}
\end{center}
ή :
\begin{center}
{\en\tt T:=tangent(G, 1.2)}
\end{center}
Για την εξίσωση της εφαπτομένης γραμμής, εισάγετε :
\begin{center}{\en\tt equation(T)}\end{center}

\subsection{Τομή ενός διδιάστατου ({\tt\textlatin{2D}}) γράφου με τους άξονες  }\index{solve}\index{resoudre}
\begin{itemize}
\item H τεταγμένη της τομής του γράφου της $f$ με τον άξονα των 
$y$ επιστρέφεται από :
\begin{center}{\en\tt f(0)}\end{center}
Πράγματι, το σημείο με συντεταγμένες $(0,f(0))$ είναι το σημείο τομής του γράφου της 
$f$ με τον άξονα των $y$,
\item Για να βρούμε την τομή του γράφου της $f$ με τον άξονα των $x$
απαιτείται να λύσουμε την εξίσωση $f(x)=0$. \\
Εάν η εξίσωση είναι  πολυωνυμική, η εντολή {\en\tt solve} θα βρει
τις ακριβείς τιμές της τετμημένης αυτών των σημείων. Είσοδος:
\begin{center}{\en\tt solve(f(x),x)}\end{center}
Διαφορετικά, μπορούμε να βρούμε αριθμητικές προσεγγίσεις μιας 
τετμημένης. Πρώτα κοιτάμε στο γράφο για μια αρχική μαντεψιά ή ένα
διάστημα που περιέχει την τομή και μετά την οριστικοποιούμε με την εντολή {\en\tt fsolve}.
\end{itemize}

\section{Γράφος ανισοτήτων με δύο μεταβλητές : \\{\tt\textlatin{ plotinequation  inequationplot}}}\index{plotinequation|textbf}\index{inequationplot|textbf}
\noindent{\en\tt plotinequation([f1(x,y)<a1,…fk(x,y)<ak],[x=x1..x2,y=y1..y2])} 
σχεδιάζει στο επίπεδο τα σημεία  των οποίων οι συντεταγμένες ικανοποιούν
τις ανισότητες δύο μεταβλητών :
\[ \left\{ \begin{array}{ccc}
f1(x,y) &<&a1 \\
& ... & \\
fk(x,y)&<&ak 
\end{array}\right., \quad
x1\leq x \leq x2, y1 \leq y \leq y2 \]
Είσοδος :
\begin{center}{\en\tt plotinequation(x\verb|^|2-y\verb|^|2<3, [x=-2..2,y=-2..2],xstep=0.1,ystep=0.1)}\end{center}
Έξοδος  :
\begin{center}{\tt το γεμισμένο τμήμα του επιπέδου που περικλείει την αρχή και περιορίζεται από την υπερβολή \en x\verb|^|2-y\verb|^|2=3}\end{center}
Είσοδος :
\begin{center}{\en\tt plotinequation([x+y>3,x\verb|^|2<y], [x-2..2,y=-1..10],xstep=0.2,ystep=0.2)}\end{center}
Έξοδος  :
\begin{center}{\tt το γεμισμένο τμήμα του επιπέδου που ορίζεται από {\en\tt -2<x<2,y<10,x+y>3,y>x\verb|^|2}}\end{center}
Σημειώστε ότι αν τα πεδία τιμών για τα $x$ και $y$ δεν συγκεκριμενοποιούνται, 
το {\en\tt Xcas} παίρνει τις προεπιλεγμένες τιμές του
{\en\tt X-,X+,Y-,Y+} που ορίζονται στις Ρυθμίσεις, δεξιά από το παράθυρο γραφικών
({\en\tt Cfg$\blacktriangleright$\mbox{\gr Ρυθμίσεις Γραφικών}}).


\section{Γράφος περιοχής κάτω από καμπύλη : \\{\tt\textlatin{ plotarea areaplot}}}\index{plotarea|textbf}\index{areaplot|textbf}\index{rectangle\_droit@{\sl rectangle\_droit}|textbf}\index{rectangle\_gauche@{\sl rectangle\_gauche}|textbf}\index{trapeze@{\sl trapeze}|textbf}\index{point\_milieu@{\sl point\_milieu}|textbf}
\begin{itemize}
\item Με δύο ορίσματα, η εντολή {\en\tt plotarea} επισκιάζει την περιοχή κάτω από μία καμπύλη.\\ 
{\en\tt plotarea(f(x),x=a..b)} σχεδιάζει και επισκιάζει την περιοχή κάτω από την καμπύλη $y=f(x)$ για 
$a<x<b$, δηλαδή το τμήμα του επιπέδου που ορίζεται από τις ανισώσεις $a<x<b$ και
$0<y<f(x)$ ή $0>y>f(x)$ ανάλογα με το πρόσημο του $f(x)$ .\\
Είσοδος :
\begin{center}{\en\tt plotarea(sin(x),x=0..2*pi)}\end{center}
Έξοδος  :
\begin{center}{\tt το τμήμα του επιπέδου που βρίσκεται ανάμεσα \\στον άξονα των {\en x} και τα δύο  τόξα του {\en\tt sin(x)}}\end{center}
\item Με τέσσερα ορίσματα, η εντολή {\en\tt plotarea}  εκτελεί μια αριθμητική προσέγγιση
της περιοχής κάτω από την καμπύλη, με μέθοδο ολοκλήρωσης που επιλέγεται από την 
ακόλουθη λίστα:\\
{\en\tt trapezoid,rectangle\_left,rectangle\_right,middle\_point}.\\
Για παράδειγμα {\en\tt plotarea(f(x),x=a..b,n,trapezoid)} 
σχεδιάζει και επισκιάζει την περιοχή $n$ τραπεζίων : το 
τρίτο όρισμα είναι ένας ακέραιος $n$, και το τέταρτο όρισμα είναι το όνομα
της αριθητικής μεθόδου ολοκλήρωσης, όταν το $[a,b]$ σπάει σε  $n$ ίσα τμήματα.\\
Είσοδος :
\begin{center}{\en\tt plotarea(x\verb|^|2,x=0..1,5,trapezoid)}\end{center}
Αν θέλετε να εμφανίσετε τον γράφο της καμπύλης σε αντίθεση
(π.χ. με έντονο κόκκινο), εισάγετε :
\begin{center}{\en\tt plotarea(x\verb|^|2,x=0..1,5,trapezoid); 
plot(x\verb|^|2,x=0..1,display=red+line\_width\_3)}\end{center}
Έξοδος :
\begin{center}{\tt τα 5 τραπέζια που χρησιμοποιούνται στην τραπεζοειδή μέθοδο και προσεγγίζουν το ολοκλήρωμα}\end{center}
Είσοδος :
\begin{center}{\en\tt plotarea((x\verb|^|2,x=0..1,5,middle\_point)}\end{center}
ή με τον γράφο της καμπύλης σε έντονο κόκκινο , εισάγετε :
\begin{center}{\en\tt plotarea(x\verb|^|2,x=0..1,5,middle\_point); plot(x\verb|^|2,x=0..1,display=red+line\_width\_3)}\end{center}
Έξοδος :
\begin{center}{\tt τα 5 ορθογώνια που χρησιμοποιούνται στην μέθοδο του μέσου σημείου 
    και προσεγγίζουν το ολοκλήρωμα}\end{center}
\end{itemize}

\section{Ισοϋψείς καμπύλες :\\ {\tt\textlatin{ plotcontour contourplot DrwCtour}}}\index{plotcontour|textbf}\index{contourplot|textbf}\index{DrwCtour|textbf}\label{sec:plotcontour}
\noindent{\en\tt plotcontour(f(x,y),[x,y])} (ή {\en\tt DrwCtour(f(x,y),[x,y])} ή \\
  {\en\tt contourplot(f(x,y),[x,y])})
σχεδιάζει τις ισοϋψείς καμπύλες της επιφάνειας που ορίζεται από $z=f(x,y)$ για $z=-10$, 
$z=-8$, .., $z=0$, $z=2$, .., $z=10$. Μπορείτε να ορίσετε τις επιθυμητές ισοϋψείς καμπύλες με μία λίστα τιμών του $z$ που δίνονται σαν τρίτο όρισμα.\\
Είσοδος :
\begin{center}{\en\tt  plotcontour(x\verb|^|2+y\verb|^|2,[x=-3..3,y=-3..3],[1,2,3], display=[green,red,black])}\end{center}
Έξοδος :
\begin{center}{\tt ο γράφος των 3 ελλείψεων {\en\tt x\verb|^|2-y\verb|^|2 = n} για {\en\tt n=1,2,3}}\end{center}
Είσοδος :
\begin{center}{\en\tt  plotcontour(x\verb|^|2-y\verb|^|2,[x,y])}\end{center}
Έξοδος :
\begin{center}{\tt ο γράφος των 11 ελλείψεων {\en\tt x\verb|^|2-y\verb|^|2 = n} για {\en\tt n=-10,-8,…,10}}\end{center}
Αν θέλετε να σχεδιάσετε την επιφάνεια σε τρισδιάστατο περιβάλλον, 
εισάγετε \\{\en\tt plotfunc(f(x,y),[x,y])}, βλέπε \ref{sec:plotfunc3}):
\begin{center}{\en\tt plotfunc( x\verb|^|2-y\verb|^|2,[x,y])}\end{center}
Έξοδος :
\begin{center}{\tt Τρισδιάστατη αναπαράσταση της {\en\tt z=x\verb|^|2+y\verb|^|2}}\end{center}

\section{Διδιάστατος γράφος με χρώματα μιας δι\-διά\-στατης συνάρτησης : 
\\{\tt\textlatin{ plotdensity  densityplot}}}
\index{plotdensity|textbf}\index{densityplot|textbf}
\noindent{{\en\tt plotdensity(f(x,y),[x,y])}  ή  {\en\tt densityplot(f(x,y),[x,y])}
σχεδιάζει τον γράφο της  $z=f(x,y)$ στο επίπεδο όπου οι τιμές του
$z$ αντιπροσωπεύονται από τα  χρώματα του ουράνιου τόξου. Το προαιρετικό όρισμα
{\en\tt z=zmin..zmax} ορίζει το πεδίο τιμών του $z$ που αντιστοιχεί στο
πλήρες ουράνιο τόξο. Αν δεν ορίζεται, συνεπάγεται από την μέγιστη
και ελάχιστη τιμή της $f$ στη διακριτοποίηση. Η διακριτοποίηση
μπορεί να ορίζεται από τα προαιρετικά ορίσματα {\en\tt xstep=...} και {\en\tt ystep=...}
ή {\en\tt nstep=...}}\\
Είσοδος :
\begin{center}{\en\tt plotdensity(x\verb|^|2-y\verb|^|2,[x=-2..2,y=-2..2], xstep=0.1,ystep=0.1)}\end{center}
Έξοδος :
\begin{center}{\tt Ένας {\en\tt 2D} γράφος όπου κάθε υπερβολή που ορίζεται από
   {\en\tt x\verb|^|2-y\verb|^|2=z} έχει ένα χρώμα από το ουράνιο τόξο}\end{center}
{\bf Σχόλιο} : Ένα ορθογώνιο που αντιπροσωπεύει την κλιμακα των χρωμάτων εμφανίζεται 
κάτω από τον γράφο .

\section{Πεπλεγμένος γράφος :\\ {\tt\textlatin{ plotimplicit  implicitplot}}}\index{plotimplicit}\index{implicitplot}\index{unfactored}
\noindent{{\en\tt plotimplicit} ή {\en\tt implicitplot} σχεδιάζει καμπύλες ή επιφάνειες 
που ορίζονται από μια πεπλεγμένη παράσταση ή εξίσωση. 
Εάν η επιλογή {\en\tt unfactored} δίνεται σαν τελευταίο όρισμα, η
αρχική παράσταση παίρνεται χωρίς τροποποίηση. Αλλιώς,
η παράσταση κανονικοποιείται, μετά αντικαθίσταται από την 
παραγοντοποίηση του αριθμητή της κανονικοποίησής της.}

Κάθε παράγοντας της παράστασης, αντιστοιχεί σε μία συνιστώσα
της πεπλεγμένης καμπύλης ή επιφάνειας. Για κάθε παράγοντα,
το {\en\tt Xcas} εξετάζει εάν είναι ολικού βαθμού
μικρότερου ή ίσου του 2, και σε αυτή την περίπτωση καλείται η {\en\tt conic} ή
{\en\tt quadric}. Διαφορετικά, καλείται ο αριθμητικός επιλυτής πεπλεγμένων. 

Προαιρετικά ορίσματα για το βήμα και τα πεδία τιμών μπορεί να περα\-στούν στον 
αριθμητικό επιλυτή πεπλεγμένων, αλλά αυτά απελευθερώνονται όταν η συνιστώσα
είναι κωνική καμπύλη ή δευτεροβάθμια επιφάνεια.

\subsection{Διδιάστατη ({\tt\textlatin{2D}}) πεπλεγμένη καμπύλη}\label{sec:implicitplot}
\begin{itemize}
\item {\en\tt plotimplicit(f(x,y),x,y)} σχεδιάζει την γραφική παράσταση της
καμπύλης που ορίζεται από την πεπλεγμένη εξίσωση $f(x,y)=0$ όπου $x$ (αντιστ $y$) 
είναι στο {\en\tt WX-, WX+} (αντίστοιχα στο {\en\tt WY-, WY+}) που ορίζονται στις Ρυθμίσεις Γραφικών {\en\tt cfg} (δεξιά από το παράθυρο γραφικών), 

\item {\en\tt plotimplicit(f(x,y),x=0..1,y=-1..1)} σχεδιάζει την γραφική παράσταση της καμπύλης που ορίζεται από την πεπλεγμένη εξίσωση $f(x,y)=0$ 
όπου $0\leq x \leq 1$ και $-1\leq y \leq 1$
\end{itemize} 
Είναι δυνατό να προσθέσουμε δύο ορίσματα για να ορίσουμε τα βήματα της διακριτοποίησης
των $x$ 
και  $y$ με {\en\tt xstep=...} και {\en\tt ystep=...}\\
Είσοδος :
\begin{center}{\en\tt plotimplicit(x\verb|^|2+y\verb|^|2-1,x,y)}\end{center}
ή :
\begin{center}{\en\tt plotimplicit(x\verb|^|2+y\verb|^|2-1,x,y,unfactored)}\end{center}
Έξοδος :
\begin{center}{\tt O μοναδιαίος κύκλος}\end{center}
Είσοδος :
\begin{center}{\en\tt plotimplicit(x\verb|^|2+y\verb|^|2-1,x,y,xstep=0.2,ystep=0.3)}\end{center}
ή :
\begin{center}{\en\tt plotimplicit(x\verb|^|2+y\verb|^|2-1,[x,y],xstep=0.2,ystep=0.3)}\end{center}
ή :
\begin{center}{\en\tt plotimplicit(x\verb|^|2+y\verb|^|2-1,[x,y], xstep=0.2,ystep=0.3,unfactored)}\end{center}
Έξοδος :
\begin{center}{\tt Ο μοναδιαίος κύκλος}\end{center}
Είσοδος :
\begin{center}{\en\tt plotimplicit(x\verb|^|2+y\verb|^|2-1,x=-2..2,y=-2..2, xstep=0.2,ystep=0.3)}\end{center}
Έξοδος :
\begin{center}{\tt Ο μοναδιαίος κύκλος}\end{center}

\subsection{Τρισδιάστατη ({\tt\textlatin{3D}}) πεπλεγμένη επιφάνεια}\label{sec:implicitplot3}
\begin{itemize}
\item {\en\tt plotimplicit(f(x,y,z),x,y,z)} σχεδιάζει την γραφική παράσταση
της επιφάνειας που ορίζεται από μια πεπλεγμένη εξίσωση $f(x,y,z)=0$, 
\item {\en\tt plotimplicit(f(x,y,z),x=0..1,y=-1..1,z=-1..1)} σχεδιάζει την επιφάνεια
που ορίζεται από την πεπλεγμένη εξίσωση $f(x,y,z)=0$, 
όπου $0\leq x \leq 1$, $-1\leq y \leq 1$ και $-1\leq z \leq 1$.
\end{itemize}
Είναι επίσης δυνατό να προσθέσουμε 3 ορίσματα για να ορίσουμε τα βήματα διακριτοποίησης
που χρησιμοποιούνται για τα $x$, $y$ και $z$ με {\en\tt xstep=...}, {\en\tt ystep=...} και 
{\en\tt zstep=...}\\
Είσοδος :
\begin{center}{\en\tt plotimplicit(x\verb|^|2+y\verb|^|2+z\verb|^|2-1,x,y,z, xstep=0.2,ystep=0.1,zstep=0.3)}\end{center}
ή :\\
Είσοδος :
\begin{center}{\en\tt plotimplicit(x\verb|^|2+y\verb|^|2+z\verb|^|2-1,x,y,z, xstep=0.2,ystep=0.1,zstep=0.3,unfactored)}\end{center}
Έξοδος :
\begin{center}{\tt Η μοναδιαία σφαίρα}\end{center}
Είσοδος :
\begin{center}{\en\tt plotimplicit(x\verb|^|2+y\verb|^|2+z\verb|^|2-1,x=-1..1,y=-1..1,z=-1..1)}\end{center}
Έξοδος:
\begin{center}{\tt Η μοναδιαία σφαίρα}\end{center}

\section{Παραμετρικές καμπύλες και επιφάνειες : \\{\tt\textlatin{ plotparam paramplot DrawParm}}}\index{plotparam|textbf}\index{paramplot|textbf}\index{DrawParm|textbf}
\subsection{Διδιάστατη ({\tt\textlatin{2D}}) παραμετρική καμπύλη }
\noindent{{\en\tt plotparam([f(t),g(t)],t)}
ή {\en\tt plotparam(f(t)+i*g(t),t)} (αντιστ. \\
{\en\tt plotparam(f(t)+i*g(t),t=t1..t2)})
σχεδιάζει την παραμετρική παράσταση της καμπύλης
που ορίζεται από τα $x=f(t),y=g(t)$ 
με τo προεπιλεγμένο πεδίο τιμών του $t$ (αντιστ. για $t1 \leq t\leq t2$).\\
Το προεπιλεγμένο πεδίο τιμών παίρνεται όπως ορίζεται στις Ρυθμίσεις Γραφικών
(από το μενού Ρυθμίσεις).
{\en\tt plotparam} δέχεται ένα προαιρετικό όρισμα για να καθορίσει το βήμα διακριτοποίησης
για το $t$ με {\en\tt tstep=}.}\\ 
Είσοδος :
\begin{center}{\en\tt plotparam(cos(x)+i*sin(x),x) }\end{center}
ή :
\begin{center}{\en\tt plotparam([cos(x),sin(x)],x) }\end{center}
Έξοδος :
\begin{center}{\tt O μοναδιαίος κύκλος}\end{center}
Στις Ρυθμίσεις Γραφικών
(από το μενού Ρυθμίσεις) το {\en\tt t} κινείται από -4 μέχρι 1. \\Είσοδος :
\begin{center}{\en\tt plotparam(sin(t)+i*cos(t))}\end{center}
ή :
\begin{center}{\en\tt plotparam(sin(t)+i*cos(t),t=-4..1) }\end{center}
ή :
\begin{center}{\en\tt plotparam(sin(x)+i*cos(x),x=-4..1) }\end{center}
Έξοδος :
\begin{center}{\tt το τόξο {\en\tt (sin(-4)+i*cos(-4),sin(1)+i*cos(1))} του μοναδιαίου κύκλου}\end{center}
Στις Ρυθμίσεις Γραφικών (από το μενού Ρυθμίσεις)  το  {\en\tt t} κινείται από -4 to 1.\\ Είσοδος :
\begin{center}{\en\tt plotparam(sin(t)+i*cos(t),t,tstep=0.5)}\end{center}
ή :
\begin{center}{\en\tt plotparam(sin(t)+i*cos(t),t=-4..1,tstep=0.5)}\end{center}
Έξοδος :
\begin{center}{\tt Ένα πολύγωνο που προσεγγίζει το τόξο {\en\tt (sin(-4)+i*cos(-4),sin(1)+i*cos(1))} του μοναδιαίου κύκλου}\end{center}

\subsection{Τρισδιάστατη ({\tt\textlatin{3D}}) παραμετρική επιφάνεια : \\{\tt\textlatin{ plotparam paramplot DrawParm}}}\index{plotparam}\index{paramplot}\index{DrawParm}
\noindent{{\en\tt plotparam} παίρνει δύο κύρια ορίσματα,
μια λίστα τριών  
παραστάσεων ως προς δύο μεταβλητές και την λίστα των ονομάτων αυτών των μεταβλητών
όπου κάθε όνομα μεταβλητής μπορεί να αντικατασταθεί από την εξίσωση {\tt μεταβλητή=διά\-στη\-μα}
για τον ορισμό των πεδίων τιμών  των παραμέτρων.
Δέχεται ένα προαιρετικό όρισμα για τον ορισμό 
των βημάτων διακριτοποίησης των παραμέτρων $u$ και $v$ με 
{\en\tt ustep=...} και  {\en\tt vstep=...}\\
{\en\tt plotparam([f(u,v),g(u,v),h(u,v)],[u,v])} σχεδιάζει την επιφάνεια που ορίζεται από 
το πρώτο όρισμα : $x=f(u,v),y=g(u,v),z=h(u,v)$, όπου $u$ και $v$
κυμαίνονται στα πεδία που προεπιλέγονται στις Ρυθμίσεις Γραφικών.}\\
Είσοδος :
\begin{center}{\en\tt plotparam([v*cos(u),v*sin(u),v],[u,v])}\end{center}
Έξοδος :
\begin{center}{\tt Ο κώνος $x=v*\cos(u),y=v*\sin(u),z=v$}\end{center}
Για να ορίσουμε το πεδίο τιμών κάθε παραμέτρου, αντικαθιστούμε κάθε μεταβλητή
με μια εξίσωση {\tt μεταβλητή=διά\-στη\-μα} , όπως εδώ:
\begin{center}{\en\tt plotparam([v*cos(u),v*sin(u),v],[u=0..pi,v=0..3]) }\end{center}
Έξοδος:
\begin{center}{\tt Τμήμα του κώνου $x=v*\cos(u),y=v*\sin(u),z=v$}\end{center}
Είσοδος :
\begin{center}{\en\tt plotparam([v*cos(u),v*sin(u),v],[u=0..pi,v=0..3],\\ustep=0.5,vstep=0.5)}\end{center}
Έξοδος :
\begin{center}{\tt  Τμήμα του κώνου $x=v*\cos(u),y=v*\sin(u),z=v$}\end{center}
 
\section{Καμπύλη που ορίζεται από πολικές συντεταγμένες : \\{\tt\textlatin{ plotpolar polarplot DrawPol courbe\_polaire}}}\index{plotpolar|textbf}\index{polarplot|textbf}\index{DrawPol|textbf}\index{courbe\_polaire|textbf}
\noindent{Έστω $E_t$ μια παράσταση που εξαρτάται από την μεταβλητή $t$.\\
{\en\tt plotpolar($E_t$,t)} σχεδιάζει την πολική παράσταση της
καμπύλης που ορίζεται από $\rho=E_t$ για $\theta=t$, και που στις
καρτεσιανές συντενταγμένες είναι η καμπύλη $(E_t \cos(t),E_t \sin(t))$.
Το πεδίο τιμών της παραμέτρου μπορεί να οριστεί αντικαθιστώντας το δεύτερο όρισμα
με {\en\tt t=tmin..tmax}. Η παράμετρος διακριτοποίησης μπορεί να οριστεί
από ένα προαιρετικό όρισμα {\en\tt tstep=...}.}\\
Είσοδος:
\begin{center}{\en\tt  plotpolar(t,t)}\end{center}
Έξοδος :
\begin{center}{\tt Η σπείρα $\rho$=\en t}\end{center}
Είσοδος:
\begin{center}{\en\tt  plotpolar(t,t,tstep=1)}\end{center}
ή :
\begin{center}{\en\tt  plotpolar(t,t=0..10,tstep=1)}\end{center}
Έξοδος :
\begin{center}{\tt Ένα πολύγωνο που προσεγγίζει την σπείρα $\rho$=\en t}\end{center}

\section{Γράφος αναδρομικής ακολουθίας : \\{\tt\textlatin{ plotseq seqplot graphe\_suite}}}\index{plotseq}\index{seqplot}\index{graphe\_suite}\label{sec:plotseq}
\noindent Έστω $f(x)$ μια παράσταση που εξαρτάται από την μεταβλητή $x$ 
(αντίστ. $f(t)$ μια παράσταση που εξαρτάται από την μεταβλητή $t$).\\
{\en\tt plotseq($f(x)$,a,n)} (αντιστ. {\en\tt plotseq($f(t)$,t=a,n)}) σχεδιάζει την γραμμή  
$y=x$, τον γράφο της $y=f(x)$ (αντιστ $y=f(t)$) και τους $n$ πρώτους όρους της
αναδρομικής ακολουθίας που ορίζεται από: $u_0=a,\ \ u_n=f(u_{n-1})$.
Η τιμή $a$ μπορεί να αντικατασταθεί από μια λίστα τριών στοιχείων, $[a,x_-,x_+]$
όπου $x_-..x_+$ θα περαστεί σαν το πεδίο τιμών του $x$  για τον υπολογισμό του γράφου.\\ 
Είσοδος :
\begin{center}{\en\tt plotseq(sqrt(1+x),x=[3,0,5],5)}\end{center}
Έξοδος :
\begin{center}{\tt Ο γράφος της {\en\tt y=sqrt(1+x)}, της {\en\tt y=x} και των 5 πρώτων όρων της ακολουθίας {\en\tt u\_0=3} και {\en\tt u\_n=sqrt(1+u\_(n-1))}}\end{center}

\section{Πεδίο κλίσεων : {\tt\textlatin{ plotfield fieldplot}}}\index{plotfield}\index{fieldplot}
\begin{itemize}
\item Εάν $f(t,y)$ είναι μια παράσταση ως προς τις δύο μεταβλητές $t$ και $y$, 
τότε:
\begin{center}
 {\en\tt plotfield(f(t,y),[t,y])}
\end{center} 
σχεδιάζει το πεδίο κλίσεων της 
διαφορικής εξίσωσης $y'=f(t,y)$ όπου $y$ είναι μια πραγματική μεταβλητή και
όπου $t$ είναι η τετμημένη,
\item Εάν $V$ είναι ένα  
διάνυσμα δύο παραστάσεων ως προς τις δύο μεταβλητές $x,y$ που είναι ανεξάρτητες 
του χρόνου  $t$, τότε
\begin{center}
{\en\tt plotfield(V,[x,y])}
\end{center}
σχεδιάζει το διανυσματικό πεδίο $V$,
\item Το πεδίο  τιμών των $t,y$ ή των $x,y$ μπορεί να ορισθεί θέτοντας\\
{\en\tt t=tmin..tmax}, {\en\tt x=xmin..xmax}, {\en\tt y=ymin..ymax}\\
στην θέση του ονόματος της μεταβλητής.
\item Η διακριτοποίηση μπορεί να οριστεί με προαιρετικά
ορίσματα {\en\tt xstep=...}, {\en\tt ystep=...}
\end{itemize}
Είσοδος :
\begin{center}{\en\tt plotfield(4*sin(t*y),[t=0..2,y=-3..7]) }\end{center}
Έξοδος :
\begin{center}{\tt Ευθύγραμμα τμήματα με κλίση {\en\tt 4*sin(t*y)}, που αναπαριστούν εφαπτομένες, σχεδιάζονται σε διαφορετικά σημεία }\end{center}
\begin{center}{\en\tt plotfield(5*[-y,x],[x=-1..1,y=-1..1]) }
\end{center}

\section{Γράφος μιας λύσης μιας  διαφορικής εξίσωσης : \\{\tt\textlatin{ plotode odeplot}}}\index{plotode}\index{odeplot}
\noindent{Έστω $f(t,y)$ μια παράσταση ως προς δύο  μεταβλητές
$t$ και $y$.}
\begin{itemize}
\item {\en\tt plotode($f(t,y)$,[t,y],[t0,y0])} σχεδιάζει την λύση
της διαφορικής εξίσωσης $y'=f(t,y)$ που περνάει από το
σημείο {\en\tt (t0,y0)} (δηλαδή, ισχύει $y(t_0)=y_0$)
\item
Από προεπιλογή, $t$ πηγαίνει σε δύο κατευθύνσεις. Το πεδίο τιμών της $t$
μπορεί να οριστεί από το προαιρετικό όρισμα
{\en\tt t=tmin..tmax}.
\item
Μπορούμε επίσης να παρουσιάσουμε, στο χώρο ή στο επίπεδο,
την λύση της διαφορικής εξίσωσης 
$y'=f(t,y)$ όπου $y=(X,Y)$ είναι ένα διάνυσμα μεγέθους 2.
Απλώς αντικαταστήστε το $y$ με τα ονόματα των μεταβλητών $X,Y$
και την αρχική τιμή $y_0$ με τις δύο αρχικές τιμές των μεταβλητών στον χρόνο $t_0$.
\end{itemize}
Είσοδος :
\begin{center}{\en\tt plotode(sin(t*y),[t,y],[0,1]) }\end{center}
Έξοδος :
\begin{center}{\tt Ο γράφος της λύσης της {\en y$'$=sin(t,y)} που περνάει \\από το σημείο (0,1)}\end{center}
Είσοδος~:
\begin{center}
{\en\tt S:=odeplot([h-0.3*h*p, 0.3*h*p-p], [t,h,p],[0,0.3,0.7])}
\end{center}
Έξοδος, ο γράφος, στον χώρο, της λύσης της :
\[ [h,p]'=[h-0.3 h*p, 0.3 h*p-p] \quad [h,p](0)=[0.3,0.7] \]
Για να έχετε ένα διδιάστατο ({\en\tt 2-d}) γράφο (στο επίπεδο), χρησιμοποιήστε την επιλογή 
{\en\tt plane}
\begin{center}
{\en\tt S:=odeplot([h-0.3*h*p, 0.3*h*p-p], [t,h,p],[0,0.3,0.7],plane)}
\end{center}


\section{Αλληλεπιδραστικός γράφος λύσεων μιας διαφορικής εξίσωσης: \\{\tt\textlatin{ interactive\_plotode interactive\_odeplot}}}\index{interactive\_plotode}\index{interactive\_odeplot}
\noindent{Έστω $f(t,y)$ μια παράσταση ως προς τις δύο  
μεταβλητές $t$ και $y$.\\
{\en\tt interactive\_plotode(f(t,y),[t,y])} σχεδιάζει το πεδίο κλίσεων
της διαφορικής εξίσωσης $y'=f(t,y)$ σε ένα νέο παράθυρο. 
Στο παράθυρο αυτό, μπορεί κανείς να επιλέξει ένα σημείο (κάνοντας κλικ με το ποντίκι)  για να πάρει τον
γράφο της λύσης της  $y'=f(t,y)$ που περνά από αυτό το σημείο.\\
Μπορείτε να επιλέξετε περισσότερα σημεία για να εμφανίσετε 
περισσότερες λύσεις. Για να σταματήσετε πιέστε το
κουμπί {\en\tt Esc}.}\\
Είσοδος:
\begin{center}
{\en\tt interactive\_plotode(sin(t*y),[t,y])}
\end{center}
Έξοδος :
\begin{center}{\tt Το πεδίο κλίσεων της {\en\tt y$'$=sin(t,y)} με τις λύσεις που περνάνε από τα σημεία που ορίζονται με κλικ του ποντικιού. }\end{center}

\section{Εφέ κίνησης  γράφων (\tt\textlatin{2D, 3D} ή \tt\textlatin{"4D"})}
Το {\en\tt Xcas} μπορεί να εμφανίσει εφέ κίνησης  γράφων {\en\tt 2D, 3D} ή {\en\tt "4D"}. 
Αυτό γίνεται πρώτα υπολογίζοντας
μια ακολουθία γραφικών αντικειμένων, και μετά την ολοκλήρωση,
εμφανίζοντας την ακολουθία σε βρόχο.
\begin{itemize} 
\item Για να σταματήσετε ή να ξεκινήσετε (ξανά) το εφέ κίνησης, κάντε κλικ στο κουμπί 
$\blacktriangleright \mid$ (δεξιά από το παράθυρο γραφικών και δεξιά του Μενού Γραφικών {\en\tt M}).
\item
Ο χρόνος εμφάνισης κάθε γραφικού αντικειμένου ορίζεται στο {\tt{\en "}Εφέ κί\-νη\-σης{\en "}} του Μενού Γραφικών ({\en\tt M}). Επιλέξτε {\tt {\en "}Πιο γρήγορα{\en "}}, 
για να πάρετε ένα γρήγορο εφέ κίνησης.
\item
Εάν το εφέ κίνησης είναι σταματημένο, 
μπορείτε να δείτε την ακολουθία των αντικειμένων μία προς μία
κάνοντας κλικ με το ποντίκι μέσα στο γραφικό περιβάλλον.
\end{itemize}

\subsection{Εφέ κίνησης διδιάστατου ({\tt\textlatin{2D}}) γράφου~: {\tt\textlatin{ animate}}}\index{animate}
\noindent{{\en\tt animate} μπορεί να δημιουργήσει διδιάστατο εφέ κίνησης  με γράφους συναρτήσεων
μιας παράμετρου. Η παράμετρος ορίζεται σαν το τρίτο όρισμα
της εντολής
{\en\tt animate}, ενώ ο αριθμός των εικόνων σαν τέταρτο όρισμα με
{\en\tt frames=}\index{frames@{\sl frames}|textbf} {\tt αριθμός}. 
Tα υπόλοιπα ορίσματα είναι ίδια με εκείνα της εντολής {\en\tt plot}, 
βλέπε \ref{sec:plot2d}, σελ. \pageref{sec:plot2d}.}\\
Είσοδος :
\begin{center}
{\en\tt animate(sin(a*x),x=-pi..pi,a=-2..2,frames=10,color=red)}
\end{center}
Έξοδος:
\begin{center}{\tt μια ακολουθία γραφικών αναπαραστάσεων της {\en\tt y=sin($a$x)} για 
11 τιμές του $a$ μεταξύ -2 και 2}\end{center}

\subsection{Εφέ κίνησης τρισδιάστατου ({\tt\textlatin{3D}}) γράφου~: \\{\tt\textlatin{ animate3d}}}\index{animate3d}
\noindent{{\en\tt animate3d} μπορεί να δημιουργήσει τρισδιάστατο εφέ κίνησης με 
γράφους συναρτήσεων  μιας παράμετρου. Η παράμετρος ορίζεται σαν
το τρίτο όρισμα της εντολής {\en\tt animate3d}, ενώ ο αριθμός των εικόνων
σαν τέταρτο όρισμα με
{\en\tt frames=}\index{frames@{\sl frames}}{\tt α\-ριθμός}. Τα υπόλοιπα ορίσματα είναι ίδια με εκείνα της εντολής {\en\tt plotfunc}, βλέπε \ref{sec:plotfunc3}, σελ. \pageref{sec:plotfunc3}.}\\
Είσοδος :
\begin{center}
{\en\tt animate3d(x\verb|^|2+a*y\verb|^|2,[x=-2..2,y=-2..2],a=-2..2, frames=10,display=red+filled)}
\end{center}
Έξοδος :
\begin{center}{\tt μία ακολουθία γραφικών αναπαραστάσεων της {\en z=x\verb|^|2+$a$*y\verb|^|2} για 11 τιμές του $a$ μεταξύ -2 και 2}
\end{center}

\subsection{Εφέ κίνησης μιας ακολουθίας γραφικών αντικειμένων~: {\tt\textlatin{ animation}}}\index{animation}
\noindent{{\en\tt animation} δημιουργεί εφέ κίνησης σε μια
ακολουθία γραφικών αντικειμένων
με έναν ορισμένο  χρόνο εμφάνισης στην οθόνη. Η ακολουθία των αντικειμένων εξαρτάται τις πιο πολλές φορές από μία παραμέτρου και ορίζεται χρησιμοποιώντας την εντολή {\en\tt seq} 
αλλά δεν είναι υποχρεωτικό.\\
{\en\tt animation} παίρνει σαν όρισμα την ακολουθία γραφικών αντικειμένων.\\
Για να ορίσουμε μια ακολουθία γραφικών αντικειμένων με {\en\tt seq},
εισάγουμε τον ορισμό του γραφικού αντικειμένου (που εξαρτάται από
την παράμετρο), το όνομα της παραμέτρου, την ελάχιστη τιμή της, την μέγιστη 
τιμή της και προαιρετικά μια τιμή βήματος.}\\
Είσοδος :
\begin{center}{\en\tt animation(seq(plotfunc(cos(a*x),x),a,0,10))}\end{center}
Έξοδος :
\begin{center}{\tt Η ακολουθία των καμπύλων που ορίζονται από την $y=\cos(ax)$,\\ για $a=0,1,2,\dots,10$}\end{center}
Είσοδος:
\begin{center}
{\en\tt animation(seq(plotfunc(cos(a*x),x),a,0,10,0.5))}
\end{center}
ή
\begin{center}
{\en\tt animation(seq(plotfunc(cos(a*x),x),a=0..10,0.5))}
\end{center}
Έξοδος :
\begin{center}{\tt  Η ακολουθία των καμπύλων που ορίζονται από την $y=\cos(ax)$, \\για $a=0,0.5,1,1.5,\dots,10$ }\end{center}
Είσοδος :
\begin{center}{\en\tt animation(seq(plotfunc([cos(a*x),sin(a*x)],x=0..2*pi/a), a,1,10))}\end{center}
Έξοδος :
\begin{center}{\tt Η ακολουθία των δύο καμπύλων που ορίζονται από την $y=\cos(ax)$ και την $y=\sin(ax)$, για $a=1..10$ και για $x=0..2\pi/a$ }\end{center}
Είσοδος :
\begin{center}{\en\tt animation(seq(plotparam([cos(a*t),sin(a*t)], t=0..2*pi),a,1,10))}\end{center}
Έξοδος :
\begin{center}{\tt Η ακολουθία των παραμετρικών καμπύλων που ορίζονται από την $x=\cos(at)$ και την $y=\sin(at)$, για $a=1..10$ και για $t=0..2\pi$ }\end{center}
Είσοδος :
\begin{center}{\en\tt animation(seq(plotparam([sin(t),sin(a*t)], t,0,2*pi,tstep=0.01),a,1,10))}\end{center}
Έξοδος :
\begin{center}{\tt Η ακολουθία των παραμετρικών καμπύλων που ορίζονται από τις $x=\sin(t),y=\sin(at)$, για $a=0..10$ και $t=0..2\pi$}\end{center}
Είσοδος :
\begin{center}{\en\tt animation(seq(plotpolar(1-a*0.01*t\verb|^|2, t,0,5*pi,tstep=0.01),a,1,10))}\end{center}
Έξοδος :
\begin{center}{\tt Η ακολουθλια των πολικών καμπύλων που ορίζονται από την $\rho=1-a*0.01*t^2$, για $a=0..10$ και $t=0..5\pi$}\end{center}
Είσοδος :
\begin{center}{\en\tt plotfield(sin(x*y),[x,y]); animation(seq(plotode(sin(x*y),[x,y],[0,a]),a,-4,4,0.5))}\end{center}
Έξοδος :
\begin{center}{\tt Το πεδίο κλίσεων της $y'=sin(xy)$ και η ακολουθία των ολοκληρωτικών καμπύλων που περνάνε από τα σημεία $(0,a)$ για $a$=-4,-3.5,\dots,3.5,4}\end{center}
Είσοδος :
\begin{center}{\en\tt animation(seq(display(square(0,1+i*a),filled),a,-5,5))}\end{center}
Έξοδος :
\begin{center}{\tt Η ακολουθία των τετραγώνων που ορίζονται από τα σημεία 0 και {\en\tt 1+i}*$a$ για $a=-5..5$}\end{center}
Είσοδος :
\begin{center}{\en\tt animation(seq(line([0,0,0],[1,1,a]),a,-5,5))}\end{center}
Έξοδος :
\begin{center}{\tt Η ακολουθία των γραμμών που ορίζονται από τα σημεία [0,0,0] και [1,1,$a$] για $a=-5..5$}\end{center}
Είσοδος :
\begin{center}{\en\tt animation(seq(plotfunc(x\verb|^|2-y\verb|^|a,[x,y]),a=1..3))}\end{center}
Έξοδος :
\begin{center}{\tt Η ακολουθία των {\en\tt "3D"} επιφανειών που ορίζονται από την $x^2-y^a$, για $a=1..3$}\end{center}
Είσοδος :
\begin{center}{\en\tt animation(seq(plotfunc((x+i*y)\verb|^|a,[x,y], display=filled),a=1..10)}\end{center}
Έξοδος :
\begin{center}{\tt Η ακολουθία των {\en\tt "4D"} επιφανειών που ορίζονται από την $(x+i*y)^a$, για $a=0..10$ με τα χρώματα του ουράνιου τόξου} \end{center}

{\bf Σχόλιο}
Μπορούμε επίσης να ορίσουμε την ακολουθία των αντικειμένων με ένα προγραμμα. Για 
 παράδειγμα, αν θέλουμε να σχεδιάσουμε τα
ευθύγραμμα τμήματα μήκους $1,\sqrt 2,\dots,\sqrt 20$ που κατασκευάζονται με ένα
ορθογώνιο  τρίγωνο πλευράς 1 και το προηγούμενο τμήμα γράφουμε το ακόλουθο πρόγραμμα
(σημειώστε ότι υπάρχει η εντολή {\en\tt c:=evalf(..)} 
για να εξαναγκάσει προσεγγιστικούς υπολογισμούς, αλλιώς ο χρόνος υπολογισμού θα ήταν πολύ 
μεγάλος):
{\en\tt
\begin{verbatim}
 seg(n):={
 local a,b,c,j,aa,bb,L;
 a:=1;
 b:=1;
 L:=[point(1)];
 for(j:=1;j<=n;j++){
  L:=append(L,point(a+i*b));
  c:=evalf(sqrt(a^2+b^2));
  aa:=a;
  bb:=b;
  a:=aa-bb/c;
  b:=bb+aa/c;
 }
 L;}:;
\end{verbatim}
}
Έπειτα εισάγετε : 
\begin{center}{\en\tt animation(seg(20))}\end{center}
Βλέπουμε, κάθε σημείο, ένα προς ένα.\\
ή :
\begin{center}{\en\tt L:=seg(20); s:=segment(0,L[k])\$(k=0..20)}\end{center}
Βλέπουμε 21 τμήματα. Έπειτα, εισάγετε:
\begin{center}{\en\tt animation(s)}\end{center}
Βλέπουμε, κάθε ευθύγραμμο τμήμα, ένα προς ένα.



\chapter{Αριθμητικοί υπολογισμοί }\label{sec:numeric}
Οι πραγματικοί αριθμοί μπορεί να έχουν {\tt ακριβή} αναπαράσταση
(π.χ. ρη\-τοί, συμβολικές παραστάσεις
που περιλαμβάνουν τετραγωνικές ρίζες ή σταθερές όπως $\pi$, ...)
ή {\tt προσεγγιστική} αναπαράσταση, που σημαίνει ότι ο πραγματικός
αναπαρίσταται από έναν ρητό (με  παρονομαστή που είναι μια
δύναμη της βάσης της αναπαράστασης) που είναι κοντά στον πραγματικό.
Στο {\en\tt Xcas}, ο συνήθης επιστημονικός συμβολισμός χρησιμοποιείται για την
προσεγγιστική αναπαράσταση, δηλαδή το κλασματικό μέρος  ({\en\tt mantissa})
 (με μία {\tt τελίτσα} για δεκαδικό οριοθέτη, αντί για κόμμα που χρησιμοποιούμε στην Ελλάδα) που ακολουθείται προαιρετικά από το γράμμμα {\en\tt e}
και έναν ακέραιο εκθέτη.

Σημειώστε ότι ο πραγματικός αριθμός $10^{-4}$ είναι ένας ακριβής αριθμός αλλά ο
$1e-4$ είναι η προσεγγιστική αναπαράσταση του πρώτου.

\section{Αναπαράσταση κινητής υποδιαστολής.}
Σε αυτήν την ενότητα, θα δούμε πως αναπαρίστανται οι πραγματικοί αριθμοί.

\subsection{{\tt\textlatin{Digits}}}
Η μεταβλητή  {\en\tt Digits} χρησιμοποιείται για να ελέγξουμε πως αναπαρίστανται οι
πραγματικοί αριθμοί και επίσης πως εμφανίζονται.
Όταν ο ορισμένος αριθμός
των ψηφίων είναι μικρότερος ή ίσος του 14 (για παράδειγμα {\en\tt
  Digits:=14}), τότε οι αριθμοί μηχανής κινητής υποδιαστολής χρησιμοποιούνται και εμφανίζονται  χρησιμοποιώντας τον συγκεκριμένο
αριθμό ψηφίων.
Όταν η  {\en\tt Digits} είναι μεγαλύτερη από 14, το {\en\tt Xcas} χρησιμοποιεί την βιβλιοθήκη {\en\tt MPFR}, η αναπαράσταση είναι ίδια με αυτή των αριθμών μηχανής κινητής υποδιαστολής  αλλά ο αριθμός των {\en\tt bits} της
{\en\tt mantissa} δεν είναι σταθερός και το πεδίο τιμών των εκθετών είναι πολύ μεγαλύτερο.
Πιο συγκεκριμένα, ο αριθμός των {\en\tt bits} της {\en\tt mantissa} που δημιουργείται στην {\en\tt MPFR} 
είναι {\en\tt ceil(Digits*log(10)/log(2))}.

Σημειώστε ότι αν αλλάξετε την τιμή της {\en\tt Digits}, αυτό θα επηρεάσει
την δημιουργία νέων πραγματικών αριθμών που  μεταγλωττίζονται από γραμμές εντολών 
ή από προγράμματα ή από εντολές όπως {\en\tt approx}, αλλά δεν θα επηρεάσει
τους υπάρχοντες πραγματικούς αριθμούς. Γι' αυτό, οι αριθμοί μηχανής κινητής υποδιαστολής μπορεί να συνυπάρχουν
με τους  αριθμούς κινητής υποδιαστολής της {\en\tt MPFR}, και ακόμα ανάμεσα σε αυτούς τους αριθμούς κινητής υποδιαστολής της {\en\tt MPFR}, κάποιοι μπορεί να έχουν {\en\tt mantissa} με 100 {\en\tt bits} και κάποιοι μπορεί να έχουν {\en\tt mantissa} με 150 {\en\tt bits}. Εάν οι πράξεις
αναμιγνύουν διαφορετικά είδη αριθμών κινητής υποδιαστολής, τα πιο ακριβή είδη αριθμών κινητής υποδιαστολής
μετατρέπονται στο λιγότερο ακριβές είδος αριθμών κινητής υποδιαστολής.

\subsection{Αναπαράσταση με  αριθμούς μηχανής κινητής \\υποδιαστολής}
Ένας πραγματικός αναπαρίσταται από έναν αριθμό κινητής υποδιαστολής $d$, όπου
\[ d=2^\alpha*(1+m),  \quad 0<m<1, \quad -2^{10} < \alpha < 2^{10} \]
Εάν $\alpha>1-2^{10}$, τότε $m \geq 1/2$, και $d$ είναι ένας
κανονικοποιημένος αριθμός κινητής υποδιαστολής, αλλιώς ο
$d$ είναι μη κανονικοποιημένος ($\alpha=1-2^{10}$). Ο ειδικός εκθέτης $2^{10}$
χρησιμοποιείται για να αναπαραστήσει +  ή - άπειρο και {\en\tt NaN (Not a Number)}.
Ένας αριθμός μηχανής κινητής υποδιαστολής αποτελείται από 64 {\en\tt bits}:
\begin{itemize}
\item  το πρώτο {\en\tt bit} είναι για το πρόσημο του $d$ (0 για '+' και 1 για '-')
\item  τα  επόμενα 11 {\en\tt bits} αναπαριστούν τον εκθέτη, πιο συγκεκριμένα
εάν $\alpha$ είναι ο ακέραιος με αυτά τα  11 {\en\tt bits},
ο εκθέτης είναι $\alpha+2^{10}-1$, 
\item  τα τελευταία 52 {\en\tt bits} αναπαριστούν την {\en\tt mantissa} $m$, πιο συγκεκριμένα εάν 
$M$ είναι ο  ακέραιος με αυτά τα 52 {\en\tt bits}, τότε
$m=1/2+M/2^{53}$ για κανονικοποιημένους αριθμούς κινητής υποδιαστολής( {\en\tt floats}) και $m=M/2^{53}$ για
μη κανονικοποιημένους αριθμούς κινητής υποδιαστολής.
\end{itemize}
Παραδείγματα αναπαράστασης του εκθέτη:
\begin{itemize}
\item $\alpha=0$ κωδικοποιείται ως 011 1111 1111
\item $\alpha=1$ κωδικοποιείται ως 100 0000 0000
\item $\alpha=4$ κωδικοποιείται ως 100 0000 0011
\item $\alpha=5$ κωδικοποιείται ως 100 0000 0100
\item $\alpha=-1$ κωδικοποιείται ως 011 1111 1110
\item $\alpha=-4$ κωδικοποιείται ως 011 1111 1011
\item $\alpha=-5$ κωδικοποιείται ως 011 1111 1010
\item $\alpha=2^{10}$ κωδικοποιείται ως 111 1111 1111
\item $\alpha=2^{-10}-1$ κωδικοποιείται ως 000 0000 000
\end{itemize}
{\bf Σχόλιο}: $2^{-52}=0.2220446049250313e-15$

\subsection{Παραδείγματα αναπαράστασεων κανονικοποιημένων αριθμών κινητής υποδιαστολής}
\begin{itemize}
\item 3.1 :\\
Έχουμε :
\begin{eqnarray*}
3.1&=&2*(1+\frac{1}{2}+\frac{1}{2^5}+\frac{1}{2^6}+
\frac{1}{2^9}+\frac{1}{2^{10}}+....)\\
&=&2*(1+\frac{1}{2}+\sum_{k=1}^\infty(\frac{1}{2^{4*k+1}}+\frac{1}{2^{4*k+2}}) ) 
\end{eqnarray*}
και επομένως $\alpha=1$ και 
$m=\frac{1}{2}+\sum_{k=1}^\infty(\frac{1}{2^{4*k+1}}+\frac{1}{2^{4*k+2}})$.
Συνεπώς, η δεκαεξαδική και η δυαδική αναπαράσταση του 3.1 είναι:
{\en\tt
\begin{verbatim} 
40 (01000000), 8 (00001000), cc (11001100), cc (11001100), 
cc (11001100), cc (11001100), cc (11001100), cd (11001101),
\end{verbatim}
}
η τελευταία οκτάδα λήγει σε 1101, το τελευταίο {\en\tt bit} είναι 1, επειδή το ακόλουθο
ψηφίο είναι 1 (στρογγυλοποίηση προς τα πάνω).
\item  3. :\\
Έχουμε $3=2*(1+1/2)$.
Επομένως η δεκαεξαδική και η δυαδική αναπαράσταση του 3 είναι:
{\en\tt
\begin{verbatim} 
40 (01000000), 8 (00001000), 0 (00000000), 0 (00000000), 
0 (00000000), 0 (00000000), 0 (00000000), 0 (00000000) 
\end{verbatim}
}
\end{itemize}

\subsection{Διαφορά μεταξύ των αναπαραστάσεων της (3.1-3) και του 0.1}
\begin{itemize}
\item αναπαράσταση του  0.1 :\\
Έχουμε :
\[ 0.1=2^{-4}*(1+\frac{1}{2}+\frac{1}{2^4}+\frac{1}{2^5}+
\frac{1}{2^8}+\frac{1}{2^9}+...)=
2^{-4}*\sum_{k=0}^\infty (\frac{1}{2^{4*k}}+\frac{1}{2^{4*k+1}}) \]
και επομένως $\alpha=1$ και $m=\frac{1}{2}+
\sum_{k=1}^\infty (\frac{1}{2^{4*k}}+\frac{1}{2^{4*k+1}})$.
Συνεπώς, η αναπαράσταση του 0.1 είναι
{\en\tt 
\begin{verbatim}
3f (00111111), b9 (10111001), 99 (10011001), 99 (10011001),
99 (10011001), 99 (10011001), 99 (10011001), 9a (10011010), 
\end{verbatim}
}
η τελευταία οκτάδα λήγει σε 1010, πράγματι τα 2 τελεύταια ψηφία 
01 έγιναν 10  επειδή το ακόλουθο ψηφίο είναι 1 (στρογγυλοποίηση προς τα πάνω).

\item αναπαράσταση της διαφοράς {\en\tt a:=3.1-3} :\\
Ο υπολογισμός του {\en\tt a} γίνεται προσαρμόζοντας τους εκθέτες (εδώ δεν ισχύει), έπειτα αφαιρώντας την {\en\tt mantissa}, και προσαρμόζοντας τον εκθέτη του αποτελέσματος
για να έχουμε ένα κανονικοποιημένο αριθμό κινητής υποδιαστολής.
Ο εκθέτης είναι $\alpha=-4$ (που αντιστοιχεί στο $2*2^{-5}$) :
τα {\en\tt bits} της {\en\tt mantissa} ολισθαίνουν προς τα αριστερά 5 θέσεις
και έχουμε:
{\en\tt 
\begin{verbatim}
3f (00111111), b9 (10111001), 99 (10011001), 99 (10011001),
99 (10011001), 99 (10011001), 99 (10011001), 9a (10100000),
\end{verbatim}}
Επομένως,
$a>0.1$ και $a-0.1=1/2^{50}+1/2^{51}$ 
(εφόσον 100000-11010=110)
\end{itemize}
{\bf Σχόλιο}\\
Αυτός είναι και ο λόγος που
\begin{center}
{\en\tt floor(1/(3.1-3))} 
\end{center}
επιστρέφει {\en\tt 9} και όχι {\en\tt 10} όταν {\en\tt Digits:=14} ή μεγαλύτερο.


\section{Προσεγγιστική αποτίμηση : {\tt \textlatin{evalf approx}} \\και {\tt \textlatin{Digits}}}\index{evalf|textbf}\index{approx|textbf}\index{DIGITS}\index{Digits}
\noindent{{\en\tt evalf} ή {\en\tt approx} υπολογίζει μία αριθμητική προσέγγιση (εάν αυτό είναι δυνατό).}\\
Είσοδος :
\begin{center}{\en\tt evalf(sqrt(2))}\end{center}
Έξοδος, εάν στις Ρυθμίσεις {\en\tt CAS} (από το μενού {\tt Ρυθμίσεις}) έχουμε ορίσει {\en\tt Digits=7} 
(εδώ χρησιμοποιούνται αριθμοί μηχανής κινητής υποδιαστολής  και εμφανίζονται 7 ψηφία) :
\begin{center}{\tt 1.414214}\end{center}
Μπορούμε να αλλάξουμε τον αριθμό των ψηφίων κάνοντας ανάθεση τιμής
στην μεταβλητή {\en\tt DIGITS} ή {\en\tt Digits} στην γραμμή εντολών.
Είσοδος :
\begin{center}{\en\tt DIGITS:=20}\end{center}
\begin{center}{\en\tt evalf(sqrt(2))}\end{center}
Έξοδος :
\begin{center}{\en\tt 1.4142135623730950488}\end{center}
Είσοδος ({\en \tt DIGITS:=7}) : 
\begin{center}{\en\tt evalf(10\verb|^|-5)}\end{center}
Έξοδος :
\begin{center}{\en\tt 1e-05}\end{center}
Είσοδος :
\begin{center}{\en\tt evalf(10\verb|^|15)}\end{center}
Έξοδος :
\begin{center}{\en\tt 1e+15}\end{center}
Είσοδος : 
\begin{center}{\en\tt evalf(sqrt(2))*10\verb|^|-5}\end{center}
Έξοδος :
\begin{center}{\en\tt 1.414214e-05}\end{center}

\section{Αριθμητικοί αλγόριθμοι}
\subsection{Προσεγγιστική επίλυση εξισώσεων : {\tt\textlatin{ newton}}}\index{newton}
\noindent{{\en\tt newton} παίρνει σαν ορίσματα : μια παράσταση {\en\tt ex}, 
το όνομα της 
μεταβλητής αυτής της παράστασης (από προεπιλογή {\en\tt x}), και τρεις τιμές  {\en\tt a} (από προεπιλογή 
{\en\tt a=0}), {\en\tt eps} (από προεπιλογή {\en\tt eps=1e-8}) και {\en\tt nbiter} 
(από προεπιλογή {\en\tt nbiter=12}).\\
{\en\tt newton(ex,x,a,eps,nbiter)} υπολογίζει μια προσεγγιστική λύση
 {\en\tt x} της εξίσωσης {\en\tt ex=0}
χρησιμοποιώντας τον αλγόριθμο {\en\tt Newton} με σημείο εκκίνησης
{\en\tt x=a}. Ο μέγιστος αριθμός επαναλήψεων είναι {\en\tt nbiter}
και η ακρίβεια είναι {\en\tt eps}.}\\
Είσοδος :
\begin{center}{\en\tt newton(x\verb|^|2-2,x,1) }\end{center}
Έξοδος :
\begin{center}{\en\tt 1.414214}\end{center}
Είσοδος :
\begin{center}{\en\tt newton(x\verb|^|2-2,x,-1) }\end{center}
Έξοδος :
\begin{center}{\en\tt -1.414214}\end{center}
Είσοδος :
\begin{center}{\en\tt newton(cos(x)-x,x,0)}\end{center}
Έξοδος :
\begin{center}{\en\tt 0.7390851}\end{center}

\subsection{Προσεγγιστικός υπολογισμός παραγώγων : \\{\tt\textlatin{ nDeriv}}}\index{nDeriv}
\noindent {\en\tt nDeriv} παίρνει σαν ορίσματα : μια παράσταση {\en\tt ex}, το όνομα της
μεταβλητής αυτής της παράστασης (από προεπιλογή {\en\tt x}), και {\en\tt h} (από προεπιλογή
{\en\tt h=0.001}).\\
{\en\tt nDeriv(ex,x,h)} υπολογίζει μια προσεγγιστική τιμή της παραγώγου της παράστασης
 {\en\tt ex} στo σημείο {\en\tt x} και επιστρέφει:
\begin{center}{\en\tt (f(x+h)-f(x+h))/2*h}\end{center}
Είσοδος :
\begin{center}{\en\tt nDeriv(x\verb|^| 2,x)}\end{center}
Έξοδος :
\begin{center}{\en\tt ((x+0.001)\verb|^|2-(x+-0.001)\verb|^|2)*500.0}\end{center}
Είσοδος :
\begin{center}{\en\tt subst(nDeriv(x\verb|^| 2,x),x=1)}\end{center}
Έξοδος :
\begin{center}{\en\tt 2}\end{center}
Είσοδος :
\begin{center}{\en\tt nDeriv(exp(x\verb|^| 2),x,0.00001)}\end{center}
Έξοδος :
\begin{center}{\en\tt (exp((x+1e-05)\verb|^|2)-exp((x+-1e-05)\verb|^|2))*50000}\end{center}
Είσοδος :
\begin{center}{\en\tt subst(exp(nDeriv(x\verb|^| 2),x,0.00001),x=1)}\end{center}
Έξοδος :
\begin{center}{\en\tt 5.43656365783}\end{center}
που είναι η προσεγγιστική τιμή της {\en\tt 2e=5.43656365692}.

\subsection{Προσεγγιστικός υπολογισμός ολοκληρωμάτων : \\{\tt\textlatin{ romberg nInt}}}\index{romberg}\index{nInt}
\noindent{{\en\tt romberg} ή {\en\tt nInt} παίρνει σαν ορίσματα : μία παράσταση  
{\en\tt ex}, το όνομα της μεταβλητής αυτής της παράστασης (από προεπιλογή {\en\tt x}), και 
δύο πραγματικές τιμές {\en\tt a,b}.\\
{\en\tt romberg(ex,x,a,b)} ή {\en\tt nInt(ex,x,a,b)} υπολογίζει μια προσεγγιστική
τιμή του ολοκληρώματος $\int_a^b ex\ dx$ χρησιμοποιώντας την μέθοδο {\en\tt Romberg}. Η προς ολοκλήρωση παράσταση
πρέπει να είναι επαρκώς κανονική για είναι ακριβής η 
προσέγγιση. Διαφορετικά, {\en\tt romberg} επιστρέφει μια λίστα πραγματικών τιμών,
που προέρχεται από την εφαρμογή της μεθόδου
{\en\tt Romberg} (το πρώτο στοιχείο της λίστας είναι προσέγγιση με
τον κανόνα του τραπεζοειδούς, τα επόμενα προέρχονται από τη  εφαρμογή 
του τύπου {\en\tt Euler-Mac Laurin} για την αφαίρεση διαδοχικών αρτίων δυνάμεων του
βήματος του τραπεζοειδούς κανόνα).}\\
Είσοδος :
\begin{center}{\en\tt romberg(exp(x\verb|^|2),x,0,1)}\end{center}
Έξοδος:
\begin{center}{\en\tt 1.462652}\end{center}

\subsection{Προσεγγιστική επίλυση της {\tt\textlatin{y$'$=f(t,y)}} : {\tt\textlatin{ odesolve}}}\index{odesolve|textbf}
\begin{itemize}
\item Έστω $f$ μια συνάρτηση από το $\mathbb R^2$ στο $\mathbb R$.\\
 {\en\tt odesolve(f(t,y),[t,y],[t0,y0],t1)} ή\\
{\en\tt odesolve(f(t,y),t=t0..t1,y,y0)} ή\\
{\en\tt odesolve(t0..t1,f,y0)} ή\\
{\en\tt odesolve(t0..t1,(t,y)->f(t,y),y0)}\\
επιστρέφει μια προσεγγιστική τιμή του $y(t1)$ όπου $y(t)$ είναι   
η λύση του:
\[ y'(t)=f(t,y(t)), \quad  y(t0)=y0 \]
\item {\en\tt odesolve} δέχεται ένα προαιρετικό όρισμα για
τη διακριτοποίηση του {\en\tt t} ({\tt {\en tstep}=τιμή}). 
Αυτή η τιμή περνιέται σαν αρχική τιμή βήματος στον αριθμητικό επιλυτή
της {\en\tt GSL (Gnu Scientific Library)}, αλλά μπορεί  να αλλαχθεί από τον
επιλυτή. Χρησιμοποιείται επίσης για να ελέγχεται ο αριθμός των επαναλήψεων
του επιλυτή με τον τύπο {\en\tt 2*(t1-t0)/tstep} (εάν ο αριθμός των επαναλήψεων
υπερβαίνει αυτή την τιμή, ο επιλυτής θα σταματήσει σε  $t<t1$).
\item {\en\tt odesolve} δέχεται σαν προαιρετικό όρισμα την {\en\tt curve}.
Σε αυτήν την περίπτωση, 
{\en\tt odesolve} επιστρέφει την λίστα όλων των τιμών [$t,[y(t)]$]
που έχουν υπολογιστεί.
\end{itemize}
Είσοδος :
\begin{center}{\en\tt odesolve(sin(t*y),[t,y],[0,1],2)}\end{center}
ή :
\begin{center}{\en\tt odesolve(sin(t*y),t=0..2,y,1)}\end{center}
ή :
\begin{center}{\en\tt odesolve(0..2,(t,y)->sin(t*y),1)}\end{center}
ή ορίστε τη συνάρτηση :
\begin{center}{\en\tt f(t,y):=sin(t*y)} \end{center}
και εισάγετε :
\begin{center}{\en\tt odesolve(0..2,f,1)}\end{center}
Έξοδος :
\begin{center}{\en\tt [1.822413]}\end{center}
Είσοδος :
\begin{center}{\en\tt odesolve(0..2,f,1,tstep=0.3)}\end{center}
Έξοδος :
\begin{center}{\en\tt [1.822413]}\end{center}
Είσοδος :
\begin{center}{\en\tt odesolve(sin(t*y),t=0..2,y,1,tstep=0.5)}\end{center}
Έξοδος :
\begin{center}{\en\tt [1.822413]}\end{center}
Είσοδος :
\begin{center}{\en\tt odesolve(sin(t*y),t=0..2,y,1,tstep=0.5,curve)}\end{center}
Έξοδος :
\begin{center}{\en\tt [[0.0,[1.0]],[0.3906,[1.078118]],[0.7609631,[1.309724]],}\end{center}\begin{center}{\en\tt [1.070868,[1.604761]],[1.393346,[1.864171]]]}\end{center}


\subsection{Προσεγγιστική επίλυση του συστήματος {\tt\textlatin{v$'$=f(t,v)}}: \\{\tt\textlatin{ odesolve}}}\index{odesolve}
\begin{itemize}
\item Εάν $v$ είναι ένα διάνυσμα
μεταβλητών  $[x1,..,xn]$ και εάν η  $f$ δίνεται από ένα διάνυσμα παρaστάσεων
{\en\tt [e1,...,en]} ως προς $t$ και $[x1,..,xn]$,
και εάν η αρχική τιμή του $v$ στο {\en\tt t0}
είναι το διάνυσμα $[x10,...,xn0]$ τότε η εντολή
\begin{center}
{\en\tt odesolve([e1,..,en],t=t0..t1,[x1,...,xn],
[x10,...,xn0])} 
\end{center}
επιστρέφει μια προσεγγιστική τιμή του $v$ στο $t=t1$.
Με το προαιρετικό όρισμα {\en\tt curve}, η εντολή {\en\tt odesolve}  επιστρέφει την λίστα των 
ενδιάμεσων τιμών [$t,v(t)$] που υπολογίζονται από τον επιλυτή. 

Παράδειγμα, για να λύσουμε το σύστημα
{\en\tt 
\begin{eqnarray*}
x'(t) &=&-y(t)\\
y'(t)&=&x(t)
\end{eqnarray*}
}
Είσοδος :
\begin{center}
{\en\tt odesolve([-y,x],t=0..pi,[x,y],[0,1])}\end{center}
Έξοδος :
\begin{center}{\en\tt  [-1.797364e-12,-1.0]}\end{center}

\item  Εάν η $f$ είναι μια συνάρτηση από το $\mathbb R \times \mathbb R^n$ στο 
$\mathbb R^n$.\\
{\en\tt odesolve(t0..t1,(t,v)->f(t,v),v0)} ή
{\en\tt odesolve(t0..t1,f,v0)}\\
υπολογίζει μια προσεγγιστική τιμή της $v(t1)$ όπου το διάνυσμα $v(t)$
στο $\mathbb R^n$ είναι η λύση της
\[ v'(t)=f(t,v(t)), \quad v(t0)=v0 \]
Με το προαιρετικό όρισμα {\en\tt curve}, η {\en\tt odesolve} επιστρέφει τη λίστα των   
ενδιάμεσων τιμών [$t,v(t)$] που υπολογίζεται από τον {\en\tt solver}. 

Παράδειγμα, για να λύσουμε το σύστημα :\\
{\en\tt 
\begin{eqnarray*}
x'(t) &=&-y(t)\\
y'(t)&=&x(t)
\end{eqnarray*}
}
Είσοδος :
\begin{center}{\en\tt odesolve(0..pi,(t,v)->[-v[1],v[0]],[0,1])}\end{center}
ή ορίστε την συνάρτηση:
\begin{center}{\en\tt f(t,v):=[-v[1],v[0]]}\end{center}
και μετά εισάγετε : 
\begin{center}{\en\tt odesolve(0..pi,f,[0,1])}\end{center}
Έξοδος :
\begin{center}{\en\tt  [-8.931744e-14,-1.0]}\end{center}
Εναλλακτικά εισάγετε :
\begin{center}{\en\tt odesolve(0..pi/4,f,[0,1],curve)}\end{center}
Έξοδος:
\begin{center}{\en\tt  [[0.0,[0.0,1.0]],[0.2781,[-0.2745291,0.9615788]], [0.4781,[-0.4600931,0.8878707]], [0.6781,[-0.6273145,0.778766]], [0.7853982,[-0.7071068,0.7071068]]]}\end{center}
\end{itemize}


\section{Επίλυση εξισώσεων με {\tt\textlatin{ fsolve nSolve}}}\index{fsolve}\index{nSolve}
\noindent {\en\tt fsolve} ή {\en\tt nSolve} λύνει αριθμητικές εξισώσεις
(σε αντίθεση με την {\en\tt solve} ή την {\en\tt proot}, δεν περιορίζεται σε πολυωνυμικές
εξισώσεις) της μορφής:
\[ f(x)=0, \quad x \in ]a,b[ \]
{\en\tt fsolve} ή  {\en\tt nSolve} δέχεται ένα τελευταίο προαιρετικό όρισμα,
το όνομα ενός επαναληπτικού αλγορίθμου  που χρησιμοποιείται από τον επιλυτή.
Οι διαφορετικές μέθοδοι εξηγούνται στην συνέχεια.

\subsection{{\tt\textlatin{fsolve}} ή {\tt\textlatin{nSolve}} με την επιλογή {\tt\textlatin{ bisection\_solver}}}\index{bisection\_solver@{\sl bisection\_solver}|textbf}
Αυτός ο αλγόριθμος διχοτόμησης είναι ο απλούστερος αλλά γενικά ο
πιο αργός. 
Περικλείει  την  ρίζα μιας συνάρτησης σε ένα διάστημα. 
Κάθε επανάληψη, χωρίζει το διάστημα σε δύο τμήματα. Υπολογίζουμε την τιμή στο μεσαίο σημείο . 
Το πρόσημο της συνάρτησης σε αυτό το σημείο, μας δίνει το ημιδιάστημα
στο οποίο θα γίνει η επόμενη επανάληψη.\\
Είσοδος :
\begin{center}{\en\tt fsolve((cos(x))=x,x,-1..1,bisection\_solver)}\end{center}
Έξοδος :
\begin{center}{\en\tt [0.739085078239,0.739085137844]}\end{center}

\subsection{{\tt\textlatin{fsolve}} ή {\tt\textlatin{nSolve}} με την επιλογή {\tt\textlatin{ brent\_solver}}}\index{brent\_solver{\sl brent\_solver}|textbf}\index{color@{\sl }|textbf}
Η μέθοδος {\en\tt Brent} παρεμβάλλει την $f$ σε τρία σημεία , βρίσκει την τομή
της παρεμβολής με τον άξονα των $x$, υπολογίζει το πρόσημο
της $f$ σε αυτό το σημείο και διαλέγει το διάστημα όπου το πρόσημο αλλάζει.
Είναι γενικά γρηγορότερη από την διχοτόμηση.\\
Είσοδος :
\begin{center}{\en\tt fsolve((cos(x))=x,x,-1..1,brent\_solver)}\end{center}
Έξοδος :
\begin{center}{\en\tt [0.739085133215,0.739085133215]}\end{center}

\subsection{{\tt\textlatin{fsolve}} ή {\tt\textlatin{nSolve}} με την επιλογή {\tt\textlatin{ falsepos\_solver}}}\index{falsepos\_solver{\sl falsepos\_solver}|textbf}
Ο αλγόριθμος {\en\tt "false position"} είναι ένα επαναληπτικός αλγόριθμος βασισμένος  στη γραμμική
παρεμβολή : υπολογίζουμε τη τιμή της $f$ στην τομή της γραμμής 
$(a,f(a))$, $(b,f(b))$ με τον άξονα των $x$ . Η τιμή μας δίνει το τμήμα του  
διαστήματος που περιέχει την ρίζα , και στο οποίο εκτελείται μια νέα επανάληψη.\\
Η σύγκλιση είναι γραμική αλλά  γενικά γρηγορότερη από την διχοτόμηση.\\
Είσοδος :
\begin{center}{\en\tt fsolve((cos(x))=x,x,-1..1,falsepos\_solver)}\end{center}
Έξοδος :
\begin{center}{\en\tt [0.739085133215,0.739085133215]}\end{center}

\subsection{{\tt\textlatin{fsolve}} ή {\tt\textlatin{nSolve}} με την επιλογή {\tt\textlatin{ newton\_solver}}}\index{newton\_solver{\sl newton\_solver}|textbf}
{\en\tt newton\_solver} είναι η βασική μέθοδος {\en\tt Newton}.
Ο αλγόριθμος ξεκινάει με αρχική τιμή  $x_0$, μετά βρίσκουμε την τομή
 $x_1$, της εφαπτομένης του $x_0$ στον γράφο της $f$, με τον άξονα των $x$ , και 
η επόμενη επανάληψη γίνεται με  $x_1$ αντί για  $x_0$.
Η ακολουθία των σημείων $x_i$ ορίζεται από
\[ x_0=x_0, \quad x_{n+1}=x_n-\frac{f(x_n)}{f'(x_n)} \]
Εάν η μέθοδος {\en\tt Newton} συγκλίνει, η σύγκλιση είναι  τετραγωνική για  
ρίζες πολλαπλότητας  1.\\
Είσοδος :
\begin{center}{\en\tt fsolve((cos(x))=x,x,0,newton\_solver)}\end{center}
Έξοδος :
\begin{center}{\en\tt 0.739085133215}\end{center}

\subsection{{\tt\textlatin{fsolve}} ή {\tt\textlatin{nSolve}} με την επιλογή {\tt\textlatin{ secant\_solver}}}\index{secant\_solver{\sl secant\_solver}|textbf}
Η μέθοδος της τέμνουσας ({\en\tt secant}) είναι απλοποιημένη έκδοση της μεθόδου {\en\tt Newton}.
Ο υπολογισμός του  $x_1$ γίνεται χρησιμοποιώντας την μέθοδο {\en\tt Newton}.
Ο υπολογισμός του $f'(x_n), n>1$ γίνεται προσεγγιστικά. 
Η μέθοδος χρησιμοποιείται όταν ο υπολογισμός
της παραγώγου κοστίζει ακριβά:
\[ x_{i+1} = x_i-\frac{ f(x_i)}{f'_{est}}, \quad 
f'_{est} = \frac{f(x_i) - f(x_{i-1})}{(x_i - x_{i-1})}
\]
Η σύγκλιση για ρίζες πολλαπλότητας 1
είναι της τάξης $(1 + \sqrt5)/2 \approx 1.62... $.\\
Είσοδος :
\begin{center}{\en\tt fsolve((cos(x))=x,x,-1..1,secant\_solver)}\end{center}
Έξοδος :
\begin{center}{\en\tt 0.739085133215}\end{center}
Είσοδος :
\begin{center}{\en\tt fsolve((cos(x))=x,x,0,secant\_solver)}\end{center}
Έξοδος :
\begin{center}{\en\tt 0.739085133215}\end{center}

\subsection{{\tt\textlatin{fsolve}} ή {\tt\textlatin{nSolve}} με την επιλογή {\tt\textlatin{ steffenson\_solver}}}\index{steffenson\_solver{\sl steffenson\_solver}|textbf}
Η μέθοδος {\en\tt Steffenson} είναι γενικά η γρηγορότερη μέθοδος.\\
Συνδυάζει την μέθοδο {\en\tt Newton} με μια $\Delta^{2}$ {\en\tt Aitken} επιτάχυνση : 
με την μέθοδο {\en\tt Newton} , παίρνουμε την ακολουθία $x_i$ and και η επιτάχυσνη της σύγκλισης δίνει
την ακολουθία {\en\tt Steffenson}  
\[ R_i =x_i - \frac{(x_{i+1} - x_i)^2}{ (x_{i+2} - 2 x_{i+1} + x_{i})} \]
Είσοδος :
\begin{center}{\en\tt fsolve(cos(x)=x,x,0,steffenson\_solver)}\end{center}
Έξοδος :
\begin{center}{\en\tt  0.739085133215}\end{center}

\section{Επίλυση συστημάτων με {\tt\textlatin{ fsolve}}}\index{fsolve}
Tο {\en\tt Xcas} παρέχει έξι μεθόδους (από την {\en\tt GSL})
για την επίλυση αριθμητικών συστημάτων εξισώσεων 
της μορφής $f(x)=0$:
\begin{itemize}
\item Τρεις μέθοδοι χρησιμοποιούν τον  Ιακωβιανό πίνακα ({\en\tt jacobian}) $f'(x)$ και τα ονόματά τους τελειώνουν σε 
{\en\tt j\_solver}. 
\item
Οι άλλες τρεις  μέθοδοι προσεγγίζουν την  $f'(x)$ και χρησιμοποιούν μόνο
$f$.
\end{itemize}
Όλες οι μέθοδοι χρησιμοποιούν μια επανάληψη τύπου {\en\tt Newton} 
\[ x_{n+1}=x_n-{f'(x_n)}^{-1}*f(x_n) \]
Οι τέσσερις μέθοδοι {\en\tt hybrid*\_solver} χρησιμοποιούν επίσης μια μέθοδο
μεγίστης καθόδου  όταν η επανάληψη {\en\tt Newton} θα έκανε ένα πολύ μεγάλο βήμα.
Το μήκος του βήματος υπολογίζεται χωρίς κλιμάκωση  
για {\en\tt hybrid\_solver} και {\en\tt hybridj\_solver}
ή με κλιμάκωση (υπολογίζεται από $f'(x_n)$) για \\
{\en\tt hybrids\_solver} και {\en\tt hybridsj\_solver}.

\subsection{{\tt\textlatin{fsolve}} με την επιλογή {\tt\textlatin{ dnewton\_solver}}}\index{dnewton\_solver{\sl dnewton\_solver}|textbf}
\noindent{
 Είσοδος :
\begin{center}{\en\tt fsolve([x\verb|^|2+y-2,x+y\verb|^|2-2],[x,y],[2,2],dnewton\_solver)}\end{center}
Έξοδος  :
\begin{center}{\en\tt [1.0,1.0]}\end{center}
}

\subsection{{\tt\textlatin{fsolve}} με την επιλογή {\tt\textlatin{ hybrid\_solver}}}\index{hybrid\_solver{\sl hybrid\_solver}|textbf}
\noindent{
 Είσοδος :
\begin{center}{\en\tt fsolve([x\verb|^|2+y-2,x+y\verb|^|2-2],[x,y],[2,2],}\end{center}
\begin{center}{\en\tt cos(x)=x,x,0,hybrid\_solver)}\end{center}
Έξοδος  :
\begin{center}{\en\tt [1.0,1.0]}\end{center}
}

\subsection{{\tt\textlatin{fsolve}} με την επιλογή {\tt\textlatin{ hybrids\_solver}}}\index{hybrids\_solver{\sl hybrids\_solver}|textbf}
\noindent{
 Είσοδος :
\begin{center}{\en\tt fsolve([x\verb|^|2+y-2,x+y\verb|^|2-2],[x,y],[2,2],hybrids\_solver)}\end{center}
Έξοδος  :
\begin{center}{\en\tt [1.0,1.0]}\end{center}
}

\subsection{{\tt\textlatin{fsolve}} με την επιλογή {\tt\textlatin{ newtonj\_solver}}}\index{newtonj\_solver{\sl newtonj\_solver}|textbf}
\noindent{
 Είσοδος :
\begin{center}{\en\tt fsolve([x\verb|^|2+y-2,x+y\verb|^|2-2],[x,y],[0,0],newtonj\_solver)}\end{center}
Έξοδος  :
\begin{center}{\en\tt [1.0,1.0]}\end{center}
}

\subsection{{\tt\textlatin{fsolve}} με την επιλογή {\tt\textlatin{ hybridj\_solver}}}\index{hybridj\_solver{\sl hybridj\_solver}|textbf}
\noindent{
 Είσοδος :
\begin{center}{\en\tt fsolve([x\verb|^|2+y-2,x+y\verb|^|2-2],[x,y],[2,2],hybridj\_solver)}\end{center}
Έξοδος  :
\begin{center}{\en\tt  [1.0,1.0]}\end{center}
}

\subsection{{\tt\textlatin{fsolve}} με την επιλογή {\tt\textlatin{ hybridsj\_solver}}}\index{hybridsj\_solver{\sl hybridsj\_solver}|textbf}
\noindent{
 Είσοδος :
\begin{center}{\en\tt fsolve([x\verb|^|2+y-2,x+y\verb|^|2-2],[x,y],[2,2],hybridsj\_solver)}\end{center}
Έξοδος  :
\begin{center}{\en\tt  [1.0,1.0]}\end{center}
}

\section{Αριθμητικές ρίζες πολυωνύμων : {\tt\textlatin{ proot}}}\index{proot}
\noindent{{\en\tt proot} παίρνει σαν όρισμα ένα {\en\tt squarefree} πολυώνυμο,
είτε σε συμβολική μορφή είτε σαν μια λίστα
πολυωνυμικών συντελεστών (γραμμένων σε φθίνουσα σειρά).\\
{\en\tt proot} επιστρέφει μια λίστα των αριθμητικών ριζών αυτού του πολυωνύμου.}\\
Για να βρείτε τις αριθμητικές ρίζες της $P(x)=x^3+1$, εισάγετε :
\begin{center}{\en\tt proot([1,0,0,1]) }\end{center}
ή :
\begin{center}{\en\tt proot(x\verb|^|3+1) }\end{center}
Έξοδος  :
\begin{center}{\en\tt [-1.0,0.5+0.866025403784*i,0.5-0.866025403784*i]}\end{center}
Για να βρείτε τις αριθμητικές ρίζες του $x^2-3$, εισάγετε :
\begin{center}{\en\tt proot([1,0,-3])}\end{center}
ή  :
\begin{center}{\en\tt proot(x\verb|^|2-3)}\end{center}
Έξοδος  :
\begin{center}{\en\tt [-1.73205080757,1.73205080757]}\end{center} 
%proot([1,0,-15,0,90,0,-270,0,405,0,-243])


\section{Αριθμητική παραγοντοποίηση πίνακα : \\{\tt\textlatin{ cholesky qr lu svd}}}
Αριθμητικές παραγοντοποιήσεις πίνακα του
 {\en\tt
\begin{itemize}
\item Cholesky,
\item QR,
\item LU,
\item svd,
\end{itemize}
}
περιγράφονται στην ενότητα \ref{sec:factormatrice}.

\chapter{Μονάδες και φυσικές σταθερές}\label{sec:unit}
Το μενού {\tt Φυσ} περιλαμβάνει:
\begin{itemize}
\item τις φυσικές σταθερές (στο υπομενού {\tt Σταθερά}), 
\item τις  συναρτήσεις για μετατροπές μονάδων
(στο υπομενού {\tt Μετατροπέας}), 
\item τα διάφορα προθέματα μονάδων (στο υπομενού {\tt Προθέματα Μονάδων}) 
\item τις μονάδες οργανομένες κατά κατηγορία
\end{itemize}

\section{Μονάδες}
\subsection{Συμβολισμός των μονάδων}\index{\_|textbf}
Ένα αντικείμενο μονάδας έχει δύο μέρη : έναν πραγματικό αριθμό και την παράσταση της μονάδας (μιας μόνο μονάδας  ή ενός πολλαπλασιαστικού συνδυασμού --- διαφόρων --- μονάδων). Τα δύο μέρη συνδέονται με τον
χαρακτήρα {\en\tt \_} ({\en\tt underscore}). Για παράδειγμα γράφουμε {\en\tt 2\_m} για 2 μέτρα.
Για σύνθετες μονάδες, πρέπει να χρησιμοποιηθούν παρενθέσεις, π.χ. {\en\tt 1\_(m*s)}.\\
Εάν ένα πρόθεμα μπει μπροστά από την μονάδα τότε η μονάδα  πολλαπλασιάζεται επί μία δύναμη του
10. Για παράδειγα {\en\tt k} ή {\en\tt K} για {\en\tt kilo} (υποδηλώνουν έναν πολλαπλασιασμό επί
$10^3$), {\en\tt D} για {\en\tt deca} (υποδηλώνουν έναν πολλαπλασιασμό επί $10$), {\en\tt d} για 
{\en\tt deci} (υποδηλώνουν ένα πολλαπλασιασμό επί $10^{-1}$) κλπ...\\ 
Είσοδος :
\begin{center}{\en\tt 10.5\_m}\end{center}
Έξοδος :
\begin{center}{\tt ένα αντικείμενο μονάδας με τιμή 10.5 μέτρα }\end{center}
Είσοδος :
\begin{center}{\en\tt 10.5\_km}\end{center}
Έξοδος :
\begin{center}{\tt ένα αντικείμενο μονάδας με τιμή 10.5 χιλιόμετρα}\end{center}

\subsection{Υπολογισμός με  μονάδες}
Το {\en\tt Xcas} εκτελεί τις συνήθεις αριθμητικές πράξεις ({\en\tt +, -, *, /, \verb|^|}) σε
αντι\-κείμενα μονάδας. Διαφορετικές μονάδες μπορεί να χρησιμοποιηθούν, αλλά πρέπει να είναι συμβατές
για + και -. Το αποτέλεσμα είναι ένα αντικείμενο μονάδας
\begin{itemize}
\item για τον 
πολλαπλασιασμό και την διαίρεση δύο αντικειμένων μονάδας  {\en\tt \_u1} και {\en\tt \_u2} η μονάδα του αποτελέσματος γράφεται
{\en\tt \_(u1*u2)} ή {\en\tt \_(u1/u2)}. 
\item  για την πρόσθεση ή την αφαίρεση συμβατών αντικειμένων μονάδας, 
το αποτέλεσμα εκφράζεται με την ίδια μονάδα που εκφράζεται ο {\tt πρώτος} όρος της πράξης.
\end{itemize}
Είσοδος :
\begin{center}{\en\tt 1\_m+100\_cm}\end{center}
Έξοδος :
\begin{center}{\en\tt 2\_m}\end{center}
Είσοδος :
\begin{center}{\en\tt 100\_cm+1\_m}\end{center}
Έξοδος :
\begin{center}{\en\tt 200\_cm}\end{center}
Είσοδος :
\begin{center}{\en\tt 1\_m*100\_cm}\end{center}
Έξοδος :
\begin{center}{\en\tt 1\_m\verb|^|2}\end{center}

\subsection{Μετατροπή μονάδων σε μονάδες {\tt\textlatin{MKSA}} : {\tt\textlatin{ mksa}}}\index{mksa}
\noindent{{\en\tt mksa} μετατρέπει ένα αντικείμενο μονάδας σε ένα αντικείμενο μονάδας
γραμμένο στο δειθνές μετρικό  σύστημα {\en\tt MKSA}.}\\ 
\noindent{
 Είσοδος :
\begin{center}{\en\tt mksa(15\_C)}\end{center}
Έξοδος :
\begin{center}{\en\tt 15.0\_(s*A)}\end{center}
}

\subsection{Μετατροπή μονάδων : {\tt\textlatin{ convert}}}\index{convert}\label{sec:convertunit}
\noindent{{\en\tt convert} μετατρέπει μονάδες : το πρώτο όρισμα είναι ένα αντικείμενο μονάδας και το δεύτερο όρισμα είναι η νέα μονάδα  (που πρέπει να είναι συμβατή).}\\
Είσοδος :
\begin{center}{\en\tt convert(1\_h,\_s) }\end{center}
Έξοδος :
\begin{center}{\en\tt 3600\_s}\end{center}
Είσοδος :
\begin{center}{\en\tt convert(3600\_s,\_h) }\end{center}
Έξοδος :
\begin{center}{\en\tt 1\_h}\end{center}

\subsection{Παραγοντοποίηση μονάδων : {\tt\textlatin{ ufactor}}}\index{ufactor|textbf}
\noindent{{\en\tt ufactor} παραγοντοποιεί μια μονάδα σε ένα αντικείμενο μονάδας : το πρώτο 
όρισμα είναι ένα αντικείμενο μονάδας και το δεύτερο όρισμα είναι η μονάδα προς παραγοντοποίηση.\\ 
Το αποτέλεσμα είναι ένα αντικείμενο μονάδας που πολλαπλασιάζεται με τις υπόλοιπες μονάδες {\en\tt MKSA}.}\\
Είσοδος :
\begin{center}{\en\tt ufactor(3\_J,\_W) }\end{center}
Έξοδος :
\begin{center}{\en\tt 3\_(W*s)}\end{center}
Είσοδος :
\begin{center}{\en\tt ufactor(3\_W,\_J) }\end{center}
Έξοδος :
\begin{center}{\en\tt 3\_(J/s)}\end{center}

\subsection{Απλοποίηση μονάδων : {\tt\textlatin{ usimplify}}}\index{usimplify}
\noindent{{\en\tt usimplify} απλοποιεί μία μονάδα σε ένα αντικείμενο μονάδας.}\\
Είσοδος :
\begin{center}{\en\tt usimplify(3\_(W*s))}\end{center}
Έξοδος :
\begin{center}{\en\tt 3\_J}\end{center}

\subsection{Προθέματα μονάδων}
Μπορείτε να εισάγετε ένα  πρόθεμα μονάδας μπροστά από μία μονάδα για να υποδηλώσετε μία δύναμη του 10.\\
Ο ακόλουθος πίνακας δίνει τα διαθέσιμα προθέματα:
{\en\tt 
\begin{center}
\begin{tabular}{|l|c|r||l|c|r|}
\hline
Prefix & Name & (*10\verb|^|) n & Prefix & Name & (*10\verb|^|) n \\
\hline
Y & yota & 24 & d & deci & -1\\
Z & zeta & 21 & c & cent & -2\\
E & exa & 18 & m & mili & -3\\
P & peta & 15 & mu & micro &-6\\
T & tera & 12 & n & nano & -9\\
G & giga & 9 & p & pico & -12\\
M & mega & 6 & f & femto & -15\\
k or K & kilo & 3 & a & atto & -18\\
h or H & hecto & 2 & z & zepto & -21\\
D & deca & 1 & y & yocto &-24\\
\hline
\end{tabular}
\end{center}
}
{\bf Σχόλιο}\\
Δεν μπορείτε να χρησιμοποιήσετε ένα πρόθεμα με μία ενσωματωμένη μονάδα εάν το αποτέλεσμα δίνει μία άλλη 
ενσωματωμένη μονάδα.\\
Για παράδειγμα, 
{\en\tt 1\_a} , αλλά {\en\tt 1\_Pa} είναι ένα {\en\tt pascal} και όχι
{\en\tt 10\verb|^|15\_a}.
 
\section{Σταθερές}
\subsection{Συμβολισμός φυσικών σταθερών}\index{\_}
Εάν θέλετε να χρησιμοποιήσετε μια φυσική σταθερά μέσα στο {\en\tt Xcas}, βάλτε
το όνομά της μεταξύ δύο χαρακτήρων {\en\tt \_} 
({\en\tt underscore}). Μη συγχέετε τις φυσικές σταθερές με τις συμβολικές σταθερές, 
για παράδειγμα, $e,\pi$ είναι συμβολικές σταθερές ενώ {\en\tt \_c\_,\_NA\_} είναι φυσικές
σταθερές.\\
Είσοδος :
\begin{center}{\en\tt \_c\_ }\end{center}
Έξοδος, η ταχύτητα του φωτός στο κενό :
\begin{center}{\en\tt 299792458\_m*s\verb|^|-1}\end{center}
Είσοδος:
\begin{center}{\en\tt \_NA\_ }\end{center}
Έξοδος, αριθμός {\en\tt Avogadro} :
\begin{center}{\en\tt 6.0221367e23\_gmol\verb|^|-1}\end{center}

\subsection{Βιβλιοθήκη Σταθερών}
Οι φυσικές σταθερές είναι στο μενού {\tt Φυσ}, στο υπομενού {\tt Σταθερά}.
% ή επίσης στο μενού {\en\tt Help}.\\ 
Ο ακόλουθος πίνακας δίνει την Βιβλιοθήκη Σταθερών  :

\begin{center}
\begin{tabular}{|l|l|}
\hline
Όνομα & Περιγραφή\\
\hline
{\en\tt \_NA\_} & {\tt  αριθμός \en Avogadro}\\
{\en\tt \_k\_} & {\tt  σταθερά \en Boltzmann}\\
{\en\tt \_Vm\_} & {\tt γραμμομοριακός όγκος}\\
{\en\tt \_R\_} & {\tt παγκόσμια σταθερά αερίων}\\
{\en\tt \_StdT\_} & {\tt κανονική θερμοκρασία} \\
{\en\tt \_StdP\_} & {\tt κανονική πίεση}\\
{\en\tt \_sigma\_} & {\tt σταθερά \en Stefan-Boltzmann}\\
{\en\tt \_c\_} & {\tt ταχύτητα του φωτός στο κενό}\\
{\en\tt \_epsilon0\_} & {\tt διηλεκτρική σταθερά}\\
{\en\tt \_mu0\_} &{\tt  μαγνητική διαπερατότητα}\\
{\en\tt \_g\_} &{\tt επιτάχυνση της βαρύτητας}\\
{\en\tt \_G\_} &{\tt σταθερά βαρύτητας}\\
{\en\tt \_h\_} &{\tt σταθερά \en Planck}\\
{\en\tt \_hbar\_} &{\tt σταθερά \en Dirac}\\
{\en\tt \_q\_} &{\tt φορτίο ηλεκτρονίου}\\
{\en\tt \_me\_} &{\tt μάζα ηρεμίας ηλεκτρονίου}\\
{\en\tt \_qme\_} &{\tt {\en q/me} (φορτίο ηλεκτρονίου/μάζα)} \\
{\en\tt \_mp\_} &{\tt μάζα ηρεμίας πρωτονίου}\\
{\en\tt \_mpme\_} &{\tt {\en mp/me} (μάζα πρωτονίου/μάζα ηλεκτρονίου)} \\
{\en\tt \_alpha\_} &{\tt σταθερά λεπτής υφής}\\
{\en\tt \_phi\_} & {\tt κβάντο μαγνητικής ροής}\\
{\en\tt \_F\_} & {\tt  σταθερά \en Faraday}\\
{\en\tt \_Rinfinity\_} & {\tt  σταθερά \en Rydberg}\\
{\en\tt \_a0\_} &{\tt  ακτίνα \en Bohr}\\
{\en\tt \_muB\_} &{\tt σταθερά μαγνητικής ορμής ηλεκτρονίου \en Bohr magneton}\\
{\en\tt \_muN\_} &{\tt σταθερά μαγνητικής ορμής \en nuclear magneton}\\
{\en\tt \_lambda0\_} & {\tt μήκος κύματος φωτονίου\en (ch/e)}\\
{\en\tt \_f0\_} &{\tt συχνότητα φωτονίου\en (e/h)}\\
{\en\tt \_lambdac\_} &{\tt μήκος κύματος \en Compton} \\
{\en\tt \_rad\_} &{\tt 1 ακτίνιο}\\
{\en\tt \_twopi\_} &{\tt {\en 2*pi} ακτίνια}\\
{\en\tt \_angl\_} &{\tt γωνία 180 μοιρών}\\
{\en\tt \_c3\_} &{\tt σταθερά μετατόπισης \en Wien}\\
{\en\tt \_kq\_} & {\tt {\en k/q (Boltzmann}/φορτίο ηλεκτρονίου)}\\
{\en\tt \_epsilon0q\_} &{\tt {\en epsilon0/q }(διηλεκτρική σταθερά / φορτίο ηλεκτρονίου)}\\
{\en\tt \_qepsilon0\_} &{\tt {\en q*epsilon0 }(διηλεκτρική σταθερά * φορτίο ηλεκτρονίου)}\\
{\en\tt \_epsilonsi\_} &{\tt διηλεκτρική σταθερά πυριτίου}\\
{\en\tt \_epsilonox\_} &{\tt διηλεκτρική σταθερά διοξειδίου του πυριτίου}\\
{\en\tt \_I0\_} &{\tt ένταση αναφοράς}\\
\hline
\end{tabular}
\end{center}

Για να πάρετε την τιμή της σταθεράς, εισάγετε το όνομα της σταθεράς στη γραμμή εντολών 
του {\en\tt Xcas} και πατείστε {\en\tt enter} (μην ξεχάσετε να βάλετε 
{\en\tt \_} στην αρχή και στο τέλος του ονόματος της σταθεράς).


\en



\end{document}
