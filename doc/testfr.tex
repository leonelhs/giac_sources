% Pour compiler ce fichier, vous devez d'abord installer
% hevea: http://hevea.inria.fr
% giac.tex: http://www-fourier.ujf-grenoble.fr/~parisse/giac/giac.tex
% copier hevea.sty dans le repertoire courant
% ensuite vous pouvez tester avec la commande hevea test 
\documentclass[a4paper,11pt]{article}
%\textwidth 11,8 cm
%\textheight 17 cm
\textheight 23 cm
\usepackage{graphicx}
\usepackage{amsmath}
\usepackage{amsfonts}
\usepackage{amssymb}
\usepackage{stmaryrd}
\usepackage{makeidx}
\usepackage{times}
\usepackage[utf8]{inputenc}
\usepackage[T1]{fontenc}
\usepackage[francais]{babel}
\usepackage{latexsym}
\usepackage{graphicx}
%\usepackage{pst-plot}
\usepackage{ifpdf}
\ifpdf
 \usepackage[pdftex,colorlinks]{hyperref}
\else
 \usepackage[ps2pdf,
            breaklinks=true,
            colorlinks=true,
            linkcolor=red,
            citecolor=green
            ]{hyperref}
\fi

\input{giacfr.tex}
%HEVEA\htmlfoot{Retour \`a la page principale de \ahref{http://www-fourier.ujf-grenoble.fr/\~parisse/giac_fr.html}{Giac/Xcas}.}
%HEVEA\htmlhead{Retour \`a la page principale de \ahref{http://www-fourier.ujf-grenoble.fr/\~parisse/giac_fr.html}{Giac/Xcas}.}


\newtheorem{thm}{Théorème}
\newtheorem{defn}[thm]{D\'efinition}
\newtheorem{prop}[thm]{Proposition}
\newtheorem{lemma}[thm]{Lemme}
\newtheorem{example}[thm]{Exemple}


\newcommand{\R}{{\mathbb{R}}}
\newcommand{\C}{{\mathbb{C}}}
\newcommand{\Z}{{\mathbb{Z}}}
\newcommand{\N}{{\mathbb{N}}}
\newcommand{\Q}{{\mathbb{Q}}}
\newcommand{\tr}{\mbox{tr\,}}


\title {Exemple d'utilisation de giac dans un fichier \LaTeX.}
\author{B. Parisse\\Institut Fourier\\UMR 5582 du CNRS
\\Université de Grenoble I}

\date{Juillet 2015}


\begin{document}
\maketitle

Ce source \LaTeX\ illustre l'utilisation du moteur de calcul formel Giac
lorsqu'on le compile en HTML avec {\tt hevea} (test\'e avec \verb|hevea 2.23|).
La commande \verb|\loadgiacjs| ou \verb|\loadgiacjsonline|
doit figurer une fois dans le document, selon que l'on va utiliser le moteur
de calcul \verb|giac.js| depuis une installation de Xcas sur le disque dur
ou par t\'el\'echargement sur le serveur de l'Institut Fourier.
%\loadgiacjs
\loadgiacjsonline

Une commande en ligne avec r\'eponse en mode texte ou sous forme de graphique 2d 
avec \verb|\giacinput|, ici \verb|\giacinput{factor(x^4-1)}|~:\\
\giacinput{factor(x^4-1)}\\
La m\^eme avec un argument de style optionnel de la ligne de commande\\ 
\verb|\giacinput[style="width:200px;height:20px;font-size:large"]{plot(sin(x))}|\\
\giacinput[style="width:200px;height:20px;font-size:large"]{plot(sin(x))}

Un bouton avec une commande qui sera appliqu\'ee \`a un argument avec
\verb|\giaccmd|, ici \verb|\giaccmd{factor}{x^4-1}|~:\\
\giaccmd{factor}{x^4-1}\\
accepte aussi un argument de style optionnel~:
\giaccmd[style="width:200px;height:20px;font-size:large"]{factor}{x^4-1}

La m\^eme chose avec un programme ou tout autre commande de plusieurs lignes
avec \verb|\begin{giacprog}...\end{giacprog}|~:
\begin{giacprog}
f(x):={
  local y;
  si x<0 alors y:=-x; sinon y=x; fsi;
  return y;
}
\end{giacprog}

Une commande en ligne avec r\'eponse en MathML avec
\verb|\giacinputmath{}| ou \verb|\giaccmdmath{}{}|~:
\giacinputmath{factor(x^10-1)}
\giacinputmath[style="width:200px;height:20px;font-size:large"]{factor(x^10-1)}
\giaccmdmath{factor}{x^4-1}
\giaccmdmath[style="width:200px;height:20px;font-size:large"]{factor}{x^4-1}

Une commande hors-ligne avec r\'eponse en MathML
avec \verb|\giacinputbigmath{}| ou \verb|\giaccmdbigmath{}{}|~:
\giacinputbigmath{factor(x^100-1)}
\giacinputbigmath[style="width:600px;height:20px;font-size:large"]{factor(x^100-1)}
\giaccmdbigmath{factor}{x^100-1}
\giaccmdbigmath[style="width:600px;height:20px;font-size:large"]{factor}{x^100-1}

Un lien vers Xcas au Soleil (appellation faisant r\'ef\'erence aux
applications dans les nuages qui n\'ecessitent beaucoup plus de
ressources serveur)
\giaclink{http://www-fourier.ujf-grenoble.fr/\%7eparisse/xcasfr.html#+1,2&+[1,2]&+{1,2}&+poly1[1,2]&+idn(3)&}

Un curseur
\giacslider{a}{-5}{5}{0.1}{0}{plot(sin(a*x))}
\end{document}
